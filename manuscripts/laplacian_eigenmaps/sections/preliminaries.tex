\section{Preliminaries}
\label{sec:setup_main_results}

We begin in Sections~\ref{subsec:regression_laplacian_eigenmaps}-\ref{subsec:laplacian_eigenmaps} by precisely defining the models (random design points, Sobolev-smooth regression functions) and methods (Laplacian eigenmaps) under consideration. Then in Section~\ref{subsec:spectral_projection}, we connect Sobolev spaces to Laplacian eigenmaps using the spectrum of a density-dependent Laplace-Beltrami operator, and show in Propositions~\ref{prop:spectral_series_estimation} and~\ref{prop:spectral_series_testing} that projection methods which use the eigenfunctions of this operator are statistically optimal.

\subsection{Nonparametric regression over Sobolev spaces}
\label{subsec:regression_laplacian_eigenmaps}

We will always operate in the usual setting of nonparametric regression with random design. We observe independent random samples $(X_1,Y_1),\ldots,(X_n,Y_n)$; the design points $X_1,\ldots,X_n$ are sampled from a distribution $P$ with support $\mc{X} \subseteq \Rd$, and the responses follow the signal plus noise model
\begin{equation}
\label{eqn:model}
Y_i = f_0(X_i) + w_i,
\end{equation}
with regression function $f_0: \mc{X} \to \Reals$, and $w_i \sim N(0,1)$ independent Gaussian noise.)
% (AG 8/4/21): Removed the comment regarding noise with variance = \sigma^2. 

We now formulate two models, which differ in the assumed nature of the support $\mc{X}$ of the design distribution $P$: the \emph{flat Euclidean} and \emph{manifold} models.

\paragraph{Flat Euclidean model.}
In Definitions~\ref{def:model_flat_euclidean}-\ref{def:zero_trace_sobolev_space}, we collect the assumptions we make when working under the flat Euclidean model. We begin by giving some regularity conditions on the design.

\begin{definition}[Flat Euclidean model]
	\label{def:model_flat_euclidean}
	 The support $\mc{X}$ of the design distribution $P$ is an open, connected, and bounded subset of $\Rd$, with Lipschitz boundary. The distribution $P$ admits a Lipschitz density $p$ with respect to the $d$-dimensional Lebesgue measure $\nu$, which is bounded away from $0$ and $\infty$,
	\begin{equation*}
	0 < p_{\min} \leq p(x) \leq p_{\max} < \infty, \quad \textrm{for all $x \in \mc{X}$.}
	\end{equation*}
\end{definition}
% (AG 8/13/21): I got rid of the sentence ``The data are sampled according to (1).'' It seemed repetitive. 
At various points we will also assume that the density $p \in C^k(\mc{X})$; when we make such assumptions we will state them explicitly.

We model the regression function as belonging to an order-$s$ Sobolev space, and being bounded in Sobolev norm.
\begin{definition}[Sobolev space on a flat Euclidean domain]
	\label{def:sobolev_space}
	For an integer $s \geq 1$, a function $f \in L^2(\mc{X})$ belongs to the Sobolev space $H^s(\mc{X})$ if for all $|\alpha| \leq s$, the weak derivatives $D^{\alpha}f$ exists and satisfy $D^{\alpha}f \in L^2(\mc{X})$. The $j$th order semi-norm for $f \in H^s(\mc{X})$ is $|f|_{H^j(\mc{X})} := \sum_{|\alpha| = j}\|D^{\alpha}f\|_{\Leb^2(\mc{X})}$, and the corresponding norm
	\begin{equation*}
	\|f\|_{H^s(\mc{X})}^2 := \|f\|_{\Leb^2(\mc{X})}^2 + \sum_{j = 1}^{s} |f|_{H^j(\mc{X})}^2,
	\end{equation*}
	induces the Sobolev ball
	\begin{equation*}
	H^s(\mc{X};M) := \bigl\{f \in H^s(\mc{X}): \|f\|_{H^s(\mc{X})} \leq M\bigr\}.
	\end{equation*} 
\end{definition}
Finally when $s > 1$ we will assume that $f_0$ satisfies a zero-trace boundary condition. Recall that $H^s(\mc{X})$ can alternatively be defined as the completion of $C^{\infty}(\mc{X})$ in the Sobolev norm $\|\cdot\|_{H^s(\mc{X})}$. The zero-trace Sobolev spaces are defined in a similar fashion, as the completion of $C_c^{\infty}(\mc{X})$ in the same norm.

\begin{definition}[Zero-trace Sobolev space]
	\label{def:zero_trace_sobolev_space}
	A function $f \in H^s(\mc{X})$ belongs to the zero-trace Sobolev space $H_0^s(\mc{X})$ if there exists a sequence $f_1,f_2,\ldots$ of functions in $C_c^{\infty}(\mc{X})$ such that
	\begin{equation*}
	\lim_{k \to \infty}\|f_k - f\|_{H^s(\mc{X})} = 0.
	\end{equation*}
	The normed ball $H_0^{s}(\mc{X};M) := H_0^{s}(\mc{X}) \cap H^{s}(\mc{X};M)$.
\end{definition}
Boundary conditions play an important role in the analysis of spectral methods, as we explain further in Section~\ref{subsec:spectral_projection}. For now, we limit ourselves to pointing out that for functions $f \in C^\infty(\mc{X})$, the zero-trace condition can be stated more concretely, as implying that $\partial^{k}f/\partial{\bf n}^k(x) = 0$ for each $k = 0,\ldots,s - 1$, and for all $x \in \partial\mc{X}$. (Here $\partial/(\partial {\bf n})$ is the partial derivative operator in the direction of the normal vector $\mathbf{n}$.)
\paragraph{Manifold model.}
As in the flat Euclidean case, we start with some regularity conditions on the design.
\begin{definition}[Manifold model]
	\label{def:model_manifold}
	The support $\mc{X}$ of the design distribution $P$ is a closed, connected, and smooth Riemannian manifold (without boundary) embedded in $\Rd$, of intrinsic dimension $1 \leq m < d$, and with a positive injectivity radius $\mathrm{inj}(\mc{X}) > 0$. The design distribution $P$ admits a Lipschitz density $p$ with respect to the volume form $d\mu$ induced by the Riemannian structure of $\mc{X}$, which is bounded away from $0$ and $\infty$,
	\begin{equation*}
	0 < p_{\min} \leq p(x) \leq p_{\max} < \infty, \quad \textrm{for all $x \in \mc{X}$.}
	\end{equation*}
\end{definition}
In the above, recall that the injectivity radius $\mathrm{inj}(\mc{X})$ of an $m$-dimensional Riemannian manifold $\mc{X}$ is the maximum $\delta > 0$ such that the exponential map $\exp_x: B_m(0,\delta) \subset T_x(\mc{X}) \to B_{\mc{X}}(x,\delta) \subset \mc{X}$ is a diffeomorphism for all $x \in \mc{X}$.

There are several equivalent ways to define Sobolev spaces on smooth Riemannian manifolds. We will stick with a definition that parallels our setup in the flat Euclidean setting as much as possible. To do so, we first recall the notion of partial derivatives on a manifold, which are defined with respect to a local coordinate system. Letting $r_1,\ldots,r_m$ be the standard basis of $\Reals^m$, for a given chart $(\phi,U)$ (meaning an open set $U \subseteq \mc{X}$, and a smooth mapping $\phi: U \to \Reals^m$) we write $\phi =: (x_1,\ldots,x_m)$ in local coordinates, meaning $x_i = r_i \circ \phi$. Then the partial derivative $\partial f/\partial x_i$ of a function $f$ with respect to $x_i$ at $x \in U$ is
\begin{equation*}
\frac{\partial f}{\partial x_i}(x) := \frac{\partial(f \circ \phi^{-1})}{\partial r_i}\bigl(\phi(x)\bigr).
\end{equation*}
The right hand side should be interpreted in the weak sense of derivative. As before, we use the multi-index notation $D^{\alpha}f := \partial^{|\alpha|}f/\partial^{\alpha_1}x_1\ldots\partial^{\alpha_m}x_m$. 

\begin{definition}[Sobolev space on a manifold]
	\label{def:sobolev_space_manifold}
	A function $f \in \Leb^2(\mc{X})$ belongs to the Sobolev space $H^{s}(\mc{X})$ if for all $\abs{\alpha} \leq s$, the weak derivatives $D^{\alpha}f$ exist and satisfy  $D^{\alpha}f \in \Leb^2(\mc{X})$. The $j$th order semi-norm $|f|_{H^j(\mc{X})}$, the norm $\|f\|_{H^s(\mc{X})}$, and the ball $H^s(\mc{X};M)$ are all defined as in Definition~\ref{def:sobolev_space}.
\end{definition}
The partial derivatives $D^{\alpha}f$ will depend on the choice of local coordinates, and so will the resulting Sobolev norm~$\|f\|_{H^s(\mc{X})}$. However, for our purposes the important point is that regardless of the choice of local coordinates the resulting norms will be equivalent\footnote{Recall that norms $\|\cdot\|_1$ and $\|\cdot\|_2$ on a space $\mc{F}$ are said to be equivalent if there exist constants $c$ and $C$ such that
	\begin{equation*}
	c \|f\|_1 \leq \|f\|_2 \leq C \|f\|_1 \quad \textrm{for all $f \in \mc{F}$.}
	\end{equation*}} 
and so the ultimate Sobolev space $H^s(\mc{X})$ is independent of the choice of local coordinates. For more information regarding manifolds and Sobolev spaces defined thereupon, see~\cite{lee2013} and~\cite{hebey1996}.

\subsection{Laplacian eigenmaps}
\label{subsec:laplacian_eigenmaps}
We now formally define the estimator and test statistic we study. Both are derived from eigenvectors of a graph Laplacian.  For a positive, symmetric kernel $\eta: [0,\infty) \to [0,\infty)$, and a radius parameter $\varepsilon > 0$, let $G = ([n],W)$ be the neighborhood graph formed over the design points $\{X_1,\ldots,X_n\}$, with a weighted edge $W_{ij} = \eta(\|X_i - X_j\|/\varepsilon)$ between vertices $i$ and $j$. Then the 
\emph{neighborhood graph Laplacian} $L_{n,\varepsilon}: \Reals^n \to \Reals$ is defined by its action on vectors $u \in \Reals^n$ as
\begin{equation}
\label{eqn:neighborhood_graph_laplacian}
\bigl(L_{n,\varepsilon}u\bigr)_i := \frac{1}{n\varepsilon^{2 + \mathrm{dim}(\mc{X})}} \sum_{j = 1}^{n} \bigl(u_i - u_j\bigr) \eta\biggl(\frac{\|X_i - X_j\|}{\varepsilon}\biggr).
\end{equation}
(Here $\mathrm{dim}(\mc{X})$ stands for the dimension of $\mc{X}$. It is equal to $d$ under the assumptions of Model~\ref{def:model_flat_euclidean}, and equal to $m$ under the assumptions of Model~\ref{def:model_manifold}. The pre-factor $(n\varepsilon^{2 + \mathrm{dim}(\mc{X})})^{-1}$ ensures non-degenerate stable limits as $n \to \infty, \varepsilon \to 0$). Written in standard coordinates we have $(n\varepsilon^{\dim(\mc{X}) + 2}) \cdot L_{n,\varepsilon} = D - W$, where $D \in \Reals^{n \times n}$ is the diagonal degree matrix, $D_{ii} = \sum_{i = 1}^{n} W_{ij}$.

The graph Laplacian is a positive semi-definite matrix, and admits the eigendecomposition $L_{n,\varepsilon} = \sum_{k = 1}^{n} \lambda_k v_k v_k^{\top}$, where for each $k = 1,\ldots,n$ the eigenvalue-eigenvector pair $(\lambda_k,v_k)$ satisfies
\begin{equation*}
L_{n,\varepsilon}v_k = \lambda_k v_k, \quad \|v_k\|_2^2 = 1.
\end{equation*}
We will assume without loss of generality that each eigenvalue $\lambda$ of $L_{n,\varepsilon}$ has algebraic multiplicity $1$, and so we can index the eigenpairs $(\lambda_1,v_1),\ldots,(\lambda_n,v_n)$ in ascending order of eigenvalue, $0 = \lambda_1 < \ldots < \lambda_n$. 

The Laplacian eigenmaps estimator $\wh{f}$ simply projects the response vector ${\bf Y} = (Y_1,\ldots,Y_n)$ onto the first $K$ eigenvectors of $L_{n,\varepsilon}$: letting $V_K \in \Reals^{n \times K}$ be the matrix with columns $V_{K,k} = v_k$, we have that
\begin{equation}
\label{eqn:laplacian_eigenmaps_estimator}
\wh{f} := \sum_{k = 1}^{K} \dotp{{\bf Y}}{v_k}_{2} v_k = V_K V_K^{\top} {\bf Y}.
\end{equation} 
If $\wh{f}$ is a reasonable estimate of $f_0$, then the Laplacian eigenmaps test statistic
\begin{equation}
\label{eqn:laplacian_eigenmaps_test}
\wh{T} := \|\wh{f}\|_n^2 = \frac{1}{n} {\bf Y}^{\top} V_K V_K^{\top} {\bf Y}
\end{equation}
is in turn a reasonable estimate of $\|f_0\|_{P}^2$, and can be used in the \emph{signal detection} problem to distinguish whether or not $f_0 = 0$.

% AG 8/12/21: Commented out due to overlap with footnote in new introduction.
% It may be helpful to comment briefly on the term ``Laplacian eigenmaps'', which we use a bit differently than is typical in the literature. Laplacian eigenmaps typically refers to an algorithm for embedding, which maps each design point $X_1,\ldots,X_n$ to $\Reals^K$ according to $X_i \mapsto (v_{1,i}, \ldots, v_{K,i})$. Viewing this embedding as a feature map, we can then interpret the estimator $\wh{f}$ as the least-squares solution to a linear regression problem with responses $Y_1,\ldots,Y_n$ and features $v_1,\ldots,v_K$. Often, the Laplacian eigenmaps embedding is viewed as a tool for dimensionality reduction, wherein it is implicitly assumed that $K$ is much smaller than $d$. We will neither explicitly nor implicitly take $K < d$; indeed, the embedding perspective is not particularly illuminating in what follows, and we do not henceforth make reference to it. Instead, we use ``Laplacian eigenmaps'' to directly refer to the estimator $\wh{f}$ or test statistic $\wh{T}$. 

\subsection{Spectral Projection using Continuum Eigenfunctions}
\label{subsec:spectral_projection}
Laplacian Eigenmaps serves as a data-dependent alternative to more classical spectral projection methods. Classically, spectral projection methods operate by projecting the response vector ${\bf Y}$ onto a subspace spanned by eigenfunctions of some differential operator. For instance, projecting ${\bf Y}$ onto the eigenfunctions of the Laplacian operator $\Delta = \sum_{i = 1}^{d} \partial^2f/\partial x_i^2$ corresponds to estimating $f_0$ using noisy empirical Fourier coefficients.

We will focus on a density-dependent altnerative to the Laplacian,  $\Delta_P:C^2(\mc{X}) \to L^2(\mc{X})$,
\begin{equation}
\label{eqn:laplace_beltrami}
\Delta_Pf := -\frac{1}{p} \mathrm{div}(p^2\nabla f),
\end{equation}
which has enumerable eigenvalue/eigenfunction pairs $(\lambda_k(\Delta_P),\psi_k),(\lambda_2(\Delta_P),\psi_2),\ldots$ defined according to
\begin{equation}
\label{eqn:laplace_beltrami_eigenproblem}
\Delta_P\psi = \lambda(\Delta_P) \psi, \quad \frac{\partial}{\partial{\bf n}}\psi = 0~~\textrm{on $\partial \mc{X}$,}
\end{equation}
and sorted as usual in ascending order of eigenvalue.\footnote{For formal justification of why~\eqref{eqn:laplace_beltrami_eigenproblem} has discrete spectrum under either Model~\ref{def:model_flat_euclidean} or~\ref{def:model_manifold}, see~\cite{garciatrillos18,trillos2019}.} Formally, the classical spectral series estimator and test statistic are\footnote{Technically speaking, $\wt{f}$ is not the projection of ${\bf Y}$ onto the subspace of $L^2(P_n)$ spanned by $\psi_1,\ldots,\psi_K$, since $\dotp{\psi_k}{\psi_{\ell}}_n \neq 0$. However, these eigenfunctions are very close to orthogonal in $L^2(P_n)$, orthogonalization will have a negligible impact on the overall statistical properties, and so for simplicity we consider the methods as defined in~\eqref{eqn:classical_spectral_projection}.}
\begin{equation}
\label{eqn:classical_spectral_projection}
\wt{f}(x) := \frac{1}{n}\sum_{k = 1}^{K} \dotp{{\bf Y}}{\psi_k}_{n} \psi_k(x),\quad\textrm{and}\quad \wt{T} := \|\wt{f}\|_n^2.
\end{equation}

We focus on the particular differential operator $\Delta_P$ because the spectrum of $\Delta_P$ is the population-level limit of graph Laplacian spectra: for any fixed $k \in \mathbb{N}$, as the number of samples $n \to \infty$ and the graph radius $\varepsilon \to 0$,
\begin{equation}
\lambda_k \to \lambda_k(\Delta_P), \quad\textrm{and}\quad \Bigl\|\frac{1}{n}v_k - \psi_k\Bigr\|_n^2 \to 0.
\end{equation}
This holds in either the flat Euclidean (Model~\ref{def:model_flat_euclidean}) or manifold (Model~\ref{def:model_manifold}) setups~\citep{garciatrillos18,trillos2019}. It follows that for any fixed number of eigenvectors $K \in \mathbb{N}$, both $\wh{f} \to \wt{f}$ and $\wh{T} \to \wt{T}$, and in this way Laplacian Eigenmaps can be formally tied to classical spectral projection methods for regression.
 
On the other hand, the eigenvectors and eigenfunctions of $\Delta_P$ can also be used to give a spectral definition of Sobolev spaces on general domains $\mc{X}$. Consider the ellipsoid
\begin{equation}
\label{eqn:sobolev_ellipsoid}
\mc{H}^{s}(\mc{X}) := \Bigl\{\sum_{k = 1}^{\infty} a_k \psi_k \in L^2(\mc{X}):  \sum_{k = 1}^{\infty} a_k^2 \lambda_{k}(\Delta_P) \leq M^2 \Bigr\},
\end{equation}
equipped with the norm $\|\sum_{k = 1}^{\infty} a_k \psi_k\|_{\mc{H}^s(\mc{X})}^2 = \sum_{k = 1}^{\infty} a_k^2 \lambda_{k}(\Delta_P)$. Under appropriate regularity conditions $\mc{H}^s(\mc{X})$ consists of functions $f \in H^s(\mc{X})$ which also satisfy some additional boundary conditions. For instance, assuming Model~\ref{def:model_flat_euclidean}, $p \in C^{\infty}(\mc{X})$ and $\partial \mc{X} \in C^{1,1}$, \citet{dunlop2020} show that for any $s \geq 1$, the ellipsoid $\mc{H}^{2s}(\mc{X})$ satisfies
\begin{equation}
\label{eqn:sobolev_ellipsoid_to_sobolev_ball}
\mc{H}^{2s}(\mc{X}) = 
\biggl\{f \in H^{2s}(\mc{X}): \frac{\partial \Delta_P^rf}{\partial {\bf n}} = 0~\textrm{on}~\partial\mc{X},~~\textrm{for all $0 \leq r \leq s - 1$} \biggr\},
\end{equation}
and likewise $\mc{H}^{2s + 1}(\mc{X}) = \mc{H}^{2s}(\mc{X}) \cap H^{2s + 1}(\mc{X})$ for any $s \geq 0$; additionally, the norms $\|\cdot\|_{\mc{H}^s(\mc{X})}$ and $\|\cdot\|_{H^s(\mc{X})}$ are equivalent.

This spectral formulation suggests that classical spectral projection methods---and in turn Laplacian Eigenmaps---are natural choices for regression over Sobolev spaces. Furthermore, since classical spectral projection methods serve, in a sense, as the infinite-data limit of Laplacian eigenmaps, the statistical properties of the former---which are generally speaking better understood and easier to derive---should intuitively shed some light on the properties of the latter. To this end, we now give a pair of results, Propositions~\ref{prop:spectral_series_estimation} and~\ref{prop:spectral_series_testing}, which establish that classical spectral projection methods are statistically optimal for nonparametric estimation and testing over Sobolev ellipsoids. These results are generalizations of previously known upper bounds on the estimation and testing error of classical spectral methods~\citep{tsybakov08,ingster2009}, but they hold under less stringent (nonparametric) conditions on the design distribution $P$. This latter point is important, since as we have argued one attractive quality of Laplacian Eigenmaps is its adaptivity to general design distributions.

\paragraph{Estimation.}
We begin with an upper bound on the $L^2(P)$ risk of $\wt{f}$.
\begin{proposition}
	\label{prop:spectral_series_estimation}
	Suppose data is observed according to Model~\ref{def:model_flat_euclidean}, and that the density $p$ is known.  Suppose additionally that $\partial \mc{X} \in C^{1,1}$, $p \in C^{\infty}(\mc{X})$, $f_0 \in \mc{H}^{s}(\mc{X};M)$ and $\|f_0\|_P^2 \leq 1$. Then there exists a constant $C$ which does not depend on $f_0,M$ or $n$ such that the following statement holds: if the spectral projection estimator $\wt{f}$ is computed with parameter $K = \floor{M^2n}^{d/(2s + d)}$, then
	\begin{equation}
	\label{eqn:spectral_series_estimation}
	\Ebb\bigl[\|\wt{f} - f_0\|_P^2\bigr] \leq C \min\Bigl\{M^2\bigl(M^2n\bigr)^{-2s/(2s + d)}, M^2\Bigr\}.
	\end{equation}
\end{proposition}
When the Sobolev ball radius $M \asymp 1$---which automatically implies $\|f_0\|_{P}^2 \lesssim 1$ as well---the upper bound in~\eqref{eqn:spectral_series_estimation} is on the order of $n^{-2s/(2s + d)}$, which matches the well-known minimax rate for estimation over Sobolev classes (see e.g.~\cite{wasserman2006,tsybakov08} and references therein). 

% (AG 8/19/21): I would like to strengthen this statement to not require $M \asymp 1$, but I do not know where I can find the rate M^2(M^2n)^{-2s/(2s + d)} for all s and d combinations. In fact, I don't even know of a paper that states the M^2(M^2n)^{-2s/(2s + d)} rate for L^2(P) risk in random design.

We now give the proof of Proposition~\ref{prop:spectral_series_estimation}. The structure of the analysis, which is fairly classical and straightforward, can be usefully compared to our analysis of Laplacian Eigenmaps (see Section~\ref{subsec:analysis}).

\paragraph{Proof of Proposition~\ref{prop:spectral_series_estimation}.}
We decompose risk into squared bias and variance,
\begin{equation}
\label{pf:spectral_series_estimation_0}
\Ebb \|\wt{f} - f_0\|_P^2 = \Ebb\| \Ebb[\wt{f}]  - f_0\|_P^2 + \Ebb\| \wt{f} - \Ebb[\wt{f}]\|_P^2.
\end{equation}
Since the eigenfunctions $\{\psi_k\}$ form an orthonormal basis of $L^2(P)$, and $f_0 \in \mc{H}^{s}(\mc{X}) \subseteq L^2(P)$, we can write the squared bias in terms of squared Fourier coefficients of $f_0$, leading to the following upper bound,
\begin{equation*}
\|f_0 - \Ebb[\wt{f}]\|_P^2 = \sum_{k = K + 1}^{\infty}  \dotp{f_0}{\psi_k}_P^2 \leq  \frac{1}{\{\lambda_{K + 1}(\Delta_P)\}^s} \sum_{k = K + 1}^{\infty} \{\lambda_{k + 1}(\Delta_P)\}^s \dotp{f_0}{\psi_k}_P^2 \leq \frac{\|f_0\|_{\mc{H}^s(\mc{X})}}{\{\lambda_{K + 1}(\Delta_P)\}^s}.
\end{equation*}
On the other hand, the variance term can be written as the sum of the variance of each empirical Fourier coefficient, and subsequently by the law of total variance we derive that
\begin{align}
\label{pf:spectral_series_estimation_2}
\Ebb\| \wt{f} - \Ebb[\wt{f}]\|_P^2 = \sum_{k = 1}^{K} \Var\Bigl[\dotp{{\bf Y}}{\psi_k}_n\Bigr] & = \sum_{k = 1}^{K} \Var\Bigl[\Ebb[\dotp{Y}{\psi_k}_n|{\bf X}\Bigr] + \Ebb\Bigl[\Var[\dotp{Y}{\psi_k}_n|{\bf X}\Bigr] \nonumber \\
& = \sum_{k = 1}^{K} \Var\Bigl[\dotp{f_0}{\psi_k}_n\Bigr] + \frac{1}{n}\Ebb\Bigl[\|\psi_k\|_n^2\Bigr] \nonumber \\
& \leq \frac{K}{n} + \frac{1}{n}\sum_{k = 1}^{K}\Ebb\Bigl[\Bigl(f_0(X)\psi_k(X)\Bigr)^2\Bigr].
\end{align}
Consequently,
\begin{equation}
\label{pf:spectral_series_estimation_1}
\Ebb \|\wt{f} - f_0\|_P^2 \leq \frac{\|f_0\|_{\mc{H}^s(\mc{X})}^2}{\bigl[\lambda_{K + 1}(\Delta_P)\bigr]^s} + \frac{K}{n} + \frac{1}{n}\Ebb\Bigl[(f_0(X))^2 \cdot \sum_{k = 1}^{K} (\psi_k(X))^2\Bigr].
\end{equation}
The claim of the proposition then follows from variants of two classical results in spectral geometry. The first is a Weyl's Law asymptotic scaling of the eigenvalues of $\Delta_P$ due to~\cite{dunlop2020}; formally, there exist constants $c$ and $C$ (which will depend on $P$ and $d$) such that
\begin{equation}
\label{eqn:weyl}
ck^{2/d} \leq \lambda_k(\Delta_P) \leq Ck^{2/d}\quad\textrm{for all $k \in \mathbb{N}$, $k \geq 2$}.
\end{equation}
The second is a local analog to Weyl's Law, which says that there exists a constant $C$ (again depending on $P$ and $d$) such that
\begin{equation}
\label{eqn:local_weyl}
\sup_{x \in \mc{X}}\biggl\{\sum_{k = 1}^{K} \bigl(\psi_k(x)\bigr)^2\biggr\} \leq CK \quad\textrm{for all $K \in \mathbb{N}$}.
\end{equation}
Equation~\eqref{eqn:local_weyl} follows straightforwardly from Theorem 17.5.3 of~\cite{hormander1973}. Plugging the upper bounds~\eqref{eqn:weyl} and~\eqref{eqn:local_weyl} back into~\eqref{pf:spectral_series_estimation_1} shows that
\begin{equation}
\label{pf:spectral_series_estimation_3}
\Ebb \|\wt{f} - f_0\|_P^2 \leq C\biggl(\frac{\|f_0\|_{\mc{H}^s(\mc{X})}^2}{(K + 1)^{2s/d}} + \frac{K}{n}\biggr),
\end{equation}
and setting $K = \floor{(1 + \|f_0\|_{P}^2)M^2n}^{d/(2s + d)}$, which balances squared bias and variance, then yields the claim.
\qed.

\paragraph{Testing.}
In the goodness-of-fit testing problem, one asks for a test function---formally, a Borel measurable function $\phi$ that takes values in $\{0,1\}$--- which can distinguish between the hypotheses
\begin{equation}
\mathbf{H}_0: f_0 = f_0^{\star}, ~~\textrm{versus}~~ \mathbf{H}_a: f_0 \in \mc{H}^{s}(\mc{X};M) \setminus \{f_0^{\star}\}.
\end{equation} 
To fix ideas, here and throughout we focus on the signal detection problem, meaning the special case where $f_0^{\star} = 0$.\footnote{This is without loss of generality since all the test statistics we consider are easily modified to handle the case when $f_0^{\ast}$ is not $0$, by simply subtracting $f_0^{\ast}(X_i)$ from each observation $Y_i$, with no change in the analysis.}

In this case, the classical spectral projection test $\wt{\varphi} = \1\{\wt{T} \geq K/N + \sqrt{2K/an}\}$ has bounded Type I error, $\Ebb_{0}[\wt{\varphi}] \leq a (1 + o(1))$. Proposition~\ref{prop:spectral_series_testing} gives an upper bound on the Type II error that holds uniformly over all $f_0 \in \mc{H}^s(\mc{X};M)$ for which $\|f_0\|_P^2$ is sufficiently large.
\begin{proposition}
	\label{prop:spectral_series_testing}
	Suppose data is observed according to Model~\ref{def:model_flat_euclidean}, and that the density $p$ is known.  Suppose additionally that $\partial \mc{X} \in C^{1,1}$, $p \in C^{\infty}(\mc{X})$, $f_0 \in \mc{H}^{s}(\mc{X};M)$ and $\|f_0\|_{\Leb^4(\mc{X})}^4 \leq 1$. Then there exists a constant $C$ which does not depend on $f_0,M$ or $n$ such that the following statement holds: if the spectral projection test $\wt{\varphi}$ is computed with parameter $K = \floor{M^2n}^{2d/(4s + d)}$, and if
	\begin{equation}
	\label{eqn:spectral_series_testing}
	\|f_0\|_P^2 \geq C\min\{M^2(M^2n)^{-4s/(4s + d)}, M^2\}
	\end{equation}
	then the Type I error is bounded, $\Ebb_{f_0}[1 - \phi] \leq b$.
\end{proposition}
Focusing again on the special case where the Sobolev ball radius $M \asymp 1$---which automatically implies $\|f_0\|_{\Leb^4(\mc{X})}^4 \lesssim 1$ when $s > d/4$, due to a Sobolev embedding theorem---the upper bound in~\eqref{eqn:spectral_series_testing} is $n^{-4s/(4s + d)}$, matching the usual minimax critical radius over Sobolev spaces (for more details on goodness-of-fit testing see~\cite{ingster2009,ingster2012}). On the other hand, when $s \leq d/4$ the problem fundamentally changes in two ways. First, Sobolev space $H^s(\mc{X})$ no longer continuously embeds into $L^4(\mc{X})$, and indeed the minimax testing rates in this case are unknown. Second, if one explicitly assumes $f_0 \in L^4(\mc{X};1)$ then the minimax critical radius for regression testing is on the order of $n^{-1/2}$ \citep{guerre02}, and is achieved by a test using the naive statistic $\|{\bf Y}\|_n^2$. In other words, the regression testing problem over Sobolev spaces fundamentally changes when $s \leq d/4$, and hereafter when we discuss testing we will limit our consideration to $s > d/4$. 
\paragraph{Proof of Proposition~\ref{prop:spectral_series_testing}.}
We briefly lay out the main ideas needed to prove Proposition~\ref{prop:spectral_series_testing}, following the lead of~\cite{ingster2009} who prove a similar result in the special case where $P$ is the uniform distribution over $\mc{X} = [0,1]^d$, and referring to that work for more details.

We begin by computing the first two moments of the test statistic $\wt{T}$. The expectation is
\begin{align*}
\Ebb[\wt{T}] = \frac{(n - 1)}{n} \sum_{k = 1}^{K} \dotp{f_0}{\psi_K}_P^2 + \frac{K}{n} + \Ebb[(f_0(X))^2 \dot \sum_{k = 1}^{K} (\psi_k(X))^2],
\end{align*}
from which we clearly see that under the alternative,
\begin{equation*}
\Ebb_{f_0}[\wt{T}] \geq \frac{\|f_0\|_{\mc{H}^s(\mc{X})}^2}{\lambda_{K + 1}(\Delta_{P})^s} + K/n.
\end{equation*}
To compute the variance, we decompose $\wt{T} = \wt{T}_{1,1} + \wt{T}_{1,2} + \wt{T}_{1,3} + \wt{T}_2$ into the sum of 3 U-statistics and the remaining diagonal terms; letting $\kappa(x,x') = \sum_{k = 1}^{K} \psi_k(x) \psi_k(x')$ be the equivalent kernel, we have that
\begin{align*}
T_{1,1} & = \frac{1}{n^2} \sum_{1 \leq i \neq j \leq n} w_i w_j \kappa(X_i,X_j),\quad && T_{1,2} = \frac{1}{n^2} \sum_{1 \leq i \neq j \leq n} \bigl(w_i f_0(X_j) + w_jf_0(X_i)\bigr)\kappa(X_i,X_j) \\
T_{1,3} & = \frac{1}{n^2} \sum_{1 \leq i \neq j \leq n} f_0(X_i) f_0(X_j) \kappa(X_i,X_j), \quad && T_2 = \frac{1}{n^2} \sum_{i = 1}^{n} Y_i^2 \kappa(X_i,X_i).
\end{align*}
The variances of each statistic can be found by routine computation (see~\cite{ingster2009}), and in particular satisfy the upper bounds
\begin{align*}
\Var(T_{1,1}) & \leq \frac{2K}{n^2}, \quad && \Var(T_{1,2}) \overset{\mathrm{(i)}}{\leq} \frac{C}{n}\|f_0\|_P^2\\
\Var(T_{1,3}) & \overset{\mathrm{(ii)}}{\leq} C\biggl(\frac{K}{n}\|f_0\|_P^4 + \frac{K}{n^2}\|f_0\|_{L^4(\mc{X})}^4\biggr), \quad && \Var(T_{2}) \overset{\mathrm{(iii)}}{\leq} \frac{CK^2}{n^3}\biggl(1 + \|f_0\|_{L^4(\mc{X})}^4\biggr)
\end{align*}
where $\mathrm{(i)}-\mathrm{(iii)}$ hold due to local Weyl's law, meaning the upper bound on $\sum_{k = 1}^{K} (\psi_k(x))^2$ given in~\eqref{eqn:local_weyl}. Upper bounds on Type I and Type II error,
\begin{equation*}
\Ebb_0[\wt{\varphi}] \leq \biggl(1 + CK/n^2\biggr)a, \quad \Ebb_{f_0}[1 - \wt{\varphi}] \leq \frac{C(K/n^2 + \|f_0\|_P^2 + K/n \|f_0\|_P^4 + K/n^2 \|f_0\|_{L^4(\mc{X})}^4)}{(\sum_{k = 1}^{K} \dotp{f_0}{\psi_k}_P^2 - \sqrt{2K/an})^2},
\end{equation*}
follow from Chebyshev's inequality. It can be verified that so long as
\begin{equation}
\label{pf:spectral_series_test}
\|f_0\|_P^2 \geq C\Biggl(\frac{\|f_0\|_{\mc{H}^s(\mc{X})}^2}{\lambda_{K + 1}(\Delta_{P})^s} + \sqrt{\frac{2K}{n}}\biggl(\sqrt{\frac{1}{a}} + \sqrt{\frac{1}{b}}\biggr)\Biggr)
\end{equation}
for a sufficiently large constant $C$, then $\Ebb_{f_0}[1 - \wt{\varphi}] \leq b$. The two summands in~\eqref{pf:spectral_series_test} are bias and standard deviation terms, respectively; choosing $K = \floor{M^2n}^{2d/(4s + d)}$ balances these two terms, and leads to~\eqref{eqn:spectral_series_testing}. \qed
% (AG 8/19/21): Same comment as above, regarding the role of M in random design regression testing.

The main takeaway from Propositions~\ref{prop:spectral_series_estimation} and~\ref{prop:spectral_series_testing} is that spectral projection methods for nonparametric regression achieve optimal rates of convergence, when the regression function $f_0$ is Sobolev smooth and the design density $p$ is known a priori and satisfies an appropriate notion of smoothness\footnote{The assumption $p \in C^{\infty}(\mc{X})$ could likely be weakened, but since this would not substantially add to the main points of Propositions~\ref{prop:spectral_series_estimation} and~\ref{prop:spectral_series_testing}, we do not pursue the details further.}. As we will see, Laplacian eigenmaps achieves comparable rates of convergence when $p$ is sufficiently smooth but potentially unknown, and in this sense both learns and leverages the design distribution in a way that more classical spectral projection methods do not.

Of course, it is worth pointing out that other methods besides Laplacian eigenmaps are statistically optimal for nonparametric regression even when $p$ is unknown. We comment more on some of these in Section~\ref{sec:discussion}, after we have derived our major results regarding Laplacian Eigenmaps.