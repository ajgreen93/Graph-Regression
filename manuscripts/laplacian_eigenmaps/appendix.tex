\noindent 

\section{Graph-dependent error bounds}
\label{sec:fixed_graph_error_bounds}
In this section, we adopt the fixed design perspective; or equivalently, condition on $X_i = x_i$ for $i = 1,\ldots,n$. Let $G = \bigl([n],W\bigr)$ be a fixed graph on $\{1,\ldots,n\}$ with Laplacian matrix $L = \sum_{k = 1}^{n}\lambda_k v_k v_k^{\top}$; the eigenvectors have unit empirical norm, $\|v_k\|_n^2 = 1$. The randomness thus all comes from the responses 
\begin{equation}
\label{eqn:fixed_graph_regression_model}
Y_i = f_{0}(x_i) + w_i
\end{equation}
where the noise variables $w_i$ are independent $N(0,1)$. In the rest of this section, we will mildly abuse notation and write $f_0 = (f_0(x_1),\ldots,f_0(x_n)) \in \Reals^n$. We will also write ${\bf Y} = (Y_1,\ldots,Y_n)$.

\subsection{Upper bound on Estimation Error of Laplacian Eigenmaps}

\begin{lemma}
	\label{lem:fixed_graph_estimation}
	For any integer $s > 0$, and any integer $0 \leq K \leq n$, the Laplacian eigenmaps estimator $\wh{f}$ of~\eqref{eqn:laplacian_eigenmaps_estimator} satisfies
	\begin{equation}
	\label{eqn:fixed_graph_estimation}
	\|\wh{f} - f_0\|_n^2 \leq \frac{\dotp{L^sf_0}{f_0}_n}{\lambda_{K + 1}^s} + \frac{5K}{n};
	\end{equation}
	this is guaranteed if $K = 0$, and otherwise holds with probability at least $1 - \exp(-K)$ if $1 \leq K \leq n$. 
\end{lemma}
\paragraph{Proof (of Lemma~\ref{lem:fixed_graph_estimation}).}
	By the triangle inequality,
	\begin{equation}
	\label{pf:fixed_graph_estimation_1}
	\|\wh{f} - f_0\|_n^2 \leq 2\Bigl(\|\mathbb{E}\wh{f} - f_0\|_n^2 + \|\wh{f} - \mathbb{E}\wh{f}\|_n^2\Bigr).
	\end{equation}
	The first term in~\eqref{pf:fixed_graph_estimation_1} (approximation error) is non-random, since the design is fixed. The expectation $\mathbb{E}\wh{f} = \sum_{k = 1}^{K} \dotp{v_k}{f_0}_n v_k$, so that
	\begin{equation*}
	\|\mathbb{E}\wh{f} - f_0\|_n^2 = \Bigl\|\sum_{k = K + 1}^{n} \dotp{v_k}{f_0}_n v_k\Bigr\|_n^2 = \sum_{k = K + 1}^n \dotp{v_k}{f_0}_n^2.
	\end{equation*}
	In the above, the last equality relies on the fact that $v_k$ are orthonormal in $L^2(P_n)$. Using the fact that the eigenvalues are in increasing order, we obtain
	\begin{equation*}
	\sum_{k = K + 1}^n \dotp{v_k}{f_0}_n^2 \leq \frac{1}{\lambda_{K + 1}^s} \sum_{k = K + 1}^n \lambda_k^s \dotp{v_k}{f_0}_n^2 \leq \frac{\dotp{L^sf_0}{f_0}_n}{\lambda_{K + 1}^s}.
	\end{equation*}
	
	If $K = 0$, $\wh{f} = \Ebb{\wh{f}} = 0$, and the second term in~\eqref{pf:fixed_graph_estimation_1} is $0$. Otherwise the second   in~\eqref{pf:fixed_graph_estimation_1} (estimation error) is random. Observe that $\dotp{v_k}{\varepsilon}_n \overset{d}{=} Z_k/\sqrt{n}$, where $(Z_1,\ldots,Z_n) \sim N(0,I_{n \times n})$. Again using the orthonormality of the eigenvectors $v_k$, we have
	\begin{equation*}
	\|\wh{f} - \mathbb{E}\wh{f}\|_n^2 = \sum_{k = 1}^{K} \dotp{v_k}{\varepsilon}_n^2 \overset{d}{=} \frac{1}{n}\sum_{k = 1}^{K} Z_k^2.
	\end{equation*}
	Thus $\|\wh{f} - \mathbb{E}\wh{f}\|_n^2$ is equal to $1/n$ times a $\chi^2$ distribution with $K$ degrees of freedom. Consequently, it follows from a result of \citep{laurent00} that
	\begin{equation*}
	\Pbb\biggl(\|\wh{f} - \mathbb{E}\wh{f}\|_n^2 \geq \frac{K}{n} + 2\frac{\sqrt{K}}{n}\sqrt{t} + \frac{2t}{n}\biggr) \leq \exp(-t).
	\end{equation*}
	Setting $t = K$ completes the proof of the lemma.

\subsection{Upper bound on Testing Error of Laplacian Eigenmaps}

Let $\wh{T} = \sum_{k = 1}^{K} \dotp{{\bf Y}}{v_k}_n^2$, and let $\varphi = \1\{\wh{T} \geq t_a\}$. In the following Lemma, we upper bound the Type I and Type II error of the test $\varphi$.

\begin{lemma}
	\label{lem:fixed_graph_testing}
	Suppose we observe $(Y_1,x_1),\ldots,(Y_n,x_n)$ according to~\eqref{eqn:fixed_graph_regression_model}.
	\begin{itemize}
		\item If $f_0 = 0$, then $\Ebb_0[\varphi] \leq a$.
		\item Suppose $f_0 \neq 0$ satisfies
		\begin{equation}
		\label{eqn:fixed_graph_testing_critical_radius}
		\|f_0\|_n^2 \geq \frac{\dotp{L^sf_0}{f_0}_n}{\lambda_{K + 1}^s} + \frac{\sqrt{2K}}{n}\biggl[2\sqrt{\frac{1}{a}} + \sqrt{\frac{2}{b}} + \frac{32}{bn}\biggr],
		\end{equation}
		for some $s \in \mathbb{N}\setminus \{0\}$. Then $\Ebb_{f_0}[1 - \phi] \leq b$.
	\end{itemize}
\end{lemma}
\paragraph{Proof (of Lemma~\ref{lem:fixed_graph_testing}).}
We first compute the expectation and variance of $\wh{T}$, then apply Chebyshev's inequality to upper bound the Type I and Type II error.

\underline{\emph{Expectation}.}
Recall that $\wh{T} = \sum_{k = 1}^{K} \dotp{Y}{v_k}_n^2$. Expanding the square gives
\begin{equation*}
\Ebb[\wh{T}] = \sum_{k = 1}^{K} \Ebb[\dotp{Y}{v_k}_n^2] = \sum_{k = 1}^{K} \dotp{f_0}{v_k}_n^2 + \Ebb[2\dotp{f_0}{v_k}_n\dotp{\varepsilon}{v_k}_n + \dotp{\varepsilon}{v_k}_n^2] = \frac{K}{n} + \sum_{k = 1}^{K} \dotp{f_0}{v_k}_n^2.
\end{equation*}
Thus $\Ebb[\wh{T}] - t_a = \sum_{k = 1}^{K} \dotp{f_0}{v_k}_n^2 - \sqrt{2K}/n \cdot \sqrt{1/a}$. Furthermore, it is a consequence of~\eqref{eqn:fixed_graph_testing_critical_radius} that 
\begin{equation}
\label{pf:fixed_graph_testing_1}
\sum_{k = 1}^{K} \dotp{f_0}{v_k}_n^2 - \frac{\sqrt{2K}}{n}\sqrt{1/a} \geq \|f_0\|_n^2 - \frac{\dotp{L^sf_0}{f_0}_n}{\lambda_{K + 1}^s} - \frac{\sqrt{2K}}{n}\sqrt{1/a} \geq \frac{\sqrt{2K}}{n}\biggl[\sqrt{\frac{1}{a}} + \sqrt{\frac{2}{b}} + \frac{32}{bn}\biggr].
\end{equation} 

\underline{\emph{Variance}.}
Recall from the proof of Lemma~\ref{lem:fixed_graph_estimation} that $\dotp{\varepsilon}{v_k}_n \overset{d}{=} Z_k/\sqrt{n}$ for $(Z_1,\ldots,Z_n) \sim N(0,I_{n \times n})$. Expanding the square, and recalling that $\Cov[Z,Z^2] = 0$ for Gaussian random variables, we have that
\begin{equation*}
\Var\bigl[\dotp{{\bf Y}}{v_k}_n^2\bigr] = \Var\biggl[\frac{2}{n}\dotp{f_0}{v_k}_nZ_k + \frac{2}{n^2}Z_k^2\biggr] = \frac{4\dotp{f_0}{v_k}_n^2}{n} + \frac{2}{n^2}.
\end{equation*}
Moreover, since $\Cov[Z_k^2,Z_{\ell}^2] = 0$ for each $k = 1,\ldots,K$, we see that
\begin{equation*}
\Var\bigl[\wh{T}\bigr] = \sum_{k = 1}^{K} \Var\bigl[\dotp{{\bf Y}}{v_k}_n^2\bigr] = \frac{2K}{n^2} + \sum_{k = 1}^{K}\frac{4\dotp{f_0}{v_k}_n^2}{n}.
\end{equation*}

\underline{\emph{Bounds on Type I and Type II error}.}
The upper bound on Type I error follows immediately from Chebyshev's inequality. 

The upper bound on Type II error also follows from Chebyshev's inequality. We observe that~\eqref{eqn:fixed_graph_testing_critical_radius} implies $\Ebb_{f_0}[\wh{T}] = t_a$, and apply Chebyshev's inequality to deduce
\begin{equation*}
\Pbb_{f_0}\bigl(\wh{T} < t_a\bigr) \leq \Pbb_{f_0}\Bigl(|\wh{T} - \Ebb_{f_0}[\wh{T}]|^2 > |\Ebb_{f_0}[\wh{T}] - t_a|^2\Bigr) \leq \frac{\Var\bigl[\wh{T}\bigr]}{\bigl[\Ebb_{f_0}[\wh{T}] - t_a\bigr]^2} = \frac{2K/n^2 + 4/n\sum_{k = 1}^{K}\dotp{f_0}{v_k}_n^2}{\bigl[\Ebb_{f_0}[\wh{T}] - t_a\bigr]^2}.
\end{equation*}
Thus we have upper bounded the Type II error by the sum of two terms, each of which are no more than $1/(2b)$, as we now show. For the first term, after noting that~\eqref{pf:fixed_graph_testing_1} implies $\Ebb_{f_0}[\wh{T}] - t_a \geq \sqrt{2K}/n \cdot \sqrt{2/b}$, the upper bound follows:
\begin{equation*}
\frac{2K/n^2}{\bigl[\Ebb_{f_0}[\wh{T}] - t_a\bigr]^2} \leq \frac{b}{2}.
\end{equation*}
On the other hand, for the second term we use~\eqref{pf:fixed_graph_testing_1} in two ways: first to conclude that $\Ebb_{f_0}[\wh{T}] - t_a \geq 1/2 \cdot \sum_{k = 1}^{K}\dotp{f_0}{v_k}_n^2$, and second to obtain
\begin{equation*}
\frac{4\sum_{k = 1}^{K}\dotp{f_0}{v_k}_n^2}{n\bigl[\Ebb_{f_0}[\wh{T}] - t_a\bigr]^2} \leq \frac{4\sum_{k = 1}^{K}\dotp{f_0}{v_k}_n^2}{n\bigl(\sum_{k = 1}^{K}\dotp{f_0}{v_k}_n^2/2\bigr)^2} \leq \frac{16}{n\sum_{k = 1}^{K}\dotp{f_0}{v_k}_n^2} \leq \frac{b}{2}.
\end{equation*}

\section{Analysis of Spectral Series Estimator}
\label{sec:analysis_spectral_series_estimator}
In this section we prove Proposition~\ref{prop:spectral_series_estimation}. As mentioned in our main text, this proof relies on three facts: (i) a continuous embedding of $H_0^s(\mc{X})$ into $\mc{H}^s(\mc{X})$, (ii) the Weyl's Law asymptotic scaling $\lambda_k(\Delta_P) \asymp k^{2/d}$, and (iii) a local Weyl's Law scaling $\sum_{k = 1}^{K} \bigl(\psi_k(x)\bigr)^2 \lesssim K$. (Recall that $(\lambda_k(\Delta_P),\psi_k)$ are defined to be solutions to~\eqref{eqn:laplacian_eigenfunction_neumann}). We record the precise results we need in the following three lemmas. 

The first result follows immediately from Lemma~17 of \textcolor{blue}{(Dunlop)}. Recall the spectral Sobolev class $\mc{H}^s(\mc{X})$ defined in~\eqref{eqn:spectral_sobolev_space}. Equip $\mc{H}^s(\mc{X})$ with the norm
\begin{equation*}
\|f\|_{\mc{H}^s(\mc{X})}^2 := \sum_{k = 1}^{\infty} \bigl[\dotp{f}{\psi_k}_P\bigr]^2 \cdot [\lambda_k(\Delta_P)]^2.
\end{equation*}
\begin{lemma}[cf. Lemma~17 of \textcolor{blue}{(Dunlop)}]
	\label{lem:zero_trace_spectral_sobolev}
	Suppose Model~\ref{def:model_flat_euclidean}, and additionally that $p \in C^{\infty}(\mc{X})$ and $\partial \mc{X} \in C^{1,1}$. Then for any $s \in \mathbb{N}$, we have $H_0^s(\mc{X}) \subseteq \mc{H}^s(\mc{X})$. Moreover, there exists a constant $C$ such that for any $f \in H_0^s(\mc{X})$ we have
	\begin{equation}
	\label{eqn:zero_trace_spectral_sobolev}
	\|f\|_{\mc{H}^s(\mc{X})} \leq C\|f\|_{H^s(\mc{X})}.
	\end{equation}
\end{lemma}

The particular version of Weyl's Law we use is also due to \textcolor{blue}{(Dunlop)}. 
\begin{lemma}[cf. Lemma 28 of \textcolor{blue}{(Dunlop)}]
	\label{lem:weyl}
	Suppose Model~\ref{def:model_flat_euclidean}. Then there exist positive constants $c$ and $C$ such that for all $k \geq 2$, 
	\begin{equation}
	\label{eqn:weyl}
	ck^{2/d} \leq \lambda_k(\Delta_P) \leq Ck^{2/d}.
	\end{equation}
\end{lemma}
As we discuss in our main text, the asymptotic scaling $\lambda_k(\Delta_P) \asymp k^{2/d}$ also plays a key role in our analysis of Laplacian eigenmaps. We point out that although Lemma 28 of \textcolor{blue}{(Dunlop)} assumes $p \in C^{\infty}(\mc{X})$ and $\partial \mc{X} \in C^{1,1}$, the proof uses only the hypotheses of Model~\ref{def:model_flat_euclidean}: that $p$ is bounded away from $0$ and $\infty$, and that $\partial \mc{X}$ is Lipschitz. 

Finally, we need a local version of Weyl's Law, due to \textcolor{blue}{(H\"{o}rmander)}.
\begin{lemma}[cf. Theorem 17.5.3 of \textcolor{blue}{H\"{o}rmander}]
	\label{lem:local_weyl}
	Suppose Model~\ref{def:model_flat_euclidean}, and additionally that $p \in C^{\infty}(\mc{X})$ and $\partial \mc{X} \in C^{1,1}$. Then there exists a positive constant $C$ such that for all $K \in \mathbb{N} \setminus {0}$,
	\begin{equation}
	\label{eqn:local_weyl}
	\sup_{x \in \mc{X}}\biggl\{\sum_{k = 1}^{K} \bigl(\psi_k(x)\bigr)^2\biggr\} \leq CK.
	\end{equation}
\end{lemma}
Translated into our notation, Theorem 17.5.3 of \textcolor{blue}{(H\"{o}rmander)} says that
\begin{equation*}
\sup_{x \in \mc{X}}\biggl\{\sum_{k: \lambda_k(\Delta_P) \leq \lambda_K(\Delta_P)}  \bigl(\psi_k(x)\bigr)^2\biggr\} \leq C \bigl[\lambda_K(\Delta_P)\bigr]^{d/2},
\end{equation*}
and from here~\eqref{eqn:local_weyl} follows from~\eqref{eqn:weyl}. \textcolor{blue}{(H\"{o}rmander)} proves the result only when $\partial \mc{X} \in C^{\infty}$, but the proof goes through unchanged when $\partial \mc{X} \in C^1$, since $\partial \mc{X} \in C^1$ is sufficient to apply the relevant Sobolev embedding theorem (Lemma 17.5.2 of~\textcolor{blue}{(H\"{o}rmander)}) that is the key to proving Lemma~\ref{lem:local_weyl}.

We note that the assumptions $p \in C^{\infty}(\mc{X})$ and $\partial\mc{X} \in C^{1,1}$ used in Lemmas~\ref{lem:zero_trace_spectral_sobolev} and~\ref{lem:local_weyl} can likely be removed. This would allow us to remove the same assumptions in the statement of Proposition~\ref{prop:spectral_series_estimation}. Since this does not change the main point of the proposition, we do not pursue the details further.

\paragraph{Proof (of Proposition~\ref{prop:spectral_series_estimation}).}
We decompose risk into squared bias and variance,
\begin{equation}
\label{pf:spectral_series_estimation_0}
\Ebb \|\wt{f} - f_0\|_P^2 = \Ebb\| \Ebb[\wt{f}]  - f_0\|_P^2 + \Ebb\| \wt{f} - \Ebb[\wt{f}]\|_P^2.
\end{equation}
Then some standard arguments (which we give below) give the following upper bound:
\begin{equation}
\label{pf:spectral_series_estimation_1}
\Ebb \|\wt{f} - f_0\|_P^2 \leq \frac{\|f_0\|_{\mc{H}^s(\mc{X})}^2}{\bigl[\lambda_{K + 1}(\Delta_P)\bigr]^s} + \frac{K}{n} + \frac{1}{n}\Ebb\Bigl[(f_0(X))^2 \cdot \sum_{k = 1}^{K} (\psi_k(X))^2\Bigr].
\end{equation}
The claim of the proposition follows from~\eqref{pf:spectral_series_estimation_1} and Lemmas~\ref{lem:zero_trace_spectral_sobolev}-\ref{lem:local_weyl}, upon taking $K = \floor{(M^2n)^{d/(2s + d)}}$. (Here we use the convention $\sum_{k = 1}^{0}c_k = 0$ for any sequence $(c_k)$.) \qed

The upper bound~\eqref{pf:spectral_series_estimation_1} is to be compared with~\eqref{eqn:estimation_biasvariance}, which gives upper bounds on the (design-dependent) squared bias and variance of Laplacian eigenmaps. We see~\eqref{eqn:estimation_biasvariance} has terms analogous to the first two terms on the right hand side of~\eqref{pf:spectral_series_estimation_1}, but not the third. The relevance of this third term to $L^2$ estimation problems is discussed by~\cite{birge2008}.

\paragraph{Proof (of \eqref{pf:spectral_series_estimation_1}).}
Since $\{\psi_k\}$ are an orthonormal basis of $L^2(P)$, and $f_0 \in H_0^{s}(\mc{X}) \subseteq L^2(P)$, we have $f_0 = \sum_{k = 1}^{\infty} \dotp{f_0}{\psi_k}_P \psi_k$ in $L^2(P)$. This allows us to expand the squared bias in terms of squared Fourier coefficients of $f_0$, and gives the following upper bound,
\begin{equation*}
\|f_0 - \Ebb{\wt{f}}\|_P^2 = \sum_{k = K + 1}^{\infty}  \dotp{f_0}{\psi_k}^2 \leq  \frac{1}{\lambda_{K + 1}(\Delta_P)^s} \sum_{k = K + 1}^{\infty} \lambda_{k + 1}(\Delta_P)^s \dotp{f_0}{\psi_k}^2 \leq \frac{\|f_0\|_{\mc{H}^s}}{\bigl[\lambda_{K + 1}(\Delta_P)\bigr]^s}.
\end{equation*}
On the other hand, the variance term can be written as the sum of the variance of each empirical Fourier coefficient,
\begin{equation}
\label{pf:spectral_series_estimation_2}
\Ebb\| \wt{f} - \Ebb[\wt{f}]\|_P^2 = \sum_{k = 1}^{K} \Var\Bigl[\dotp{{\bf Y}}{\psi_k}_n\Bigr] 
\end{equation}
Then from the law of total variance,
\begin{equation*}
 \Var\bigl[\dotp{{\bf Y}}{\psi_k}_n\bigr] = \Var\Bigl[\Ebb[\dotp{Y}{\psi_k}_n|{\bf X}\Bigr] + \Ebb\Bigl[\Var[\dotp{Y}{\psi_k}_n|{\bf X}\Bigr] = \Var\Bigl[\dotp{f_0}{\psi_k}_n\Bigr] + \frac{1}{n}\Ebb\Bigl[\|\psi_k\|_n^2\Bigr].
\end{equation*}
Finally, $\Ebb \|\psi_k\|_n^2 = \|\psi_k\|_P^2 = 1$, 
\begin{equation*}
\Var\Bigl[\dotp{f_0}{\psi_k}_n\Bigr] = \frac{1}{n} \Var\Bigl[f_0(X)\psi_k(X)\Bigr] \leq \frac{1}{n} \Ebb\Bigl[\Bigl(f_0(X)\psi_k(X)\Bigr)^2\Bigr],
\end{equation*}
and plugging back into~\eqref{pf:spectral_series_estimation_2} yields the claim.

\section{Graph Sobolev semi-norm, flat Euclidean domain}
\label{sec:graph_quadratic_form_euclidean}
In this section we prove Proposition~\ref{prop:graph_seminorm_ho}. The proposition will follow from several intermediate results.
\begin{enumerate}
	\item~In Section~\ref{subsec:decomposition_graph_seminorm}, we show that
	\begin{equation}
	\label{pf:graph_seminorm_ho_1}
	\dotp{L_{n,\varepsilon}^sf}{f}_n \leq \frac{1}{\delta} \dotp{L_{P,\varepsilon}^sf}{f}_{P} + \frac{C\varepsilon^2}{\delta n\varepsilon^{2 + d}}M^2.
	\end{equation}
	with probability at least $1 - 2\delta$. 
	
	We term the first term on the right hand side the \emph{non-local Sobolev semi-norm}, as it is a kernelized approximation to the Sobolev semi-norm $\dotp{\Delta_P^sf}{f}_{P}$. The second term on the right hand side is a pure bias term, which as we will see is negligible compared to the non-local Sobolev semi-norm as long as $\varepsilon \ll n^{-1/(2(s -1 + d))}$. 
	\item~In Section~\ref{subsec:approximation_error_nonlocal_laplacian}, we show that when $x$ is sufficiently in the interior of $\mc{X}$, then $L_{P,\varepsilon}^kf(x)$ is a good approximation to $\Delta_P^kf(x)$, as long as $f \in H^{s}(\mc{X})$ and $p \in C^{s - 1}(\mc{X})$ for some $s \geq 2k + 1$. 
	\item~In Section~\ref{subsec:boundary_behavior_nonlocal_laplacian}, we show that when $x$ is sufficiently near the boundary of $\mc{X}$, then $L_{P,\varepsilon}^kf(x)$ is close to $0$, as long as $f \in H_0^{s}(\mc{X})$ for some $s > 2k$.
	\item~In Section~\ref{subsec:estimate_nonlocal_seminorm}, we use the results of the preceding two sections to show that if $f \in H_0^s(\mc{X};M)$ and $p \in C^{s - 1}(\mc{X})$, there exists a constant $C$ which does not depend on $f$ such that
	\begin{equation}
	\label{pf:graph_seminorm_ho_2}
	\dotp{L_{P,\varepsilon}^sf}{f}_{P} \leq CM^2.
	\end{equation}
\end{enumerate}
Finally, in Section~\ref{subsec:integrals} we provide some assorted estimates used in Sections~\ref{subsec:decomposition_graph_seminorm}. 

\paragraph{Proof (of Proposition~\ref{prop:graph_seminorm_ho}).}
Proposition~\ref{prop:graph_seminorm_ho} follows immediately from~\eqref{pf:graph_seminorm_ho_1} and~\eqref{pf:graph_seminorm_ho_2}. \qed

One note regarding notation: suppose a function $g \in H^{\ell}(U)$, where $\ell \in \mathbb{N}$ and $U$ is an open set. Let $V$ be another open set, compactly contained within $U$. Then we will use the notation $g \in H^{\ell}(V)$ to mean that the restriction $\restr{g}{V}$ of $g$ to $V$ belongs to $H^{\ell}(V)$.

\subsection{Decomposition of graph Sobolev semi-norm}
\label{subsec:decomposition_graph_seminorm}

In Lemma~\ref{lem:graph_seminorm_bias}, we decompose the graph Sobolev semi-norm (a V-statistic) into an unbiased estimate of the non-local Sobolev semi-norm (a U-statistic), and a pure bias term. We establish that the pure bias term will be small (in expectation) relative to the U-statistic whenever $\varepsilon$ is sufficiently small.
\begin{lemma}
	\label{lem:graph_seminorm_bias}
	For any $f \in L^2(\mc{X})$, the graph Sobolev semi-norm satisfies
	\begin{equation}
	\label{eqn:graph_seminorm_bias_1}
	\dotp{L_{n,\varepsilon}^sf}{f}_{n} = U_{n,\varepsilon}^{(s)}(f) + B_{n,\varepsilon}^{(s)}(f),
	\end{equation}
	such that $\mathbb{E}[U_{n,\varepsilon}^{(s)}(f)] = (n - s - 1)!/n! \cdot \dotp{L_{P,\varepsilon}^sf}{f}_P$. If additionally $f \in H^1(\mc{X};M)$ and $\varepsilon \geq n^{-1/d}$, then the bias term $B_{n,\varepsilon}^{(s)}(f)$ satisfies
	\begin{equation}
	\label{eqn:graph_seminorm_bias_2}
	\mathbb{E}\bigl[|B_{n,\varepsilon}^{(s)}(f)|\bigr] \leq \frac{C\varepsilon^2}{\delta n\varepsilon^{2 + d}}M^2.
	\end{equation}
\end{lemma}
Then~\ref{pf:graph_seminorm_ho_1} follows immediately from Lemma~\ref{lem:graph_seminorm_bias}, by Markov's inequality.
\paragraph{Proof (of Lemma~\ref{lem:graph_seminorm_bias}).}
We begin by introducing some notation. We will use bold notation $\bj = (j_1,\ldots,j_s)$ for a vector of indices where $j_i \in [n]$ for each $i$. We write $[n]^s$ for the collection of all such vectors, and $(n)^s$ for the subset of such vectors with no repeated indices. Finally, we write $D_if$ for a kernelized difference operator,
\begin{equation*}
D_if(x) := \bigl(f(x) - f(X_i)\bigr) \eta\biggl(\frac{\|X_i - x\|}{\varepsilon}\biggr),
\end{equation*}
and we let $D_{\bj}f(x) := \bigl(D_{j_1}\circ \cdots \circ D_{j_s}f\bigr)(x)$.

With this notation in hand, it is easy to represent $\dotp{L_{n,\varepsilon}^sf}{f}_{n}$ as the sum of a U-statistic and a bias term,
\begin{align*}
\dotp{L_{n,\varepsilon}^sf}{f}_{n} & = \frac{1}{n} \sum_{i = 1}^{n} L_{n,\varepsilon}^sf(X_i) \cdot f(X_i) \\
& = \underbrace{\frac{1}{n^{s + 1}\varepsilon^{s(d + 2)}} \sum_{{i\bf j} \in (n)^{s + 1}}D_{\bj}f(X_i) \cdot f(X_i)}_{=:U_{n,\varepsilon}^{(s)}(f)} + \underbrace{\frac{1}{n^{s + 1}\varepsilon^{s(d + 2)}} \sum_{\substack{i\bj \in \\ [n]^{s + 1}\setminus (n)^{s + 1}}} D_{\bj}f(X_i) \cdot f(X_i)}_{=:B_{n,\varepsilon}^{(s)}(f)}
\end{align*}
When the indices of $i\bj$ are all distinct, it follows straightforwardly from the law of iterated expectation that
\begin{equation*}
\mathbb{E}[ D_{\bj}f(X_i) \cdot f(X_i)] = \varepsilon^{s(d + 2)}\mathbb{E}[L_{P,\varepsilon}^sf(X_i) \cdot f(X_i)] = \dotp{L_{P,\varepsilon}^sf}{f}_{P}, 
\end{equation*}
which in turn implies $\mathbb{E}[U_{n,\varepsilon}^{(s)}(f)] = (n - s - 1)!/n! \cdot \dotp{L_{P,\varepsilon}^sf}{f}_P$. 

It remains to show~\eqref{eqn:graph_seminorm_bias_2}. By adding and subtracting $f(X_{\bj_1})$, we obtain by symmetry that
\begin{equation*}
\sum_{\substack{i\bj \in \\ [n]^{s + 1}\setminus (n)^{s + 1}}} D_{\bj}f(X_i) \cdot f(X_i) = \frac{1}{2} \cdot \sum_{\substack{i\bj \in \\ [n]^{s + 1}\setminus (n)^{s + 1}}} D_{\bj}f(X_i) \cdot \bigl(f(X_i) - f(X_{\bj_1})\bigr),
\end{equation*}
and consequently
\begin{equation*}
\Ebb\Bigl[\sum_{\substack{i\bj \in \\ [n]^{s + 1}\setminus (n)^{s + 1}}} D_{\bj}f(X_i) \cdot f(X_i)\Bigr] \leq \frac{1}{2} \cdot \sum_{\substack{i\bj \in \\ [n]^{s + 1}\setminus (n)^{s + 1}}} \Ebb\Bigl[\bigl|D_{\bj}f(X_i)\bigr| \cdot \bigl|f(X_i) - f(X_{\bj_1})\bigr|\Bigr].
\end{equation*}
In Lemma~\ref{lem:graph_seminorm_bias2}, we show that if $f \in H^1(\mc{X};M)$, then for any $i\bj \in [n]^{s + 1}$ which contains a total of $k + 1$ distinct indices, 
\begin{equation*}
\Ebb\Bigl[\bigl|D_{\bj}f(X_i)\bigr| \cdot \bigl|f(X_i) - f(X_{\bj_1})\bigr|\Bigr] \leq C_1 \varepsilon^{2 + kd} M^2.
\end{equation*}
This shows us that the expectation of $|B_{n,\varepsilon}^s(f)|$ can bounded from above by the sum over several different terms, as follows:
\begin{align*}
\Ebb\Bigl[|B_{n,\varepsilon}^s(f)|\Bigr] & \leq C_1\frac{\varepsilon^2}{n\varepsilon^{2s}}M^2 \sum_{\substack{i\bj \in \\ [n]^{s + 1}\setminus (n)^{s + 1}}} \frac{1}{(n\varepsilon^d)^s}  \varepsilon^{(|i\bj| - 1)d} \\
& \leq C_1\frac{\varepsilon^2}{n\varepsilon^{2s}}M^2  \sum_{k = 1}^{s - 1} \frac{(n\varepsilon^d)^k}{(n\varepsilon^d)^s}n.
\end{align*}
Finally, we note that by assumption $n\varepsilon^d \geq 1$, so that in the above sum the factor of $(n\varepsilon^d)^k$ is largest when $k = s- 1$. We conclude that
\begin{equation*}
\Ebb\Bigl[|B_{n,\varepsilon}^s(f)|\Bigr] \leq C_1 (s - 1) \frac{\varepsilon^2}{n\varepsilon^{2s + d}}M^2,
\end{equation*}
which is the desired result.

\subsection{Approximation error of non-local Laplacian}
\label{subsec:approximation_error_nonlocal_laplacian}

In this section, we establish the convergence $L_{P,\varepsilon}^kf \to \sigma_{\eta}^k\Delta_P^kf$ as $\varepsilon \to 0$. More precisely, we give an upper bound on the squared difference between $L_{P,\varepsilon}^kf$ and  $\sigma_{\eta}^k\Delta_P^kf$ as a function of $\varepsilon$. The bound holds for all $x \in \mc{X}_{k\varepsilon}$, and $f \in H^{s}(\mc{X})$, as long as $s \geq 2k + 1$.  \textcolor{red}{(TODO): Make sure that $\sigma_{\eta}$ is defined somewhere.}
\begin{lemma}
	\label{lem:approximation_error_nonlocal_laplacian}
	Assume Model~\ref{def:model_flat_euclidean}. Let $s \in \mathbb{N} \setminus \{0,1\}$, suppose that $f \in H^s(\mc{X};M)$, and if $s > 1$ suppose that $p \in C^{s - 1}(\mc{X})$. Let $L_{P,\varepsilon}$ be define with respect to a kernel $\eta$ that satisfies~\ref{asmp:kernel_flat_euclidean}. Then there exist constants $C_1$ and $C_2$ that do not depend on $f$, such that each of the following statements hold.
	\begin{itemize}
		\item If $s$ is odd and $k = (s - 1)/2$, then
		\begin{equation}
		\label{eqn:approximation_error_nonlocal_laplacian_1}
		\|L_{P,\varepsilon}^kf - \Delta_P^kf\|_{L^2(\mc{X}_{k\varepsilon})} \leq C_1 M \varepsilon
		\end{equation}
		\item If $s$ is even and $k = (s - 2)/2$, then
		\begin{equation}
		\label{eqn:approximation_error_nonlocal_laplacian_2}
		\|L_{P,\varepsilon}^kf - \Delta_P^kf\|_{L^2(\mc{X}_{k\varepsilon})} \leq C_2 M \varepsilon^2.
		\end{equation}
	\end{itemize}
\end{lemma}
We remark that when $k = 1$ and $f \in C^3(\mc{X})$ or $C^4(\mc{X})$, statements of this kind are well known \textcolor{red}{(references)}, and indeed stronger results---with $L^{\infty}(\mc{X})$ norm replacing $L^2(\mc{X})$ norm---hold. When dealing with the iterated Laplacian, and functions $f$ which are regular only in the Sobolev sense, the proof is somewhat more lengthy, but the spirit of the result is largely the same.
 
\paragraph{Proof (of Lemma~\ref{lem:approximation_error_nonlocal_laplacian}).}
Throughout this proof, we shall assume that $f$ and $p$ are smooth functions, meaning they belong to $C^{\infty}(\mc{X})$. This is without loss of generality, since $C^{\infty}(\mc{X})$ is dense in both $H^s(\mc{X})$ and $C^{s - 1}(\mc{X})$, and since both sides of the inequalities~\eqref{eqn:approximation_error_nonlocal_laplacian_1} and~\eqref{eqn:approximation_error_nonlocal_laplacian_2} are continuous with respect to $\|\cdot\|_{H^s(\mc{X})}$ and $\|\cdot\|_{C^{s - 1}(\mc{X})}$ norms.

We will actually prove a more general set of statements than contained in Lemma~\ref{lem:approximation_error_nonlocal_laplacian}, more general in the sense that they give estimates for all $k$, rather than simply the particular choices of $k$ given above. In particular, we will prove that the following two statements hold for any $s \in \mathbb{N}$ and any $k \in \mathbb{N} \setminus \{0\}$. 
\begin{itemize}
	\item If $k \geq s/2$, then for every $x \in \mc{X}_{k\varepsilon}$, 
	\begin{equation}
	\label{pf:approximation_error_nonlocal_laplacian_0}
	L_{P,\varepsilon}^kf(x) = g_s(x) \varepsilon^{s - 2k}
	\end{equation}
	for a function $g_s$ that satisfies
	\begin{equation}
	\label{pf:approximation_error_nonlocal_laplacian_0.5}
	\|g_s\|_{L^2(\mc{X}_{k\varepsilon})} \leq C \|p\|_{C^{q}(\mc{X})}^k M 
	\end{equation}
	where $q = 1$ if $s =0$ or $s = 1$, and otherwise $q = s - 1$. 
	\item If $k < s/2$, then for every $x \in \mc{X}_{k\varepsilon}$,
	\begin{equation}
	\label{pf:approximation_error_nonlocal_laplacian_1}
	L_{P,\varepsilon}^kf(x) = \sigma_{\eta}^k \cdot \Delta_{P}^kf(x) + \sum_{j = 1}^{\floor{(s - 1)/2} - k} g_{2(j + k)}(x)\varepsilon^{2j} + g_{s}(x) \varepsilon^{s - 2k}.
	\end{equation}
	for functions $g_j$ that satisfy
	\begin{equation}
	\label{pf:approximation_error_nonlocal_laplacian_1.5}
	\|g_j\|_{H^{s - j}(\mc{X}_{k\varepsilon})} \leq C \|p\|_{C^{s - 1}(\mc{X})}^k M.
	\end{equation}
\end{itemize}
In the statement above, recall that $H^0(\mc{X}_{k\varepsilon}) = L^2(\mc{X}_{k\varepsilon})$. Additionally, note that we may speak of the pointwise behavior of derivatives of $f$ because we have assumed that $f$ is a smooth function. Observe that~\eqref{eqn:approximation_error_nonlocal_laplacian_1} follows upon taking $k = \floor{(s - 1)/2}$ in~\eqref{pf:approximation_error_nonlocal_laplacian_1}, whence we have
\begin{equation*}
\bigl(L_{P,\varepsilon}^kf(x) - \sigma_{\eta}^k \Delta_{P}^kf(x)\bigr)^2 = \varepsilon^2 \bigl(g_s(x)\bigr)^2
\end{equation*}
for some $g_s \in \Leb^2(\mc{X}_{k\varepsilon},C \cdot M \cdot \|p\|_{C^{s - 1}(\mc{X})})$, and integrating over $\mc{X}_{k\varepsilon}$ gives the desired result. \eqref{eqn:approximation_error_nonlocal_laplacian_2} follows from~\eqref{pf:approximation_error_nonlocal_laplacian_1} in an identical fashion. 

It thus remains establish~\eqref{pf:approximation_error_nonlocal_laplacian_1}, and~\eqref{pf:approximation_error_nonlocal_laplacian_0} which is an important part of proving~\eqref{pf:approximation_error_nonlocal_laplacian_1}. We will do so by induction on $k$. Note that throughout, we will let $g_j$ refer to functions which may change from line to line, but which always satisfy~\eqref{pf:approximation_error_nonlocal_laplacian_1.5}. 

\underline{\textit{Proof of~\eqref{pf:approximation_error_nonlocal_laplacian_0} and~\eqref{pf:approximation_error_nonlocal_laplacian_1}, base case.}}

We begin with the base case, where $k = 1$. Again, we point out that although desired result is known when $s = 3$ or $s = 4$, and $f$ is regular in the H\"{o}lder sense, we require estimates for all $s \in \mathbb{N}$ when $f$ is regular in the Sobolev sense.

When $s = 0$, the inequality~\eqref{pf:approximation_error_nonlocal_laplacian_0} is implied by Lemma~\ref{lem:l2estimate_nonlocal_laplacian}.  When $s \geq 1$, we proceed using Taylor expansion. For any $x \in \mc{X}_{\varepsilon}$, we have that $B(x,\varepsilon) \subseteq \mc{X}$. Thus for any $x' \in B(x,\varepsilon)$, we may take an order $s$ Taylor expansion of $f$ around $x' = x$, and an order $q$ Taylor expansion of $p$ around $x' = x$, where $q = 1$ if $s = 1$, and otherwise $q = s - 1$. (See Section~\ref{subsec:taylor_expansion} for a review of the notation we use for Taylor expansions, as well as some properties that we make use of shortly.) This allows us to express $L_{P,\varepsilon}f(x)$ as the sum of three terms,
\begin{align*}
L_{P,\varepsilon}f(x) & = \frac{1}{\varepsilon^{d + 2}}\sum_{j_1 = 1}^{s - 1} \sum_{j_2 = 0}^{q - 1}\frac{1}{j_1!j_2!}  \int_{\mc{X}} \bigl(d_x^{j_1}f\bigr)(x' - x) \bigl(d_x^{j_2}p\bigr)(x' - x) \eta\biggl(\frac{\|x' - x\|}{\varepsilon}\biggr) \,dx' \quad + \\
& \quad \frac{1}{\varepsilon^{d + 2}}\sum_{j = 1}^{s - 1} \frac{1}{j!} \int_{\mc{X}} \bigl(d_x^jf\bigr)(x' - x)  r_{x'}^{q}(x;p) \eta\biggl(\frac{\|x' - x\|}{\varepsilon}\biggr) \,dx' \quad  + \\
& \quad \frac{1}{\varepsilon^{d + 2}} \int_{\mc{X}} r_{x'}^j(x;f) \eta\biggl(\frac{\|x' - x\|}{\varepsilon}\biggr) \,dP(x').
\end{align*}
Here we have adopted the convention that $\sum_{j = 1}^{0} = 0$. 

Changing variables to $z = (x' - x)/\varepsilon$, we can rewrite the above expression as 
\begin{align*}
L_{P,\varepsilon}f(x) & = \frac{1}{\varepsilon^{2}}\sum_{j_1 = 1}^{s - 1} \sum_{j_2 = 0}^{q - 1}\frac{\varepsilon^{j_1 + j_2}}{j_1!j_2!}  \int d_x^{j_1}f(z) d_x^{j_2}p(z) \eta\bigl(\|z\|\bigr) \,dz \quad + \\
& \quad \frac{1}{\varepsilon^{2}} \sum_{j = 1}^{s - 1} \frac{\varepsilon^j}{j!} \int d_x^jf(z)  r_{zh + x}^{q}(x;p) \eta\bigl(\|z\|\bigr) \,dz \quad  + \\
& \quad \frac{1}{\varepsilon^{2}} \int r_{zh + x}^j(x;f) \eta\bigl(\|z\|\bigr) p(zh + x)\,dz \\
& := G_1(x) + G_2(x) + G_3(x).
\end{align*}
We now separately consider each of $G_1(x),G_2(x)$ and $G_3(x)$. We will establish that if $s = 1$ or $s = 2$, then $G_1(x) = 0$, and otherwise if $s \geq 3$ that
\begin{equation*}
G_1(x) = \sigma_{\eta}\Delta_Pf(x) + \sum_{j = 1}^{\floor{(s - 1)/2} - 1}g_{2(j + 1)}(x)\varepsilon^{2j} + g_{s}(x)\varepsilon^{s - 2}.
\end{equation*}
On the other hand, we will establish that if $s = 1$ then $G_2(x) = 0$, and otherwise for $s \geq 2$
\begin{equation}
\label{pf:approximation_error_nonlocal_laplacian_2}
\|G_2\|_{\Leb^2(\mc{X}_{\varepsilon})} \leq C \varepsilon^{s - 2} M \|p\|_{C^{s - 1}(\mc{X})};
\end{equation}
this same estimate will hold for $G_3$ for all $s \geq 1$. Together these will imply~\eqref{pf:approximation_error_nonlocal_laplacian_0} and~\eqref{pf:approximation_error_nonlocal_laplacian_1}. 

\emph{Estimate on $G_1(x)$.}
If $s = 1$, then $s - 1 = 0$, and so $G_1(x) = 0$. We may therefore suppose $s \geq 2$. Recall that
\begin{equation}
G_1(x) = \sum_{j_1 = 1}^{s - 1} \sum_{j_2 = 0}^{q - 1} \frac{\varepsilon^{j_1 + j_2 - 2}}{j_1!j_2!}  \underbrace{\int_{B(0,1)} d_x^{j_1}f(z) d_x^{j_2}p(z) \eta(\|z\|) \,dz}_{:= g_{j_1,j_2}(x)} \label{pf:approximation_error_nonlocal_laplacian_3}
\end{equation}
The nature of $g_{j_1,j_2}(x)$ depends on the sum $j_1 + j_2$. Since $d_x^{j_1}f d_x^{j_2}$ is an order $j_1 + j_2$ (multivariate) monomial, we have (see Section~\ref{subsec:taylor_expansion}) that whenever $j_1 + j_2$ is odd,
\begin{equation*}
g_{j_1,j_2}(x) = \int_{\mc{X}} d_x^{j_1}f(z) d_x^{j_2}p(z) \eta(\|z\|) \,dz = 0.
\end{equation*}
In particular this is the case when $j_1 = 1$ and $j_2 = 0$. Thus when $s = 2$,  $G_1(x) = g_{1,0}(x) = 0$. On the other hand if $s \geq 3$, then the lowest order terms in~\eqref{pf:approximation_error_nonlocal_laplacian_3} are those where $j_1 + j_2 = 2$, so that either $j_1 = 1$ and $j_2 = 1$, or $j_1 = 2$ and $j_2 = 0$. We have that
\begin{align*}
g_{1,1}(x) + \frac{1}{2}g_{2,0}(x) & = \int_{\mc{X}} d_x^{1}f(z) d_x^{1}p(z) \eta(\|z\|) \,dz + \frac{p(x)}{2} \int_{\mc{X}} d_x^{2}f(z) \eta(\|z\|) \,dz \\
& = \sum_{i_1 = 1}^{d} \sum_{i_2 = 1}^{d} D^{e_{i_1}}f(x) D^{e_{i_2}}p(x) \int_{\mc{X}} z^{e_{i_1} + e_{i_2}} \eta(\|z\|) \,dz + \frac{p(x)}{2} \sum_{i_1 = 1}^{d} \sum_{i_2 = 1}^{d}D^{e_{i_2}+e_{i_2}}f(x)\int_{\mc{X}} z^{e_{i_1} + e_{i_2}} \eta(\|z\|) \,dz\\
& = \sum_{i = 1}^{d} D^{e_{i}}f(x) D^{e_{i}}p(x) \int_{\mc{X}} z^2 \eta(\|z\|) \,dz + \frac{p(x)}{2} \sum_{i = 1}^{d} D^{2e_{i}}f(x)\int_{\mc{X}} z^2 \eta(\|z\|) \,dz\\ 
& = \sigma_{\eta}\Delta_Pf(x),
\end{align*}
which is the leading term order term. Now it remains only to deal with the higher-order terms, where $j_1 + j_2 > 2$, and where it suffices to show that each function $g_{j_1,j_2}$ satisfies~\eqref{pf:approximation_error_nonlocal_laplacian_1.5} for $j = \min\{j_1 + j_2 - 2,s - 2\}$. It is helpful to write $g_{j_1,j_2}$ using multi-index notation, 
\begin{align*}
g_{j_1,j_2}(x) = \sum_{|\alpha_1| = j_1} \sum_{|\alpha_2| = j_2} D^{\alpha_1}f(x) D^{\alpha_2}p(x) \int_{B(0,1)} z^{\alpha_1 + \alpha_2} \eta(\|z\|) \,dz,
\end{align*}
where we note that $|\int_{B(0,1)} z^{\alpha_1 + \alpha_2} \eta(\|z\|) \,dz| < \infty$ for all $\alpha_1, \alpha_2$, by the assumption that $\eta$ is Lipschitz on its support. Finally, by H\"{o}lder's inequality we have that
\begin{align*}
\|D^{\alpha_1}f D^{\alpha_2}p\|_{H^{s - (j + 2)}(\mc{X})} & \leq \|D^{\alpha_1}f\|_{H^{s - (j + 2)}(\mc{X})} \|D^{\alpha_2}p\|_{C^{s - (j + 2)}(\mc{X})} \\
& \leq \|D^{\alpha_1}f\|_{H^{s - j_1}(\mc{X})} \|D^{\alpha_2}p\|_{C^{s - (j_2 + 1)}(\mc{X})} \\
& \leq M \cdot \|p\|_{C^{s - 1}(\mc{X})},
\end{align*}
and summing over all $|\alpha_1| = j_1$ and $|\alpha_2| = j_2$ establishes that $g_{j_1,j_2}$ satisfies~\eqref{pf:approximation_error_nonlocal_laplacian_1.5}.

\emph{Estimate on $G_2(x)$.}
Note immediately that $G_2(x) = 0$ if $s = 1$. Otherwise if $s \geq 2$, then $q = s - 1$. Recalling that $|r_{x + z\varepsilon}^{s - 1}(x; p)| \leq C\varepsilon^{s - 1}\|p\|_{C^{s - 1}(\mc{X})}$ for any $z \in B(0,1)$, and that $d_x^jf(\cdot)$ is a $j$-homogeneous function, we have that
\begin{align}
|G_2(x)| & \leq \sum_{j = 1}^{s - 1} \frac{\varepsilon^{j - 2}}{j!}\int_{B(0,1)} \Bigl|\bigl(d_x^{j}f\bigr)(z)\Bigr| \cdot |r_{x + z\varepsilon}^{s - 1}(x;p)| \cdot \eta(\|z\|) \,dz \nonumber \\
& \leq C\varepsilon^{s - 2}\|p\|_{C^{s - 1}(\mc{X})} \sum_{j = 1}^{s - 1} \frac{1}{j!} \int_{B(0,1)} \Bigl|\bigl(d_x^{j}f\bigr)(z)\Bigr| \cdot \eta(\|z\|) \,dz \label{pf:approximation_error_nonlocal_laplacian_4}.
\end{align}
Furthermore, for each $j = 1,\ldots,s - 1$ convolution of $d_x^jf$ with $\eta$ only decreases the $\Leb^2(\mc{X}_{\varepsilon})$ norm, meaning
\begin{equation}
\label{pf:approximation_error_nonlocal_laplacian_5}
\begin{aligned}
\int_{\mc{X}_{\varepsilon}} \biggl(\int_{B(0,1)} \Bigl|\bigl(d_x^{j}f\bigr)(z)\Bigr| \cdot \eta(\|z\|) \,dz\biggr)^2 \,dx & \leq \int_{\mc{X}_{\varepsilon}} \biggl(\int_{B(0,1)} \Bigl|\bigl(d_x^jf\bigr)(z)\Bigr|^2 \eta(\|z\|)\,dz \biggr) \cdot \biggl(\int_{B(0,1)} \eta(\|z\|) \,dz \biggr) \,dx \\
& \leq \int_{B(0,1)} \int_{\mc{X}_{\varepsilon}} \Bigl[\bigl(d^jf\bigr)(x)\Bigr]^2 \eta(\|z\|) \,dx  \,dz \\
& \leq \|d^jf\|_{\Leb^2(\mc{X_{\varepsilon}})}^2.
\end{aligned}
\end{equation}
In the above, we have used both that $|d_x^jf(z)| \leq |d^jf(x)|$ for all $z \in B(0,1)$, and that the kernel is normalized so that $\int \eta(\|z\|) \,dz = 1$. 
Combining this with~\eqref{pf:approximation_error_nonlocal_laplacian_4}, we conclude that
\begin{align*}
\int_{\mc{X}_{\varepsilon}} |G_2(x)|^2 \,dx & \leq C \Bigl(\varepsilon^{s - 2}\|p\|_{C^{s - 1}(\mc{X})}\Bigr)^2 \sum_{j = 1}^{s - 1} \int_{\mc{X}_{\varepsilon}}\biggl(\frac{1}{j!} \int_{B(0,1)} \Bigl|\bigl(d_x^{j}f\bigr)(z)\Bigr| \cdot \Bigl|\eta(\|z\|)\Bigr| \,dz\biggr)^2 \,dx \\
& \leq C \Bigl(\varepsilon^{s - 2}\|p\|_{C^{s - 1}(\mc{X})}\Bigr)^2 \sum_{j = 1}^{s - 1} \|d^ju\|_{\Leb^2(\mc{X_{\varepsilon}})}^2,
\end{align*}
establishing the desired estimate.

\emph{Estimate on $G_3(x)$.}
Applying the Cauchy-Schwarz inequality, we deduce a pointwise upper bound on $|G_3(x)|^2$,
\begin{align*}
|G_3(x)|^2 & \leq \biggl(\frac{p_{\max}}{\varepsilon^2}\biggr)^2 \cdot \biggl(\int_{B(0,1)} \bigl|r_{x + \varepsilon z}^s(x;u)\bigr|^2 \eta(\|z\|)\,dz\biggr) \cdot \biggl(\int_{B(0,1)} \eta(\|z\|) \,dz\biggr) \\
& \leq \biggl(\frac{p_{\max}}{\varepsilon^2}\biggr)^2 \int_{B(0,1)} \bigl|r_{x + \varepsilon z}^s(x;u)\bigr|^2 \eta(\|z\|) \,dz.
\end{align*}
Applying this pointwise over all $x \in \mc{X}_{\varepsilon}$ and integrating, we obtain
\begin{align*}
\int_{\mc{X}_{\varepsilon}} |G_3(x)|^2 \,dx & \leq \biggl(\frac{p_{\max}}{\varepsilon^2}\biggr)^2 \int_{\mc{X}_{\varepsilon}} \int_{B(0,1)} \bigl|r_{x + \varepsilon z}^s(x;f)\bigr|^2 \eta(\|z\|) \,dz \,dx \\
& = \biggl(\frac{p_{\max}}{\varepsilon^2}\biggr)^2 \int_{B(0,1)} \int_{\mc{X}_{\varepsilon}} \bigl|r_{x + \varepsilon z}^s(x;f)\bigr|^2 \eta(\|z\|) \,dx \,dz \\
& \leq \biggl(\frac{p_{\max}\varepsilon^s}{\varepsilon^2}\biggr)^2  \|d^sf\|_{L^2(\mc{X}_{\varepsilon})}^2,
\end{align*}
with the last inequality following from~\eqref{eqn:sobolev_remainder_term}. Noting that $p_{\max} = \|p\|_{C^0(\mc{X})} \leq \|p\|_{C^{s - 1}(\mc{X})}$, we see that this is a sufficient bound on $\|G_3\|_{\Leb^2(\mc{X}_{\varepsilon})}$.

\underline{\textit{Proof of~\eqref{pf:approximation_error_nonlocal_laplacian_0} and~\eqref{pf:approximation_error_nonlocal_laplacian_1}, induction step.}}
We now assume that~\eqref{pf:approximation_error_nonlocal_laplacian_0} and~\eqref{pf:approximation_error_nonlocal_laplacian_1} hold for all order up to some $k$, and show that they then hold for order $k + 1$ as well. The proof is relatively straightforward, once we introduce a bit of notation. Namely, for any $\ell,j \in \mathbb{N}$ such that $1 \leq j \leq \ell \leq$, we will use $g_j^{\ell}$ to refer to a function satisfying
\begin{equation}
\label{pf:approximation_error_nonlocal_laplacian_6}
\|g_j^{\ell}\|_{H^{\ell - j}(\mc{X}_{(k + 1)\varepsilon})} \leq C \|p\|_{C^{q}(\mc{X})}^{k + 1} M.
\end{equation}
Note that $g_j^{\ell}(x) = g_{(s - \ell) + j}(x)$, so that $g_j^{s}(x) = g_j(x)$. As before, the functions $g_j^{\ell}$ may change from line to line, but will always satisfy~\eqref{pf:approximation_error_nonlocal_laplacian_6}. We immediately illustrate the purpose of this notation. Suppose $g \in H^{\ell}(\mc{X}_{k\varepsilon}; C \|p\|_{C^{q}(\mc{X})}^k M)$ for some $\ell \leq s$. If $\ell \leq 2$, then by the inductive hypothesis, it follows that for any $x \in \mc{X}_{(k + 1)\varepsilon}$
\begin{equation}
\label{pf:approximation_error_nonlocal_laplacian_7}
L_{P,\varepsilon}g(x) = g_{\ell}^{\ell}(x) \varepsilon^{\ell - 2}.
\end{equation} 
On the other hand if $2 < \ell \leq s$, then by the inductive hypothesis, it follows that for any $x \in \mc{X}_{(k + 1)\varepsilon}$,
\begin{equation}
\label{pf:approximation_error_nonlocal_laplacian_8}
L_{P,\varepsilon}g(x) = \sigma_{\eta} \Delta_Pg(x) + \sum_{j = 1}^{\floor{(\ell - 1)/2} - 1} g_{2j + 2}^{\ell}(x) \varepsilon^{2j} + g_{\ell}^{\ell}(x) \varepsilon^{\ell - 2}.
\end{equation}

\emph{Proof of \eqref{pf:approximation_error_nonlocal_laplacian_0}.} If $s \leq 2(k + 1)$, then by the inductive hypothesis it follows that for all $x \in \mc{X}_{k\varepsilon}$, we have $L_{P,\varepsilon}^kf(x) = g_{s}(x) \cdot \varepsilon^{s - 2k}$, for some $g_s \in L^2(\mc{X}_{k\varepsilon}, C\|p\|_{C^{s - 1}(\mc{X})}^k M)$. Note that we may know more about $L_P^kf(x)$ than simply that it is bounded in $L^2$-norm, but a bound in $L^2$-norm suffices. In particular, from such a bound along with~\eqref{pf:approximation_error_nonlocal_laplacian_7} we deduce that for any $x \in \mc{X}_{(k + 1)\varepsilon}$,
\begin{equation}
\label{pf:approximation_error_nonlocal_laplacian_8.5}
L_{P,\varepsilon}^{k + 1}f(x) = \bigl(L_{P,\varepsilon} \circ L_{P,\varepsilon}^k f)(x)= L_{P,\varepsilon} g_s(x)\varepsilon^{s - 2k} = g_{s}^{s}(x) \varepsilon^{s - 2(k + 1)},
\end{equation}
establishing~\eqref{pf:approximation_error_nonlocal_laplacian_0}. 

\emph{Proof of \eqref{pf:approximation_error_nonlocal_laplacian_1}.} If $s > 2(k + 1)$, then by the inductive hypothesis we have that for all $x \in \mc{X}_{k\varepsilon}$, 
\begin{equation*}
L_{P,\varepsilon}^kf(x) = \sigma_{\eta}^k \Delta_P^kf(x) + \sum_{j = 1}^{\floor{(s - 1)/2} - k} g_{2(j + k)}(x) \varepsilon^{2j} + g_s(x) \varepsilon^{s - 2k}.
\end{equation*}
Thus for any $x \in \mc{X}_{(k + 1)\varepsilon}$, 
\begin{equation*}
L_{P,\varepsilon}^{k + 1}f(x) = \bigl(L_{P,\varepsilon} \circ L_{P,\varepsilon}^k f\bigr)(x) = \sigma_{\eta}^k L_{P,\varepsilon}\Delta_P^kf(x) + \sum_{j = 1}^{\floor{(s - 1)/2} - k} L_{P,\varepsilon}g_{2(j + k)}(x) \varepsilon^{2j} + L_{P,\varepsilon}g_s(x) \varepsilon^{s - 2k}
\end{equation*}
There are three terms on the right hand side of this equality, and we now analyze each separately.
\begin{enumerate}
	\item Noting that $\Delta_P^kf \in H^{s - 2k}(\mc{X}; C\|p\|_{C^{s - 1}(\mc{X})}^kM)$, we use~\eqref{pf:approximation_error_nonlocal_laplacian_8} to derive that
	\begin{align}
	L_{P,\varepsilon}\Delta_P^kf(x) & = \sigma_{\eta} \Delta_P^{k + 1}f(x) + \sum_{j = 1}^{(s - 2k - 1)/2 - } g_{2j + 2}^{s - 2k}(x)\varepsilon^{2j} + g_{s - 2k}^{s - 2k}(x) \varepsilon^{s - 2k - 2} \nonumber \\
	& = \sigma_{\eta} \Delta_P^{k + 1}f(x) + \sum_{j = 1}^{(s - 1)/2 - (k + 1)} g_{2(k + 1 + j)}(x)\varepsilon^{2j} + g_{s}(x) \varepsilon^{s - 2(k + 1)}, \label{pf:approximation_error_nonlocal_laplacian_9}
	\end{align}
	where in the second equality we have simply used the fact $g_j^{\ell}(x) = g_{(s - \ell) + j}(x)$ to rewrite the equation.
	\item Suppose $j < \floor{(s - 1)/2} - k$. Then we use~\eqref{pf:approximation_error_nonlocal_laplacian_8} to derive that
	\begin{align*}
	L_{P,\varepsilon}g_{2(j + k)}(x) & = \sigma_{\eta}\Delta_P g_{2(j + k)}(x) + \sum_{i = 1}^{\floor{(s - 2j - 2k - 1)/2} - 1} g_{2(i + 1)}^{s - 2(j + k)}(x)\varepsilon^{2i} + g_{s - 2(j + k)}^{s - 2(j + k)}(x) \varepsilon^{s - 2(j + k + 1)} \\
	& = g_{2(j + k + 1)}(x) + \sum_{i = 1}^{\floor{(s - 1)/2} - (j + k + 1)} g_{2(i + j + k + 1)}(x)\varepsilon^{2i} + g_{s}(x) \varepsilon^{s - 2(j + k + 1)},
	\end{align*}
	where in the second equality we have again used $g_j^{\ell}(x) = g_{(s - \ell) + j}(x)$, and also written $\sigma_{\eta} \Delta_Pf = g_{2}^{s - 2(j + k)} = g_{2(j + k + 1)}$, since the particular dependence on the Laplacian $\Delta_P$ will not matter. From here, multiplying by $\varepsilon^{2j}$, we conclude that
	\begin{align}
	\varepsilon^{2j} L_{P,\varepsilon}g_{2(j + k)}(x) & = g_{2(j + k + 1)}(x) \varepsilon^{2j} + \sum_{i = 1}^{\floor{(s - 1)/2} - (j + k + 1)} g_{2(i + j + k + 1)}(x)\varepsilon^{2(i + j)} + g_{s}(x) \varepsilon^{s - 2(k + 1)} \nonumber \\ 
	& = g_{2(j + k + 1)}(x) \varepsilon^{2j} + \sum_{m = 1}^{\floor{(s - 1)/2} - (k + 1)}  g_{2(m + k + 1)}(x)\varepsilon^{2m} + g_{s}(x) \varepsilon^{s - 2(k + 1)} \label{pf:approximation_error_nonlocal_laplacian_10},
	\end{align}
	with the second equality following upon changing variables to $m = i + j$. 
	
	On the other hand if $j = \floor{(s - 1)/2} - k$, then the calculation is much simpler,
	\begin{equation}
	\label{pf:approximation_error_nonlocal_laplacian_11}
	\varepsilon^{2j} L_{P,\varepsilon}g_{2(j + k)}(x) = g_{s - 2(j + k)}^{s - 2(j + k)}(x) \varepsilon^{2j} \varepsilon^{s - 2(j + k) - 2} = g_s(x) \varepsilon^{s - 2(k + 1)}.
	\end{equation}
	\item Finally, it follows immediately from~\eqref{pf:approximation_error_nonlocal_laplacian_8} that
	\begin{equation}
	\label{pf:approximation_error_nonlocal_laplacian_12}
	L_{P,\varepsilon}g_s(x) \varepsilon^{s - 2k} = g_{s}(x) \varepsilon^{s - 2(k + 1)}.
	\end{equation}
\end{enumerate}
Plugging~\eqref{pf:approximation_error_nonlocal_laplacian_9}-\eqref{pf:approximation_error_nonlocal_laplacian_12} back into~\eqref{pf:approximation_error_nonlocal_laplacian_8.5} proves the claim.

\subsection{Boundary behavior of non-local Laplacian}
\label{subsec:boundary_behavior_nonlocal_laplacian}

In Lemma~\ref{lem:approximation_error_nonlocal_laplacian_boundary}, we establish that if $f$ is Sobolev smooth of order $s > 2k$ and zero-trace, then near the boundary of $\mc{X}$ the non-local Laplacian $L_{P,\varepsilon}^kf$ is close to $0$ in the $\Leb^2$-sense.
\begin{lemma}
	\label{lem:approximation_error_nonlocal_laplacian_boundary}
	Assume Model~\ref{def:model_flat_euclidean}. Let $s,k \in \mathbb{N}$. Suppose that $f \in H_0^{s}(\mc{X};M)$. Then there exist numbers $c,C > 0$ that do not depend on $M$, such that for all $\varepsilon < c$, 
	\begin{equation*}
	\|L_{P,\varepsilon}^kf\|_{L^2(\partial_{k\varepsilon}\mc{X})}^2 \leq C \varepsilon^{2(s - 2k)}M^2.
	\end{equation*}
\end{lemma}

\paragraph{Proof (of Lemma~\ref{lem:approximation_error_nonlocal_laplacian_boundary})}
Applying Lemma~\ref{lem:l2estimate_nonlocal_laplacian}, we have that
\begin{equation*}
\|L_{P,\varepsilon}^kf\|_{L^2(\partial_{k\varepsilon}(\mc{X}))}^2 \leq \frac{(Cp_{\max})^{2}}{\varepsilon^4} \|L_{P,\varepsilon}^{k - 1}f\|_{L^2(\partial_{k\varepsilon}(\mc{X}))}^2 \leq \cdots \leq \frac{(Cp_{\max})^{2}}{\varepsilon^{4k}} \|f\|_{L^2(\partial_{k\varepsilon}(\mc{X}))}^2   
\end{equation*}
Thus it remains to show that for all $\varepsilon < c$,
\begin{equation}
\label{pf:approximation_error_nonlocal_laplacian_boundary_0}
\|f\|_{L^2(\partial_{k\varepsilon}(\mc{X}))}^2 = \int_{\partial_{k\varepsilon}(\mc{X})} \bigl(f(x)\bigr)^2 \,dx \leq C_1 \varepsilon^{2s} \|f\|_{H^s(\mc{X})}^2.
\end{equation}
We will build to~\eqref{pf:approximation_error_nonlocal_laplacian_boundary_0} by a series of intermediate steps, following the same rough structure as the proof of Theorem 18.1 in \citet{leoni2017}. For simplicity, we will take $k = 1$; the exact same proof applies to the general case upon assuming $\varepsilon < c/k$.

\underline{\textit{Step 1: Local Patch.}}
To begin, we assume that for some $c_0 > 0$ and a Lipschitz mapping $\phi: \Reals^{d - 1} \to [-c_0,c_0]$, we have that $f \in C_c^{\infty}(U_{\phi}(c_0))$, where 
\begin{equation*}
U_{\phi}(c_0) = \Bigl\{y \in Q(0,c_0): \phi(y_{-d}) \leq y_d\Bigr\}, 
\end{equation*}
and here $Q(0,c_0)$ is the $d$-dimensional cube of side length $c_0$, centered at $0$. We will show that for all $0 < \varepsilon < c_0$, and for the tubular neighborhood $V_{\phi}(\varepsilon) = \{y \in Q(0,c_0): \phi(y_{-d}) \leq y_d \leq \phi(y_{-d}) + \varepsilon\}$, we have that
\begin{equation*}
\int_{V_{\phi}(\varepsilon)} |f(x)|^2 \,dx \leq C\varepsilon^{2s} \|f\|_{H^s(U_{\phi}(c_0))}^2.
\end{equation*}
For a given $y = (y',y_d) \in V_{\phi}(\varepsilon)$, let $y_0 = (y',\phi(y'))$. Taking the Taylor expansion of $f(y)$ around $y = y_0$ because $u$ is compactly supported in $V_{\phi}$ it follows that,
\begin{align*}
f(y) & = f(y_0) + \sum_{j = 1}^{s - 1} \frac{1}{j!} D^{je_d}f(y_0) \bigl(y_d - \phi(y')\bigr)^j + \frac{1}{(s - 1)!}\int_{\phi(y')}^{y_d} (1 - t)^{s - 1} D^{se_d}f(y',z) \bigl(y_d - z\bigr)^{s - 1} \,dz \Longrightarrow\\
|f(y)| & \leq C\varepsilon^{s - 1}\int_{\phi(y')}^{y_d} \bigl|D^{se_d}f(y',z)\bigr| \,dz. 
\end{align*}
Consequently, by squaring both sides and applying Cauchy-Schwarz, we have that
\begin{equation*}
|f(y)|^2 \leq C\varepsilon^{2(s - 1)} \biggl(\int_{\phi(y')}^{y_d} \bigl|D^{se_d}f(y',z)\bigr| \,dz\biggr)^2 \leq C\varepsilon^{2s - 1} \int_{\phi(y')}^{y_d} \bigl|D^{se_d}f(y',z)\bigr|^2 \,dz.
\end{equation*}
Applying this bound for each $y \in V_{\phi}(\varepsilon)$, and then integrating, we obtain
\begin{align}
\int_{V_{\phi}(\varepsilon)} |f(y)|^2 \,dy & \leq \int_{Q_{d - 1}(c_0)} \int_{\phi(y')}^{\phi(y') + \varepsilon} |f(y',y_d)|^2 \,dy_d \,dy' \nonumber \\
& \leq C\varepsilon^{2s - 1}\int_{Q_{d - 1}(c_0)}  \int_{\phi(y')}^{\phi(y') + \varepsilon} \int_{\phi(y')}^{y_d} \bigl|D^{se_d}f(y',z)\bigr|^2 \,dz \,dy_d \,dy' \label{pf:approximation_error_nonlocal_laplacian_boundary_1}
\end{align}
where we have written $Q_{d - 1}(0,c_0)$ for the $d - 1$ dimensional cube of side length $c_0$, centered at $0$. Exchanging the order of the inner two integrals then gives
\begin{align*}
\int_{\phi(y')}^{\phi(y') + \varepsilon} \int_{\phi(y')}^{y_d} \bigl|D^{se_d}f(y',z)\bigr|^2 \,dz \,dy_d & = \int_{\phi(y')}^{\phi(y') + \varepsilon} \int_{z}^{\varepsilon} \bigl|D^{se_d}f(y',z)\bigr|^2 \,dy_d \,dz \\
& \leq C \varepsilon \int_{\phi(y')}^{\phi(y') + \varepsilon} \bigl|D^{se_d}f(y',z)\bigr|^2 \,dz \\
& \leq C \varepsilon \int_{\phi(y')}^{c_0} \bigl|D^{se_d}f(y',z)\bigr|^2 \,dz.
\end{align*}
Finally, plugging back into~\eqref{pf:approximation_error_nonlocal_laplacian_boundary_1}, we conclude that
\begin{equation*}
\int_{V_{\phi}(\varepsilon)} |f(y)|^2 \,dy \leq C \varepsilon^{2s} \int_{Q_{d - 1}(0,c_0)} \int_{\phi(y')}^{c_0} \bigl|D^{se_d}f(y',z)\bigr|^2 \,dz \,dy' \leq C \varepsilon^{2s} |u|_{H^s(U_{\phi}(c_0))}^2.
\end{equation*}

\underline{\textit{Step 2: Rigid motion of local patch.}} Now, suppose that at a point $x_0 \in \partial \mc{X}$, there exists a rigid motion $T: \Rd \to \Rd$ for which $T(x_0) = 0$, and a number $C_0$ such that for all $\varepsilon \cdot C_0 \leq c_0$, 
\begin{equation*}
T\bigl(Q_{T}(x_0,c_0) \cap \partial_{\varepsilon}\mc{X}\bigr) \subseteq V_{\phi}\bigl(C_0\varepsilon\bigr) \quad\textrm{and}\quad T\bigl(Q_T(x_0,c_0) \cap \mc{X}\bigr) = U_{\phi}(c_0).
\end{equation*}
Here $Q_{T}(x_0,c_0))$ is a (not necessarily coordinate-axis-aligned) cube of side length $c_0)$, centered at $x_0$. Define $v(y) := f(T^{-1}(y))$ for $y \in U_{\phi}(c_0)$. If $u \in C_c^{\infty}(\mc{X})$, then $v \in C_c^{\infty}(U_{\phi}(c_0))$, and moreover $\|v\|_{H^s(U_{\phi}(c_0))}^2 = \|f\|_{H^s(Q_{T}(x_0,c_0) \cap \mc{X})}^2$. Therefore, using the upper bound that we derived in Step 1,
\begin{equation*}
\int_{V_{\phi}(C_0 \cdot \varepsilon)} |v(y)|^2 \,dy \leq C \varepsilon^{2s} \|v\|_{H^s(U_{\phi}(c_0))}^2,
\end{equation*}
we conclude that
\begin{align*}
\int_{Q_{T}(x_0,c_0) \cap \partial_{\varepsilon}\mc{X}} |f(x)|^2 \,dx & = \int_{T(Q_T(x_0,c_0)) \cap \partial_{\varepsilon}\mc{X})} |v(y)|^2 \,dy \\
& \leq \int_{V_{\phi}(C_0 \cdot \varepsilon)} |v(y)|^2 \,dy \\
& \leq C \varepsilon^{2s} \|v\|_{H^s(U_{\phi}(c_0))}^2 = C \varepsilon^{2s} \|f\|_{H^s(Q_{T}(x_0,c_0)) \cap \mc{X})}^2 \leq C \varepsilon^{2s} \|f\|_{H^s(\mc{X})}^2.
\end{align*}

\underline{\textit{Step 3: Lipschitz domain}}.
Finally, we deal with the case where $\mc{X}$ is assumed to be an open, bounded subset of $\Rd$, with Lipschitz boundary. In this case, at every $x_0 \in \partial \mc{X}$, there exists a rigid motion $T_{x_0}: \Rd \to \Rd$ such that $T_{x_0}(x_0) = 0$, a number $c_0(x_0)$, a Lipschitz function $\phi_{x_0}:\Reals^{d - 1} \to [-c_0,c_0]$, and a number $C_0(x_0)$, such that for all $\varepsilon \cdot C_0(x_0) \leq c_0(x_0)$,
\begin{equation*}
T\bigl(Q_{T}(x_0,c_0(x_0)) \cap \partial_{\varepsilon}\mc{X}\bigr) \subseteq V_{\phi}\bigl(C_0(x_0) \cdot \varepsilon\bigr) \quad\textrm{and}\quad T\bigl(Q_T(x_0,c_0(x_0)) \cap \mc{X}\bigr) = U_{\phi}(c_0(x_0)).
\end{equation*}
Therefore for every $x_0 \in \partial \mc{X}$, it follows from the previous step that
\begin{equation*}
\int_{Q_{T_{x_0}}(x_0,c_0(x_0)) \cap \partial_{\varepsilon}\mc{X}} |f(x)|^2 \,dx \leq C(x_0) \varepsilon^{2s} \|f\|_{H^s(\mc{X})}^2,
\end{equation*}
where on the right hand side $C(x_0)$ is a constant that may depend on $x_0$, but not on $u$ or $\varepsilon$.

We conclude by taking a collection of cubes that covers $\partial_{\varepsilon}\mc{X}$ for all $\epsilon$ sufficiently small. First, we note that by a compactness argument there exists a finite subset of the collection of cubes $\{Q_{T_{x_0}}(x_0,c_0(x_0)/2): x_0 \in \partial\mc{X} \}$ which covers $\partial \mc{X}$, say $Q_{T_{x_1}}(x_1,c_0(x_1)/2),\ldots, Q_{T_{x_N}}(x_N,c_0(x_N)/2)$. Then, for any $\varepsilon \leq \min_{i = 1,\ldots,N} c_0(x_i)/2$, it follows from the triangle inequality that
\begin{equation*}
\partial_{\varepsilon}\mc{X} \subseteq \bigcup_{i = 1}^{N} Q_{T_{x_i}}(x_i, c_0(x_i)).
\end{equation*}
As a result,
\begin{equation*}
\int_{\partial_{\varepsilon}\mc{X}} |f(x)|^2 \leq \sum_{i = 1}^{N} \int_{Q_{T_{x_i}}(x_i, c_0(x_i)) \cap \partial_{\varepsilon}(\mc{X})} |f(x)|^2 \leq  \varepsilon^{2s} \|f\|_{H^s(\mc{X})}^2 \sum_{i = 1}^{N}C_0(x_i),
\end{equation*}
which proves the claim of~\eqref{pf:approximation_error_nonlocal_laplacian_boundary_0}.

\subsection{Estimate of non-local Sobolev seminorm}
\label{subsec:estimate_nonlocal_seminorm}

Now, we use the results of the preceding two sections to prove~\eqref{pf:graph_seminorm_ho_2}. We will divide our analysis in two cases, depending on whether $s$ is odd or even, but before we do this we state some facts that will be applicable to both cases. First, we  recall that $L_{P,\varepsilon}$ is self-adjoint in $L^2(P)$, meaning $\dotp{L_{P,\varepsilon}f}{g}_{P} = \dotp{f}{L_{P,\varepsilon}g}_{P}$ for all $f, g \in L^2(P)$. We also recall the definition of the Dirichlet energy $E_{P,\varepsilon}(f;\mc{X})$,
\begin{equation}
\label{eqn:dirichlet_energy}
\dotp{L_{P,\varepsilon}f}{f}_{P} = \frac{1}{\varepsilon^{d + 2}}\int_{\mc{X}} \int_{\mc{X}} \bigl(f(x) - f(x')\bigr)^2 \eta\biggl(\frac{\|x' - x\|}{\varepsilon}\biggr) \,dP(x') \,dP(x) =: E_{P,\varepsilon}(f;\mc{X}).
\end{equation}
Finally, we recall a result of \textcolor{blue}{(Green 21)}: there exist constants $c_0$ and $C_0$ which do not depend on $M$, such that for all $\varepsilon < c_0$ and for any $f \in H^1(\mc{X};M)$,
\begin{equation}
\label{pf:estimate_nonlocal_seminorm_1}
E_{P,\varepsilon}(f;\mc{X}) \leq C_0 M^2.
\end{equation}
\paragraph{Case 1: $s$ odd.}
Suppose $s$ is odd, so that $s \geq 3$. Taking $k = (s - 1)/2$, we use the self-adjointness of $L_{P,\varepsilon}$ to relate the non-local semi-norm $\dotp{L_{P,\varepsilon}^sf}{f}_{P}$ to a non-local Dirichlet energy,
\begin{equation*}
\dotp{L_{P,\varepsilon}^sf}{f}_P = \dotp{L_{P,\varepsilon}^{k + 1}f}{L_{P,\varepsilon}^{k}f}_P = E_{P,\varepsilon}(L_{P,\varepsilon}^{k}f;\mc{X}).
\end{equation*}
We now separate this energy into integrals over $\mc{X}_{k\varepsilon}$ and $\partial_{k\varepsilon}(\mc{X})$,
\begin{align}
E_{P,\varepsilon}(L_{P,\varepsilon}^{k}f;\mc{X}) & = \frac{1}{\varepsilon^{d + 2}}\Biggl\{\int_{\mc{X}_{k\varepsilon}} \int_{\mc{X}_{k\varepsilon}} \bigl(L_{P,\varepsilon}^kf(x) - L_{P,\varepsilon}^kf(x')\bigr)^2 \eta\biggl(\frac{\|x' - x\|}{\varepsilon}\biggr) \,dP(x') \,dP(x) \nonumber \\
& \quad + \int_{\partial_{k\varepsilon}\mc{X}} \int_{\partial_{k\varepsilon}\mc{X}} \bigl(L_{P,\varepsilon}^kf(x) - L_{P,\varepsilon}^kf(x')\bigr)^2 \eta\biggl(\frac{\|x' - x\|}{\varepsilon}\biggr) \,dP(x') \,dP(x)\Biggr\} \nonumber \\
& := E_{P,\varepsilon}(L_{P,\varepsilon}^{k}f;\mc{X}_{k\varepsilon}) + E_{P,\varepsilon}(L_{P,\varepsilon}^{k}f;\partial_{k\varepsilon}\mc{X}) \label{pf:estimate_nonlocal_seminorm_1.5}
\end{align}
and upper bound each energy separately. For the first term, we add and substract $\sigma_{\eta}^k\Delta_P^kf(x)$ and $\sigma_{\eta}^k\Delta_P^kf(x')$ within the integrand, then use the triangle inequality and \textcolor{red}{the symmetry between $x$ and $x'$} to deduce that
\begin{equation}
\label{pf:estimate_nonlocal_seminorm_2}
E_{P,\varepsilon}(L_{P,\varepsilon}^{k}f;\mc{X}_{k\varepsilon}) \leq 3 \sigma_{\eta}^{2k} E_{P,\varepsilon}(\Delta_P^kf;\mc{X}_{k\varepsilon}) + \frac{2}{\varepsilon^{d + 2}}\int_{\mc{X}_{k\varepsilon}} \int_{\mc{X}_{k\varepsilon}} \bigl(L_{P,\varepsilon}^kf(x) - \sigma_{\eta}^k \Delta_P^kf(x)\bigr)^2 \eta\biggl(\frac{\|x' - x\|}{\varepsilon}\biggr) \,dP(x') \,dP(x).
\end{equation}
Noticing that $\Delta_P^kf \in H^1(\mc{X};\|p\|_{C^{s - 1}(\mc{X})}^kM)$, we use~\eqref{pf:estimate_nonlocal_seminorm_1} to conclude that $E_{P,\varepsilon}(\Delta_P^kf;\mc{X}_{k\varepsilon}) \leq C_0M^2$. On the other hand, it follows from Assumption~\ref{asmp:kernel_flat_euclidean} and~\eqref{eqn:approximation_error_nonlocal_laplacian_1} that
\begin{align*}
\frac{2}{\varepsilon^{d + 2}}\int_{\mc{X}_{k\varepsilon}} \int_{\mc{X}_{k\varepsilon}} \bigl(L_{P,\varepsilon}^kf(x) - \sigma_{\eta}^k \Delta_P^kf(x)\bigr)^2 \eta\biggl(\frac{\|x' - x\|}{\varepsilon}\biggr) \,dP(x') \,dP(x) & \leq \frac{2p_{\max}}{\varepsilon^{2}}\int_{\mc{X}_{k\varepsilon}} \bigl(L_{P,\varepsilon}^kf(x) - \sigma_{\eta}^k \Delta_P^kf(x)\bigr)^2 \,dP(x) \\
& \leq C_1M^2.
\end{align*}
Plugging these two bounds into~\eqref{pf:estimate_nonlocal_seminorm_2} gives the desired upper bound on $E_{P,\varepsilon}(L_{P,\varepsilon}^{k};\mc{X}_{k\varepsilon})$. 

For the second term in~\eqref{pf:estimate_nonlocal_seminorm_1.5}, we apply Lemmas~\ref{lem:dirichlet_estimate_nonlocal_laplacian} and~\ref{lem:approximation_error_nonlocal_laplacian_boundary} and conclude that,
\begin{equation*}
E_{P,\varepsilon}(L_{P,\varepsilon}^{k}f;\partial_{k\varepsilon}\mc{X}) \leq \frac{4p_{\max}^2}{\varepsilon^2}\|L_{P,\varepsilon}^kf\|_{L^2(\partial_{k\varepsilon}\mc{X})} \leq C M^2.
\end{equation*}
\paragraph{Case 2: $s$ even.}
If $s \in \mathbb{N}$ is even, $s \geq 2$, then letting $k = (s - 2)/2$, the  self-adjointness of $L_{P,\varepsilon}$ implies
\begin{equation*}
\dotp{L_{P,\varepsilon}^sf}{f}_P = \|L_{P,\varepsilon}^{k + 1}f\|_P^2.
\end{equation*}
As in the first case, we divide the integral up into the interior region $\mc{X}_{k\varepsilon}$ and the boundary region $\partial_{k\varepsilon}\mc{X}$,
\begin{equation}
\label{pf:estimate_nonlocal_seminorm_3}
\|L_{P,\varepsilon}^{k + 1}f\|_P^2 \leq p_{\max} \|L_{P,\varepsilon}^{k + 1}f\|_{\Leb^2(\mc{X})}^2 \leq p_{\max}\biggl\{\int_{\mc{X}_{k\varepsilon}} \bigl(L_{P,\varepsilon}^{k + 1}f(x)\bigr)^2 \,dP(x) + \int_{\partial_{k\varepsilon}\mc{X}} \bigl(L_{P,\varepsilon}^{k + 1}f(x)\bigr)^2 \,dP(x)\biggr\},
\end{equation}
and upper bound each term separately. For the first term, adding and subtracting $\sigma_{\eta}^k \Delta_P^kf(x)$ gives
\begin{align*}
\int_{\mc{X}_{k\varepsilon}} \bigl(L_{P,\varepsilon}^{k + 1}f(x)\bigr)^2 \,dP(x) & \leq 2\int_{\mc{X}_{k\varepsilon}} \bigl(L_{P,\varepsilon}\Delta_P^kf(x)\bigr)^2 \,dP(x)  + 2 \int_{\mc{X}_{k\varepsilon}} \Bigl(L_{P,\varepsilon}\bigl(L_{P,\varepsilon}^kf- \sigma_{\eta}\Delta_P^kf\bigr)(x)\Bigr)^2 \,dP(x) \\
& \overset{(i)}{\leq} CM^2  + 2 \int_{\mc{X}_{k\varepsilon}} \Bigl(L_{P,\varepsilon}\bigl(L_{P,\varepsilon}^kf- \sigma_{\eta}\Delta_P^kf\bigr)(x)\Bigr)^2 \,dP(x) \\
& \overset{(ii)}{\leq} CM^2  + \frac{Cp_{\max}^2}{\varepsilon^{2}} \|L_{P,\varepsilon}^kf- \sigma_{\eta}\Delta_P^kf\|_{L^2(\mc{X}_{k\varepsilon})}^2 \\
& \overset{(iii)}{\leq} CM^2,
\end{align*}
with $(i)$ following from~\eqref{pf:approximation_error_nonlocal_laplacian_0} since $\Delta_P^kf \in H^2(\mc{X};M\|p\|_{C^{s - 1}(\mc{X})}^l)$, $(ii)$ following from Lemma~\ref{lem:l2estimate_nonlocal_laplacian}, and $(iii)$ following from~\eqref{eqn:approximation_error_nonlocal_laplacian_2}.

Then Lemma~\ref{lem:approximation_error_nonlocal_laplacian_boundary} shows that the second term in~\eqref{pf:estimate_nonlocal_seminorm_3} satisfies
\begin{equation*}
\int_{\partial_{k\varepsilon}\mc{X}} \bigl(L_{P,\varepsilon}^{k + 1}f(x)\bigr)^2 \,dP(x) \leq CM^2.
\end{equation*}

\subsection{Assorted integrals}
\label{subsec:integrals}

\textcolor{red}{(TODO): The language ``...for a Borel set $U \subseteq \mc{X}$...'' should be checked.}

\begin{lemma}
	\label{lem:l2estimate_nonlocal_laplacian}
	Assume Model~\ref{def:model_flat_euclidean}. Suppose $f \in L^2(U;M)$ for a Borel set $U \subseteq \mc{X}$, and let $L_{P,\varepsilon}$ be defined with respect to a kernel $\eta$ that satisfies~\ref{asmp:kernel_flat_euclidean}. Then there exists a constant $C$ which does not depend on $f$ or $M$ such that
	\begin{equation}
	\label{eqn:l2estimate_nonlocal_laplacian}
	\|L_{P,\varepsilon}f\|_{\Leb^2(U)} \leq \frac{2 p_{\max}}{\varepsilon^2} \|f\|_{L^2(U)}
	\end{equation}
\end{lemma}

\begin{lemma}
	\label{lem:dirichlet_estimate_nonlocal_laplacian}
	Assume Model~\ref{def:model_flat_euclidean}. Suppose $f \in L^2(U;M)$ for a Borel set $U \subseteq \mc{X}$, and let $L_{P,\varepsilon}$ be defined with respect to a kernel $\eta$ that satisfies~\ref{asmp:kernel_flat_euclidean}. Then there exists a constant $C$ which does not depend on $f$ or $M$ such that
	\begin{equation}
	\label{eqn:dirichlet_estimate_nonlocal_laplacian}
	E_{P,\varepsilon}(f;U) \leq \frac{4 p_{\max}^2}{\varepsilon^2} \|f\|_{L^2(U)}^2
	\end{equation}
\end{lemma}

\begin{lemma}
	\label{lem:graph_seminorm_bias2}
	Assume Model~\ref{def:model_flat_euclidean}. Suppose $f \in H^1(\mc{X};M)$, and let $D_if$ be defined with respect to a kernel $\eta$ that satisfies~\ref{asmp:kernel_flat_euclidean}. Then there exists a constant $C$ which does not depend on $f$ or $M$, such that for any $i \in [n]$ and $\bj \in [n]^s$,
	\begin{equation*}
	\Ebb\Bigl[|D_{\bj}f(X_i)| \cdot |f(X_i) - f(X_{\bj_1})|\Bigr] \leq C \varepsilon^{2 + dk}M^2,
	\end{equation*}
	where $k + 1$ is the number of distinct indices in $i\bj$. 
\end{lemma}

\paragraph{Proof (of Lemma~\ref{lem:l2estimate_nonlocal_laplacian}).}
We fix a version of $f \in \Leb^2(U)$, so that we may speak of its pointwise values.

At a given point $x \in U$, we can upper bound $|L_{P,\varepsilon}f(x)|^2$ using the Cauchy-Schwarz inequality as follows,
\begin{align*}
|L_{P,\varepsilon}f(x)|^2 & \leq \biggl(\frac{p_{\max}}{\varepsilon^{2 + d}}\biggr)^2 \Biggl(\int_U \bigl(|f(x')| + |f(x)|\bigr)^2 \eta\biggl(\frac{\|x' - x\|}{\varepsilon}\biggr) \,dx'\Biggr)^2 \\
& \leq \biggl(\frac{p_{\max}}{\varepsilon^{2 + d}}\biggr)^2 \Biggl(\int_U \bigl(|f(x')| + |f(x)|\bigr)^2 \eta\biggl(\frac{\|x' - x\|}{\varepsilon}\biggr) \,dx' \cdot \int \eta\biggl(\frac{\|x' - x\|}{\varepsilon}\biggr) \,dx' \Biggr) \\
& = \frac{p_{\max}^2}{\varepsilon^{4 + d}} \int_U \bigl(|f(x')| + |f(x)|\bigr)^2 \eta\biggl(\frac{\|x' - x\|}{\varepsilon}\biggr) \,dx'.
\end{align*}
The equality follows by the assumption $\int_{\Rd} \eta(\|z\|)\,dx = 1$ in~\ref{asmp:kernel_flat_euclidean}. Integrating over all $x \in U$, it follows from the triangle inequality that
\begin{align}
\|L_{P,\varepsilon}\|_{\Leb^2(U)}^2 & \leq \frac{2 p_{\max}^2}{\varepsilon^{4 + d}} \int_{U} \int_{U} \bigl(|f(x')|^2 + |f(x)|^2\bigr) \eta\biggl(\frac{\|x' - x\|}{\varepsilon}\biggr) \,dx' \,dx \nonumber \\
& \leq \frac{2 p_{\max}^2}{\varepsilon^{4 + d}} \int_{U} \int_{U} \bigl(|f(x')|^2 + |f(x)|^2\bigr) \eta\biggl(\frac{\|x' - x\|}{\varepsilon}\biggr) \,dx' \,dx. \label{pf:l2estimate_nonlocal_laplacian_1}
\end{align}
Finally, using Fubini's Theorem we determine that
\begin{equation}
\int_{U} \int_{U} \bigl(|f(x')|^2 + |f(x)|^2\bigr) \eta\biggl(\frac{\|x' - x\|}{\varepsilon}\biggr) \,dx' \,dx = 2 \int_{U} \int_{U} |f(x)|^2 \eta\biggl(\frac{\|x' - x\|}{\varepsilon}\biggr) \,dx \leq 2 \varepsilon^d \int_{U} |f(x)|^2 \,dx = 2\varepsilon^d \|f\|_{\Leb^2(U)}^2,
\label{pf:l2estimate_nonlocal_laplacian_2}
\end{equation}
and by combining~\eqref{pf:l2estimate_nonlocal_laplacian_1} and~\eqref{pf:l2estimate_nonlocal_laplacian_2} we conclude that
\begin{equation*}
\|L_{P,\varepsilon}\|_{\Leb^2(U)}^2 \leq \frac{4p_{\max}^2}{\varepsilon^4} \|f\|_{\Leb^2(U)}^2. 
\end{equation*}

\paragraph{Proof (of Lemma~\ref{lem:dirichlet_estimate_nonlocal_laplacian}).}
We have
\begin{equation*}
E_{P,\varepsilon}(f) = \frac{1}{\varepsilon^{2 + d}} \int_{U} \int_{U} \bigl(f(x) - f(x')\bigr)^2 \eta\biggl(\frac{\|x' - x\|}{\varepsilon}\biggr) \,dP(x') \,dP(x) \leq \frac{2p_{\max}^2}{\varepsilon^{2 + d}} \int_{U} \int_{U} \bigl(|f(x)|^2 + |f(x')|^2\bigr) \eta\biggl(\frac{\|x' - x\|}{\varepsilon}\biggr) \,dx' \,dx,
\end{equation*}
and the claim follows from~\eqref{pf:l2estimate_nonlocal_laplacian_2}.

\paragraph{Proof (of Lemma~\ref{lem:graph_seminorm_bias2}).}
Let $G_{n,\varepsilon}[X_{i\bj}]$ be the subgraph induced by vertices $X_i, X_{\bj_1},\ldots,X_{\bj_s}$. We make two observations. First, in order for $|D_{\bj}f(X_i)| \cdot |f(X_i)  - f(X_j)|$ to be non-zero, it must be the case that the subgraph $G_{n,\varepsilon}[X_{i\bj}]$ is connected. Second, noting that for any indices $i$ and $j$,
\begin{equation*}
|D_{ij}f(x)| \leq \Bigl(|D_jf(X_i)| + |D_jf(x)|\Bigr)\|\eta\|_{\infty}, 
\end{equation*}
a straightforward inductive argument implies that 
\begin{equation*}
|D_{\bj}f(X_i)| \leq s\|\eta\|_{\infty}^s \sum_{j \in i\bj} |D_{\bj_s}f(X_j)|.
\end{equation*}
Combining these two observations, we reduce the task to upper bounding the product of two (first-order) differences,
\begin{align*}
\mathbb{E}\Bigl[ |D_{\bj}f(X_i)| |f(X_i) - f(X_{\bj_1})| \Bigr] & = \mathbb{E}\Bigl[ |D_{\bj}f(X_i)| |f(X_i) - f(X_{\bj_1})| \cdot \1\bigl\{G_{n,\varepsilon}[X_{i\bj}]~~\textrm{is connected.} \bigr\} \Bigr] \\
& \leq s\|\eta\|_{\infty}^s \sum_{j \in i\bj} \mathbb{E}\Bigl[ |D_{\bj_s}f(X_j)| \cdot |f(X_i) - f(X_{\bj_1})| \cdot \1\bigl\{G_{n,\varepsilon}[X_{i\bj}]~~\textrm{is connected.} \bigr\}   \Bigr] \\
& \leq s\|\eta\|_{\infty}^s \sum_{j \in i\bj} \mathbb{E}\Bigl[ |f(X_j) - f(X_{\bj_s})| \cdot |f(X_i) - f(X_{\bj_1})| \cdot \1\bigl\{G_{n,\varepsilon}[X_{i\bj}]~~\textrm{is connected.} \bigr\}   \Bigr] 
\end{align*}
Next, from the Cauchy-Schwarz inequality we have that for any $j \in \bj$,
\begin{align*}
& \mathbb{E}\Bigl[ |f(X_j) - f(X_{\bj_s})| \cdot |f(X_i) - f(X_{\bj_1})| \cdot \1\bigl\{G_{n,\varepsilon}[X_{i\bj}]~~\textrm{is connected.} \bigr\}   \Bigr] \\
& \quad \leq \sqrt{\mathbb{E}\Bigl[ |f(X_j) - f(X_{\bj_s})|^2 \cdot \1\bigl\{G_{n,\varepsilon}[X_{i\bj}]~~\textrm{is connected.} \bigr\} \Bigr]} \cdot \sqrt{\mathbb{E}\Bigl[ |f(X_j) - f(X_{\bj_s})|^2 \cdot \1\bigl\{G_{n,\varepsilon}[X_{i\bj}]~~\textrm{is connected.} \bigr\} \Bigr]} \\
& \quad = \mathbb{E}\Bigl[ |f(X_j) - f(X_{i})|^2 \cdot \1\bigl\{G_{n,\varepsilon}[X_{i\bj}]~~\textrm{is connected.} \bigr\} \Bigr],
\end{align*}
with the equality following since each $X_i$ are identically distributed. Marginalizing out the contribution of all indices in $\bj$ not equal to $i$ or $j$ gives
\begin{align}
\mathbb{E}\Bigl[ |f(X_j) - f(X_{i})|^2 \cdot \1\bigl\{G_{n,\varepsilon}[X_{i\bj}]~~\textrm{is connected.} \bigr\} \Bigr] & \leq \bigl((s + 1)p_{\max}\nu_d\varepsilon^d\bigr)^{|i\bj\setminus\{j \cup i\}|} \cdot \mathbb{E}\Bigl[ |f(X_j) - f(X_{i})|^2 \1\{\|X_i - X_j\| \leq \varepsilon\}\Bigr] \nonumber \\
& \leq \bigl((s + 1)p_{\max}\nu_d\varepsilon^d\bigr)^{|i\bj\setminus\{j \cup i\}|} \cdot p_{\max}^2 \nu_d \varepsilon^{2 + d} M^2 \label{pf:graph_seminorm_bias2_1}
\end{align}
with the second inequality following from the proof of Lemma~1 in \textcolor{blue}{(Green 2021).} Finally, we notice that $|i\bj\setminus \{i \cup j\}| + 1 = k$, so that~\eqref{pf:graph_seminorm_bias2_1} gives the desired result.

\section{Graph Sobolev semi-norm, manifold domain}
\label{sec:graph_quadratic_form_manifold}
In this section we prove Proposition~\ref{prop:graph_seminorm_manifold}. Note that when $s = 1$, the upper bound~\eqref{eqn:graph_seminorm_manifold} follows immediately from Lemma~\ref{lem:dirichlet_energy_sobolev} and Markov's inequality. 

On the other hand when $s = 2$ or $s = 3$, we prove Proposition~\ref{prop:graph_seminorm_manifold} by first establishing some intermediate results, many of which are analogous to results we have already shown in the flat Euclidean case. Indeed, in some ways the proof will be simpler in the manifold setting than in the flat Euclidean case: there is no boundary, and we do not need to analyze the iterated nonlocal Laplacian $L_{P,\varepsilon}^j$ for $j > 1$. 

That being said, as mentioned in our main text, in the manifold setting there is some extra error induced by using Euclidean rather than geodesic distance. We upper bound this error by comparing $L_{P,\varepsilon}$ to an alternative nonlocal Laplacian $\wt{L}_{P,\varepsilon}$, which is defined with respect to geodesic distance. Precisely, let $d_{\mc{X}}(x,x')$ denote the geodesic distance between $x,x' \in \mc{X}$, and define
\begin{equation*}
\wt{L}_{P,\varepsilon}f(x) := \int_{\mc{X}} \bigl(f(x') - f(x)\bigr) \eta \biggl(\frac{d_{\mc{X}}(x',x)}{\varepsilon}\biggr) p(x') \,dx'.
\end{equation*}

We show the following results, each of which hold under the same assumptions as Proposition~\ref{prop:graph_seminorm_manifold}.
\begin{itemize}
	\item In Section~\ref{subsec:manifold_decomposition_graph_seminorm} we show that the graph Sobolev seminorm $\dotp{L_{n,\varepsilon}^sf}{f}_n$ is upper bounded by the sum of a nonlocal seminorm and a pure bias term: specifically, with probability at least $1 - 2\delta$,
	\begin{equation}
	\label{pf:graph_seminorm_manifold_1}
	\dotp{L_{n,\varepsilon}^sf}{f}_n \leq \frac{\dotp{L_{P,\varepsilon}^sf}{f}_P}{\delta} + C_1\frac{\varepsilon^2}{n\varepsilon^{2s + m}}M^2.
	\end{equation}
	This upper bound is essentially the same as~\eqref{pf:graph_seminorm_ho_1}, but with the intrinsic dimension $m$ taking the place of the ambient dimension $d$. The pure bias term will be of at most constant order when $\varepsilon \gtrsim n^{-1/(2(s-1) + m)}$. 
	\item In Section~\ref{subsec:error_euclidean_distance}, we show that the error incurred by using the ``wrong'' metric is negligible. Precisely, we find that
	\begin{equation}
	\label{eqn:nonlocal_laplacian_geodesic_error}
	\|L_{P,\varepsilon}f - \wt{L}_{P,\varepsilon}f\|_{L^2(\mc{X})}^2 \leq C_2 \varepsilon^2 |f|_{H^1(\mc{X})}^2.
	\end{equation}
	\item In Section~\ref{subsec:manifold_approximation_error_nonlocal_laplacian}, we analyze the approximation error of $\wt{L}_{P,\varepsilon}$. We show that when $f \in H^2(\mc{X})$ and $p \in C^1(\mc{X})$, 
	\begin{equation}
	\label{eqn:nonlocal_laplacian_approximation_error_manifold_l2}
	\|\wt{L}_{P,\varepsilon}f\|_{\Leb^2(\mc{X})}^2 \leq C_3 \|f\|_{H^2(\mc{X})}^2,
	\end{equation}
	whereas if $f \in H^3(\mc{X})$ and $p \in C^2(\mc{X})$, 
	\begin{equation}
	\label{eqn:nonlocal_laplacian_approximation_error_manifold_sobolev}
	\|\wt{L}_{P,\varepsilon}f - \sigma_{\eta}\Delta_Pf\|_{\Leb^2(\mc{X})}^2 \leq C_3 \varepsilon^2 \|f\|_{H^3(\mc{X})}^2.
	\end{equation}
	\item In Section~\ref{subsec:manifold_estimate_nonlocal_seminorm}, we use the results of the preceding two sections to show that if $f \in H^s(\mc{X})$ and $p \in C^{s - 1}(\mc{X})$, then 
	\begin{equation}
	\label{eqn:manifold_nonlocal_seminorm}
	\dotp{L_{P,\varepsilon}^sf}{f}_P \leq C_4\|f\|_{H^s(\mc{X})}^2.
	\end{equation}
	\item In Section~\ref{subsec:manifold_integrals} we state some technical results used in the previous sections.
\end{itemize}
We point out that when $f$ is H\"{o}lder smooth, results analogous to~\eqref{eqn:nonlocal_laplacian_approximation_error_manifold_sobolev} have been established in \citet{calder2019}. When $f$ is Sobolev smooth, our analysis (which relies heavily on Taylor expansions) is largely similar, except that the remainder term in the relevant Taylor expansion will be bounded in $L^2(\mc{X})$ norm rather than $L^{\infty}(\mc{X})$ norm. This is analogous to the situation in the flat Euclidean model.

\paragraph{Proof (of Proposition~\ref{prop:graph_seminorm_manifold}).} Follows immediately from~\eqref{pf:graph_seminorm_manifold_1} and~\eqref{eqn:manifold_nonlocal_seminorm}. \qed

\subsection{Decomposition of graph Sobolev seminorm}
\label{subsec:manifold_decomposition_graph_seminorm}
The proof of~\eqref{pf:graph_seminorm_manifold_1} is identical to the proof of~\eqref{pf:graph_seminorm_ho_1}, except substituting the intrinsic dimension $m$ for ambient dimension $d$, and using Lemma~\ref{lem:manifold_graph_seminorm_bias2} rather than Lemma~\ref{lem:graph_seminorm_bias2}.

\subsection{Error due to Euclidean Distance}
\label{subsec:error_euclidean_distance}

In this section, we prove~\eqref{eqn:nonlocal_laplacian_geodesic_error}. By applying Cauchy-Schwarz we obtain an upper bound on $|L_{P,\varepsilon}f(x) - \wt{L}_{P,\varepsilon}f(x)|^2$:
\begin{align}
\bigl[L_{P,\varepsilon}f(x) - \wt{L}_{P,\varepsilon}f(x)\bigr]^2 & \leq \frac{p_{\max}^2}{\varepsilon^{2(2 + m)}} \int_{\mc{X}} \bigl[f(x') - f(x)\bigr]^2 \biggl|\eta\biggl(\frac{\|x' - x\|}{\varepsilon}\biggr) - \eta\biggl(\frac{d_{\mc{X}}(x',x)}{\varepsilon}\biggr)\biggr| \,d\mu(x') \nonumber \\
& \quad \cdot \int_{\mc{X}} \biggl|\eta\biggl(\frac{\|x' - x\|}{\varepsilon}\biggr) - \eta\biggl(\frac{d_{\mc{X}}(x',x)}{\varepsilon}\biggr)\biggr| \,d\mu(x') \nonumber \\
& = \frac{1}{\varepsilon^{2(2 + m)}} A_1(x) \cdot A_2(x) \label{pf:error_euclidean_distance_0}
\end{align} 
Thus we have upper bounded $|L_{P,\varepsilon}f(x) - \wt{L}_{P,\varepsilon}f(x)|^2$ by the product of two terms, each of which we now suitably bound. 

To do so, we will use the following estimates, from Proposition 4 of \cite{trillos2019}: letting $R$ denote the reach of $\mc{X}$, for all $\|x' - x\| \leq R/2$,
\begin{equation}
\label{eqn:distance_error}
\|x' - x\| \leq d_{\mc{X}}(x',x) \leq \|x' - x\| + \frac{8}{R^2} \|x' - x\|^3.
\end{equation}
\paragraph{Upper bound on $A_1(x)$.}
From here forward we will assume $\varepsilon < R/2$. Consequently $\eta(\|x' - x\|/\varepsilon) \geq \eta(d_{\mc{X}}(x',x)/\varepsilon)$. Furthermore, letting $L_{\eta}$ denote the Lipschitz constant of $\eta$, and setting $\wt{\varepsilon} := (1 + 27\varepsilon^2/R^2)\varepsilon$ we have that
\begin{equation*}
\biggl|\eta\biggl(\frac{\|x' - x\|}{\varepsilon}\biggr) - \eta\biggl(\frac{d_{\mc{X}}(x',x)}{\varepsilon}\biggr)\biggr| \leq \frac{L_{\eta} 8 \varepsilon^2}{R^2} \cdot \1\bigl\{d_{\mc{X}}(x',x) \leq \varepsilon\bigr\} + \|\eta\|_{\infty} \cdot 1\{\varepsilon < d_{\mc{X}}(x',x) \leq \wt{\varepsilon}\}.
\end{equation*}
Thus,
\begin{equation*}
A_1(x) \leq \frac{8L_{\eta}\varepsilon^2}{R^2}\int_{\mc{X}}\bigl[f(x') - f(x)\bigr]^2 \1\{\|x' - x\| \leq \varepsilon\} \,d\mu(x') + \|\eta\|_{\infty} \int_{\mc{X}}\bigl[f(x') - f(x)\bigr]^2 \1\bigl\{\varepsilon < d_{\mc{X}}(x',x) \leq \wt{\varepsilon}\bigr\} \,d\mu(x') \\
\end{equation*}
Integrating over $\mc{X}$, we conclude from Lemma~\ref{lem:dirichlet_energy_remainder} and Lemma~3.3 of \citep{burago2014} and  that
\begin{equation*}
\int_{\mc{X}} A_1(x) \,d\mu(x) \leq \frac{8L_{\eta}\nu_m\varepsilon^2}{R^2(m + 2)} \Bigl(1 + CmKR^2\Bigr) \varepsilon^{m + 2} |f|_{H^1(\mc{X})}^2 + C\|\eta\|_{\infty}\varepsilon^{m + 4} |f|_{H^1(\mc{X})}^2 =: C_5 \varepsilon^{m + 4}|f|_{H^1(\mc{X})}.
\end{equation*}

\paragraph{Upper bound on $A_2(x)$.}
Integrating over $x' \in \mc{X}$, we see that
\begin{align}
\int_{\mc{X}} \biggl|\eta\biggl(\frac{\|x' - x\|}{\varepsilon}\biggr) - \eta\biggl(\frac{d_{\mc{X}}(x',x)}{\varepsilon}\biggr)\biggr| \,d\mu(x') & \leq \frac{8L_{\eta}\varepsilon^2}{R^2} \int_{\mc{X}} \1\bigl\{d_{\mc{X}}(x',x)\bigr\} \,d\mu(x') + p_{\max}\|\eta\|_{\infty} \int_{\mc{X}} \1\bigl\{ \varepsilon < d_{\mc{X}}(x',x) \leq \wt{\varepsilon} \bigr\} \,d\mu(x') \nonumber \\
& = \frac{8L_{\eta}\varepsilon^2}{R^2} \cdot \mu\bigl(B(x,\varepsilon)\bigr) +  p_{\max}\|\eta\|_{\infty} \Bigl[\mu\bigl(B(x,\wt{\varepsilon})\bigr) - \mu\bigl(B(x,\varepsilon)\bigr) \Bigr]. \label{pf:error_euclidean_distance_1}
\end{align}
Equation (1.36) in \cite{trillos2019} states that
\begin{equation*}
\bigl|\mu(B_\mc{X}(x,\varepsilon))  - \omega_m \varepsilon^m\bigr|  \leq CmK\varepsilon^{m + 2},
\end{equation*}
where $K$ is an upper bound on the \textcolor{red}{sectional curvature} of $\mc{X}$. Plugging this back into~\eqref{pf:error_euclidean_distance_1}, we conclude that
\begin{align*}
\int_{\mc{X}} \biggl|\eta\biggl(\frac{\|x' - x\|}{\varepsilon}\biggr) - \eta\biggl(\frac{d_{\mc{X}}(x',x)}{\varepsilon}\biggr)\biggr| \,d\mu(x') & \leq  \frac{8L_{\eta}\varepsilon^2}{R^2}\Bigl[\omega_m\varepsilon^{m} + CmK\varepsilon^{m + 2}\Bigr] + \|\eta\|_{\infty} \Bigl[\omega_m(\wt{\varepsilon}^m - {\varepsilon}^m) + 2CmK\varepsilon^{m + 2}\Bigr] \\
& \leq \frac{8L_{\eta}\varepsilon^2}{R^2}\Bigl[\omega_m\varepsilon^{m} + R^2CmK\varepsilon^{m}\Bigr] + \|\eta\|_{\infty} \varepsilon^{m + 2}\Bigl[\frac{27\omega_m}{R^2} + 2CmK\Bigr] \\
& =: C_6 \varepsilon^{m + 2}.
\end{align*}

\paragraph{Putting together the pieces.}
Plugging our upper bounds on $A_1(x)$ and $A_2(x)$ back into~\eqref{pf:error_euclidean_distance_0}, we deduce that
\begin{align*}
\|\wt{L}_{P,\varepsilon}f - L_{P,\varepsilon}f\|_{L^2(\mc{X})}^2 & \leq \frac{1}{\varepsilon^{2(2 + m)}} \int_{\mc{X}} A_1(x) \cdot A_2(x) \,d\mu(x) \\
& \leq \frac{C_6}{\varepsilon^{(2 + m)}} \int_{\mc{X}} A_1(x) \,d\mu(x) \\
& \leq C_5 C_6 \varepsilon^2 |f|_{H^1(\mc{X})}^2,
\end{align*}
thus proving the claimed result.

\subsection{Approximation Error of non-local Laplacian}
\label{subsec:manifold_approximation_error_nonlocal_laplacian}
Fix $x \in \mc{X}$. We begin with a pointwise estimate of $\wt{L}_{P,\varepsilon}f$, facilitated by expressing $w(v) = f(\exp_x(v))$ and $q(v) = p(\exp_x(v))$ in normal coordinates, as in \citep{calder2019}. Letting $J_x(v)$ be the Jacobian of the exponential map $\exp_x: B(0,\varepsilon) \subseteq T_x(\mc{X}) \to B_{\mc{X}}(x,\varepsilon)$ evaluated at $v \in T_x(\mc{X})$, we have
\begin{align*}
\wt{L}_{P,\varepsilon}f(x) & = \frac{1}{\varepsilon^{m + 2}} \int_{\mc{X}} \bigl(f(x') - f(x)\bigr) \eta\biggl(\frac{d_{\mc{X}}(x',x)}{\varepsilon}\biggr) \,dP(x') \\
& = \frac{1}{\varepsilon^{m + 2}} \int_{B(0,\varepsilon) \subset T_x(\mc{X})} \bigl(w(v) - w(0)\bigr) \eta\biggl(\frac{\|v\|}{\varepsilon}\biggr) J_x(v) q(v) \,dv \\
& = \frac{1}{\varepsilon^{2}}\biggl\{\int_{B(0,1)} \bigl(w(\varepsilon v) - w(0)\bigr) \eta(\|v\|) q(\varepsilon v) \,dv + \int_{B(0,1)} \bigl(w(\varepsilon v) - w(0)\bigr) \eta(\|v\|) q(\varepsilon v) \bigl(J_x(\varepsilon v) - 1\bigr) \,dv \biggr\} \\
& = A_1(x) + A_2(x)
\end{align*}
Note that $w$ and $q$ have the same smoothness properties as $f$ and $p$. Moreover, arguing exactly as we did in the flat Euclidean case, we can show that when $f \in H^2(\mc{X})$ and $p \in C^1(\mc{X})$, then
\begin{equation*}
\|A_1\|_{\Leb^2(\mc{X})}^2 \leq C \|f\|_{H^2(\mc{X})}^2
\end{equation*}
whereas if $f \in H^3(\mc{X})$ and $p \in C^2(\mc{X})$ then 
\begin{equation*}
\|A_1 - \sigma_{\eta} \Delta_Pf\|_{\Leb^2(\mc{X})}^2 \leq C\|f\|_{H^3(\mc{X})}^2 \varepsilon^2.
\end{equation*}
\textcolor{red}{(TODO): This is correct, but feels unacceptably vague. On the other hand, I don't want to have to redo the entire proof from the flat Euclidean section, since all the steps will be basically identical. Ask Ryan + Siva for guidance.}

Therefore it remains only to upper bound $A_2$ in $\Leb^2(\mc{X})$ norm. To do so, we recall (1.34) of \cite{trillos2019}: for any $\varepsilon < i_0$ and all $x \in \mc{X}$, the Jacobian $J_x(v)$ satisfies the upper bound
\begin{equation*}
|J_x(v) - 1| \leq CmK\varepsilon^2, \quad  \textrm{for all} ~~ v \in B(0,\varepsilon) \subseteq T_x(\mc{X}).
\end{equation*}
Combining this estimate with the Cauchy-Schwarz inequality, we conclude that
\begin{align*}
\|A_2\|_{\Leb^2(\mc{X})}^2 & \leq Cm^2K^2 \biggl[\int_{B(0,1)} \bigl(w(\varepsilon v) - w(0)\bigr)^2 \eta(\|v\|) q(\varepsilon v) \,dv\biggr] \cdot \biggl[\int_{B(0,1)} \eta(\|v\|) q(\varepsilon v) \,dv\biggr] \\
& \leq Cm^2K^2 \sigma_{\eta} (1 + L_q\varepsilon)  \int_{B(0,1)} \bigl(w(\varepsilon v) - w(0)\bigr)^2 \eta(\|v\|) q(\varepsilon v) \,dv \\
& \leq Cm^2K^2 \sigma_{\eta}^2 (1 + L_q\varepsilon) p_{\max}  \varepsilon^2 |f|_{H^1(\mc{X})}^2,
\end{align*}
with the final inequality following from (3.2) of~\cite{burago2014}. Combining our estimates on $A_1$ and $A_2$ yields the claim.

\subsection{Estimate of non-local Sobolev seminorm}
\label{subsec:manifold_estimate_nonlocal_seminorm}
In this subsection we establish that the upper bound~\eqref{eqn:manifold_nonlocal_seminorm} holds when $f \in H^s(\mc{X})$ and $p \in C^{s - 1}(\mc{X})$. We first consider $s = 2$, and then $s = 3$.

\paragraph{Case 1: $s = 2$.}
When $s = 2$, the triangle inequality implies that
\begin{equation*}
\dotp{L_{P,\varepsilon}^sf}{f}_P \leq 2p_{\max}\Bigl(\|L_{P,\varepsilon}f - \wt{L}_{P,\varepsilon}\|_{L^2(\mc{X})}^2 + \|\wt{L}_{P,\varepsilon}f\|_{L^2(\mc{X})}^2\Bigr)
\end{equation*}
The first term on the right hand side is upper bounded in~\eqref{eqn:nonlocal_laplacian_geodesic_error}, and the second term is upper bounded in~\eqref{eqn:nonlocal_laplacian_approximation_error_manifold_l2}. Together these estimates imply the claim.

\paragraph{Case 2: $s = 3$.}
When $s = 3$, the triangle inequality implies that
\begin{equation*}
\dotp{L_{P,\varepsilon}^sf}{f}_P = E_{P,\varepsilon}(L_{P,\varepsilon}f;\mc{X}) \leq 3\Bigl( E_{P,\varepsilon}(L_{P,\varepsilon}f - \wt{L}_{P,\varepsilon}f;\mc{X}) +  E_{P,\varepsilon}(\wt{L}_{P,\varepsilon}f - \sigma_{\eta} \Delta_Pf;\mc{X}) + \sigma_{\eta}^2 E_{P,\varepsilon}(\Delta_Pf;\mc{X})\Bigr)
\end{equation*}
We now upper bound each of the three terms on the right hand side of the above inequality. First, we note that by Lemma~\ref{lem:dirichlet_energy_l2} and~\eqref{eqn:nonlocal_laplacian_geodesic_error}, 
\begin{equation*}
 E_{P,\varepsilon}(L_{P,\varepsilon}f - \wt{L}_{P,\varepsilon}f;\mc{X}) \leq  \frac{C}{\varepsilon^2}\|L_{P,\varepsilon}f - \wt{L}_{P,\varepsilon}f\|_{\Leb^2(\mc{X})}^2 \leq C |f|_{H^1(\mc{X})}^2.
\end{equation*}
An equivalent upper bound on $E_{P,\varepsilon}(\wt{L}_{P,\varepsilon}f - \sigma_{\eta} \Delta_Pf;\mc{X})$ follows from Lemma~\ref{lem:dirichlet_energy_l2} and~\eqref{eqn:nonlocal_laplacian_approximation_error_manifold_sobolev}. Finally, we notice that $f \in H^3(\mc{X})$ and $p \in C^2(\mc{X})$ implies $\Delta_Pf \in H^1(\mc{X})$, and furthermore $|\Delta_Pf|_{H^1(\mc{X})} \leq \|p\|_{C^2(\mc{X})} \cdot \|f\|_{H^3(\mc{X})}$. We conclude from Lemma~\ref{lem:dirichlet_energy_sobolev} that
\begin{equation*}
E_{P,\varepsilon}(\Delta_Pf;\mc{X}) \leq C |\Delta_Pf|_{H^1(\mc{X})}^2 \leq C \|f\|_{H^3(\mc{X})}^2,
\end{equation*}
where in the final inequality we have absorbed $\|p\|_{C^2(\mc{X})}$ into the constant $C$. Together, these upper bounds prove the claim.

\subsection{Integrals}
\label{subsec:manifold_integrals}
Recall the Dirichlet energy $E_{P,\varepsilon}(f;\mc{X}) = \dotp{L_{P,\varepsilon}f}{f}_P$, defined in~\eqref{eqn:dirichlet_energy}. Now we establish some estimates on $E_{P,\varepsilon}(f;\mc{X})$ under Model~\ref{def:model_manifold}, and under various assumptions regarding the regularity of $f$.
\begin{lemma}
	\label{lem:dirichlet_energy_l2}
	Suppose Model~\ref{def:model_manifold}, and additionally that $f \in L^2(\mc{X})$. Then there exists a constant $C$ such that
	\begin{equation}
	\label{eqn:dirichlet_energy_l2}
	E_{P,\varepsilon}(f;\mc{X}) \leq \frac{C}{\varepsilon^2} \|f\|_{L^2(\mc{X})}^2.
	\end{equation}
\end{lemma}
\begin{lemma}
	\label{lem:dirichlet_energy_sobolev}
	Suppose Model~\ref{def:model_manifold}, and additionally that $f \in H^1(\mc{X})$. Then there exist constants $c$ and $C$ which do not depend on $f$ such that for any $0 < \varepsilon < c$,
	\begin{equation}
	\label{eqn:dirichlet_energy_sobolev}
	E_{P,\varepsilon}(f;\mc{X}) \leq C |f|_{H^1(\mc{X})}^2.
	\end{equation}
\end{lemma}

We use Lemma~\ref{lem:dirichlet_energy_remainder} to help upper bound the error incurred by using $\|\cdot\|$ rather than $d_{\mc{X}}(\cdot,\cdot)$. Recall the notation $\wt{\varepsilon} = (1 + 27\varepsilon^2/R^2)\varepsilon$, where $R$ is the reach of $\mc{X}$.
\begin{lemma}
	\label{lem:dirichlet_energy_remainder}
	Suppose Model~\ref{def:model_manifold}, and additionally that $f \in H^1(\mc{X})$. There exist constants $c$ and $C$ such that for any $\varepsilon < c$,
	\begin{equation}
	\label{eqn:dirichlet_energy_remainder}
	\int_{\mc{X}} \int_{\mc{X}} \bigl(f(x') - f(x)\bigr)^2 \1\{\varepsilon < d_{\mc{X}}(x',x) \leq \wt{\varepsilon}\} \,d\mu(x') \,d\mu(x) \leq C \varepsilon^{4 + m} \|f\|_{H^1(\mc{X})}^2
	\end{equation}
\end{lemma}

Finally, we use Lemma~\ref{lem:manifold_graph_seminorm_bias2} to show that the pure bias component of $\dotp{L_n^sf,f}_n$ is small in expectation. This is analogous to Lemma~\ref{lem:graph_seminorm_bias2}, except assuming Model~\ref{def:model_manifold} rather than Model~\ref{def:model_flat_euclidean}.
\begin{lemma}
	\label{lem:manifold_graph_seminorm_bias2}
	Assume Model~\ref{def:model_manifold}. Suppose $f \in H^1(\mc{X})$, and let $D_if$ be defined with respect to a kernel $\eta$ that satisfies~\ref{asmp:kernel_manifold}. Then there exists a constant $C$ which does not depend on $f$ or $n$, such that for any $i \in [n]$ and $\bj \in [n]^s$,
	\begin{equation*}
	\Ebb\Bigl[|D_{\bj}f(X_i)| \cdot |f(X_i) - f(X_{\bj_1})|\Bigr] \leq C \varepsilon^{2 + mk} \cdot \|f\|_{H^1(\mc{X})}^2,
	\end{equation*}
	where $k + 1$ is the number of distinct indices in $i\bj$. 
\end{lemma}

\paragraph{Proof (of Lemmas~\ref{lem:dirichlet_energy_l2} and~\ref{lem:dirichlet_energy_sobolev}).}
Define the non-local energy $\wt{E}_{P,\varepsilon}$ with respect to geodesic distance,
\begin{equation*}
\wt{E}_{P,\varepsilon}(f;{\mc{X}}) := \dotp{\wt{L}_{P,\varepsilon}f}{f}_{P} = \int_{\mc{X}} \int_{\mc{X}} \bigl(f(x') - f(x)\bigr)^2 \eta\biggl(\frac{d_{\mc{X}}(x',x)}{\varepsilon}\biggr) \,dP(x') \,dP(x).
\end{equation*}
From the lower bound in~\eqref{eqn:distance_error}, it follows that $E_{P,\varepsilon}(f;X) \leq \wt{E}_{P,\varepsilon}(f;{\mc{X}})$, and from the upper bounds $p(x) \leq p_{\max}$ and $\eta(|x|) \leq \|\eta\|_{\infty} \cdot \1\{x \in [-1,1]\}$ we further have
\begin{equation*}
\wt{E}_{P,\varepsilon}(f;{\mc{X}}) \leq p_{\max}^2 \|\eta\|_{\infty} \cdot \int_{\mc{X}} \int_{B_{\mc{X}}(\varepsilon)} \bigl(f(x') - f(x)\bigr)^2 \,d\mu(x') \,d\mu(x).
\end{equation*}
The estimates~\eqref{eqn:dirichlet_energy_l2} and~\eqref{eqn:dirichlet_energy_sobolev} then respectively follow from (3.1) and Lemma~3.3 of \cite{burago2014}.

\paragraph{Proof (of Lemma~\ref{lem:dirichlet_energy_remainder}).}
Following exactly the steps of the proof of Lemma~3.3 of \citet{burago2014}, but replacing all references to a ball of radius $r$ by references to the set difference between balls of radius $\wt{\varepsilon}$ and $\varepsilon$, we obtain that
\begin{equation*}
\int_{\mc{X}} \int_{\mc{X}} \bigl(f(x') - f(x)\bigr)^2 \1\{\varepsilon < d_{\mc{X}}(x',x) \leq \wt{\varepsilon}\} \,d\mu(x') \,d\mu(x) \leq (1 + CmK\varepsilon^2) \cdot \int_{\mc{X}} \int_{B_{m}(0,\wt{\varepsilon})} |d_x^{1}f(v)|^2 \,dv \,d\mu(x).
\end{equation*}
From (2.7) of~\citet{burago2014}, we further have
\begin{equation*}
\int_{\mc{X}} \int_{B_{m}(0,\wt{\varepsilon})} |d_x^{1}f(v)|^2 \,dv \,d\mu(x)  = \frac{\nu_m}{2 + m} (\wt{\varepsilon}^{2 + m} - \varepsilon^{2 + m}) \int_{\mc{X}} |d_x^1f|^2 \,d\mu(x) = 27\frac{\nu_m}{(2 + m)R^2} \varepsilon^{4 + m} \|d^1f\|_{\Leb^2(\mc{X})}^2. 
\end{equation*}
Recalling that $\|d^1f\|_{\Leb^2(\mc{X})}^2 \leq \|f\|_{H^1(\mc{X})}^2$, we see that this implies the claim of Lemma~\ref{lem:dirichlet_energy_remainder}.

\paragraph{Proof (of Lemma~\ref{lem:manifold_graph_seminorm_bias2}).}
The proof of Lemma~\ref{lem:manifold_graph_seminorm_bias2} is identical to the proof of Lemma~\ref{lem:graph_seminorm_bias2}, upon substituting the ambient dimension $m$ for the intrinsic dimension $d$, and using Lemma~\ref{lem:dirichlet_energy_sobolev} rather than Lemma~\ref{lem:dirichlet_estimate_nonlocal_laplacian} to establish~\eqref{pf:graph_seminorm_bias2_1}.

\section{Lower bound on empirical norm}
\label{sec:empirical_norm}
In this Section we prove Proposition~\ref{prop:empirical_norm_sobolev} (in Section~\ref{subsec:empirical_norm_sobolev}). We also prove an analogous result when $\mc{X}$ is a manifold as in Model~\ref{def:model_manifold} (in Section~\ref{subsec:empirical_norm_sobolev_manifold}).

\subsection{Proof of Proposition~\ref{prop:empirical_norm_sobolev}}
\label{subsec:empirical_norm_sobolev}
In this section we establish Proposition~\ref{prop:empirical_norm_sobolev}. As mentioned, the proof of this Proposition follows from the Gagliardo-Nirenberg interpolation inequality, and a one-sided Bernstein's inequality (Lemma~\ref{lem:one_sided_bernstein}). 

\begin{lemma}[Gagliardo-Nirenberg inequality]
	\label{lem:gagliardo_nirenberg}
	Suppose Model~\ref{def:model_flat_euclidean}, and that $f \in H^s(\mc{X})$ for some $s \geq d/4$. Then there exist constants $C_1$ and $C_2$ that do not depend on $f$, such that
	\begin{equation}
	\label{eqn:gagliardo_nirenberg}
	\|f\|_{\Leb^4(\mc{X})} \leq C_1 |f|_{H^s(\mc{X})}^{d/4s} \|f\|_{\Leb^2(\mc{X})}^{1 - d/(4s)} + C_2 \|f\|_{\Leb^2(\mc{X})}
	\end{equation}
\end{lemma}
\textcolor{red}{(TODO): Find the exact statement of this Lemma in \cite{leoni2017}.}

\paragraph{Proof (of Proposition~\ref{prop:empirical_norm_sobolev}).}
Rearranging~\eqref{eqn:gagliardo_nirenberg} and raising both sides to the $4$th power, we see that
\begin{equation*}
\frac{\Ebb[f^4(X)]}{\|f\|_P^4} \leq C \biggl(\frac{\|f\|_{\Leb^4(\mc{X})}}{\|f\|_{\Leb^2(\mc{X})}}\biggr)^4 \leq C_1\biggl(\frac{|f|_{H^s(\mc{X})}}{\|f\|_{\Leb^2(\mc{X})}}\biggr)^{d/s} + C_2,
\end{equation*}
here the constants $C_1,C_2$ are not the same as in~\eqref{eqn:gagliardo_nirenberg}. Therefore taking the constant $C$ in assumption~\eqref{eqn:empirical_norm_sobolev_1} to be sufficiently large relative to $C_1$ and $C_2$, we have that
\begin{equation*}
C_1\biggl(\frac{|f|_{H^s(\mc{X})}}{\|f\|_{\Leb^2(\mc{X})}}\biggr)^{d/s} \leq \frac{\delta n}{64},
\end{equation*} 
and consequently 
\begin{equation*}
\frac{\Ebb[f^4(X)]}{\|f\|_P^4} \leq \frac{\delta n}{8} + 8C_2^3.
\end{equation*}
The claim then follows from Lemma~\ref{lem:one_sided_bernstein}, upon taking $c = 1/(64C_2^3)$ in the statement of Proposition~\ref{prop:empirical_norm_sobolev}.

\subsection{Proof of Proposition~\ref{prop:empirical_norm_sobolev_manifold}}
\label{subsec:empirical_norm_sobolev_manifold}
The proof of Proposition~\ref{prop:empirical_norm_sobolev_manifold} follows exactly the same steps as the proof of Proposition~\ref{prop:empirical_norm_sobolev}, upon replacing Lemma~\ref{lem:gagliardo_nirenberg} by Lemma~\ref{lem:gagliardo_nirenberg_manifold}.
\begin{lemma}[(c.f Theorem~3.70 of~\citet{aubin2012})]
	\label{lem:gagliardo_nirenberg_manifold}
	Suppose Model~\ref{def:model_manifold}, and that $f \in H^s(\mc{X})$ for some $s \geq m/4$. Then there exist constants $C_1$ and $C_2$ that do not depend on $f$, such that
	\begin{equation}
	\label{eqn:gagliardo_nirenberg_manifold}
	\|f\|_{\Leb^4(\mc{X})} \leq C_1 |f|_{H^s(\mc{X})}^{m/4s} \|f\|_{\Leb^2(\mc{X})}^{1 - m/(4s)} + C_2 \|f\|_{\Leb^2(\mc{X})}.
	\end{equation}
\end{lemma}
	
\section{Proof of Main Results}
\label{sec:proofs_main_results}

\subsection{Estimation Results}

\paragraph{Proof of Theorem~\ref{thm:laplacian_eigenmaps_estimation_fo}.}
We condition on the event that the design points $X_1,\ldots,X_n$ satisfy
\begin{equation}
\label{pf:laplacian_eigenmaps_estimation_fo_1}
\dotp{L_{n,\varepsilon}f_0}{f_0}_n \leq \frac{C}{\delta}M^2 \quad \textrm{and} \quad \lambda_k \geq \min\{\lambda_k(\Delta_P), \varepsilon^{-2}\}~~\textrm{for all $2 \leq k \leq n$.}
\end{equation}
Note that by Propositions~\ref{prop:graph_seminorm_fo} and~\ref{prop:graph_eigenvalue}, these statements are both satisfied with probability at least $1 - \delta - Cn\exp\{-cn\varepsilon^d\}$. 

Conditional on~\eqref{pf:laplacian_eigenmaps_estimation_fo_1}, we have from Lemma~\ref{lem:fixed_graph_estimation} that for any $0 \leq K \leq n$,
\begin{equation*}
\|\wh{f} - f_0\|_n^2 \leq C\biggl\{\frac{M^2}{\delta (\lambda_{K + 1}(\Delta_P) \wedge \varepsilon^{-2})} + \frac{K}{n}\biggr\},
\end{equation*}
either deterministically (when $K = 0$), or with probability at least $1 - \exp(-K)$ (when $K \geq 1$). Further, from the bounds $\varepsilon \leq c_0 K^{-1/d}$ (Assumption~\ref{asmp:parameters_estimation_fo}) and $\lambda_{K + 1}(\Delta_P) \geq c (K + 1)^{2/d}$ (Weyl's Law) we can simply the above expression to the following,
\begin{equation}
\label{pf:laplacian_eigenmaps_estimation_fo_2}
\|\wh{f} - f_0\|_n^2 \leq C\biggl\{\frac{M^2}{\delta}(K + 1)^{-2/d} + \frac{K}{n}\biggr\}.
\end{equation}
We now upper bound the right hand side of~\eqref{pf:laplacian_eigenmaps_estimation_fo_2}, based on the value of $K$ chosen in~\ref{asmp:parameters_estimation_fo}.  When possible we choose $K = \floor{M^2n}^{d/(2 + d)}$ to balance bias and variance, in which case~\eqref{pf:laplacian_eigenmaps_estimation_fo_2} implies
\begin{equation*}
\|\wh{f} - f_0\|_n^2 \leq \frac{C}{\delta} M^2 (M^2n)^{-2/(2 + d)}.
\end{equation*}
If $M^2 < n^{-1}$, then we take $K = 1$, and from~\eqref{pf:laplacian_eigenmaps_estimation_fo_2} we get
\begin{equation*}
\|\wh{f} - f_0\|_n^2 \leq \frac{C}{n\delta}.
\end{equation*}
Finally if $M > n^{1/d}$, we take $K = n$. In this case, we note that $\wh{f}(X_i) = Y_i$ for all $i = 1,\ldots,n$, and it immediately follows that
\begin{equation*}
\|\wh{f} - f_0\|_n^2 = \frac{1}{n}\sum_{i = 1}^{n} w_i^2 \leq 5,
\end{equation*}
with probability at least $1 - \exp(-n)$. Combining these three separate cases yields the conclusion of Theorem~\ref{thm:laplacian_eigenmaps_estimation_fo}. 


\paragraph{Proof of Theorem~\ref{thm:laplacian_eigenmaps_estimation_ho}.}
Follows identically to the proof of Theorem~\ref{thm:laplacian_eigenmaps_estimation_fo}, except substituting $L_{n,\varepsilon}^s$ for $L_{n,\varepsilon}$, $\lambda_k^s$ for $\lambda_k$, and using Proposition~\ref{prop:graph_seminorm_ho} rather than Proposition~\ref{prop:graph_seminorm_fo} and Assumption~\ref{asmp:parameters_estimation_ho} rather than Assumption~\ref{asmp:parameters_estimation_fo}.

\paragraph{Proof of Theorem~\ref{thm:laplacian_eigenmaps_estimation_manifold}.}
Follows identically to the proof of Theorem~\ref{thm:laplacian_eigenmaps_estimation_fo}, substituting $L_{n,\varepsilon}^s$ for $L_{n,\varepsilon}$, $\lambda_k^s$ for $\lambda_k$, and using Proposition~\ref{prop:graph_seminorm_manifold} rather than Proposition~\ref{prop:graph_seminorm_fo}, Proposition~\textcolor{red}{(?)} rather than Proposition~\ref{prop:graph_eigenvalue}, and Assumption~\ref{asmp:parameters_estimation_manifold} rather than Assumption~\ref{asmp:parameters_testing_fo}.

\subsection{Testing Results}

\paragraph{Proof of Theorem~\ref{thm:laplacian_eigenmaps_testing_fo}.}
We have already upper bounded the Type I error of $\varphi$ in Lemma~\ref{lem:fixed_graph_testing}, and it remains to upper bound the Type II error. To do so, we condition on the event that the design points $X_1,\ldots,X_n$ satisfy,
\begin{equation}
\label{pf:laplacian_eigenmaps_testing_fo_1}
\dotp{L_{n,\varepsilon}f_0}{f_0}_n \leq \frac{C}{\delta}M^2,\quad \textrm{and} \quad \lambda_k \geq \min\{\lambda_k(\Delta_P), \varepsilon^{-2}\}~~\textrm{for all $2 \leq k \leq n$,}
\end{equation}
as well as that
\begin{equation}
\label{pf:laplacian_eigenmaps_testing_fo_2}
\|f_0\|_n^2 \geq \frac{1}{2}\|f_0\|_P^2.
\end{equation}
Note that by Propositions~\ref{prop:graph_seminorm_fo} and~\ref{prop:graph_eigenvalue}, both statements in~\eqref{pf:laplacian_eigenmaps_testing_fo_1} are satisfied with probability at least $1 - \delta - Cn\exp\{-cn\varepsilon^d\}$. Additionally, by Proposition~\ref{prop:empirical_norm_sobolev} and the assumption in~\eqref{eqn:laplacian_eigenmaps_testing_criticalradius_fo} that $\|f_0\|_P^2 \geq CM^2/(bn^{2/d})$, the one-sided inequality~\eqref{pf:laplacian_eigenmaps_testing_fo_2} follows with probability at least $1 - \exp\{-(cn \wedge 1/b)\}$. Setting $\delta = b/3$ and taking $n \geq N$ to be sufficiently large, the bottom line is that both~\eqref{pf:laplacian_eigenmaps_testing_fo_1} and~\eqref{pf:laplacian_eigenmaps_testing_fo_2} are together satisfied with probability at least $1 - b/2$.

Now, to complete the proof of Theorem~\ref{thm:laplacian_eigenmaps_testing_fo}, we would like to invoke Lemma~\ref{lem:fixed_graph_testing}, and conclude that conditional on $X_1,\ldots,X_n$ satisfying~\eqref{pf:laplacian_eigenmaps_testing_fo_1} and~\eqref{pf:laplacian_eigenmaps_testing_fo_2}, our test $\varphi$ will equal $1$ with probability at least $1 - b/2$. To use Lemma~\ref{lem:fixed_graph_testing}, we will need to establish that~\eqref{eqn:fixed_graph_testing_critical_radius} is satisfied, which we now show. 

On the one hand, we have that the right hand side of~\eqref{eqn:fixed_graph_testing_critical_radius} is upper bounded, 
\begin{align*}
\frac{\dotp{L_{n,\varepsilon}f_0}{f_0}_n}{\lambda_{K + 1}} + \frac{\sqrt{2K}}{n}\biggl[2\sqrt{\frac{1}{a}} + \sqrt{\frac{2}{b}} + \frac{32}{bn}\biggr] & \leq C\biggl(\frac{M^2}{b \min\{\lambda_{K+1}(\Delta_P), \varepsilon^{-2}\}} + \frac{\sqrt{2K}}{n}\biggl[\sqrt{\frac{1}{a}} + \frac{1}{b}\biggr]\biggr) \\
& \leq C\biggl(\frac{M^2}{b}K^{-2/d} + \frac{\sqrt{2K}}{n}\biggl[\sqrt{\frac{1}{a}} + \frac{1}{b}\biggr]\biggr)
\end{align*}
with the second inequality following by the assumption $\varepsilon \leq K^{-1/d}$ and Weyl's Law. On the other hand, we have that $\|f_0\|_n^2 \geq \|f_0\|_P^2/2$. Consequently, to prove Theorem~\ref{thm:laplacian_eigenmaps_testing_fo}, it remains only to verify that
\begin{equation}
\label{pf:laplacian_eigenmaps_testing_fo_3}
\|f_0\|_P^2 \geq C\biggl(\frac{M^2}{b}K^{-2/d} + \frac{\sqrt{2K}}{n}\biggl[\sqrt{\frac{1}{a}} + \frac{1}{b}\biggr]\biggr).
\end{equation}
As in the estimation case, we can further upper bound the right hand side of~\eqref{pf:laplacian_eigenmaps_testing_fo_3}, depending on the value of $K$ chosen in~\ref{asmp:parameters_testing_fo}. The classical case is $K = (M^2n)^{d/(2 + d)}$, in which case~\eqref{pf:laplacian_eigenmaps_testing_fo_3} is satisfied as long as
\begin{equation*}
\|f_0\|_P^2 \geq CM^2(M^2n)^{-4/(4 + d)}\biggl[\sqrt{\frac{1}{a}} + \frac{1}{b}\biggr]
\end{equation*}
If $M^2 < n^{-1}$, then we take $K = 1$, and~\eqref{pf:laplacian_eigenmaps_testing_fo_3} is satisfied whenever
\begin{equation*}
\|f_0\|_P^2 \geq \frac{C}{n}\biggl[\sqrt{\frac{1}{a}} + \frac{1}{b}\biggr].
\end{equation*}
Finally if $M > n^{1/d}$, we take $K = n$, and~\eqref{pf:laplacian_eigenmaps_testing_fo_3} is satisfied if
\begin{equation*}
\|f_0\|_P^2 \geq C\biggl(\frac{M^2}{n^{2/d}b} + n^{-1/2}\biggl[\sqrt{\frac{1}{a}} + \frac{1}{b}\biggr]\biggr).
\end{equation*}
We conclude by observing that~\eqref{eqn:laplacian_eigenmaps_testing_criticalradius_fo} implies each of these three inequalities, and thus implies~\eqref{pf:laplacian_eigenmaps_testing_fo_3}.

\paragraph{Proof of Theorem~\ref{thm:laplacian_eigenmaps_testing_ho}.}
Follows identically to the proof of Theorem~\ref{thm:laplacian_eigenmaps_estimation_fo}, except substituting $L_{n,\varepsilon}^s$ for $L_{n,\varepsilon}$, $\lambda_k^s$ for $\lambda_k$, and using Proposition~\ref{prop:graph_seminorm_ho} rather than Proposition~\ref{prop:graph_seminorm_fo} and Assumption~\ref{asmp:parameters_testing_ho} rather than Assumption~\ref{asmp:parameters_testing_fo}.

\paragraph{Proof of Theorem~\ref{thm:laplacian_eigenmaps_testing_manifold}.}
Follows identically to the proof of Theorem~\ref{thm:laplacian_eigenmaps_estimation_fo}, except substituting $L_{n,\varepsilon}^s$ for $L_{n,\varepsilon}$, $\lambda_k^s$ for $\lambda_k$, and using Proposition~\ref{prop:graph_seminorm_manifold} rather than Proposition~\ref{prop:graph_seminorm_fo}, Proposition~\ref{prop:graph_eigenvalue_manifold} rather than Proposition~\ref{prop:graph_eigenvalue}, Proposition~\ref{prop:empirical_norm_sobolev_manifold} rather than Proposition~\ref{prop:empirical_norm_sobolev}, and Assumption~\ref{asmp:parameters_testing_manifold} rather than Assumption~\ref{asmp:parameters_testing_fo}.

\paragraph{Proof of Theorem~\ref{thm:laplacian_eigenmaps_testing_ho_suboptimal}.}
Note that our choices of $K$ and $\varepsilon$ ensure  that~\eqref{pf:laplacian_eigenmaps_testing_fo_1} (with $L_{n,\varepsilon}^s$ replacing $L_{n,\varepsilon}$) and~\eqref{pf:laplacian_eigenmaps_testing_fo_2} are satisfied with probability at least $1 - b/2$. Proceeding as in the proof of Theorem~\ref{thm:laplacian_eigenmaps_testing_fo}, we upper bound the right hand side of~\eqref{eqn:fixed_graph_testing_critical_radius},
\begin{align*}
\frac{\dotp{L_{n,\varepsilon}f_0}{f_0}_n}{\lambda_{K + 1}} + \frac{\sqrt{2K}}{n}\biggl[2\sqrt{\frac{1}{a}} + \sqrt{\frac{2}{b}} + \frac{32}{bn}\biggr] & \leq C\biggl(\frac{M^2}{b \min\{\lambda_{K+1}(\Delta_P), \varepsilon^{-2}\}} + \frac{\sqrt{2K}}{n}\biggl[\sqrt{\frac{1}{a}} + \frac{1}{b}\biggr]\biggr) \\
& \leq C\biggl(\frac{M^2}{b}\varepsilon^2 + \frac{\sqrt{2K}}{n}\biggl[\sqrt{\frac{1}{a}} + \frac{1}{b}\biggr]\biggr).
\end{align*}
Unlike in the proof of Theorem~\ref{thm:laplacian_eigenmaps_testing_fo}, we note that in this case $\varepsilon^2 \leq C\lambda_K(\Delta_P)$ rather than vice versa. From here, proceeding as in the proof of Theorem~\ref{thm:laplacian_eigenmaps_testing_fo} gives the claimed result.

\section{Analysis of kernel smoothing}
\label{subsec:kernel_smoothing}
In this section we prove Lemma~\ref{lem:kernel_smoothing_insample} (in Section~\ref{subsec:pf_kernel_smoothing_insample}) and Lemma~\ref{lem:kernel_smoothing_bias} (in Section~\ref{subsec:pf_kernel_smoothing_bias}). In Section~\ref{subsec:eigenmaps_beats_kernel_smoothing}, we give a sequence of design densities and regression functions $\{p^{(n)}(x), f_0^{(n)}(x)\}_{n \in \mathbb{N}}$ for which the estimator $T_{n,h}\wc{f}$ (extension of Laplacian eigenmaps by kernel smoothing) strictly outperforms directly kernel smoothing the responses, in the sense that
\begin{equation*}
\lim_{n \to \infty} \sup_{h'} \frac{\Ebb \|T_{n,h}\wh{f} - f_0\|_P^2}{\Ebb \|T_{h',n}Y - f_0\|_P^2 } = 0.
\end{equation*}
We begin with some preliminary estimates in Section~\ref{subsec:kernel_smoothing_preliminaries}, which will ease the subsequent analysis.
\subsection{Some preliminary estimates}
\label{subsec:kernel_smoothing_preliminaries}
In certain parts the analysis of this section will overlap with Section~\ref{sec:graph_quadratic_form_euclidean}, where we upper bounded the non-local graph-Sobolev seminorm of a function $f$ in terms of the Sobolev norm of $f$. To see why this should be, note that for an function $f$ and point $x \in \mc{X}$, we have
\begin{equation*}
T_{P,h}f(x) - f(x) = \frac{1}{d_{Q,h}(x)} \int \bigl(f(x') - f(x)\bigr) \psi\biggl(\frac{\|x' - x\|}{h}\biggr) \,dQ(x') = \frac{h^{d + 2}}{d_{P,h}(x)} L_{P,h}f(x).
\end{equation*}
This expression reflects the known fact that the bias operator of kernel smoothing is equal to the non-local Laplacian, up to a rescaling by the population degree functional $d_{P,h}(x)$.  In the second equality, we are using the notation $L_{P,h}f(x)$ exactly as defined in~\eqref{eqn:nonlocal_laplacian}, but with the kernel $\psi$ instead of $\eta$. Note that $\psi$ satisfies all the same assumptions as $\eta$, except that of positivity; when $\psi$ is a higher-order kernel it may take negative values. 

Now we provide a lower bound on $d_{P,h}(x)$ that holds uniformly over all $x \in \mc{X}$. Recall that by assumption the density $p$ is Lipschitz. Letting $L_p$ denote the Lipschitz constant of $p$, we have that
\begin{align*}
d_{P,h}(x) & = \int \psi\biggl(\frac{\|x' - x\|}{h}\biggr) p(x') \,dx' \\
& = h^d \int \psi(\|z\|) p(hz + x) \1\Bigl\{ hz + x \in \mc{X} \Bigr\} \,dz \\
& \geq h^d p(x) \int \psi(\|z\|) \1\Bigl\{ hz + x  \in \mc{X}\Bigr\} \,dz - L_p h^{d + 1} \|\psi\|_{\infty} \nu_d.
\end{align*}
Since by assumption $\mc{X}$ has Lipschitz boundary, setting $c_0$ to be a sufficiently small constant in~\ref{asmp:bandwidth}, we can further deduce that $\int \psi(\|z\|) \1\{hz + x \in \mc{X}\} \,dz \geq 1/3$, and consequently that
\begin{equation}
\label{eqn:degree_lower_bound}
d_{P,h}(x) \geq \frac{p(x)}{3} h^d \geq \frac{p_{\min}}{3}h^d \quad \textrm{for all $x \in \mc{X}$.}
\end{equation}

\subsection{Proof of Lemma~\ref{lem:kernel_smoothing_insample}}
\label{subsec:pf_kernel_smoothing_insample}

To begin with, we apply the triangle inequality to upper bound $\|T_{n,h}\wc{f} - f_0\|_P$ by the sum of two terms,
\begin{equation}
\label{pf:kernel_smoothing_insample_1}
\|T_{n,h}\wc{f} - f_0\|_P \leq \|T_{n,h}(\wc{f} - f_0)\|_P + \|T_{n,h}f_0 - f_0\|_P.
\end{equation}
We proceed by separately upper bounding each term on the right hand side of~\eqref{pf:kernel_smoothing_insample_1}. We will show that
\begin{equation}
\label{pf:kernel_smoothing_insample_2}
\|T_{n,h}(\wc{f} - f_0)\|_P^2 \leq C \|\wc{f} - f_0\|_n^2
\end{equation}
and that 
\begin{equation}
\label{pf:kernel_smoothing_insample_3}
\|T_{n,h}f_0 - f_0\|_P^2 \leq \frac{C}{\delta} \cdot \frac{h^2}{nh^d} |f|_{H^1(\mc{X})}^2 + \frac{C}{\delta}\|T_{h,P}f_0 - f_0\|_P^2,
\end{equation}
each with probability at least $1 - C\exp(-cnh^d)$. Together these will imply the claim. 

\underline{\emph{Proof of~\eqref{pf:kernel_smoothing_insample_2}}.}
Fix $x \in \mc{X}$. By the Cauchy-Schwarz inequality we have
\begin{align*}
\Bigl[T_{n,h}\bigl(\wc{f} - f_0\bigr)(x)\Bigr]^2 & = \Biggl[\frac{1}{d_{n,h}(x)^2}\int \psi\biggl(\frac{\|x' - x\|}{h}\biggr) \cdot \bigl(\wc{f}(x') - f_0(x')\bigr) \,dP_n(x')\Biggr]^2 \\
& \leq \Biggl[\frac{1}{d_{n,h(x)}^2} \int \biggl|\psi\biggl(\frac{\|x' - x\|}{h}\biggr)\biggr| \,dP_n(x')\Biggr] \cdot \Biggl[\int \biggl|\psi\biggl(\frac{\|x' - x\|}{h}\biggr)\biggr| \cdot \bigl(\wc{f}(x') - f_0(x')\bigr)^2 \,dP_n(x')\Biggr] \\
& = \frac{d_{n,h}^{+}(x)}{|d_{n,h}(x)|} \cdot \frac{1}{|d_{n,h}(x)|} \Biggl[\int \biggl|\psi\biggl(\frac{\|x' - x\|}{h}\biggr)\biggr| \cdot \bigl(\wc{f}(x') - f_0(x')\bigr)^2 \,dP_n(x')\Biggr].
\end{align*}
In the last line all we have done is written $d_{n,h}^{+}(x)$ for the degree functional computed with respect to the kernel $|\psi|$, recalling that $\psi$ may take negative values so $d_{n,h}^{+}(x)$ may not be equal to $d_{n,h}(x)$.

Now we integrate over $x \in \mc{X}$ to get 
\begin{align}
\|T_{n,h}(\wc{f} - f_0)\|_P^2 & = \int \Bigl[T_{n,h}\bigl(\wc{f} - f_0\bigr)(x)\Bigr]^2 \,dP(x) \nonumber \\
& \leq \int \int \frac{d_{n,h}^{+}(x)}{|d_{n,h}(x)|} \cdot \frac{1}{|d_{n,h}(x)|} \biggl|\psi\biggl(\frac{\|x' - x\|}{h}\biggr)\biggr| \cdot \bigl(\wc{f}(x') - f_0(x')\bigr)^2 \,dP_n(x') \,dP(x) \nonumber \\
& \leq \sup_{x \in \mc{X}} \frac{d_{n,h}^{+}(x)}{|d_{n,h}(x)|} \cdot \int \int  \frac{1}{|d_{n,h}(x)|} \biggl|\psi\biggl(\frac{\|x' - x\|}{h}\biggr)\biggr| \cdot \bigl(\wc{f}(x') - f_0(x')\bigr)^2 \,dP(x) \,dP_n(x') \nonumber \\
& \leq \sup_{x \in \mc{X}} \frac{d_{n,h}^{+}(x) d_{P,h}^+(x)}{|d_{n,h}(x)|^2} \cdot \|\wc{f} - f_0\|_n^2. \label{pf:kernel_smoothing_insample_4}
\end{align}
Thus we have reduced the problem to showing that the various degree functionals $d_{n,h}^{+}, d_{P,h}^{+}$ and $d_{n,h}$ all put similar weight on a given point $x$. We use~\eqref{eqn:uniform_bound_empirical_degree_2}, which gives a uniform multiplicative bound on deviations of the empirical degree around its mean, to conclude that with probability at least $1 - C\exp\{-cnh^d\}$,
\begin{equation*}
d_{n,h}(x) \geq \frac{1}{2}d_{P,h}(x)~~ \textrm{and} ~~ d_{n,h}^{+}(x) \leq \frac{3}{2} d_{P,h}^{+}(x) \quad\textrm{for all $x \in \mc{X}$.}
\end{equation*}
Therefore,
\begin{equation*}
\sup_{x \in \mc{X}} \frac{d_{n,h}^{+}(x) d_{P,h}^+(x)}{|d_{n,h}(x)|^2} \leq 6 \cdot \sup_{x \in \mc{X}} \frac{|d_{P,h}^{+}(x)|^2}{|d_{P,h}(x)|^2} \leq 36 \biggl(\frac{\|\psi\|_{\infty} p_{\max} \nu_d}{p_{\min}}\biggr)^2
\end{equation*}
with the second inequality following from~\eqref{eqn:degree_lower_bound}. Plugging this back into~\eqref{pf:kernel_smoothing_insample_4} gives the claim.

\underline{\emph{Proof of~\eqref{pf:kernel_smoothing_insample_3}}.}
At a given point $x \in \mc{X}$, we have
\begin{align*}
T_{n,h}f_0(x) - f_0(x) & = \frac{1}{d_{n,h}(x)} \sum_{i = 1}^{n} \bigl(f_0(X_i) - f_0(x)\bigr) \psi\biggl(\frac{\|X_i - x\|}{h}\biggr) \\
& = \frac{d_{P,h}(x)}{d_{n,h}(x)} \cdot \frac{1}{nd_{P,h}(x)} \sum_{i = 1}^{n} \bigl(f_0(X_i) - f_0(x)\bigr) \psi\biggl(\frac{\|X_i - x\|}{h}\biggr).
\end{align*}
Thus,
\begin{equation}
\label{pf:kernel_smoothing_insample_5}
\Bigl[T_{n,h}f_0(x) - f_0(x)\Bigr]^2 = \biggl[\frac{d_{P,h}(x)}{d_{n,h}(x)}\biggr]^2 \cdot \biggl[\underbrace{\frac{1}{nd_{P,h}(x)}\sum_{i = 1}^{n} \bigl(f_0(X_i) - f_0(x)\bigr) \psi\biggl(\frac{\|X_i - x\|}{h}\biggr)}_{:= \wt{L}_{n,h}f_0(x)}\biggr]^2
\end{equation}
In the proof of~\eqref{pf:kernel_smoothing_insample_2} we have already given an upper bound on the ratio of population to empirical degree, which implies that
\begin{equation*}
\sup_{x \in \mc{X}} \biggl[\frac{d_{P,h}(x)}{d_{n,h}(x)}\biggr]^2 \leq 4,
\end{equation*}
with probability at least $1 - C\exp\{-cnh^d\}$. On the other hand, we note that the second term in the product in~\eqref{pf:kernel_smoothing_insample_5} has expectation
\begin{equation*}
\Ebb\Bigl[\wt{L}_{n,h}f_0(x)\Bigr] = T_{P,h}f_0(x) - f_0(x),
\end{equation*} 
and variance 
\begin{equation*}
\Var\Bigl[\wt{L}_{n,h}f_0(x)\Bigr] \leq \frac{1}{n(d_{P,h}(x))^2} \Ebb\biggl[(f_0(X) - f_0(x))^2 \cdot \biggl|\psi\biggl(\frac{\|X - x\|}{h}\biggr)\biggr|^2\biggr].
\end{equation*}
Integrating with respect to $P$ gives
\begin{align*}
\Ebb\biggl[\int \Bigl(\wt{L}_{n,h}f_0(x)\Bigr)^2 \,dP(x)\biggr] & = \int \Ebb\Bigl[\Bigl(\wt{L}_{n,h}f_0(x)\Bigr)^2\Bigr] \,dP(x) \\
& \leq \| T_{P,h}f_0 - f_0\|_P^2 + \frac{1}{n}\int \int \frac{1}{\bigl(d_{P,h}(x)\bigr)^2} \bigl(f_0(x') - f_0(x)\bigr)^2 \cdot \biggl|\psi\biggl(\frac{\|x' - x\|}{h}\biggr)\biggr|^2 \,dP(x') \,dP(x) \\
& \leq \| T_{P,h}f_0 - f_0\|_P^2 + \frac{3h^2}{p_{\min}n} \wt{E}_{P,h}(f_0;\psi^2).
\end{align*}
In the final inequality we have used the lower bound on $d_{P,h}(x)$ from~\eqref{eqn:degree_lower_bound}, and written $E_{P,h}(f_0;\psi^2)$ for the non-local Dirichlet energy defined with respect to the kernel $\psi^2$. 

Putting the pieces together, we conclude that
\begin{align*}
\|T_{n,h}f_0(x) - f_0(x)\|_P^2 & = \int \bigl(T_{n,h}f_0(x) - f_0(x)\bigr)^2 \,dP(x) \\
& \leq \sup_{x \in \mc{X}} \biggl[\frac{d_{P,h}(x)}{d_{n,h}(x)}\biggr]^2 \cdot \int \Bigl(\wt{L}_{n,h}f_0(x)\Bigr)^2 \,dP(x) \\
& \overset{(i)}{\leq} 4\frac{\| T_{P,h}f_0 - f_0\|_P^2}{\delta} + \frac{12h^2}{\delta p_{\min}nh^d} {E}_{P,h}(f_0;\psi^2) \\
& \overset{(ii)}{\leq}  4\frac{\| T_{P,h}f_0 - f_0\|_P^2}{\delta} + \frac{Ch^2}{\delta p_{\min}nh^d} |f_0|_{H^1(\mc{X})}^2,
\end{align*}
with probability at least $1 - \delta - C\exp(-cnh^d)$. In $(i)$ we have used Markov's inequality, and in $(ii)$ we have applied the estimate~\eqref{pf:estimate_nonlocal_seminorm_1} to the non-local Dirichlet energy ${E}_{P,h}(f_0;\psi^2)$. This establishes~\eqref{pf:kernel_smoothing_insample_3}.

\subsection{Kernel smoothing bias}
\label{subsec:pf_kernel_smoothing_bias}
Lemma~\ref{lem:kernel_smoothing_bias} gives the necessary upper bounds on the bias of kernel smoothing.
\begin{lemma}
	\label{lem:kernel_smoothing_bias}
	Suppose Model~\ref{def:model_flat_euclidean}, and that the kernel smoothing operator $T_{P,h}$ is computed with respect to a kernel $\eta$ that satisfies~\ref{asmp:kernel}.
	\begin{itemize}
		\item If $f_0 \in H^1(\mc{X})$, then there exists a constant $C$ which does not depend on $f_0$ such that
		\begin{equation*}
		\|T_{P,h}f_0 - f_0\|_P^2 \leq C h^{2} |f|_{H^1(\mc{X})}^2.
		\end{equation*}
		\item If $f_0 \in H_0^{s}(\mc{X})$, $p \in C^{s - 1}(\mc{X})$, and $\eta$ satisfies~\ref{asmp:ho_kernel}, then there exists a constant $C$ which does not depend on $f_0$ such that
		\begin{equation*}
		\|T_{P,h}f_0 - f_0\|_P^2 \leq C h^{2s} |f|_{H^s(\mc{X})}^2.
		\end{equation*}
	\end{itemize}
\end{lemma}
We separately prove the first-order ($s = 1$) and higher-order ($s > 1$) parts of Lemma~\ref{lem:kernel_smoothing_bias}. In both cases, the proof will rely heavily on results already established regarding the non-local Laplacian $L_{P,h}$ and non-local Dirichlet energy $E_{P,h}$, which we recall are given for a kernel function $\mc{K}$ by
\begin{equation*}
L_{P,h}f(x) = \frac{1}{h^{d + 2}} \int \bigl(f(x') - f(x)\bigr)\mc{K}\biggl(\frac{\|x' - x\|}{h}\biggr)\,dP(x'),
\end{equation*}
and $E_{P,h}(f;\mc{K}) = \dotp{L_{P,h}f}{f}_{P}$, respectively. 

\paragraph{Proof of Lemma~\ref{lem:kernel_smoothing_bias}, $s = 1$.}
Using the conclusions from Section~\ref{subsec:kernel_smoothing_preliminaries}, we have that
\begin{equation}
\label{pf:kernel_smoothing_bias_1}
\|T_{P,h}f - f\|_P^2 \leq \frac{9h^4}{p_{\min}^2} \int \bigl[L_{P,h}f(x)\bigr]^2 \,dP(x).
\end{equation}
By the Cauchy-Schwarz inequality, we have that
\begin{align*}
\int \bigl[L_{P,h}f(x)\bigr]^2 \,dP(x) & = \frac{1}{h^{2d + 4}}\int \biggl[\int \bigl(f(x') - f(x)\bigr)\psi\biggl(\frac{\|x' - x\|}{h}\biggr) \,dP(x')\biggr]^2 \,dP(x) \\
& \leq \frac{C}{h^{d + 4}} \int \int \bigl(f(x') - f(x)\bigr)^2 \cdot \biggl|\psi\biggl(\frac{\|x' - x\|}{h}\biggr)\biggr| \,dP(x') \,dP(x) \\
& = \frac{C}{h^{2}} E_{P,h}(f;|\psi|).
\end{align*}
Applying the estimate~\eqref{pf:estimate_nonlocal_seminorm_1} to the non-local Dirichlet energy ${E}_{P,h}(f;|\psi|)$ and plugging back into~\eqref{pf:kernel_smoothing_bias_1} gives the claimed result.

\paragraph{Proof of Lemma~\ref{lem:kernel_smoothing_bias}, $s > 1$.}
Proceeding from~\eqref{pf:kernel_smoothing_bias_1}, we separate the integral into the portion sufficiently in the interior of $\mc{X}$ and that near the boundary, obtaining
\begin{equation}
\label{pf:kernel_smoothing_bias_2}
\|T_{P,h}f - f\|_P^2 \leq\frac{9p_{\max}h^4}{p_{\min}^2}\Bigl(\|L_{P,h}f\|_{L^2(\mc{X}_h)}^2 + \|L_{P,h}f\|_{L^2(\partial_{h}(\mc{X}))}^2\Bigr).
\end{equation}
In Lemma~\ref{lem:approximation_error_nonlocal_laplacian_boundary}, we established a sufficient upper bound on the second term,
\begin{equation*}
\|L_{P,h}f\|_{L^2(\partial_{h}(\mc{X}))}^2 \leq Ch^{2(s - 2)} \|f\|_{H^s(\mc{X})}^2.
\end{equation*}
Thus it remains to upper bound the first term. Here we recall that at a given $x \in \mc{X}_h$, we can write
\begin{align*}
L_{P,h}f(x) & = \frac{1}{h^{2}}\sum_{j_1 = 1}^{s - 1} \sum_{j_2 = 0}^{q - 1}\frac{h^{j_1 + j_2}}{j_1!j_2!}  \int d_x^{j_1}f(z) d_x^{j_2}p(z) \psi\bigl(\|z\|\bigr) \,dz \quad + \\
& \quad \frac{1}{h^{2}} \sum_{j = 1}^{s - 1} \frac{h^j}{j!} \int d_x^jf(z)  r_{zh + x}^{q}(x;p) \psi\bigl(\|z\|\bigr) \,dz \quad  + \\
& \quad \frac{1}{h^{2}} \int r_{zh + x}^j(x;f) \psi\bigl(\|z\|\bigr) p(zh + x)\,dz \\
& = G_1(x) + G_2(x) + G_3(x).
\end{align*}
(Here $q = s - 1$.) 

We have already given sufficient upper bounds on $\|G_j\|_{L^2(\mc{X}_{h})}$ for $j = 2,3$ in~\eqref{pf:approximation_error_nonlocal_laplacian_2}.  Thus it remains only to upper bound $\|G_1\|_{L^2(\mc{X}_h)}$. Recall the expansion of $G_1$ from~\eqref{pf:approximation_error_nonlocal_laplacian_3},
\begin{equation*}
G_1(x) = \sum_{j_1 = 1}^{s - 1} \sum_{j_2 = 0}^{q - 1} \frac{h^{j_1 + j_2 - 2}}{j_1!j_2!}  \underbrace{\int_{B(0,1)} d_x^{j_1}f(z) d_x^{j_2}p(z) \eta(\|z\|) \,dz}_{:= g_{j_1,j_2}(x)}.
\end{equation*}
Noting that $d_x^{j_1} \cdot d_x^{j_2}$ is a degree-$(j_1 + j_2)$ multivariate polynomial, and recalling that $\psi$ is an order-$s$ kernel, we have that 
\begin{equation*}
\int g_{j_1,j_2}(z) \psi\bigl(\|z\|\bigr) \,dz = 0,\quad \textrm{for all $j_1,j_2$ such that $j_1 + j_2 < s$.}
\end{equation*}
Otherwise, derivations similar to those used in the proof of Lemma~\ref{lem:approximation_error_nonlocal_laplacian} imply that
\begin{equation*}
\|g_{j_1,j_2}\|_{L^2(\mc{X}_h)} \leq C \|f\|_{H^s(\mc{X})} \|p\|_{C^{s - 1}(\mc{X})},\quad \textrm{for all $j_1,j_2$ such that $j_1 + j_2 \geq s$,}
\end{equation*}
from which it follows that
\begin{equation*}
\|G_1\|_{L^2(\mc{X}_h)}^2 \leq C  h^{2(s - 2)} \|f\|_{H^s(\mc{X})} \|p\|_{C^{s - 1}(\mc{X})}.
\end{equation*}
Together these upper bounds on $\|G_j\|_{L^2(\mc{X}_j)}$ for $j = 1,2,3$ imply that
\begin{equation*}
\|L_{P,h}f\|_{\mc{X}_h}^2 \leq Ch^{2(s - 2)} \|f\|_{H^s(\mc{X})}^2,
\end{equation*}
and plugging this back into~\eqref{pf:kernel_smoothing_bias_2} yields the claim.

\subsection{Comparison of Laplacian eigenmaps to kernel smoothing}
\label{subsec:eigenmaps_beats_kernel_smoothing}

In this section, we give a sequence of densities $p^{(n)}$ and regression functions $f_0^{(n)}$ for which the $L^2(P)$ error of the two-stage estimator $T_{n,h}\wh{f}$ is stochastically dominated by the $L^2(P)$ error of applying kernel smoothing directly to the responses. We begin by precisely setting up the problem and reviewing the two estimators we will compare. Then we give and prove our formal claim.

\paragraph{Setup.}
As usual, we assume the design points $(X_1,\ldots,X_n)$ are sampled independently from a distribution $P$, and the responses
\begin{equation*}
Y_i = f_0(X_i) + \varepsilon_i,
\end{equation*}
where the noise terms $\varepsilon_i \sim N(0,1)$ are sampled independently, and are also independent of the design points.

Unlike in our main text, we shall assume a specific (parametric) form for $p = p^{(n)}$ and $f_0 = f_0^{(n)}$, that changes as a function of $n$. More specifically, using the notation
\begin{equation*}
r := \frac{(\log n)^2}{n}, \quad Q_1 := [0,1/2 - r], \quad Q_2 := [1/2 + r,1],
\end{equation*}
we take the domain $\mc{X}^{(n)} = Q_1 \cup Q_2$, and 
\begin{equation}
\label{def:model_cluster_assumption}
p^{(n)}(x) := \frac{1}{1 - 2r}\1\bigl\{x \in Q_1 \cup Q_2\bigr\}, \quad f_0^{(n)}(x) := \theta \cdot \Bigl(\1\bigl\{x \in Q_1\bigr\} - \1\bigl\{x \in Q_2\bigr\}\Bigr),
\end{equation}
where $\theta \in \Reals$. Note that as $n \to \infty$, the density $p^{(n)}$ is converging (in, say, $L^2([0,1])$ norm) to uniform distribution on $[0,1]$, and the regression function $f_0^{(n)}$ to a step function. 

Our goal is to estimate $f_0$ using an estimator $f$, in such a way that the resulting $L^2(P)$ error $\|f - f_0\|_P^2$ is small. We shall consider two estimators: Laplacian eigenmaps + extension by kernel smoothing, and direct kernel smoothing of the responses. In the notation of our main text, these are respectively $T_{n,h}\wh{f}$ and $T_{n,h'}{\bf Y}$. For simplicity, in both cases we shall consider only the boxcar kernel,
\begin{equation}
\label{asmp:boxcar_kernel}
\eta(z) = \psi(z) = \1\{z \leq 1\}.
\end{equation}

\paragraph{Laplacian eigenmaps + kernel smoothing beats direct kernel smoothing.}
Now we are in a position to state our result. Let $P^{(n)}$ be the distribution with density $p^{(n)}$, i.e. the uniform probability distribution on $Q_1 \cup Q_2$. 
\begin{proposition}
	\label{prop:eigenmaps_beats_kernel_smoothing}
	Suppose $(X_1,Y_1),\ldots(X_n,Y_n)$ are sampled according to~\eqref{def:model_cluster_assumption}. 
	\begin{itemize}
		\item Compute the Laplacian eigenmaps estimator $\wh{f}$ using a kernel $\eta$ which satisfies~\eqref{asmp:boxcar_kernel}, number of eigenvectors $K = 2$, and radius $\varepsilon = ((\log n)^2/n)/2$. Compute the kernel smoothed extension $T_{n,h}\wh{f}$ using a kernel $\psi$ which satisfies~\eqref{asmp:boxcar_kernel}, and bandwidth $h = ((\log n)^2/n)/2$. Then,
		\begin{equation}
		\label{eqn:eigenmaps_beats_kernel_smoothing_1}
		\Ebb \|T_{n,h}\wh{f} - f_0^{(n)}\|_{L^2(P^{(n)})}^2 \leq \frac{8 (\log n)^2 + 6 (\log n)^2 \theta^2}{n}.
		\end{equation}
		\item Compute the kernel smoothing estimator $T_{n,h}{\bf Y}$, using a kernel $\psi$ which satisfies~\eqref{asmp:boxcar_kernel}. Then
		\begin{equation}
		\label{eqn:eigenmaps_beats_kernel_smoothing_2}
		\inf_{h' > 0} \Ebb\Bigl[\|T_{n,h'}{\bf Y} - f_0^{(n)}\|_{L^2(P^{(n)})}^2\Bigr] \geq \min\biggl\{ \frac{1}{16(\log n)^2}, \frac{\theta}{\sqrt{32 n}} \biggr\}.
		\end{equation}
	\end{itemize}
\end{proposition}
The statements~\eqref{eqn:eigenmaps_beats_kernel_smoothing_1} and~\eqref{eqn:eigenmaps_beats_kernel_smoothing_2} are finite sample and hold for any $\theta$. Taking a sequence $(\theta_n)_{n \in \mathbb{N}}$ such that $\theta = o(\sqrt{n})$ and $\theta = \omega((\log n)^2/\sqrt{n})$ implies that 
\begin{equation*}
\lim_{n \to \infty} \sup_{h'}  \frac{\Ebb \|T_{n,h}\wh{f} - f_0^{(n)}\|_{P^{(n)}}^2}{	\Ebb \|T_{n,h'}{\bf Y} - f_0^{(n)}\|_{P^{(n)}}^2 } = 0.
\end{equation*}

\paragraph{Proof of Proposition~\ref{prop:eigenmaps_beats_kernel_smoothing}.}
First we prove~\eqref{eqn:eigenmaps_beats_kernel_smoothing_1}, then~\eqref{eqn:eigenmaps_beats_kernel_smoothing_2}.

\underline{\emph{Proof of~\eqref{eqn:eigenmaps_beats_kernel_smoothing_1}}.}
We begin by showing that with high probability the eigenvectors $v_1,v_2$ respect the cluster structure of $p$. Formally, letting $u_1 = (\1\{X_i \in Q_1\})_{i \in [n]}$, and likewise $u_2 = (\1\{X_i \in Q_2\})_{i \in [n]}$, we establish that
\begin{equation}
\label{pf:eigenmaps_beats_kernel_smoothing_1}
\Pbb\Bigl(\mathrm{span}\{v_1,v_2\} = \mathrm{span}\{u_1,u_2\}\Bigr) \geq 1 - \frac{1}{n}.
\end{equation}
Note that~\eqref{pf:eigenmaps_beats_kernel_smoothing_1} is equivalent to the statement that the neighborhood graph $G_{n,\varepsilon}$ consists of exactly two connected components: one consisting of all design points $X_i \in Q_1$, and the other consisting of all design points $X_i \in Q_2$. Of course, since $h < r$ and $\eta$ is compactly supported on $[0,1]$, it follows that no $X_i \in Q_1$ and $X_j \in Q_2$ can be connected. On the other hand, using an elementary concentration argument (stated in Lemma~\ref{lem:balls_in_bins}) and the triangle inequality, we deduce that with probability at least $1 - 1/n$ there exists a path in $G_{n,\varepsilon}$ between each $X_i, X_j \in Q_1$, and likewise between each $X_i,X_j \in Q_2$. This establishes~\eqref{pf:eigenmaps_beats_kernel_smoothing_1}.

Now we condition on the ``good'' event $\mc{E}$ that the design points $X_1,\ldots,X_n$ satisfy~\eqref{eqn:balls_in_bins}, and therefore that $\mathrm{span}\{v_1,v_2\} = \mathrm{span}\{u_1,u_2\}$. Consider the empirical mean $\wb{Y}_Q := \frac{1}{\sharp\{Q \cup {\bf X}\}} \sum_{i: X_i \in Q} Y_i$. Since $\mathrm{span}\{v_1,v_2\} = \mathrm{span}\{u_1,u_2\}$, the estimator $\wh{f}$ will be piecewise constant on $Q_1$ and $Q_2$, and in fact we have that
\begin{equation*}
\wh{f} = \wb{Y}_{Q_1}u_1 + \wb{Y}_{Q_2}u_2.
\end{equation*}
Moreover, by~\eqref{eqn:balls_in_bins} every $x \in Q_1$ is within distance $h$ of at least one $X_i \in {\bf X} \cup Q_1$ and likewise every $x \in Q_2$ is within distance $h$ of at least one $X_j \in {\bf X} \cup Q_2$. Consequently, the kernel smoothening of $\wh{f}$ also exhibits a piecewise constant structure,
\begin{equation*}
T_{n,h}\wh{f}(x) = \wb{Y}_{Q_1}\1\{x \in Q_1\} + \wb{Y}_{Q_2}\1\{x \in Q_2\},\quad \textrm{for all}~~x \in Q_1 \cup Q_2.
\end{equation*}
Letting
\begin{equation*}
\|T_{n,h}\wh{f} - f_0^{(n)}\|_{L^2(P^{(n)})}^2 = \frac{1}{2}\Bigl((\wb{Y}_{Q_1} - \theta)^2 + (\wb{Y}_{Q_2} + \theta)^2\Bigr),
\end{equation*}
it follows from Lemma~\ref{lem:chi_square_bound} that conditional on $\mc{E}$,
\begin{equation}
\label{pf:eigenmaps_beats_kernel_smoothing_0}
\|T_{n,h}\wh{f} - f_0^{(n)}\|_{L^2(P^{(n)})}^2 \leq \frac{4(\log n)^2}{n}.
\end{equation}

Now we derive a crude upper bound on $\|T_{n,h}\wh{f} - f_0^{(n)}\|_{L^2(P^{(n)})}$ that will suffice to control the error conditional on $\mc{E}^c$. Note that if $d_{n,h}(x) = 0$ then
$T_{n,h}\wh{f}(x) = 0$, and $\bigl(T_{n,h}\wh{f}(x) - f_0(x)\bigr)^2 = [f_0(x)]^2$. On the other hand, if $d_{n,h}(x) \geq 1$, then by Jensen's inequality it follows that
\begin{align*}
\bigl(T_{n,h}\wh{f}(x) - f_0(x)\bigr)^2 & \leq \biggl(\frac{1}{d_{n,h}(x)} \sum_{i = 1}^{n} (\wh{f}_i - f_0(x)) \cdot \1\{\|X_i - x\| \leq h \}\biggr)^2 \\
& \leq \frac{1}{d_{n,h}(x)} \sum_{i = 1}^{n} (\wh{f}_i - f_0(x))^2 \cdot \1\{\|X_i - x\| \leq h \} \\
& \leq 2[f_0(x)]^2 + \frac{2}{d_{n,h}(x)} \sum_{i = 1}^{n} (\wh{f}_i)^2 \cdot \1\{\|X_i - x\| \leq h\},
\end{align*}
and integrating over all $x \in \mc{X}^{(n)}$ gives
\begin{align*}
\|T_{n,h}\wh{f} - f_0\|_{\Leb^2(P^{(n)})}^2 & \leq 2 \|f_0\|_{\Leb^2(P^{(n)})}^2 + \sum_{i = 1}^{n} (\wh{f}_i)^2 \biggl(\int \frac{\1\{\|X_i - x\| \leq h \}}{d_{n,h}(x)} \,dP(x)\biggr) \\
& \leq 2 \|f_0\|_{\Leb^2(P^{(n)})}^2 + 2h\sum_{i = 1}^{n} (\wh{f}_i)^2 \\
& = 2 \|f_0\|_{\Leb^2(P^{(n)})}^2 + 2 (\log n)^2 \|\wh{f}\|_n^2.
\end{align*}
We can further upper bound the empirical norm of $\|\wh{f}\|_n^2$ using the Cauchy-Schwarz inequality:
\begin{equation*}
\|\wh{f}\|_n^2 \leq \frac{2}{n}\sum_{i = 1}^{n}\dotp{Y}{v_1}_n^2v_{1,i}^2 + \dotp{{\bf Y}}{v_2}_n^2v_{2,i}^2 \leq 2\bigl(\dotp{{\bf Y}}{v_1}_n^2 + \dotp{{\bf Y}}{v_2}_n^2\bigr) \leq 4 \|{\bf Y}\|_n^2. 
\end{equation*}
Noting that $\Ebb[\|{\bf Y}\|_n^2|{\bf X}] = \|f_0\|_n^2 + 1/n$, and that $\max_x |f_0(x)| = \theta$, we conclude that
\begin{equation*}
\Ebb\Bigl[\|T_{n,h}\wh{f} - f_0\|_{\Leb^2(P^{(n)})}^2 \cdot \1\{\mc{E}^c\}\Bigr] \leq \Ebb\Bigl[\Bigl(2 \|f_0\|_{\Leb^2(P^{(n)})}^2 + 4 (\log n)^2 \|{\bf Y}\|_n^2\Bigr) \cdot \1\{\mc{E}^c\} \Bigr] \leq \frac{2 \theta^2}{n} + \frac{4 (\log n)^2(\theta^2 + 1)}{n}.
\end{equation*}
Combining this with~\eqref{pf:eigenmaps_beats_kernel_smoothing_0} implies~\eqref{eqn:eigenmaps_beats_kernel_smoothing_1}.

\underline{\emph{Proof of~\eqref{eqn:eigenmaps_beats_kernel_smoothing_2}}.}
Fix $x \in \mc{X}^{(n)}$. A standard argument using the law of iterated expectation implies the following lower bound on the pointwise risk in terms of squared-bias and variance-like quantities,
\begin{equation*}
\Ebb\Bigl[\Bigl(T_{n,h'}{\bf Y}(x) - f_0(x)\Bigr)^2\Bigr] \geq \Ebb\biggl[\Bigl(f_0(X) - f_0(x)\Bigr)^2|X \in B(x,h')\biggr] + \Ebb\biggl[\frac{1}{d_{n,h'}(x)}\biggr].
\end{equation*}
The variance term can be lower bounded quite simply for any $x \in \mc{X}^{(n)}$; noting that $\sup_{x} p^{(n)}(x) < 2$ and $\nu(B(x,h') \cap \mc{X}^{(n)}) \leq 2h'$, it follows by Jensen's inequality that 
\begin{equation*}
\Ebb\biggl[\frac{1}{d_{n,h'}(x)}\biggr] \geq \frac{1}{\Ebb[d_{n,h'}(x)]} \geq \frac{1}{4nh'}.
\end{equation*}
On the other hand the squared bias term is quite large for $x$ close to $1/2$. Precisely, if $h' \geq 4r$ then a simple calculation implies
\begin{equation*}
\Ebb[(f_0(X) - f_0(x))^2|X \in B(x,h')] \geq \frac{\theta^2}{8} \quad \textrm{for all}~x \in [(1 - h'/2)_{+}, 1/2 - r].
\end{equation*}
Combining these lower bounds on variance and squared bias terms and integrating over $x' \in \mc{X}$, we deduce the following: if $h' \leq 4r$, then
\begin{equation*}
\Ebb\Bigl[\|T_{n,h'}{\bf Y} - f_0^{(n)}\|_{L^2(P^{(n)})}^2\Bigr] \geq \frac{1}{16(\log n)^2},
\end{equation*}
whereas if $h' > 4r$  then
\begin{equation*}\Ebb\Bigl[\|T_{n,h'}{\bf Y} - f_0^{(n)}\|_{L^2(P^{(n)})}^2\Bigr] \geq \frac{1}{4nh'} + \frac{\theta^2h'}{32}.
\end{equation*}
In the latter case, setting the derivative equal to $0$ shows that the right hand side is always at least $\theta/\sqrt{32 n}$, and taking the minimum over the two cases then yields~\eqref{eqn:eigenmaps_beats_kernel_smoothing_2}.

\paragraph{A Useful Lemma.}
Let $m = 4n/\log^2n$. Furthermore, for $i = 0,\ldots,m - 1$, let
\begin{equation*}
Q_{i1} = [i/m,(i + 1)/m] \cdot (1/2 - r), \quad Q_{i2} = 1/2 + [i/m,(i + 1)/m] \cdot (1/2 - r).
\end{equation*}
\begin{lemma}
	\label{lem:balls_in_bins}
	Suppose $(X_1,Y_1),\ldots(X_n,Y_n)$ are sampled according to~\eqref{def:model_cluster_assumption}. For any $\log(n) > 16$, we have that 
	\begin{equation}
	\label{eqn:balls_in_bins}
	\Pbb\Bigl(\sharp\{Q_{ij} \cup {\bf X}\} > 0 ~~\textrm{for all $i = 1,\ldots,m - 1$ and $j = 1,2$} \Bigr) \geq 1 - \frac{1}{n}.
	\end{equation}
\end{lemma}
Let $\varepsilon = h = \log^2n/2n$. Note that by construction,~\eqref{eqn:balls_in_bins} implies that any points $x$ and $x'$ in adjacent intervals $Q_{ij}$ and $Q_{i'j}$ must be connected in $G_{n,\varepsilon}$. It also implies that $d_{n,h}(x) > 0$ for every $x \in Q_1 \cup Q_2$.

\paragraph{Proof (of Lemma~\ref{lem:balls_in_bins}).}
For each $Q_{ij}$, we have that 
\begin{equation*}
\Pbb\bigl(\sharp\{Q_{ij} \cup {\bf X}\} = 0\bigr) = (1 - 1/(2m))^{n} \leq \exp\{-n/2m\}.
\end{equation*}
Thus by a union bound,
\begin{equation*}
\Pbb\Bigl(\sharp\{Q_{ij} \cup {\bf X}\} = 0 ~~\textrm{for any $i = 1,\ldots,m - 1$ and $j = 1,2$} \Bigr) \leq 2m\exp\{-n/2m\} = \frac{8n}{\log^2n}\exp\{-\log^2n/8\},
\end{equation*}
and the claim follows from some basic algebra. 

\section{Miscellaneous}
Here we give assorted helpful Lemmas used at various points in the above proofs. We also review notation and relevant facts regarding Taylor expansion.

\subsection{Concentration Inequalities}
Lemma~\ref{lem:chi_square_bound} controls the deviation of a chi-squared random variable. It is from~\cite{laurent00}.
\begin{lemma}
	\label{lem:chi_square_bound}
	Let $\xi_1,\ldots,\xi_N$ be independent $N(0,1)$ random variables, and let $U := \sum_{k = 1}^{N} a_k(\xi_k^2 - 1)$.  Then for any $t > 0$,
	\begin{equation*}
	\Pbb\Bigl[U \geq 2 \norm{a}_2 \sqrt{t} + 2 \norm{a}_{\infty}t\Bigr] \leq \exp(-t).
	\end{equation*}
	In particular if $a_k = 1$ for each $k = 1,\ldots,N$, then
	\begin{equation*}
	\Pbb\Bigl[U\geq 2\sqrt{N t} + 2t\Bigr] \leq \exp(-t).
	\end{equation*}
\end{lemma}

Lemma~\ref{lem:one_sided_bernstein} is an immediate consequence of the one-sided Bernstein's inequality (14.23) in \cite{wainwright2019}.
\begin{lemma}[One-sided Bernstein's inequality]
	\label{lem:one_sided_bernstein}
	Let $X, X_1,\ldots,X_n \sim P$, and $f$ satisfy $\Ebb[f^4(X)] < \infty$. Then
	\begin{equation*}
	\|f\|_n^2 \geq \frac{1}{2}\|f\|_P^2,
	\end{equation*}
	with probability at least $1 - \exp\bigl(-n/8 \cdot \|f\|_P^4 /\Ebb[f^4(X)]\bigr)$.
\end{lemma}

In the proof of Lemma~\ref{lem:kernel_smoothing_insample}, we require uniform control of the empirical degree functional $d_{n,h}(x)$ over all $x \in \mc{X}$. Such a result is available to us because the kernel $\psi$ is Lipschitz on its support, so that the class of functions $\{\psi((x - \cdot)/h): h \in \Rd\}$ has finite VC dimension. The precise estimate we use is due to \textcolor{blue}{(Gine + Guillou 2002)}. 
\begin{lemma}[Uniform bound for empirical degree.]
	\label{lem:uniform_bound_empirical_degree}
	Suppose Model~\ref{def:model_flat_euclidean}. For a kernel $\psi$ satisfying~\ref{asmp:kernel} and bandwidth $h$ satisfying~\ref{asmp:bandwidth}, there exist constants $c, C, c_1$ and $C_1$ which do not depend on $h$ or $n$ such that
	\begin{equation*}
	\Pbb\biggl(\sup_{x \in \Rd} n \cdot \Bigl|d_{n,h}(x) - d_{P,h}(x)\Bigr| > t\biggr) \leq C\exp\biggl(-c \frac{t^2}{nh^d}\biggr),
	\end{equation*}
	for any $t \in \Reals$ satisfying
	\begin{equation}
	\label{eqn:uniform_bound_empirical_degree_1}
	C_1\sqrt{nh^d \log(1/h)} \leq t \leq c_1 nh^d.
	\end{equation}
\end{lemma}
Now we translate Lemma~\ref{lem:uniform_bound_empirical_degree} into a multiplicative bound, which will be more useful for our purposes. 
Recall the lower bound on $d_{P,h}(x)$ given in~\eqref{eqn:degree_lower_bound}. By setting $C_0$ to be a sufficiently large constant in~\ref{asmp:bandwidth}, we can ensure that choosing $t = n h^d p_{\min}/6$ satisfies both the inequalities~\eqref{eqn:uniform_bound_empirical_degree_1}. For this choice of $t$ it follows from~\eqref{eqn:degree_lower_bound} and Lemma~\ref{lem:uniform_bound_empirical_degree} that
\begin{equation}
\label{eqn:uniform_bound_empirical_degree_2}
\sup_{x \in \Rd} \biggl|\frac{d_{n,h}(x) - d_{P,h}(x)}{d_{P,h}(x)}\biggr| \leq \sup_{x \in \Rd} \biggl|\frac{d_{n,h}(x) - d_{P,h}(x)}{2t/n}\biggr| \leq \frac{1}{2}
\end{equation}
with probability at least $1 - C\exp(-cnh^d)$. This is the form of the result we use in the proof of Lemma~\ref{lem:kernel_smoothing_insample}.

\subsection{Taylor expansion}
\label{subsec:taylor_expansion}
We begin with some notation that allows us to concisely derivatives. For a given $z \in \Rd$ and $s$-times differentiable function $f: \mc{X} \to \Reals$, we denote $\bigl(d_x^sf\bigr)(z) := \sum_{\abs{\alpha} = s} D^{\alpha}f(x) z^{\alpha}$. We also write $d^sf := \sum_{\abs{\alpha} = j} D^{\alpha}f$. We point out that in the first-order case $d_x^1f$ is the differential of $f$ at $x \in \mc{X}$, while $d^1f$ is the divergence of $f$.

Let $u$ be a function which is $s$ times continuously differentiable at all $x \in \mc{X}$, for $k \in \mathbb{N}\setminus\{0\}$. Suppose that for some $h > 0$, $x \in \mc{X}_{h}$ and $x' \in B(x,h)$. We write the order-$s$ Taylor expansion of $u(x')$ around $x' = x$ as
\begin{equation*}
u(x') = u(x) + \sum_{j = 1}^{s - 1} \frac{1}{j!}\bigl(d_x^{j}u\bigr)(x' - x) + r_{x'}^{s}(x;u)
\end{equation*}
For notational convenience we have adopted the convention that $\sum_{j = 1}^{0} a_j = 0$.  Thus $\bigl(d_x^{j}f\bigr)(z)$ is a degree-$j$ polynomial---and so a $j$-homogeneous function---in $z$, meaning for any $t \in \Reals$,
\begin{equation*}
\bigl(d_x^{j}f\bigr)(tz) = t^{j} \cdot \bigl(d_x^{j}f\bigr)(z).
\end{equation*}
The remainder term $r_{x'}$ is given by
\begin{equation*}
r_{x'}^s(x;f) = \frac{1}{(j - 1)!} \int_{0}^{1}(1 - t)^{j - 1} \bigl(d_{x + t(x' - x)}^{s}f\bigr)(x' - x) \,dt,
\end{equation*}
where we point out that the integral makes sense because $x + t(x' - x) \in B(x,h) \subseteq \mc{X}$. We now give estimates on the remainder term in both sup-norm and $L^2(\mc{X}_{h})$ norm, each of which hold for any $z \in B(0,1)$. In sup-norm, we have that 
\begin{equation*}
\sup_{x \in \mc{X}_{h}}|r_{x + hz}^j(x;f)| \leq C h^j \|f\|_{C^j(\mc{X})},
\end{equation*}
whereas in $L^2(\mc{X}_{h})$ norm we have,
\begin{equation}
\label{eqn:sobolev_remainder_term}
\int_{\mc{X}_{h}} \bigl|r_{x + thz}^j(x;f)\bigr|^2 \,dx \leq h^{2j} \int_{\mc{X}_{h}} \int_{0}^{1} |d_{x + thz}^jf(z)|^2 \,dt \,dx \leq h^{2j} \|d^jf\|_{\Leb^2(\mc{X})}^2.
\end{equation}
In the last inequality 

Finally, we recall some facts regarding the interaction between smoothing kernels and polynomials.  Let $q_j(z)$ be an arbitrary degree-$j$ (multivariate) polynomial. If $\eta$ is a radially symmetric kernel and $j$ is odd, then by symmetry it follows that
\begin{equation*}
\int_{B(0,1)} q_j(z) \eta(\|z\|) \,dz = 0.
\end{equation*}
On the other hand, if $\psi$ is an order-$s$ kernel for some $s > j$, then by converting to polar coordinates we can verify that
\begin{equation*}
\int_{B(0,1)} q_j(z) \eta(\|z\|) \,dz = 0.
\end{equation*}

\textcolor{red}{(TODO): Check with Ryan and Siva that (i) they believe these facts, and (ii) these facts do not require additional justification.}

\subsection{Upper bound on $L^2(P_n)$ risk}
\label{subsec:upper_bound_l2pn_risk}
\textcolor{red}{(TODO)}