\documentclass{article}
\usepackage{amsmath}
\usepackage{amsfonts, amsthm, amssymb}
\usepackage{graphicx}
\usepackage{hyperref}
\hypersetup{
	colorlinks=true,
	linkcolor=blue,
	citecolor=blue
}
\usepackage[parfill]{parskip}
\usepackage{algpseudocode}
\usepackage{algorithm}
\usepackage{enumerate}
\usepackage[shortlabels]{enumitem}
\usepackage{fullpage}
\usepackage{mathtools}
\usepackage{tikz}
\usepackage{bm}
\usepackage{multirow}
\usepackage[font={small,it}]{caption}

\usepackage{natbib}
\renewcommand{\bibname}{References}
\renewcommand{\bibsection}{\subsubsection*{\bibname}}

\DeclareFontFamily{U}{mathx}{\hyphenchar\font45}
\DeclareFontShape{U}{mathx}{m}{n}{<-> mathx10}{}
\DeclareSymbolFont{mathx}{U}{mathx}{m}{n}
\DeclareMathAccent{\wb}{0}{mathx}{"73}
\DeclareMathAccent{\wc}{0}{mathx}{"71}

\DeclarePairedDelimiterX{\norm}[1]{\lVert}{\rVert}{#1}
\DeclarePairedDelimiterX{\seminorm}[1]{\lvert}{\rvert}{#1}

\newcommand{\eqdist}{\ensuremath{\stackrel{d}{=}}}
\newcommand{\Graph}{\mathcal{G}}
\newcommand{\Reals}{\mathbb{R}}
\newcommand{\Identity}{\mathbb{I}}
\newcommand{\Xsetistiid}{\overset{\text{i.i.d}}{\sim}}
\newcommand{\convprob}{\overset{p}{\to}}
\newcommand{\convdist}{\overset{w}{\to}}
\newcommand{\Expect}[1]{\mathbb{E}\left[ #1 \right]}
\newcommand{\Risk}[2][P]{\mathcal{R}_{#1}\left[ #2 \right]}
\newcommand{\Prob}[1]{\mathbb{P}\left( #1 \right)}
\newcommand{\iset}{\mathbf{i}}
\newcommand{\jset}{\mathbf{j}}
\newcommand{\myexp}[1]{\exp \{ #1 \}}
\newcommand{\abs}[1]{\left \lvert #1 \right \rvert}
\newcommand{\restr}[2]{\ensuremath{\left.#1\right|_{#2}}}
\newcommand{\ext}[1]{\widetilde{#1}}
\newcommand{\set}[1]{\left\{#1\right\}}
\newcommand{\seq}[1]{\set{#1}_{n \in \N}}
\newcommand{\floor}[1]{\left\lfloor #1 \right\rfloor}
\newcommand{\Var}{\mathrm{Var}}
\newcommand{\Cov}{\mathrm{Cov}}
\newcommand{\diam}{\mathrm{diam}}

\newcommand{\emC}{C_n}
\newcommand{\emCpr}{C'_n}
\newcommand{\emCthick}{C^{\sigma}_n}
\newcommand{\emCprthick}{C'^{\sigma}_n}
\newcommand{\emS}{S^{\sigma}_n}
\newcommand{\estC}{\widehat{C}_n}
\newcommand{\hC}{\hat{C^{\sigma}_n}}
\newcommand{\vol}{\text{vol}}
\newcommand{\spansp}{\mathrm{span}~}
\newcommand{\1}{\mathbf{1}}

\newcommand{\Linv}{L^{\Xsetagger}}
\DeclareMathOperator*{\argmin}{argmin}
\DeclareMathOperator*{\argmax}{argmax}

\newcommand{\emF}{\mathbb{F}_n}
\newcommand{\emG}{\mathbb{G}_n}
\newcommand{\emP}{\mathbb{P}_n}
\newcommand{\F}{\mathcal{F}}
\newcommand{\D}{\mathcal{D}}
\newcommand{\R}{\mathcal{R}}
\newcommand{\Rd}{\Reals^d}
\newcommand{\RD}{\Reals^D}
\newcommand{\Nbb}{\mathbb{N}}

%%% Vectors
\newcommand{\thetast}{\theta^{\star}}
\newcommand{\betap}{\beta^{(p)}}
\newcommand{\betaq}{\beta^{(q)}}
\newcommand{\vardeltapq}{\varDelta^{(p,q)}}
\newcommand{\lambdavec}{\boldsymbol{\lambda}}
\newcommand{\bj}{{\bf j}}

%%% Matrices
\newcommand{\X}{X} % no bold
\newcommand{\Y}{Y} % no bold
\newcommand{\Z}{Z} % no bold
\newcommand{\Lgrid}{L_{\grid}}
\newcommand{\Xsetgrid}{D_{\grid}}
\newcommand{\Linvgrid}{L_{\grid}^{\Xsetagger}}
\newcommand{\Lap}{{\bf L}}
\newcommand{\NLap}{{\bf N}}
\newcommand{\PLap}{{\bf P}}
\newcommand{\Id}{I}

%%% Sets and classes
\newcommand{\Xset}{\mathcal{X}}
\newcommand{\Sset}{\mathcal{S}}
\newcommand{\Hclass}{\mathcal{H}}
\newcommand{\Pclass}{\mathcal{P}}
\newcommand{\Leb}{L}
\newcommand{\mc}[1]{\mathcal{#1}}

%%% Distributions and related quantities
\newcommand{\Pbb}{\mathbb{P}}
\newcommand{\Ebb}{\mathbb{E}}
\newcommand{\Qbb}{\mathbb{Q}}
\newcommand{\Ibb}{\mathbb{I}}

%%% Operators
\newcommand{\Tadj}{T^{\star}}
\newcommand{\Xsetive}{\mathrm{div}}
\newcommand{\Xsetif}{\mathop{}\!\mathrm{d}}
\newcommand{\gradient}{\mathcal{D}}
\newcommand{\Hessian}{\mathcal{D}^2}
\newcommand{\dotp}[2]{\langle #1, #2 \rangle}
\newcommand{\Dotp}[2]{\Bigl\langle #1, #2 \Bigr\rangle}

%%% Misc
\newcommand{\grid}{\mathrm{grid}}
\newcommand{\critr}{R_n}
\newcommand{\Xsetx}{\,dx}
\newcommand{\Xsety}{\,dy}
\newcommand{\Xsetr}{\,dr}
\newcommand{\Xsetxpr}{\,dx'}
\newcommand{\Xsetypr}{\,dy'}
\newcommand{\wt}[1]{\widetilde{#1}}
\newcommand{\wh}[1]{\widehat{#1}}
\newcommand{\ol}[1]{\overline{#1}}
\newcommand{\spec}{\mathrm{spec}}
\newcommand{\LE}{\mathrm{LE}}
\newcommand{\LS}{\mathrm{LS}}
\newcommand{\SM}{\mathrm{SM}}
\newcommand{\OS}{\mathrm{FS}}
\newcommand{\PLS}{\mathrm{PLS}}

%%% Order of magnitude
\newcommand{\soom}{\sim}

% \newcommand{\span}{\textrm{span}}

\newtheoremstyle{alden}
{6pt} % Space above
{6pt} % Space below
{} % Body font
{} % Indent amount
{\bfseries} % Theorem head font
{.} % Punctuation after theorem head
{.5em} % Space after theorem head
{} % Theorem head spec (can be left empty, meaning `normal')

\theoremstyle{alden} 


\newtheoremstyle{aldenthm}
{6pt} % Space above
{6pt} % Space below
{\itshape} % Body font
{} % Indent amount
{\bfseries} % Theorem head font
{.} % Punctuation after theorem head
{.5em} % Space after theorem head
{} % Theorem head spec (can be left empty, meaning `normal')

\theoremstyle{aldenthm}
\newtheorem{theorem}{Theorem}
\newtheorem{conjecture}{Conjecture}
\newtheorem{lemma}{Lemma}
\newtheorem{example}{Example}
\newtheorem{corollary}{Corollary}
\newtheorem{proposition}{Proposition}
\newtheorem{assumption}{Assumption}
\newtheorem{remark}{Remark}


\theoremstyle{definition}
\newtheorem{definition}{Definition}[section]

\theoremstyle{remark}

\begin{document}
\title{Minimax-optimal Laplacian Eigenmaps regression over Sobolev Spaces with Neighborhood Graphs}
\author{Alden Green}
\date{\today}
\maketitle

\documentclass{article}

%%% Begin Ryan's template
\usepackage[utf8]{inputenc} % allow utf-8 input
\usepackage[T1]{fontenc}    % use 8-bit T1 fonts
\usepackage{booktabs}       % professional-quality tables
\usepackage{nicefrac}       % compact symbols for 1/2, etc.
\usepackage{microtype}      % microtypography
\usepackage{times}          % times font

\usepackage[round]{natbib}
\usepackage{amssymb,amsmath,amsthm,bbm}
\usepackage[margin=1in]{geometry}
\usepackage{verbatim,float,url,dsfont}
\usepackage{graphicx,subfigure,psfrag}
\usepackage{algorithm,algorithmic}
\usepackage{mathtools,enumitem}
\usepackage[colorlinks=true,citecolor=blue,urlcolor=blue,linkcolor=blue]{hyperref}
\usepackage{multirow}

% Theorems and such
\newtheorem{theorem}{Theorem}
\newtheorem{lemma}{Lemma}
\newtheorem{corollary}{Corollary}
\newtheorem{proposition}{Proposition}
\theoremstyle{definition}
\newtheorem{remark}{Remark}
\newtheorem{definition}{Definition}

% Assumption
\newtheorem*{assumption*}{\assumptionnumber}
\providecommand{\assumptionnumber}{}
\makeatletter
\newenvironment{assumption}[2]{
	\renewcommand{\assumptionnumber}{Assumption #1#2}
	\begin{assumption*}
		\protected@edef\@currentlabel{#1#2}}
	{\end{assumption*}}
\makeatother

% Widebar
\makeatletter
\newcommand*\rel@kern[1]{\kern#1\dimexpr\macc@kerna}
\newcommand*\widebar[1]{%
	\begingroup
	\def\mathaccent##1##2{%
		\rel@kern{0.8}%
		\overline{\rel@kern{-0.8}\macc@nucleus\rel@kern{0.2}}%
		\rel@kern{-0.2}%
	}%
	\macc@depth\@ne
	\let\math@bgroup\@empty \let\math@egroup\macc@set@skewchar
	\mathsurround\z@ \frozen@everymath{\mathgroup\macc@group\relax}%
	\macc@set@skewchar\relax
	\let\mathaccentV\macc@nested@a
	\macc@nested@a\relax111{#1}%
	\endgroup
}
\makeatother

% Min and max
\newcommand{\argmin}{\mathop{\mathrm{argmin}}}
\newcommand{\argmax}{\mathop{\mathrm{argmax}}}
\newcommand{\minimize}{\mathop{\mathrm{minimize}}}
\newcommand{\st}{\mathop{\mathrm{subject\,\,to}}}

% Shortcuts
\def\R{\mathbb{R}}

%%% End Ryan's template

%%% Begin Alden's additions
\newcommand{\Ebb}{\mathbb{E}}
\newcommand{\Pbb}{\mathbb{P}}
\newcommand{\dotp}[2]{\langle #1, #2 \rangle}
\newcommand{\wt}[1]{\widetilde{#1}}
\newcommand{\wh}[1]{\widehat{#1}}
\newcommand{\mc}[1]{\mathcal{#1}}
\newcommand{\Reals}{\mathbb{R}} % Same thing as Ryan's \R
\newcommand{\Rd}{\Reals^d}
\newcommand{\wb}[1]{\widebar{#1}}
\newcommand{\floor}[1]{\left\lfloor #1 \right\rfloor}
\newcommand{\Var}{\mathrm{Var}}
\newcommand{\Cov}{\mathrm{Cov}}
\newcommand{\1}{\mathbf{1}}
\newcommand{\bj}{{\bf j}}
\newcommand{\restr}[2]{\ensuremath{\left.#1\right|_{#2}}}

\DeclareFontFamily{U}{mathx}{\hyphenchar\font45}
\DeclareFontShape{U}{mathx}{m}{n}{<-> mathx10}{}
\DeclareSymbolFont{mathx}{U}{mathx}{m}{n}
\DeclareMathAccent{\wc}{0}{mathx}{"71}
%%% End Alden's Additions

\begin{document}
	\begin{center} {\Large{\bf{Convergence Rates of Principal Components Regression with Laplacian Eigenmaps}}}
		
		\vspace*{.3cm}
		
		{\large{
				\begin{center}
					Alden Green~~~~~ Sivaraman Balakrishnan~~~~~ Ryan J. Tibshirani\\
					\vspace{.2cm}
				\end{center}
				
				
				\begin{tabular}{c}
					Department of Statistics and Data Science \\
					Carnegie Mellon University
				\end{tabular}
				
				\vspace*{.2in}
				
				\begin{tabular}{c}
					\texttt{\{ajgreen,siva,ryantibs\}@stat.cmu.edu}
				\end{tabular}
		}}
		
		\vspace*{.2in}
		
		\today
		\vspace*{.2in}
	\end{center}

	\section{Introduction}

	Laplacian Eigenmaps (LE)~\citep{belkin03a} is a method for nonlinear dimensionality reduction and data representation. Given scattered data points $\{X_1,\ldots,X_n\} \subset \Reals^d$, LE maps each $X_i$ to a vector $(v_{i,1},\ldots,v_{i,K})$ according to the following steps.
	\begin{enumerate}
		\item First, LE forms a \emph{neighborhood graph} $G = (V,W)$ over the points $\{X_1,\ldots,X_n\}$. The graph $G$ is an undirected, weighted graph, with vertices $V = \{X_1,\ldots,X_n\}$, and weighted edges $W_{ij}$ which correspond to the proximity between points $X_i$ and $X_j$, say in Euclidean distance.(For a formal definition see Section~\ref{})
		\item Next, LE forms an (unweighted) \emph{graph Laplacian} matrix $L \in \Reals^{n \times n}$, a symmetric and diagonally dominant matrix with diagonal elements $L_{ii} = \sum_{j = 1}^{n} W_{ij}$, and off-diagonal elements $L_{ij} = -W_{ij}$. 
		\item Finally, LE takes the eigendecomposition $L = \sum_{k = 1}^{n} \lambda_k v_k v_k^{\top}$, and outputs the vectors $(v_{1,i},\ldots,v_{K,i}) \in \Reals^K$ for each $i = 1,\ldots,n$.
	\end{enumerate} 
	
	A natural way to use LE is by taking the vectors $\{(v_{i,1},\ldots,v_{i,K})\}_{i = 1}^{n}$ to be features in a downstream supervised learning algorithm. In this paper, we study a simple method along these lines: Principal Components Regression with Laplacian-Eigenmaps (PCR-LE), a method for nonparametric regression which operates by running ordinary least squares (OLS) using the features learned by LE. Given pairs of design points and responses $(X_1,Y_1),\ldots, (X_n,Y_n)$, PCR-LE computes an estimate $\wh{f} \in \Reals^n$,
	\begin{equation}
	\label{eqn:pcr-le}
	\wh{f} := \argmin_{f \in \mathrm{span}\{v_1,\ldots,v_K\}} \|{\bf Y} - f\|_2^2 = V_K V_K^{\top} {\bf Y},
	\end{equation}
	where ${\bf Y} = (Y_1,\ldots,Y_n) \in \Reals^n$ is the vector of responses, $\|\cdot\|_2$ denotes the usual Euclidean norm in $\Reals^n$, and the equality in \eqref{eqn:pcr-le} is due to the orthogonality of eigenvectors with respect to Euclidean inner product. 
	
	LE has been practically very successful, and by now has been used for various statistical tasks such as spectral clustering, manifold learning, level-set estimation, semi-supervised learning, etc. At this point there exists a rich literature \citep{koltchinskii2000,belkin07,vonluxburg2008,burago2014,shi2015,singer2017,garciatrillos18,trillos2019, calder2019, cheng2021,dunson2021} explaining this practical success from a theoretical perspective. Loosely speaking, these works model the design points as being independent samples from a distribution $P$ with density $p$, and show that in this case the eigenvectors of the graph Laplacian $L$ are good empirical approximations of population-level objects. These population-level objects are the eigenfunctions $\psi_k$---meaning solutions, along with eigenvalues $\rho_k$, to the equation $\Delta_P \psi_k = \rho_k \psi_k$--- of the density-weighted Laplacian operator
	\begin{equation}
	\label{eqn:density-weighted-laplace}
	\Delta_Pf := -\frac{1}{p}~ \mathrm{div}(p^2 \nabla f).
	\end{equation}  
	(Here $\mathrm{div}$ stands for the divergence operator, and $\nabla$ for the gradient.) These eigenfunctions in turn characterize various interesting structural aspects of $p$, such as the location and number of high- and low-density regions, the shape and intrinsic dimension of its support, and so forth.
	
	These aforementioned works explain what kind of representation LE learns, and the accuracy with which it learns this representation. However, this theory does not address the convergence rates of PCR-LE. That is the major question we answer in this paper. We adopt the classical model of nonparametric regression with random design: we observe independent pairs $(X_1,Y_1),\ldots,(X_n,Y_n)$ of design points and responses, where the design points $\{X_1,\ldots,X_n\}$ are sampled independently from an unknown distribution $P$ supported on $\mc{X} \subseteq \Rd$, and the responses follow a signal plus Gaussian noise model,
	\begin{equation}
	\label{eqn:model}
	Y_i = f_0(X_i) + w_i, \quad w_i \sim N(0,1),
	\end{equation}
	with noise $w_i$ independent of $X_i$. The task is to learn the regression function $f_0$, which is unknown but assumed to belong to a Sobolev space $H^s(\mc{X})$. We consider two settings: one where $\mc{X}$ is a full-dimensional domain, and the other where $\mc{X}$ is a low-dimensional submanifold of $\Rd$. In each setting, we derive upper bounds which imply that the PCR-LE estimate $\wh{f}$, and a test using the statistic $T = \|\wh{f}\|_2^2$, are statistically optimal methods for two classical problems in nonparametric regression: estimation and goodness-of-fit testing. 
	
	At first glance, the optimality of PCR-LE is somewhat surprising. As a function of the sample size $n$, known rates of convergence of LE are much slower than the minimax-optimal rates of convergence over $H^s(\mc{X})$ (roughly speaking $n^{-2s/(2s + d)}$ for estimation and $n^{-4s/(4s + d)}$ for testing). This is not due to suboptimal theory for LE, but rather reflects a fundamental fact: regression using learned features does not rely on accurately learning the features. It is essential to employ modes of analysis which exploit this fact in order to derive sharp rates of convergence for PCR-LE. We explain this in more detail after summarizing our main results.
	
	\paragraph{Sobolev spaces and spectral series regression.}
	To analyze PCR-LE, we work in a classical situation where the regression function is assumed to belong to a (Hilbert-)Sobolev space. For an open domain $\mc{X} \subseteq \Rd$, the Sobolev space $H^s(\mc{X})$ consists of all functions $f \in L^2(\mc{X})$ which are $s$-times weakly differentiable, with all order-$s$ partial derivatives $D^{\alpha}f \in L^2(\mc{X})$. We study regression over Sobolev spaces in part because generally speaking, the minimax rates are well-understood; as mentioned before, as a function of the sample size $n$ they are $n^{-2s/(2s + d)}$ for estimation, and $n^{-4s/(4s + d)}$ for testing. For this reason, regression over Sobolev spaces is a good setting in which to see whether PCR-LE measures up to simpler and more classical minimax-optimal approaches, which have strong theoretical guarantees but are less often used in practice.
	
	Moreover, we view PCR-LE as being particularly well-suited for regression over Sobolev spaces, due to their close connection with \emph{spectral series regression}. Spectral series regression computes empirical Fourier coefficients and truncates to the lowest frequency eigenfunctions, producing the estimate
	\begin{equation}
	\label{eqn:population-level_spectral_series}
	\wt{a}_k = \frac{1}{n}\sum_{i = 1}^{n} Y_i \psi_k(X_i), \quad \wt{f}(x) = \sum_{k = 1}^{K} \wt{a}_k \psi_k(x).
	\end{equation} 
	Under appropriate boundary conditions, the Sobolev spaces $H^s(\mc{X})$ can roughly be characterized as consisting of those functions $f = \sum_{k} a_k \psi_k \in L^2(\mc{X})$ for which the generalized Fourier coefficients $\{a_k\}_{k = 1}^{\infty}$ satisfy the decay condition $\sum_{k} a_k \rho_k^2 < \infty$. Heuristically, this decay condition justifies the truncation in~\eqref{eqn:population-level_spectral_series}---since the truncation incurs only a limited amount of bias for any $f_0 \in H^s(\mc{X})$---and for this reason spectral series regression over Sobolev spaces has been studied since \textcolor{red}{(?)}, shown to be optimal for estimation by \textcolor{red}{(?)}, and for goodness-of-fit testing by \textcolor{red}{(?)}.
	
	PCR-LE serves as an empirical approximation to spectral series regression, since the eigenvectors $v_k$ are empirical approximations to eigenfunctions $\psi_k$. Viewed in this light, a major advantage of PCR-LE is that it operates independently of the design distribution $P$. In the contrast, the spectral series estimator defined in~\eqref{eqn:population-level_spectral_series} relies on diagonalizing the density-weighted Laplacian $\Delta_P$. In our context, where $P$ is treated as unknown,~\eqref{eqn:population-level_spectral_series} must be viewed as an oracle method, and to emphasize this  we henceforth refer to~\eqref{eqn:population-level_spectral_series} as \emph{population-level spectral series regression}. On the other hand, naturally PCR-LE incurs some extra error by using an empirical approximation to the underlying basis $\{\psi_k\}$, and our work shows that in many cases, this extra error is small enough that it does not change the overall rate of convergence.
	
	\subsection{Main contributions}
	Summarized succinctly, our main contribution is to theoretically analyze nonparametric regression with PCR-LE and establish upper bounds which imply that, in many cases, this method achieves optimal rates of convergence over Sobolev spaces.
	
	\paragraph{Rates of convergence: population-level spectral series regression.}
	As we have already mentioned, the minimax-optimal rates over Sobolev spaces are generally well-known, as are upper bounds for population-level spectral series methods which match these rates. However, we could not find precisely stated results applying to our setting, where in particular
	\begin{enumerate}
		\item We consider subcritical Sobolev spaces $H^s(\mc{X})$, where the smoothness parameter $s$ satisfies $s < d/2$ and so $H^s(\mc{X})$ does not continuously embed into the space of continuous functions $C^0(\mc{X})$, and 
		\item We consider general design distributions $P$, which may satisfy certain regularity conditions but are not limited to being, say, the uniform distribution over $[0,1]^d$. 
	\end{enumerate}
	For completeness, we analyze population-level spectral series methods in this general setting, and establish upper bounds showing that such methods converge at the ``usual'' rates of $n^{-2s/(2s + d)}$ for estimation and $n^{-4s/(4s + d)}$ for testing. This analysis relies heavily on certain asymptotic properties of the continuum eigenfunctions $\psi_k$ and eigenvalues $\rho_k$, which hold for quite general second-order differential operators $\mc{L}$ including the density-weighted Laplacian $\mc{L} = \Delta_P$.
	 
	\paragraph{Rates of convergence: PCR-LE.}
	The rest of our results consist of various upper bounds on the rates of convergence for the PCR-LE estimator $\wh{f}$, and a test using the statistic $T = \|\wh{f}\|_2^2$. These upper bounds quantify two important properties of PCR-LE: first, that it can take advantage of smooth higher-order derivatives, and second that it can adapt to low intrinsic dimension of the design distribution, each in an optimal manner. We first model the design distribution $P$ as having support $\mc{X}$ which is a full-dimensional set in $\Rd$. In this case, our main contributions are as follows:
	\begin{itemize}
		\item  Over a ball in the Sobolev space $H^{s}(\mc{X})$, we establish that the PCR-LE estimator $\wh{f}$ has in-sample mean-squared error of at most on the order of $n^{-2s/(2s + d)}$, for any number of derivatives $s \in \mathbb{N}$ and dimension $d$. 
		\item We show that a test based on the statistic $\|\wh{f}\|_2^2$ has a squared critical radius on the order of $n^{-4s/(4s + d)}$, for any number of derivatives $s \in \mathbb{N}$ and dimension $d \in \{1,2,3,4\}$. 
	\end{itemize}
	We then consider the behavior of PCR-LE when the data satisfies a \emph{manifold hypothesis}, meaning the design distribution is supported on an (unknown) domain $\mc{X}$ which is a submanifold of $\Rd$ of intrinsic dimension $m \in \mathbb{N}, m < d$. In this case, our upper bounds imply that:
	\begin{itemize}
		\item When $f_0 \in H^s(\mc{X})$ the PCR-LE estimator has in-sample mean squared error of at most $n^{-2s/(2s + m)}$, when $s \in \{1,2,3\}$ and for any $m \in \mathbb{N}$. 
		\item A test based on the statistic $\|\wh{f}\|_2^2$ has a squared critical radius on the order of $n^{-4s/(4s + m)}$, when $s \in \{1,2,3\}$ and $m \in \{1,2,3,4\}$.
	\end{itemize}
	We note that under the manifold hypothesis, to the best of our knowledge the minimax rates for Sobolev spaces $H^s(\mc{X})$ under the assumption of random design with design distribution supported on an \emph{unknown} manifold have not been worked out. Our upper bounds confirm that these rates are the same as for regression over a known manifold, at least for the values of $s$ and $m$ mentioned above.
	
	In all these cases, our bounds also depend optimally on the radius $M$ of the Sobolev ball under consideration. However, for some values of $s$ (number of derivatives) and $d$ (dimension), there do exist gaps between our upper bounds on the error of PCR-LE and the minimax rates. Although we do not give corresponding lower bounds verifying the tightness of our analysis, we believe these gaps reflect the true behavior of the method rather than some looseness in our analysis, and we comment more on this at relevant parts in the text. For completeness, we summarize all of our upper bounds---those which match the minimax rates, and those which do not---in Tables~\ref{tbl:estimation_rates} and~\ref{tbl:testing_rates}.
	\begin{table}
		\begin{center}
			\begin{tabular}{p{.2\textwidth} | p{.14\textwidth} p{.12\textwidth} }
				Smoothness order & Flat Euclidean (Model~\ref{def:model_flat_euclidean}) & Manifold (Model~\ref{def:model_manifold}) \\
				\hline
				$s \leq 3$ & ${\bf n^{-2s/(2s + d)}}$ & ${\bf n^{-2s/(2s + m)}}$ \\
				$s > 3$  & ${\bf n^{-2s/(2s + d)}}$ & $n^{-6/(6 + m)}$
			\end{tabular}
		\end{center}
		\caption{Summary of Laplacian eigenmaps estimation rates over Sobolev balls. Bold font marks minimax optimal rates. In each case, rates hold for all $d \in \mathbb{N}$ (under Model~\ref{def:model_flat_euclidean}), and for all $m \in \mathbb{N}, 1 < m < d$ (under Model~\ref{def:model_manifold}). Although we suppress it for simplicity, in all cases when the Laplacian eigenmaps estimator is optimal, the dependence of the error rate on the radius $M$ of the Sobolev ball is also optimal.}
		\label{tbl:estimation_rates}
	\end{table}
	
	\begin{table}
		\begin{center}
			\begin{tabular}{p{.175\textwidth} p{.175\textwidth} | p{.14\textwidth} p{.12\textwidth} }
				Smoothness order & Dimension & Flat Euclidean (Model~\ref{def:model_flat_euclidean}) & Manifold (Model~\ref{def:model_manifold}) \\
				\hline
				\multirow{2}{*}{$s = 1$} & $\dim(\mc{X}) < 4$ & ${\bf n^{-4s/(4s + d)}}$ & ${\bf n^{-4s/(4s + m)}}$ \\
				& $\dim(\mc{X}) \geq 4$ & ${\bf n^{-1/2}}$ & ${\bf n^{-1/2}}$ \\
				\hline
				\multirow{3}{*}{$s = 2$ or $3$} & $\dim(\mc{X}) \leq 4$  & ${\bf n^{-4s/(4s + d)}}$ & ${\bf n^{-4s/(4s + m)}}$ \\
				& $4 <\dim(\mc{X}) < 4s$  & $n^{-2s/(2(s - 1) + d)}$ & $n^{-2s/(2(s - 1) + m)}$\\
				& $\dim(\mc{X}) \geq 4s$ & ${\bf n^{-1/2}}$ & ${\bf n^{-1/2}}$ \\
				\hline
				\multirow{3}{*}{$s > 3$} & $\dim(\mc{X}) \leq 4$ & ${\bf n^{-4s/(4s + d)}}$ & $n^{-12/(12 + d)}$ \\
				& $4 < \dim(\mc{X}) < 4s$ & $n^{-2s/(2(s - 1) + d)}$ & $n^{-6/(4 + m)}$ \\
				& $\dim(\mc{X}) \geq 4s$ & ${\bf n^{-1/2}}$ & ${\bf n^{-1/2}}$ \\
			\end{tabular}
		\end{center}
		\caption{Summary of Laplacian eigenmaps testing rates over Sobolev balls. Bold font marks minimax optimal rates. Rates when $d > 4s$ assume that $f_0 \in L^4(\mc{X})$, and depend on $\|f_0\|_{L^4(\mc{X})}$. Although we suppress it for simplicity, in all cases when othe Laplacian eigenmaps test is optimal, the dependence of the error rate on the radius $M$ of the Sobolev ball is also optimal.}
		\label{tbl:testing_rates}
	\end{table}
	
	\paragraph{Technical contributions.}
	One of our primary technical contributions is a strategy for analyzing regression methods using learned features which leads to sharp rates of convergence. 
	
	At an abstract level PCR-LE is a two-stage algorithm: the first stage (LE) uses the observed design points $\{X_1,\ldots,X_n\}$ to learn a feature representation ($X_i \mapsto (v_{1,i},\ldots,v_{K,i})$) which is an empirical approximation to some ideal population-level representation ($X_i \mapsto (\psi_{1}(X_i),\ldots,\psi_{K}(X_i))$);  the second stage then applies a simple method (OLS) to the learned features, and obtains an estimate. At first blush, one might expect the error for PCR-LE to be likewise decomposed into two parts: first, the error with which the features are learned (i.e. error due to sampling variation in $X$), and the second being error with which, given ideal population-level features, the function is learned (i.e. error due to noise in $Y$.) 
	
	Crucially, our analysis \emph{does not} work in this way. This is important because, as already mentioned, all known upper bounds on the error with which the learned features (Laplacian eigenvectors) approximate their population-level limits (continuum Laplacian eigenfunctions) are much larger than the minimax rates of convergence over $H^s(\mc{X})$. For example,
	\begin{itemize}
		\item The best known rates of convergence for graph Laplacian eigenvectors (due to~\cite{cheng2021}) are
		\begin{equation}
		\label{eqn:eigenvector_convergence}
		\max_{1 \leq k \leq K}\|\sqrt{n}v_k - \psi_k\|_n^2 \leq C_{K} n^{-2/(4 + d)},
		\end{equation}
		where the constant $C_K$ depends on $K$ in an unknown manner.
		\item The dependence of $C_K$ on $K$ is relevant for our interests where $K$ is growing with $n$. The earlier works of \cite{burago2014,trillos2019} give upper bounds which, although implying weaker rates of convergence as a function of $n$, do keep track of $C_K$, and suggest that it should depend on the inverse of the spectral gap 
		\begin{equation}
		\label{eqn:spectral_gap}
		C_K = \frac{C}{\rho_{K + 1} - \rho_K} \geq C K^{1 - 2/d},
		\end{equation}
		where $C$ is a constant depending only on the dimension $d$, and the latter inequality follows from the Weyl's Law scaling $\rho_K \asymp K^{2/d}$.
	\end{itemize}
	Although these upper bounds may not reflect the true rate of convergence of graph Laplacian eigenvectors---this is still an active area of research, and no lower bounds are known---it seems very unlikely that the true rate matches even the estimation minimax rate $n^{-2s/(2s + d)}$, which after all approaches the dimension-free rate $1/n$ for large values of $s$. 
	
	Instead of relying on convergence of eigenvectors to eigenfunctions, our analysis proceeds via a bias-variance decomposition at the level of the graph. Focusing for simplicity on estimation, this is 
	\begin{equation}
	\label{eqn:bias_variance_estimation}
	\|f_0 - \wh{f}\|_n^2 \leq 2\bigl(\|f_0 - V_KV_K^{\top}f_0\|_n^2 + \|V_KV_K^{\top}({\bf Y} - f_0)\|_n^2\bigr) 
	\end{equation}
	The second term in~\eqref{eqn:bias_variance_estimation} is the variance, and as usual for OLS estimates depends only on the degrees of freedom $\mathrm{df}(\wh{f}) = \mathrm{tr}(V_KV_K^{\top}) = K$. More surprisingly, the first term (squared-bias) can also be upper bounded without appealing to the eigenfunctions $\psi_k$. Letting $S = L^{\dagger}$ be the pseudo-inverse of the graph Laplacian,
	\begin{equation}
	\begin{aligned}
	\|(I - V_KV_K^{\top})f_0\|_n^2 & = \|(I - V_KV_K^{\top})S^{s/2}L^{s/2}f_0\|_n^2 \leq \|(I - V_KV_K^{\top})S^{s/2}\|_{op}^2 \|L^{s/2}f_0\|_n^2 = \frac{f_0^{\top} L^s f_0}{n \lambda_{K + 1}^{s}}.
	\end{aligned}
	\end{equation}
	Thus we obtain an upper bound on the in-sample sample mean squared error $\|\wh{f} - f_0\|_n^2$ that is \emph{independent of the error with which graph Laplacian eigenvectors $v_k$ approximate population-level eigenfunctions $\psi_k$}. 
	
	Instead, our upper bound on the error of PCR-LE is determined by a pair of graph functionals: the quadratic form $f_0^{\top}L^s f_0$, and the graph Laplacian eigenvalue $\lambda_{K + 1}$. Theoretically speaking, this brings two advantages over directly analyzing convergence of eigenvectors to eigenfunctions. First, the graph functionals can be more tightly controlled than the eigenvectors $v_k$; for example, the error  Moreover, in order to obtain rates of convergence we do not require that these functionals themselves converge to population-level limits, but only that they be stochastically bounded on the right order. To derive our ultimate upper bounds we use some existing results regarding neighborhood graph Laplacians, and prove some new ones which may be of independent interest. 
		
	A last observation regarding the difference between feature learning and regression using learned features. To balance the bias and variance terms in~\eqref{eqn:bias_variance_estimation}, we will end up using $K(n) := n^{d/(2s + d)}$ eigenvectors. Although this is the usual choice for spectral series regression over order-$s$ Sobolev spaces, in the context of PCR-LE it leads to a striking conclusion. Namely, examining~\eqref{eqn:eigenvector_convergence} and~\eqref{eqn:spectral_gap}, we see that for all dimensions $d \geq 3$, known upper bounds on $\|\sqrt{n} v_{K(n)} - \psi_{K(n)}\|_n^2$ do not converge to $0$ as $n \to \infty$. In words, it is not merely the case that PCR-LE converges at a faster rate than LE itself, but that PCR-LE can profitably use eigenvectors for regression which may be inconsistent estimates of their population-level limits. 
	
	To summarize, our work demonstrates, broadly speaking, that regression using learned features can be analyzed independent of the accuracy of the features themselves. Regression using learned features is a general and widely applied paradigm, and we believe this observation may have consequences outside of its application to PCR-LE in this work.
	
	\subsection{Related work}
	
	Much of the work regarding regression using neighborhood graph Laplacians deals with \emph{semi-supervised learning}, where in addition to the labeled data $(X_1,Y_1),\ldots,(X_n,Y_n)$ one observes unlabeled points $(X_{n + 1},\ldots,X_{N})$. The landmark paper of \cite{zhu2003semisupervised} proposed to interpolate the observed values by~\emph{harmonic extension}, i.e. compute $L_N$ the Laplacian matrix corresponding to a graph formed over all design points $X_1,\ldots,X_N$, and then solve the constrained problem
	\begin{equation*}
	\minimize_{f \in \Reals^N} f^{\top} L_N f \quad \mathrm{subject\,\,to}~~~ f_i = Y_i~~\textrm{for $i = 1,\ldots,n$.}
	\end{equation*}
	Conventional wisdom says that harmonic extension is sensible only when the responses are noiseless, $Y_i = f_0(X_i)$, and that in the noisy setting one should instead solve the penalized formulation
	\begin{equation}
	\label{eqn:graph_laplacian_regularization_ssl}
	\wc{f} = \argmin_{f \in \Reals^N} \|{\bf Y} - f\|_n^2 + \lambda f^{\top} L_N f.
	\end{equation}
	Notwithstanding their intuitive appeal, both the constrained and penalized problems have issues when $d > 1$ and $n/N \to 0$~\citep{nadler09,calder2019b, calder2020}, and in certain cases the estimates are degenerate, meaning they are ``spiky'' at labeled data points and close to constant everywhere else. One solution to this problem is to instead use Laplacian Eigenmaps for semi-supervised learning (SSL-LE), i.e. compute the eigendecomposition $L_N = \sum_{k = 1}^{N} \lambda_k u_k u_k^{\top}$ and solve the problem
	\begin{equation}
	\label{eqn:laplacian_eigenmaps_ssl}
	\wb{f} := \argmin_{f \in \mathrm{span}\{u_1,\ldots,u_K\}} \|{\bf Y} - f\|_n^2.
	\end{equation}
	\cite{zhou2011,lee2016} analyze SSL-LE in a particular asymptotic regime where $n$ is held fixed while $N \to \infty$, and show that it achieves minimax-optimal rates of convergence over $H^s(\mc{X})$ as a function of $n$, the number of labeled points. Of course, in this regime the relevant eigenvectors of $L_n$ all converge to their continuum limits, meaning
	\begin{equation*}
	\lim_{N \to \infty} \max_{1 \leq K \leq K(n)} \|\sqrt{N} u_k - \psi_k\|_N^2 = 0,
	\end{equation*}
	and consequently $\lim_{N \to \infty} \|\wb{f} - \wt{f}\|_N^2 = 0$. Thus the analysis of SSL-LE  reduces to that of population-level spectral series regression. The supervised setting (where $N = n$) we consider in this work is very different, and analyzing PCR-LE necessitates an entirely different approach, as explained in the preceding section.
	
	There has been much less work, relatively speaking, regarding regression with neighborhood graph Laplacians in the supervised setting, which is the focus of this work. \citet{lee2016} also analyze a variant of PCR-LE, but derive suboptimal rates of convergence. \citet{trillos2020} study nonparametric regression via graph Laplacian regularization, where the estimator is computed by solving~\eqref{eqn:graph_laplacian_regularization_ssl}, but with no unlabeled data. They establish the upper bound $\max_{i = 1,\ldots,n}|\wc{f}(X_i) - f_0(X_i)| \leq C n^{-2/(2 + d)}$ under the assumption $f_0 \in C^2(\mc{X})$, which is slower than the minimax rate $n^{-4/(4 + d)}$ for this class. In a previous paper~\citep{green2021}, we (the authors) also considered nonparametric regression via graph Laplacian regularization, when the regression function $f_0 \in H^1(\mc{X})$ and loss is measured using in-sample mean squared error. We found that graph Laplacian regularization is consistent as $n \to \infty$, and optimal (using mean-squared error) over the first-order Sobolev space $H^1(\mc{X})$ when $d \in \{1,2,3,4\}$. However, graph Laplacian regularization neither takes advantage of smooth higher-order derivatives, nor is it provably optimal over $H^1(\mc{X})$ for all dimensions $d$. One of our motivations for considering PCR-LE was to find an estimator which addressed these deficiencies. In this work we indeed find that PCR-LE has much stronger optimality properties than graph Laplacian regularization. 
	
	Most work on supervised learning using graphs adopts a \emph{fixed design} perspective, treating the design points $X_1 = x_1,\ldots,X_n = x_n$ as vertices of a fixed graph, and carrying out inference with respect to the conditional mean vector $(f_0(x_1),\ldots,f_0(x_n))$. In this setting, matching upper and lower bounds have been established that certify the optimality of graph-based methods for estimation \citep{wang2016,hutter2016,sadhanala16,sadhanala17,kirichenko2017,kirichenko2018}) and testing \citep{sharpnack2010identifying,sharpnack2013b,sharpnack2013,sharpnack2015} over different ``function'' classes (in quotes because these classes really model the $n$-dimensional vector of evaluations). This setting is quite general, because the graph need not be a geometric graph defined on a vertex set which belongs to Euclidean space. On the other hand, depending on the data collection process, it may be unnatural to model the design points as being a priori fixed, and the estimand as being a vector which exhibits a discrete notion of ``smoothness'' over this fixed design. Instead, we adopt the \emph{random design} perspective, and seek to estimate a function that we assume exhibits a more classical notion of smoothness. 
	
	
% Bibliography
\bibliographystyle{plainnat}
\bibliography{../../../graph_regression_bibliography} 
	
\end{document}
\section{Preliminaries}
\label{sec:setup_main_results}

We begin in Sections~\ref{subsec:regression_laplacian_eigenmaps}-\ref{subsec:laplacian_eigenmaps} by precisely defining the models (random design points, Sobolev-smooth regression functions) and methods (Laplacian eigenmaps) under consideration. Then in Section~\ref{subsec:spectral_projection}, we connect Sobolev spaces to Laplacian eigenmaps using the spectrum of a density-dependent Laplace-Beltrami operator, and show in Propositions~\ref{prop:spectral_series_estimation} and~\ref{prop:spectral_series_testing} that projection methods which use the eigenfunctions of this operator are statistically optimal.

\subsection{Nonparametric regression over Sobolev spaces}
\label{subsec:regression_laplacian_eigenmaps}

We will always operate in the usual setting of nonparametric regression with random design. We observe independent random samples $(X_1,Y_1),\ldots,(X_n,Y_n)$; the design points $X_1,\ldots,X_n$ are sampled from a distribution $P$ with support $\mc{X} \subseteq \Rd$, and the responses follow the signal plus noise model
\begin{equation}
\label{eqn:model}
Y_i = f_0(X_i) + w_i,
\end{equation}
with regression function $f_0: \mc{X} \to \Reals$, and $w_i \sim N(0,1)$ independent Gaussian noise. 
% (AG 8/4/21): Removed the comment regarding noise with variance = \sigma^2. 

We now formulate two models, which differ in the assumed nature of the support $\mc{X}$ of the design distribution $P$: the \emph{flat Euclidean} and \emph{manifold} models.

\paragraph{Flat Euclidean model.}
In Definitions~\ref{def:model_flat_euclidean}-\ref{def:zero_trace_sobolev_space}, we collect the assumptions we make when working under the flat Euclidean model. We begin by giving some regularity conditions on the design.

\begin{definition}[Flat Euclidean model]
	\label{def:model_flat_euclidean}
	 The support $\mc{X}$ of the design distribution $P$ is an open, connected, and bounded subset of $\Rd$, with Lipschitz boundary. The distribution $P$ admits a Lipschitz density $p$ with respect to the $d$-dimensional Lebesgue measure $\nu$, which is bounded away from $0$ and $\infty$,
	\begin{equation*}
	0 < p_{\min} \leq p(x) \leq p_{\max} < \infty, \quad \textrm{for all $x \in \mc{X}$.}
	\end{equation*}
\end{definition}
% (AG 8/13/21): I got rid of the sentence ``The data are sampled according to (1).'' It seemed repetitive. 
At various points we will also assume that the density $p \in C^k(\mc{X})$ for some $k \in \mathbb{N}$; when we make such assumptions we will state them explicitly.

We model the regression function as belonging to an order-$s$ Sobolev space, and being bounded in Sobolev norm.
\begin{definition}[Sobolev space on a flat Euclidean domain]
	\label{def:sobolev_space}
	For an integer $s \geq 1$, a function $f \in L^2(\mc{X})$ belongs to the Sobolev space $H^s(\mc{X})$ if for all $|\alpha| \leq s$, the weak derivatives $D^{\alpha}f$ exists and satisfy $D^{\alpha}f \in L^2(\mc{X})$. The $j$th order semi-norm for $f \in H^s(\mc{X})$ is $|f|_{H^j(\mc{X})} := \sum_{|\alpha| = j}\|D^{\alpha}f\|_{\Leb^2(\mc{X})}$, and the corresponding norm
	\begin{equation*}
	\|f\|_{H^s(\mc{X})}^2 := \|f\|_{\Leb^2(\mc{X})}^2 + \sum_{j = 1}^{s} |f|_{H^j(\mc{X})}^2,
	\end{equation*}
	induces the Sobolev ball
	\begin{equation*}
	H^s(\mc{X};M) := \bigl\{f \in H^s(\mc{X}): \|f\|_{H^s(\mc{X})} \leq M\bigr\}.
	\end{equation*} 
\end{definition}
Finally when $s > 1$ we will assume that $f_0$ satisfies a zero-trace boundary condition. Recall that $H^s(\mc{X})$ can alternatively be defined as the completion of $C^{\infty}(\mc{X})$ in the Sobolev norm $\|\cdot\|_{H^s(\mc{X})}$. The zero-trace Sobolev spaces are defined in a similar fashion, as the completion of $C_c^{\infty}(\mc{X})$ in the same norm.

\begin{definition}[Zero-trace Sobolev space]
	\label{def:zero_trace_sobolev_space}
	A function $f \in H^s(\mc{X})$ belongs to the zero-trace Sobolev space $H_0^s(\mc{X})$ if there exists a sequence $f_1,f_2,\ldots$ of functions in $C_c^{\infty}(\mc{X})$ such that
	\begin{equation*}
	\lim_{k \to \infty}\|f_k - f\|_{H^s(\mc{X})} = 0.
	\end{equation*}
	The normed ball $H_0^{s}(\mc{X};M) := H_0^{s}(\mc{X}) \cap H^{s}(\mc{X};M)$.
\end{definition}
Boundary conditions plays an important role in the analysis of spectral methods, as we explain further in Section~\ref{subsec:spectral_projection}. For now, we limit ourselves to pointing out that for functions $f \in C^\infty(\mc{X})$, the zero-trace condition implies that $\partial^{k}f/\partial{\bf n}^k(x) = 0$ for each $k = 0,\ldots,s - 1$, and for all $x \in \partial\mc{X}$. (Here $\partial/(\partial {\bf n})$ is the partial derivative operator in the direction of the normal vector $\mathbf{n}$.)
\paragraph{Manifold model.}
As in the flat Euclidean case, we start with some regularity conditions on the design.
\begin{definition}[Manifold model]
	\label{def:model_manifold}
	The support $\mc{X}$ of the design distribution $P$ is a closed, connected, and smooth Riemannian manifold (without boundary) embedded in $\Rd$, of intrinsic dimension $1 \leq m < d$, and with a positive injectivity radius $\mathrm{inj}(\mc{X}) > 0$. The design distribution $P$ admits a Lipschitz density $p$ with respect to the volume form $d\mu$ induced by the Riemannian structure of $\mc{X}$, which is bounded away from $0$ and $\infty$,
	\begin{equation*}
	0 < p_{\min} \leq p(x) \leq p_{\max} < \infty, \quad \textrm{for all $x \in \mc{X}$.}
	\end{equation*}
\end{definition}
In the above, recall that the injectivity radius $\mathrm{inj}(\mc{X})$ of an $m$-dimensional Riemannian manifold $\mc{X}$ is the maximum $\delta > 0$ such that the exponential map $\exp_x: B_m(0,\delta) \subset T_x(\mc{X}) \to B_{\mc{X}}(x,\delta) \subset \mc{X}$ is a diffeomorphism for all $x \in \mc{X}$.

There are several equivalent ways to define Sobolev spaces on smooth Riemannian manifolds. We will stick with a definition that parallels our setup in the flat Euclidean setting as much as possible. To do so, we first recall the notion of partial derivatives on a manifold, which are defined with respect to a local coordinate system. Letting $r_1,\ldots,r_m$ be the standard basis of $\Reals^m$, for a given chart $(\phi,U)$ (meaning an open set $U \subseteq \mc{X}$, and a smooth mapping $\phi: U \to \Reals^m$) we write $\phi =: (x_1,\ldots,x_m)$ in local coordinates, meaning $x_i = r_i \circ \phi$. Then the partial derivative $\partial f/\partial x_i$ of a function $f$ with respect to $x_i$ at $x \in U$ is
\begin{equation*}
\frac{\partial f}{\partial x_i}(x) := \frac{\partial(f \circ \phi^{-1})}{\partial r_i}\bigl(\phi(x)\bigr).
\end{equation*}
The right hand side should be interpreted in the weak sense of derivative. As before, we use the multi-index notation $D^{\alpha}f := \partial^{|\alpha|}f/\partial^{\alpha_1}x_1\ldots\partial^{\alpha_m}x_m$. 

\begin{definition}[Sobolev space on a manifold]
	\label{def:sobolev_space_manifold}
	A function $f \in \Leb^2(\mc{X})$ belongs to the Sobolev space $H^{s}(\mc{X})$ if for all $\abs{\alpha} \leq s$, the weak derivatives $D^{\alpha}f$ exist and satisfy  $D^{\alpha}f \in \Leb^2(\mc{X})$. The $j$th order semi-norm $|f|_{H^j(\mc{X})}$, the norm $\|f\|_{H^s(\mc{X})}$, and the ball $H^s(\mc{X};M)$ are all defined as in Definition~\ref{def:sobolev_space}.
\end{definition}
The partial derivatives $D^{\alpha}f$ will depend on the choice of local coordinates, and so will the resulting Sobolev norm~$\|f\|_{H^s(\mc{X})}$. However, for our purposes the important point is that regardless of the choice of local coordinates the resulting norms will be equivalent\footnote{Recall that norms $\|\cdot\|_1$ and $\|\cdot\|_2$ on a space $\mc{F}$ are said to be equivalent if there exist constants $c$ and $C$ such that
	\begin{equation*}
	c \|f\|_1 \leq \|f\|_2 \leq C \|f\|_1 \quad \textrm{for all $f \in \mc{F}$.}
	\end{equation*}} 
and so the ultimate Sobolev space $H^s(\mc{X})$ is independent of the choice of local coordinates. For more information regarding manifolds and Sobolev spaces defined thereupon, see~\cite{lee2013} and~\cite{hebey1996}.

\subsection{Laplacian eigenmaps}
\label{subsec:laplacian_eigenmaps}
We now formally define the estimator and test statistic we study. Both are derived from eigenvectors of a graph Laplacian.  For a positive, symmetric kernel $\eta: [0,\infty) \to [0,\infty)$, and a radius parameter $\varepsilon > 0$, let $G = ([n],W)$ be the neighborhood graph formed over the design points $\{X_1,\ldots,X_n\}$, with a weighted edge $W_{ij} = \eta(\|X_i - X_j\|/\varepsilon)$ between vertices $i$ and $j$. Then the 
\emph{neighborhood graph Laplacian} $L_{n,\varepsilon}: \Reals^n \to \Reals$ is defined by its action on vectors $u \in \Reals^n$ as
\begin{equation}
\label{eqn:neighborhood_graph_laplacian}
\bigl(L_{n,\varepsilon}u\bigr)_i := \frac{1}{n\varepsilon^{2 + \mathrm{dim}(\mc{X})}} \sum_{j = 1}^{n} \bigl(u_i - u_j\bigr) \eta\biggl(\frac{\|X_i - X_j\|}{\varepsilon}\biggr).
\end{equation}
(Here $\mathrm{dim}(\mc{X})$ stands for the dimension of $\mc{X}$. It is equal to $d$ under the assumptions of Model~\ref{def:model_flat_euclidean}, and equal to $m$ under the assumptions of Model~\ref{def:model_manifold}. The pre-factor $(n\varepsilon^{2 + \mathrm{dim}(\mc{X})})^{-1}$ ensures non-degenerate stable limits as $n \to \infty, \varepsilon \to 0$). Written in standard coordinates we have $(n\varepsilon^{\dim(\mc{X}) + 2}) \cdot L_{n,\varepsilon} = D - W$, where $D \in \Reals^{n \times n}$ is the diagonal degree matrix, $D_{ii} = \sum_{i = 1}^{n} W_{ij}$.

The graph Laplacian is a positive semi-definite matrix, and admits the eigendecomposition $L_{n,\varepsilon} = \sum_{k = 1}^{n} \lambda_k v_k v_k^{\top}$, where for each $k = 1,\ldots,n$ the eigenvalue-eigenvector pair $(\lambda_k,v_k)$ satisfies
\begin{equation*}
L_{n,\varepsilon}v_k = \lambda_k v_k, \quad \|v_k\|_2^2 = 1.
\end{equation*}
We will assume without loss of generality that each eigenvalue $\lambda$ of $L_{n,\varepsilon}$ has algebraic multiplicity $1$, and so we can index the eigenpairs $(\lambda_1,v_1),\ldots,(\lambda_n,v_n)$ in ascending order of eigenvalue, $0 = \lambda_1 < \ldots < \lambda_n$. 

The Laplacian eigenmaps estimator $\wh{f}$ simply projects the response vector ${\bf Y}$ onto the first $K$ eigenvectors of $L_{n,\varepsilon}$: letting $V_K \in \Reals^{n \times K}$ be the matrix with columns $v_1,\ldots,v_K$, we have that
\begin{equation}
\label{eqn:laplacian_eigenmaps_estimator}
\wh{f} := \sum_{k = 1}^{K} \dotp{{\bf Y}}{v_k}_{2} v_k = V_K V_K^{\top} {\bf Y}.
\end{equation} 
If $\wh{f}$ is a reasonable estimate of $f_0$, then the Laplacian eigenmaps test statistic
\begin{equation}
\label{eqn:laplacian_eigenmaps_test}
\wh{T} := \|\wh{f}\|_n^2 = \frac{1}{n} {\bf Y}^{\top} V_K V_K^{\top} {\bf Y}
\end{equation}
is in turn a reasonable estimate of $\|f_0\|_{P}^2$, and can be used in the \emph{signal detection} problem to distinguish whether or not $f_0 = 0$.

% AG 8/12/21: Commented out due to overlap with footnote in new introduction.
% It may be helpful to comment briefly on the term ``Laplacian eigenmaps'', which we use a bit differently than is typical in the literature. Laplacian eigenmaps typically refers to an algorithm for embedding, which maps each design point $X_1,\ldots,X_n$ to $\Reals^K$ according to $X_i \mapsto (v_{1,i}, \ldots, v_{K,i})$. Viewing this embedding as a feature map, we can then interpret the estimator $\wh{f}$ as the least-squares solution to a linear regression problem with responses $Y_1,\ldots,Y_n$ and features $v_1,\ldots,v_K$. Often, the Laplacian eigenmaps embedding is viewed as a tool for dimensionality reduction, wherein it is implicitly assumed that $K$ is much smaller than $d$. We will neither explicitly nor implicitly take $K < d$; indeed, the embedding perspective is not particularly illuminating in what follows, and we do not henceforth make reference to it. Instead, we use ``Laplacian eigenmaps'' to directly refer to the estimator $\wh{f}$ or test statistic $\wh{T}$. 

\subsection{Classical Spectral Projection over Sobolev spaces}
\label{subsec:spectral_projection}
Laplacian Eigenmaps serves as a data-dependent alternative to more classical spectral projection methods constructed out of the eigenfunctions of a Laplace-Beltrami operator (or, in the special case where $\mc{X} = [0,1]^d$, out of a Fourier basis).
We focus on the following density weighted Laplace-Beltrami operator $\Delta_P:C^2(\mc{X}) \to L^2(\mc{X})$,
\begin{equation}
\label{eqn:laplace_beltrami}
\Delta_Pf := -\frac{1}{p} \mathrm{div}(p^2\nabla f),
\end{equation}
with eigenvalue/eigenfunction pairs $(\lambda_1(\Delta_P),\psi_1),(\lambda_2(\Delta_P),\psi_2),\ldots$ defined as solutions to
\begin{equation}
\label{eqn:laplace_beltrami_eigenproblem}
\Delta_P\psi = \lambda(\Delta_P) \psi, \quad \frac{\partial}{\partial{\bf n}}\psi = 0~~\textrm{on $\partial \mc{X}$,}
\end{equation}
and sorted as usual in ascending order of eigenvalue.\footnote{For formal justification of why~\eqref{eqn:laplace_beltrami_eigenproblem} has discrete spectrum under either Model~\ref{def:model_flat_euclidean} or~\ref{def:model_manifold}, see~\cite{garciatrillos18,trillos2019}.} In this case, classical spectral projection refers to the estimator and test statistic
\begin{equation}
\label{eqn:classical_spectral_projection}
\wt{f}(x) := \frac{1}{n}\sum_{k = 1}^{K} \dotp{{\bf Y}}{\psi_k}_{n} \psi_k(x),\quad\textrm{and}\quad \wt{T} := \|\wt{f}\|_n^2.
\end{equation}
The statistical behavior of these more classical methods is relevant because the spectrum of $\Delta_P$ serves as the population-level limit of graph Laplacian spectra: for any fixed $k \in \mathrm{N}$, as $n \to \infty$
\begin{equation}
\lambda_k \to \lambda_k(\Delta_P), \quad\textrm{and}\quad \Bigl\|\frac{1}{n}v_k - \psi_k\Bigr\|_n^2 \to 0,
\end{equation}
in either the flat Euclidean (Model~\ref{def:model_flat_euclidean}) or manifold (Model~\ref{def:model_manifold}) setups~\citep{garciatrillos18,trillos2019}. It follows that for any fixed $K \in \mathbb{N}$, the Laplacian eigenmaps estimator converges to the $\wh{f} \to \wt{f}$, and likewise $\wh{T} \to \wt{T}$, justifying why we view Laplacian Eigenmaps as a data-dependent alternative to classical spectral projection.
 
On the other hand, the eigenvectors and eigenfunctions of a (possibly density-weighted) Laplace-Beltrami operator such as $\Delta_P$ can also be used to give a spectral definition of Sobolev spaces. For instance, consider the ellipsoid
\begin{equation}
\label{eqn:sobolev_ellipsoid}
\mc{H}^{s}(\mc{X}) := \Bigl\{\sum_{k = 1}^{\infty} a_k \psi_k \in L^2(\mc{X}):  \sum_{k = 1}^{\infty} a_k^2 \lambda_{k}(\Delta_P) \leq M^2 \Bigr\},
\end{equation}
with corresponding norm $\|\sum_{k = 1}^{\infty} a_k \psi_k\|_{\mc{H}^s(\mc{X})}^2 = \sum_{k = 1}^{\infty} a_k^2 \lambda_{k}(\Delta_P)$. Under appropriate regularity conditions $\mc{H}^s(\mc{X})$ consists of functions $f \in H^s(\mc{X})$ which also satisfy some additional boundary conditions. For instance, assuming Model~\ref{def:model_flat_euclidean}, $p \in C^{\infty}(\mc{X})$ and $\partial \mc{X} \in C^{1,1}$, \citet{dunlop2020} show that the ellipsoid $\mc{H}^{2s}(\mc{X})$ satisfies
\begin{equation}
\label{eqn:sobolev_ellipsoid_to_sobolev_ball}
\mc{H}^{2s}(\mc{X}) = 
\biggl\{f \in H^{2s}(\mc{X}): \frac{\partial \Delta_P^rf}{\partial {\bf n}} = 0~\textrm{on}~\partial\mc{X},~~\textrm{for all $0 \leq r \leq s - 1$} \biggr\},
\end{equation}
for any $s > 0$, and likewise $\mc{H}^{2s + 1}(\mc{X}) = \mc{H}^{2s}(\mc{X}) \cap H^{2s + 1}(\mc{X})$ for any $s \geq 0$; additionally, the norms $\|\cdot\|_{\mc{H}^s(\mc{X})}$ and $\|\cdot\|_{H^s(\mc{X})}$ are equivalent.

The spectral formulation of Sobolev spaces suggest that classical spectral projection methods---and in turn Laplacian Eigenmaps---are natural choices for regression over such function classes. Further, given the similarities between Laplacian Eigenmaps and classical spectral projection methods, the statistical properties of the latter---which are better understood and easier to derive---should intuitively shed some light on the properties of the former. To this end, we now give a pair of results showing that classical spectral methods are statistically optimal for nonparametric regression over Sobolev classes with random design, \emph{so long as the design distribution is known}. These results are in line with previously known upper bounds on the estimation and testing error of classical spectral methods~\citep{tsybakov08,ingster2009}, but hold under less stringent conditions on the design distribution. This latter point is important, since our interest in Laplacian Eigenmaps stems in part from its adaptivity to general and unknown designs.

\paragraph{Estimation.}
We begin with an upper bound on the $L^2(P)$ risk of $\wt{f}$.
\begin{proposition}
	\label{prop:spectral_series_estimation}
	Suppose data is observed according to Model~\ref{def:model_flat_euclidean}, and additionally that $\partial \mc{X} \in C^{1,1}$, $p \in C^{\infty}(\mc{X})$, $f_0 \in \mc{H}^{s}(\mc{X};M)$ and $\|f_0\|_P^2 \leq 1$. Then there exists a constant $C$ which does not depend on $f_0,M$ or $n$ such that the following statement holds: if the spectral projection estimator $\wt{f}$ is computed with parameter $K = \floor{M^2n}^{d/(2s + d)} \vee 1$, then
	\begin{equation}
	\label{eqn:spectral_series_estimation}
	\Ebb\bigl[\|\wt{f} - f_0\|_P^2\bigr] \leq C \min\bigl\{M^2(M^2n)^{-2s/(2s + d)}, M^2\bigr\}.
	\end{equation}
\end{proposition}
When the Sobolev ball radius $M \asymp 1$, the upper bound in~\eqref{eqn:spectral_series_estimation} is on the order of $n^{-2s/(2s + d)}$, which matches the standard minimax estimation rate for Sobolev classes (see~\cite{wasserman2006,tsybakov08} and references therein). 

% (AG 8/19/21): I would like to strengthen this statement to not require $M \asymp 1$, but I do not know where I can find the rate M^2(M^2n)^{-2s/(2s + d)} for all s and d combinations. In fact, I don't even know of a paper that states the M^2(M^2n)^{-2s/(2s + d)} rate for L^2(P) risk in random design.

We now give the proof of Proposition~\ref{prop:spectral_series_estimation}; the structure of the analysis, which is fairly classical and straightforward, is similar to the high-level strategy we use to analyze Laplacian Eigenmaps (see Section~\ref{subsec:analysis}).

\paragraph{Proof of Proposition~\ref{prop:spectral_series_estimation}.}
We decompose risk into squared bias and variance,
\begin{equation}
\label{pf:spectral_series_estimation_0}
\Ebb \|\wt{f} - f_0\|_P^2 = \Ebb\| \Ebb[\wt{f}]  - f_0\|_P^2 + \Ebb\| \wt{f} - \Ebb[\wt{f}]\|_P^2.
\end{equation}
Since the eigenfunctions $\{\psi_k\}$ form an orthonormal basis of $L^2(P)$, and $f_0 \in H_0^{s}(\mc{X}) \subseteq L^2(P)$, we can write the squared bias in terms of squared Fourier coefficients of $f_0$, leading to the following upper bound,
\begin{equation*}
\|f_0 - \Ebb{\wt{f}}\|_P^2 = \sum_{k = K + 1}^{\infty}  \dotp{f_0}{\psi_k}^2 \leq  \frac{1}{\lambda_{K + 1}(\Delta_P)^s} \sum_{k = K + 1}^{\infty} \lambda_{k + 1}(\Delta_P)^s \dotp{f_0}{\psi_k}^2 \leq \frac{\|f_0\|_{\mc{H}^s(\mc{X})}}{\bigl[\lambda_{K + 1}(\Delta_P)\bigr]^s}.
\end{equation*}
On the other hand, the variance term can be written as the sum of the variance of each empirical Fourier coefficient, and subsequently using the law of total variance, we have
\begin{align}
\label{pf:spectral_series_estimation_2}
\Ebb\| \wt{f} - \Ebb[\wt{f}]\|_P^2 = \sum_{k = 1}^{K} \Var\Bigl[\dotp{{\bf Y}}{\psi_k}_n\Bigr] & = \sum_{k = 1}^{K} \Var\Bigl[\Ebb[\dotp{Y}{\psi_k}_n|{\bf X}\Bigr] + \Ebb\Bigl[\Var[\dotp{Y}{\psi_k}_n|{\bf X}\Bigr] \nonumber \\
& = \sum_{k = 1}^{K} \Var\Bigl[\dotp{f_0}{\psi_k}_n\Bigr] + \frac{1}{n}\Ebb\Bigl[\|\psi_k\|_n^2\Bigr] \nonumber \\
& \leq \frac{K}{n} + \sum_{k = 1}^{K}\Ebb\Bigl[\Bigl(f_0(X)\psi_k(X)\Bigr)^2\Bigr];
\end{align}
consequently,
\begin{equation}
\label{pf:spectral_series_estimation_1}
\Ebb \|\wt{f} - f_0\|_P^2 \leq \frac{\|f_0\|_{\mc{H}^s(\mc{X})}^2}{\bigl[\lambda_{K + 1}(\Delta_P)\bigr]^s} + \frac{K}{n} + \frac{1}{n}\Ebb\Bigl[(f_0(X))^2 \cdot \sum_{k = 1}^{K} (\psi_k(X))^2\Bigr].
\end{equation}
The claim of the proposition then follows from two key facts. The first is a Weyl's Law scaling of the eigenvalues of $\Delta_P$, formally
\begin{equation}
\label{eqn:weyl}
ck^{2/d} \leq \lambda_k(\Delta_P) \leq Ck^{2/d}\quad\textrm{for all $k \in \mathbb{N}\setminus \{1\}$},
\end{equation}
which is due to~\cite{dunlop2020}; the second is a local analog to Weyl's Law,
\begin{equation}
\label{eqn:local_weyl}
\sup_{x \in \mc{X}}\biggl\{\sum_{k = 1}^{K} \bigl(\psi_k(x)\bigr)^2\biggr\} \leq CK \quad\textrm{for all $K \in \mathbb{N}$},
\end{equation}
which follows straightforwardly from Theorem 17.5.3 of~\cite{hormander1973}. Plugging the upper bounds~\eqref{eqn:weyl} and~\eqref{eqn:local_weyl} back into~\eqref{pf:spectral_series_estimation_1} and applying the assumed upper bound $\Ebb[(f_0(X))^2] \leq 1$ yields
\begin{equation*}
\Ebb \|\wt{f} - f_0\|_P^2 \leq C\biggl(\frac{\|f_0\|_{\mc{H}^s(\mc{X})}^2}{(k + 1)^s} + \frac{K}{n}\biggr),
\end{equation*}
and the choice $K = \floor{M^2n}^{d/(2s + d)} \vee 1$ then yields the claim.
\qed.

\paragraph{Testing.}
In the goodness-of-fit testing problem, one asks for a test function---formally, a Borel measurable function $\phi$ that takes values in $\{0,1\}$--- which can distinguish between the hypotheses
\begin{equation}
\mathbf{H}_0: f_0 = f_0^{\star}, ~~\textrm{versus}~~ \mathbf{H}_a: f_0 \in \mc{H}^{s}(\mc{X};M) \setminus \{f_0^{\star}\}.
\end{equation} 
To fix ideas, here and throughout we focus on the signal detection problem, which is the special case of $f_0^{\star} = 0$.\footnote{This is without loss of generality since all the test statistics we consider are easily modified to handle the case when $f_0^{\ast}$ is not $0$, by simply subtracting $f_0^{\ast}(X_i)$ from each observation $Y_i$, with no change in the analysis.}

In this case, the classical spectral projection test $\wt{\varphi} = \1\{\wt{T} \geq \textcolor{red}{(?)}\}$ has bounded Type I error, $\Ebb_{0}[\wt{\varphi}] \leq a$. Proposition~\ref{prop:spectral_series_testing} gives an upper bound on the Type II error that holds uniformly over all $f_0 \in \mc{H}^s(\mc{X};M)$ which have a sufficiently large $L^2(P)$ norm.
\begin{proposition}
	\label{prop:spectral_series_testing}
	Suppose data is observed according to Model~\ref{def:model_flat_euclidean}, and additionally that $\partial \mc{X} \in C^{1,1}$, $p \in C^{\infty}(\mc{X})$ and $f_0 \in \mc{H}^{s}(\mc{X};M)$. Then there exists a constant $C$ which does not depend on $f_0,M$ or $n$ such that the following statement holds: if the spectral projection test $\wt{\varphi}$ is computed with parameter $K = \floor{M^2n}^{2d/(4s + d)}$, and if
	\begin{equation}
	\label{eqn:spectral_series_testing}
	\|f_0\|_P^2 \geq \textcolor{red}{(?)}
	\end{equation}
	then the Type I error is bounded, $\Ebb_{f_0}[1 - \phi] \leq b$.
\end{proposition}
Again, when the Sobolev ball radius $M \asymp 1$, the upper bound in~\eqref{eqn:spectral_series_testing} becomes $n^{-4s/(4s + d)}$, matching the usual minimax critical radius over Sobolev spaces (for more details on goodness-of-fit testing see~\citep{ingster2009,ingster2012}).
% (AG 8/19/21): Same comment as above.

The main takeaway from Propositions~\ref{prop:spectral_series_estimation} and~\ref{prop:spectral_series_testing} is that spectral projection methods achieve optimal rates over Sobolev classes, when $p$ satisfies only a nonparametric notion of smoothness but is known.\footnote{The assumption $p \in C^{\infty}(\mc{X})$ could likely be weakened. Alternatively, one could likely derive upper bounds assuming Model~\ref{def:model_manifold} which depend only on the intrinsic dimension $m$. Since these would not substantially add to the main points of Propositions~\ref{prop:spectral_series_estimation} and~\ref{prop:spectral_series_testing}, we do not pursue the details further.} As we will shortly see, Laplacian eigenmaps achieves similar rates of convergence when $p$ is unknown, and in this sense both learns and leverages the design distribution in a way that more classical spectral projection methods cannot. It is worth pointing out that other methods besides Laplacian eigenmaps are statistically optimal for regression over Sobolev spaces even when the design distribution is unknown. We comment more on some of these in Section~\ref{sec:discussion}, after we have derived our major results regarding Laplacian Eigenmaps.

\paragraph{In-sample mean squared error.}
As mentioned in our introduction, roughly speaking one of our main conclusions is that the Laplacian eigenmaps estimator $\wh{f}$ is minimax rate-optimal. It is worth being clear about what we do and do not mean by this statement. We do not mean that the estimator $\wh{f}$ will match the upper bound given in~\eqref{eqn:sobolev_space_minimax_estimation_rate}, since such a statement does not make sense when the estimator is defined only at the random design points $X_1,\ldots,X_n$. Instead we will measure loss using the squared $L^2(P_n)$ error. In Section~\ref{sec:out_of_sample} we show that an extension of $\wh{f}$ defined over all $\mc{X}$ has $L^2(P)$ error comparable to the $L^2(P_n)$ error of $\wh{f}$. We also believe that in the random design setting we work in, simple arguments will imply that $L^2(P_n)$ risk has the same minimax rate of convergence as $L^2(P)$ risk. We sketch such an argument in Section~\ref{sec:out_of_sample}, but do not further pursue the details. 

% (SB via AG): Move this somewhere later.
Additionally, we will not actually measure accuracy using the expectation of the loss. Rather, we will give a constant probability bound on $\|\cdot\|_n^2$. For instance, when $f_0 \in H^1(\mc{X};1)$, we will show that with probability $1 - \delta$ the loss $\|\wh{f} - f_0\|_n^2 \leq C_{\delta} n^{-2/(2 + d)}$, for a constant $C_{\delta}$ that depends on $\delta$ but not on $f_0$ or $n$. Thus we give an upper bound on the $(1 - \delta)$th quantile of $\|\wh{f} - f_0\|_n^2$, rather than an upper bound on its expectation. We explain the reason for this in Section~\ref{sec:minimax_optimal_laplacian_eigenmaps}. We also show that if $f_0$ is bounded in a larger norm---for instance, if it is H\"{o}lder rather than Sobolev smooth---then we can obtain bounds on the expected $L^2(P_n)$ loss.

There is one other subtlety introduced by the use of in-sample mean squared error. Technically speaking, elements $f \in H^s(\mc{X})$ are equivalence classes, defined only up to a set of measure zero. Thus one cannot speak of the pointwise evaluation $f_0(X_i)$, as we do by defining our target of estimation to be $f_0(X_i)$, $i=1,\ldots,n$, until one selects \emph{representatives}. When $s > d/2$, every element $f$ of $H^s(\mc{X})$ admits a continuous version $f^{\ast}$, and as is standard we set this to be our favored representative. When $s \leq d/2$, some elements in $H^s(\mc{X})$ do not have any continuous version; however they admit a \emph{quasi-continuous} version \citep{evans15} known as the \emph{precise representative}, and we use this representative. To be clear, however, it does not really matter which representative we choose. Since all versions agree except on a set of measure zero, and since $P$ is absolutely continuous with respect to Lebesgue measure (in Model~\ref{def:model_flat_euclidean}) or the volume form $d\mu$ (in Model~\ref{def:model_manifold}), with probability $1$ any two versions $g_0, h_0 \in f_0$ will satisfy $g_0(X_i) = h_0(X_i)$ for all $i = 1,\ldots,n$. The bottom line is that we can use the notation $f_0(X_i)$ without fear of ambiguity or confusion.

Finally, we note that for testing none of these comments are relevant. We will show that our test has small worst-case risk whenever $\epsilon \gtrsim \epsilon_n(H_0^s(\mc{X};M))$, thus establishing that it is a minimax optimal test in the usual sense.


\section{Minimax Optimality of Laplacian Eigenmaps}
\label{sec:minimax_optimal_laplacian_eigenmaps}

% As previously explained, Laplacian eigenmaps is a discrete and noisy approximation to a spectral projection method using the eigenfunctions of $\Delta_P$. This is particularly useful when $P$ is unknown, or when the eigenfunctions of $\Delta_P$ cannot be explicitly computed. Our goal is to show that Laplacian eigenmaps methods are rate-optimal, notwithstanding the potential extra error incurred by this approximation. In this section and the following one, we will see that this is indeed the case: the estimator $\wh{f}$, and a test using the statistic $\wh{T}$, achieve optimal estimation and goodness-of-fit testing rates over Sobolev classes.
% (AG 8/26/21): This segment is now redundant. 

In this section we give upper bounds on the error of Laplacian Eigenmaps in the flat Euclidean case, where we observe data $(X_1,Y_1),\ldots,(X_n,Y_n)$ according to Model~\ref{def:model_flat_euclidean}. We will divide our theorem statements based on the regression function $f_0$ belongs to the first order Sobolev class $H^1(\mc{X})$ or a higher-order Sobolev class ($H_0^{s}(\mc{X})$ for some $s > 1$), since the details of the two settings are somewhat different.

\subsection{First-order Sobolev classes}
\label{sec:first_order_sobolev_classes}
We begin by assuming $f_0 \in H^1(\mc{X}; M)$. We show that $\wh{f}$ and a test based on $\wh{T}$ are minimax optimal, for all values of $d$ for which the minimax rates are known, and under no additional assumptions (beyond those of Model~\ref{def:model_flat_euclidean}) on the design distribution $P$.

\paragraph{Estimation.} Laplacian eigenmaps depends on the kernel $\eta$ and two tuning parameters, the graph radius $\varepsilon$ and number of eigenvectors $K$. We now give some assumptions on each.
\begin{enumerate}[label=(K\arabic*)]
	\setcounter{enumi}{0}
	\item
	\label{asmp:kernel_flat_euclidean}
	The kernel function $\eta$ is a nonincreasing function supported on $[0,1]$. Its restriction to $[0,1]$ is Lipschitz, and $\eta(1) > 0$. Additionally, it is normalized so that
	\begin{equation*}
	\int_{\Rd} \eta(\|z\|) \,dz = 1.
	\end{equation*}
	and we assume \smash{$\sigma_{\eta} := \frac{1}{d}\int_{\Rd} \|x\|^2 \eta(\|x\|) \,dx < \infty$}.
\end{enumerate}
\begin{enumerate}[label=(P\arabic*)]
	\setcounter{enumi}{0}
	\item 
	\label{asmp:parameters_estimation_fo} 
	For constants $c_0$ and $C_0$, the graph radius $\varepsilon$ and the number of eigenvectors $K$ satisfy the following inequalities:
	\begin{equation}\\
	\label{eqn:radius_fo} 
	C_0\biggl(\frac{\log n}{n}\biggr)^{1/d} \leq \varepsilon \leq c_0\min\{1,K^{-1/d}\},
	\end{equation}
	and 
	\begin{equation}
	\label{eqn:eigenvector_estimation_fo} 
	K = \min\Bigl\{\floor{(M^2n)^{d/(2 + d)}} \vee 1, n\Bigr\}.
	\end{equation}
\end{enumerate}
We comment on these assumptions after stating our first main theorem, regarding the estimation error of Laplacian eigenmaps.
\begin{theorem}
	\label{thm:laplacian_eigenmaps_estimation_fo}
	Suppose Model~\ref{def:model_flat_euclidean}, and additionally $f_0 \in H^1(\mc{X},M)$. There are constants $c,C$ and $N$ (not depending on $f_0$, $M$ or $n$), such that the following statement holds for all $n \geq N$ and any $\delta \in (0,1)$: if the Laplacian eigenmaps estimator $\wh{f}$ is computed with a kernel $\eta$ satisfying~\ref{asmp:kernel_flat_euclidean}, and parameters $\varepsilon$ and $K$ satisfying~\ref{asmp:parameters_estimation_fo}, then
	\begin{equation}
	\label{eqn:laplacian_eigenmaps_estimation_fo}
	\|\wh{f} - f_0\|_n^2 \leq C\Bigl(\frac{1}{\delta}M^2(M^2n)^{-2/(2 + d)} \wedge 1\Bigr) \vee \frac{1}{n},
	\end{equation}
	with probability at least $1 - \delta - Cn\exp(-cn\varepsilon^d) - \exp(-K)$.
\end{theorem}
From~\eqref{eqn:laplacian_eigenmaps_estimation_fo} it follows immediately that when $M \asymp 1$, then with constant probability $\|\wh{f} - f_0\|_n^2 \lesssim n^{-2/(2 + d)}$, matching the standard minimax estimation rate over Sobolev classes.

Some other remarks:
\begin{itemize}
	\item \emph{Radius of the Sobolev ball.} If $M = M_n$ satisfies $n^{-1/2} \lesssim M_n \lesssim n^{1/d}$, then Theorem~\ref{thm:laplacian_eigenmaps_estimation_fo} implies that $\|\wh{f} - f_0\|_n^2 \lesssim M^2(M^2n)^{-2/(2 + d)}$ with constant probability. This matches the upper bound on $\Ebb\|\wt{f} - f_0\|_P^2$ given in~\eqref{eqn:spectral_series_estimation}, including in its dependence on $M$. It also matches the known minimax rate for some related problems, including estimation in the Gaussian sequence model~\citep{johnstone2011}, and estimation with a fixed design when the dimension $d = 1$~\citep{vandergeer2000}. 
	
	When $M = o(n^{-1/2})$ then computing Laplacian eigenmaps with $K = 1$ achieves the parametric rate $\|\wh{f} - f_0\|_n^2 \lesssim n^{-1}$, and the zero-estimator $\wh{f} = 0$ achieves the better rate $\|\wh{f} - f_0\|_n^2 \lesssim M^2$. However, we do not know what the minimax rate is in this regime. On the other hand, when $M = \omega(n^{1/d})$, then computing Laplacian eigenmaps with $K = n$ achieves the rate $\|\wh{f} - f_0\|_n^2 \lesssim 1$, which is better than the rate in~\eqref{eqn:spectral_series_estimation}. This is because we are evaluating error in-sample rather than out-of-sample. However, in truth these are edge cases, which do not fall neatly into the framework of nonparametric regression. 
	
	\item \emph{In-sample error.} As we've already noted, since the Laplacian eigenmaps estimator is defined only at the design points $\{X_1,\ldots,X_n\}$, we will use the empirical norm $\|\cdot\|_n^2$ as our estimation loss, and return to the question of out-of-sample estimation later in Section~\ref{sec:out_of_sample}. 
	
	There is one subtlety introduced by the use of in-sample mean squared error. Since elements $f \in H^s(\mc{X})$ are equivalence classes, defined only up to a set of measure zero, one cannot really speak of the pointwise evaluation $f_0(X_i)$, as we do by defining our target of estimation to be $(f_0(X_1),\ldots,f_0(X_n))$, until one selects a representative of each equivalence class $f$. Implicitly, we will always pick the \emph{precise representative} $f_0^{\ast} \in f_0$ (as defined in~\cite{evans15}), and the notation ``$f_0(X_i)$'' should always be interpreted as $f_0^{\ast}(X_i)$. To be clear, however, it does not really matter which representative we choose, since all versions agree except on a set of measure zero, and so any two $g_0,h_0 \in f_0$ satisfy $g_0(X_i) = h_0(X_i)$ for all $i = 1,\ldots,n$ almost surely. For this reason we can write $f_0(X_i)$ without fear of ambiguity or confusion. 
	
	\item \emph{Tuning parameters}. The assumptions placed on the kernel function $\eta$ are needed for technical reasons. They can likely be weakened, although we note that they are already fairly general. The lower bound on $\varepsilon$ imposed by~\eqref{eqn:radius_fo} is on the order of the connectivity threshold, the smallest radius for which the resulting graph will still be connected with high probability. On the other hand, as we will see in Section~\ref{subsec:analysis}, the upper bound on $\varepsilon$ is needed to ensure that the graph eigenvalue $\lambda_K$ is of at least the same order as the continuum eigenvalue $\lambda_K(\Delta_P)$; this is essential in order to obtain a tight upper bound on the bias of $\wh{f}$.  Finally, we set $K = \floor{(M^2n)^{d/(2 + d)}}$ (when possible) to optimally trade-off bias and variance.
	
	In practice, one typically tunes hyper-parameters by cross-validation. However, because the estimator $\wh{f}$ is defined only in-sample, neither cross-validation nor any other sample-splitting method can be used to tune parameters for Laplacian eigenmaps. We return to this issue in Section~\ref{sec:out_of_sample}, when we propose an out-of-sample extension of $\wh{f}$. 
	
	% (AG 8/27/21): Ryan, you asked about AIC, but I'm not clear about whether you want me to (a) make some remark about it, (b) actually go through the work of giving guarantees, or (c) were just curious.
	\item \emph{High-probability guarantees}. The upper bound given in~\eqref{eqn:laplacian_eigenmaps_estimation_fo} holds with ``constant probability'', meaning with probability $1 - \delta - o(1)$. Under the stronger assumption that $f_0$ is $M$-Lipschitz, we can establish the same guarantee~\eqref{eqn:laplacian_eigenmaps_estimation_fo} with probability $1 - \delta^2/n - Cn\exp(-cn\varepsilon^d) - \exp(-K)$; in other words, we can give a high probability guarantee (for details see~\citep{green2021}). In this case a routine calculation shows that $\Ebb[\|\wh{f} - f_0\|_n^2]$ will also be on the some order as~\eqref{eqn:laplacian_eigenmaps_estimation_fo}. We also suspect that high-probability guarantees will hold so long as $\|\nabla f\|_{L^q(\mc{X})}$ is bounded for some sufficiently large $q < \infty$, but it remains an open question whether such guarantees can be obtained in the Sobolev case ($q = 2$) which is the focus of this work. 
\end{itemize}

\paragraph{Testing.} Consider the test $\varphi = \1\{\wh{T} \geq t_{a}\}$, where $t_{a}$ is the threshold
\begin{equation*}
t_{a} := \frac{K}{n} + \frac{1}{n}\sqrt{\frac{2K}{a}}.
\end{equation*}
This choice of threshold $t_{a}$ guarantees that $\varphi$ is a level-$a$ test. As we show in Theorem~\ref{thm:laplacian_eigenmaps_testing_fo}, when $d < 4$, $\varepsilon$ and $K$ are chosen appropriately, and the alternative $f_0$ has is sufficiently well-separated from $0$, the test $\varphi$ has Type II error of at most $b$.

\begin{enumerate}[label=(P\arabic*)]
	\setcounter{enumi}{1}
	\item 
	\label{asmp:parameters_testing_fo}
	The graph radius $\varepsilon$ satisfies~\eqref{eqn:radius_fo}, and the number of eigenvectors 
	\begin{equation}
	\label{eqn:eigenvector_testing_fo}
	K = \min\Bigl\{\floor{(M^2n)^{2d/(4 + d)}} \vee 1, n\Bigr\}.
	\end{equation}
\end{enumerate}
\begin{theorem}
	\label{thm:laplacian_eigenmaps_testing_fo}
	Fix $a,b \in (0,1)$. Suppose Model~\ref{def:model_flat_euclidean}. Then $\mathbb{E}_0[\varphi] \leq a$, i.e $\varphi$ is a level-$a$ test. Suppose additionally $f_0 \in H^1(\mc{X},M)$, and that $d < 4$. Then there exist constants $C$ and $N$ that do not depend on $f_0$, such that the following statement holds for all $n$ larger than $N$: if the Laplacian eigenmaps test $\varphi$ is computed with a kernel $\eta$ satisfying~\ref{asmp:kernel_flat_euclidean}, and parameters $\varepsilon$ and $K$ satisfying~\ref{asmp:parameters_testing_fo}, and if $f_0$ satisfies
	\begin{equation}
	\label{eqn:laplacian_eigenmaps_testing_criticalradius_fo}
	\|f_0\|_P^2 \geq C\biggl(\Bigl(M^2(M^2n)^{-4/(4 + d) } \wedge n^{-1/2}\Bigr)\biggl[\sqrt{\frac{1}{a}} + \frac{1}{b}\biggr] \vee \frac{M^2}{b n^{2/d}} \biggr) \vee \frac{1}{n},
	\end{equation}
	then $\Ebb_{f_0}[1 - \phi] \leq b$.
\end{theorem}
Although~\eqref{eqn:laplacian_eigenmaps_testing_criticalradius_fo} involves taking the maximum of several different terms, the important takeaway of Theorem~\ref{thm:laplacian_eigenmaps_testing_fo} is that if $n^{-1/2} \lesssim M \lesssim n^{(4 - d)/4d}$, then $\varphi$ has small worst-case risk as long as $f_0$ is separated from $0$ by at least $M^2(M^2n)^{-4/(4 + d)}$. In the special case $M \asymp 1$, this matches the minimax squared critical radius $n^{-4/(4 + d)}$, implying that $\varphi$ is a minimax rate-optimal test over $H^1(\mc{X};M)$ when $d \in \{1,2,3\}$. As mentioned previously, when $d \geq 4$ the first order Sobolev space $H^1(\mc{X})$ does not continuously embed into $\Leb^4(\mc{X})$, and in this case the optimal rates for regression testing over Sobolev spaces are unknown.

\subsection{Higher-order Sobolev classes}
\label{sec:higher_order_sobolev_classes}
We now consider the situation where the regression function displays some higher-order regularity, $f_0 \in H_0^s(\mc{X})$. We show that Laplacian eigenmaps methods continue to be optimal for all orders of $s$, as long as the design density is itself also sufficiently regular, $p \in C^{s - 1}(\mc{X})$. In estimation, this is the case for any dimension $d$, whereas in testing it is the case only when $d \leq 4$. 

\paragraph{Estimation.}
In order to show that $\wh{f}$ is an optimal estimator over $H_0^s(\mc{X};M)$, we will require that $\varepsilon$ be meaningfully larger than the lower bound in~\ref{asmp:parameters_estimation_fo}.
\begin{enumerate}[label=(P\arabic*)]
	\setcounter{enumi}{2}
	\item 
	\label{asmp:parameters_estimation_ho}
	For constants $c_0$ and $C_0$, the graph radius $\varepsilon$ and number of eigenvectors $K$ satisfy
	\begin{equation}
	\label{eqn:radius_ho}
	C_0\max\biggl\{\biggl(\frac{\log}{n}\biggr)^{1/d}, (M^2n)^{-1/(2(s - 1) + d)}\biggr\} \leq \varepsilon \leq c_0\min\{1, K^{-1/d}\}
	\end{equation}
	and
	\begin{equation*}
	K = \min\Bigl\{\floor{(M^2n)^{d/(2s + d)}} \vee 1,n\Bigr\}
	\end{equation*}
\end{enumerate}
Crucially, when $n$ is sufficiently large the two conditions in~\ref{asmp:parameters_estimation_ho} are guaranteed to not be mutually exclusive. This is because so long as $M^2 = \omega(n^{-1})$ then $(M^2n)^{-2/(2(s - 1) + d)} = o((M^2n)^{-2/(2s + d)})$, regardless of $s$ and $d$.
\begin{theorem}
	\label{thm:laplacian_eigenmaps_estimation_ho}
	Suppose Model~\ref{def:model_flat_euclidean}, and additionally $f_0 \in H_0^s(\mc{X},M)$ and $p \in C^{s - 1}(\mc{X})$. There exist constants $c,C$ and $N$ that do not depend on $f_0$, such that the following statement holds all for all $n$ larger than $N$ and for any $\delta \in (0,1)$: if the Laplacian eigenmaps estimator $\wh{f}$ is computed with a kernel $\eta$ satisfying~\ref{asmp:kernel_flat_euclidean}, and parameters $\varepsilon$ and $K$ satisfying~\ref{asmp:parameters_estimation_ho}, then
	\begin{equation}
	\label{eqn:laplacian_eigenmaps_estimation_ho}
	\|\wh{f} - f_0\|_n^2 \leq C\Bigl(\frac{1}{\delta}M^2(M^2n)^{-2s/(2s + d)} \wedge 1\Bigr) \vee \frac{1}{n},
	\end{equation}
	with probability at least $1 - \delta - Cn\exp(-cn\varepsilon^d) - \exp(-K)$.
\end{theorem}
Theorem~\ref{thm:laplacian_eigenmaps_estimation_ho}, in combination with Theorem~\ref{thm:laplacian_eigenmaps_estimation_fo}, implies that in the flat Euclidean setting Laplacian eigenmaps is a minimax rate-optimal estimator over Sobolev classes, for all values of $s$ and $d$. Some other remarks:
\begin{itemize}
	\item \emph{Sub-critical Sobolev spaces}. We do not require that the regularity of the Sobolev space satisfy $s > d/2$, a condition often seen in the literature. In the sub-critical regime $s \leq d/2$, the Sobolev space $H^s(\mc{X})$ is quite irregular. It is not a Reproducing Kernel Hilbert Space (RKHS), nor does it continuously embed into $C^0(\mc{X})$, much less into any H\"{o}lder space. As a result, for certain versions of the nonparametric regression problem---e.g. when loss is measured in $\Leb^{\infty}$ norm, or when the design points $\{X_1,\ldots,X_n\}$ are assumed to be fixed---in a minimax sense even consistent estimation is not possible. Likewise, certain estimators are ``off the table'', most notably RKHS-based methods such as thin-plate splines of degree $k \leq d/2$. Nevertheless, for random design regression with error measured in $\Leb^2(P)$-norm, the spectral projection estimator $\wt{f}$ obtains the standard minimax rates $n^{-2s/(2s + d)}$ for all values of $s$ and $d$. Theorems~\ref{thm:laplacian_eigenmaps_estimation_fo} and \ref{thm:laplacian_eigenmaps_estimation_ho} show that the same is true with respect to Laplacian eigenmaps, with error measured in $\Leb^2(P_n)$-norm.
	\item \emph{Smoothness of design density}. As promised,  Theorem~\ref{thm:laplacian_eigenmaps_estimation_ho} shows that Laplacian eigenmaps achieves optimal rates of convergence so long as the unknown design density $p$ is sufficiently smooth, $p \in C^{s - 1}(\mc{X})$. The requirement $p \in C^{s - 1}(\mc{X})$ is essential to showing that $\wh{f}$ enjoys the faster minimax rates of convergence when $s > 1$,  as we discuss in Section~\ref{subsec:analysis}. 
\end{itemize}

\paragraph{Testing.} The test $\varphi$ can adapt to the higher-order smoothness of $f_0$, when $\varepsilon$ and $K$ are chosen correctly.
\begin{enumerate}[label=(P\arabic*)]
	\setcounter{enumi}{3}
	\item 
	\label{asmp:parameters_testing_ho}
	The graph radius $\varepsilon$ satisfies~\eqref{eqn:radius_ho}, and the number of eigenvectors
	\begin{equation}
	\label{eqn:eigenvector_testing_ho}
	K = \min\Bigl\{\floor{(M^2n)^{2d/(4s + d)}} \vee 1, n\Bigr\}.
	\end{equation}
\end{enumerate}
When $d \leq 4$, for any value of $s \in \mathbb{N}$ when $n$ is sufficiently large it is possible to choose $\varepsilon$ and $K$ such that both~\eqref{eqn:radius_ho} and~\eqref{eqn:eigenvector_testing_ho} are satisfied, and our next theorem establishes that in this situation $\varphi$ is an optimal test.
\begin{theorem}
	\label{thm:laplacian_eigenmaps_testing_ho}
	Fix $a,b \in (0,1)$. Suppose Model~\ref{def:model_flat_euclidean}. Then $\mathbb{E}_0[\varphi] \leq a$, i.e $\varphi$ is a level-$a$ test. Suppose additionally $f_0 \in H_0^s(\mc{X},M)$, that $p \in C^{s-1}(\mc{X})$, and that $d \leq 4$. Then there exist constants $c,C$ and $N$ that do not depend on $f_0$, such that the following statement holds for all $n \geq N$: if the Laplacian eigenmaps test $\varphi$ is computed with a kernel $\eta$ satisfying~\ref{asmp:kernel_flat_euclidean}, and parameters $\varepsilon$ and $K$ satisfying~\ref{asmp:parameters_testing_ho}, and if $f_0$ satisfies
	\begin{equation}
	\label{eqn:laplacian_eigenmaps_testing_criticalradius_ho}
	\|f_0\|_P^2 \geq \frac{C}{b}\biggl(\Bigl(M^2(M^2n)^{-4s/(4s + d) } \wedge n^{-1/2}\Bigr)\biggl[\sqrt{\frac{1}{a}} + \frac{1}{b}\biggr] \vee \frac{M^2}{b n^{2s/d}} \biggr) \vee \frac{1}{n},
	\end{equation}
	then $\Ebb_{f_0}[1 - \phi] \leq b$.
\end{theorem}
Similarly to the first-order case, the main takeaway from Theorem~\ref{thm:laplacian_eigenmaps_testing_ho} is that when $M \asymp 1$, then $\varphi$ is a minimax rate-optimal test over $H_0^s(\mc{X})$. However, unlike the first-order case, when $4 < d < 4s$ the minimax testing rate over $H_0^s(\mc{X})$ is still on the order of $M^2(M^2n)^{-4s/(4s + d)}$; unfortunately, we can no longer claim that $\varphi$ is an optimal test in this regime.
\begin{theorem}
	\label{thm:laplacian_eigenmaps_testing_ho_suboptimal}
	Under the same setup as Theorem~\ref{thm:laplacian_eigenmaps_estimation_ho}, but with $4 < d < 4s$. If the Laplacian eigenmaps test $\varphi$ is computed with a kernel $\eta$ satisfying~\ref{asmp:kernel_flat_euclidean}, number of eigenvectors $K$ satisfying~\eqref{eqn:eigenvector_testing_ho}, and $\varepsilon = (M^2n)^{-1/(2(s - 1) + d)}$, and if 
	\begin{equation}
	\label{eqn:laplacian_eigenmaps_testing_criticalradius_ho_suboptimal}
	\|f_0\|_P^2 \geq \frac{C}{b}\biggl(\Bigl(M^2(M^2n)^{-2s/(2(s - 1) + d) } \wedge n^{-1/2}\Bigr)\biggl[\sqrt{\frac{1}{a}} + \frac{1}{b}\biggr] \vee \frac{M^2}{b n^{2s/d}} \biggr) \vee \frac{1}{n},
	\end{equation}
	then $\Ebb_{f_0}[1 - \phi] \leq b$.
\end{theorem}
Focusing again on the special case where $M \asymp 1$, Theorem~\ref{thm:laplacian_eigenmaps_testing_ho_suboptimal} says that $\varphi$ has have small Type II error whenever $\|f_0\|_P^2 \gtrsim n^{-2s/(2(s - 1) + d)}$ and $4 < d < 4s$. This is smaller than the estimation rate $n^{-2s/(2s + d)}$, but larger than the minimax squared critical radius $n^{-4s/(4s + d)}$. As a technical matter, the problem is that when $d > 4$ there do not exist any choices of $\varepsilon$ and $K$ which satisfy both~\eqref{eqn:radius_ho} and~\eqref{eqn:eigenvector_testing_ho}, and as a result we cannot optimally balance (our upper bounds on) testing bias and variance (defined momentarily in~\eqref{eqn:testing_biasvariance}). Although we suspect $\varphi$ is truly suboptimal when $d > 4$, the inequality in~\eqref{eqn:testing_biasvariance} gives only an upper bound on testing bias, and thus we cannot rule out that the test $\varphi$ is optimal for all $4 < d < 4s$. We leave the matter to future work.

\subsection{Analysis}
\label{subsec:analysis}

We now outline the high-level strategy we follow when proving each of Theorems~\ref{thm:laplacian_eigenmaps_estimation_fo}-\ref{thm:laplacian_eigenmaps_testing_ho_suboptimal}. We analyze the estimation error of $\wh{f}$, and the testing error of $\wh{\varphi}$, by first conditioning on the design points $X_1,\ldots,X_n$ and deriving \emph{design-dependent} bias and variance terms. For estimation, we have that with probability at least $1 - \exp(-K)$,
\begin{equation}
\label{eqn:estimation_biasvariance}
\|\wh{f} - f_0\|_n^2 \leq \underbrace{\frac{\dotp{L_{n,\varepsilon}^s f_0}{f_0}_n}{\lambda_{K}^s}}_{\textrm{bias}} + \underbrace{\frac{5K}{n} \vphantom{\frac{\dotp{L^s f_0}{f_0}_n}{\lambda_{K}^s}}}_{\textrm{variance}}.
\end{equation}
For testing, we have that $\varphi$ (which is a level-$a$ test by construction) also has small Type II Error, $\Ebb_{f_0}[1 - \phi] \leq b/2$, if 
\begin{equation}
\label{eqn:testing_biasvariance}
\|f_0\|_n^2 \geq  \underbrace{\frac{\dotp{L_{n,\varepsilon}^s f_0}{f_0}_n}{\lambda_{K}^s}}_{\textrm{bias}} + \underbrace{32\frac{\sqrt{2K}}{n}\biggl[\sqrt{\frac{1}{a}} + \frac{1}{b}\biggr]}_{\textrm{variance}}.
\end{equation}
These design-dependent bias-variance decompositions are reminiscent of the more classical bias-variance decompositions typical in the analysis of population-level spectral projection methods, but different in certain key respects. Comparing~\eqref{eqn:estimation_biasvariance} and~\eqref{eqn:testing_biasvariance} to~\eqref{pf:spectral_series_estimation_3} and~\eqref{pf:spectral_series_test}, we see that two continuum objects in the latter pair of bounds, the Sobolev norm $\|f_0\|_{\mc{H}^s(\mc{X})}^2$ and the eigenvalue $\lambda_k(\Delta_P)$, have been replaced by graph-based analogues: the graph Sobolev seminorm $\dotp{L_{n,\varepsilon}^sf_0}{f_0}_n$ and the graph Laplacian eigenvalue $\lambda_k^s$. These latter quantities, along with the empirical squared norm $\|f_0\|_n^2$, are random variables that depend on the random design points $X_1,\ldots,X_n$. We proceed to establish suitable upper and lower bounds that hold in probability. 

\paragraph{Estimates of graph quadratic forms.}
In Proposition~\ref{prop:graph_seminorm_fo} we restate an upper bound on the Dirichlet energy $\dotp{L_{n,\varepsilon}f}{f}_n$ from~\cite{green2021}. 
\begin{proposition}[Lemma 1 of~\cite{green2021}]
	\label{prop:graph_seminorm_fo}
	Suppose Model~\ref{def:model_flat_euclidean}, and additionally $f \in H^1(\mc{X})$. There exist constants $c,C$ that do not depend on $f$ or $n$ such that the following statement holds for any $\delta \in (0,1)$: if $\eta$ satisfies~\ref{asmp:kernel} and $\varepsilon < c$, then
	\begin{equation}
	\label{eqn:graph_seminorm_fo}
	\dotp{L_{n,\varepsilon}f}{f}_n \leq \frac{C}{\delta} \|f\|_{H^1(\mc{X})}^2,
	\end{equation}
	with probability at least $1 - \delta$.
\end{proposition}
Proposition~\ref{prop:graph_seminorm_fo} follows by upper bounding the expectation of $\Ebb\dotp{L_{n,\varepsilon}f}{f}_n = \dotp{L_{P,\varepsilon}f}{f}_P$---where $L_{P,\varepsilon}$ is the non-local Laplacian operator defined in~\eqref{eqn:nonlocal_laplacian}---by (a constant times) the squared Sobolev norm $\|f\|_{H^1(\mc{X})}^2$, and an application of Markov's inequality.

In this work, we establish that an analogous bound holds for $\dotp{L_{n,\varepsilon}^sf_0}{f_0}_n$ when $s > 1$. We call this quantity the \emph{graph Sobolev semi-norm}.
\begin{proposition}
	\label{prop:graph_seminorm_ho} 
	Suppose Model~\ref{def:model_flat_euclidean}, and additionally that $f \in H_0^s(\mc{X})$ and $p \in C^{s - 1}(\mc{X})$. Then there exist constants $c$ and $C$ that do not depend on $f_0$ or $n$ such that the following statement holds for any $\delta \in (0,1)$: if $\eta$ satisfies~\ref{asmp:kernel_flat_euclidean} and $Cn^{-1/(2(s - 1) + d)} < \varepsilon < c$, then
	\begin{equation}
	\label{eqn:graph_seminorm_ho}
	\dotp{L_{n,\varepsilon}^s f}{f}_n \leq \frac{C}{\delta} \|f\|_{H^s(\mc{X})}^2 ,
	\end{equation}
	with probability at least $1 - \delta$.
\end{proposition}
We now summarize the techniques used to prove Proposition~\ref{prop:graph_seminorm_ho}, which will help explain what role the conditions on $f_0$, $p$ and $\varepsilon$ play. To upper bound $\dotp{L_{n,\varepsilon}^sf}{f}_n$ in terms of $\|f\|_{H^s(\mc{X})}^2$, we introduce an intermediate quantity: the \emph{non-local Sobolev seminorm} $\dotp{L_{P,\varepsilon}^sf}{f}_{P}$. This seminorm is defined with respect to the iterated non-local Laplacian $L_{P,\varepsilon}^s = L_{P,\varepsilon} \circ \cdots \circ L_{P,\varepsilon}$, where $L_{P,\varepsilon}$ is a non-local approximation to $\Delta_P$, 
\begin{equation}
\label{eqn:nonlocal_laplacian}
L_{P,\varepsilon}f(x) := \frac{1}{\varepsilon^{d + 2}}\int_{\mc{X}}\bigl(f(z) - f(x)\bigr) \eta\biggl(\frac{\|z - x\|}{\varepsilon}\biggr) \,dP(x).
\end{equation}
Then the proof of Proposition~\ref{prop:graph_seminorm_ho} proceeds according to the following steps.
\begin{itemize}
	\item First we note that $\dotp{L_{n,\varepsilon}^s f}{f}_n$ is itself a biased estimate of the non-local seminorm $\dotp{L_{P,\varepsilon}^sf}{f}_{P}$. Specifically, $\dotp{L_{n,\varepsilon}^s f}{f}_n$ is a $V$-statistic, meaning it is the sum of an unbiased estimator of $\dotp{L_{P,\varepsilon}^sf}{f}_{P}$ (in other words, a $U$-statistic) plus some higher-order, pure bias terms. We show that these pure bias terms are negligible when $\varepsilon = \omega(n^{-1/(2(s - 1) + d)})$. 
	\item For $x$ sufficiently far from the boundary of $\mc{X}$---precisely $x \in \mc{X}$ such that $B(x,j\varepsilon) \subseteq \mc{X}$---we show that $L_{P,\varepsilon}^jf(x) \to \sigma_{\eta}^j \Delta_P^jf(x)$ as $\varepsilon \to 0$. Here $j = (s - 1)/2$ when $s$ is odd and $j = (s - 2)/2$ when $s$ is even. This step bears some resemblance to the analysis of the bias term in kernel smoothing, and requires that $p \in C^{s-1}(\mc{X})$.
	\item On the other hand for $x$ sufficiently near the boundary of $\mc{X}$, $L_{P,\varepsilon}^jf(x)$ does not in general converge to $\sigma_{\eta}^j\Delta_P^jf(x)$. Instead, we use the zero-trace property of $f$ to show that $L_{P,\varepsilon}^jf(x)$ is small.
	\item Finally, we combine the results of previous two steps to deduce an upper bound on $\dotp{L_{P,\varepsilon}^sf}{f}_{P}$ in terms of the squared Sobolev norm $\|f\|_{H^s(\mc{X})}^2$.  Roughly speaking, when $s$ is odd, $\dotp{L_{P,\varepsilon}^sf}{f}_P = E_{P,\varepsilon}(L_{P,\varepsilon}^jf) \approx \sigma_{\eta}^{2j}E_{P,\varepsilon}(\Delta_P^jf)$, whereas when $s$ is even $\dotp{L_{P,\varepsilon}^sf}{f}_P = \|L_{P,\varepsilon}L_{P,\varepsilon}^{j}f\|_{P}^2 \approx \sigma_{\eta}^{2j}\|L_{P,\varepsilon} \Delta_Pf\|_P^2$. Reasoning in this way, we can translate estimates of $L_{P,\varepsilon}^jf$ into an upper bound on the order-$s$ non-local Sobolev seminorm, even though $s > j$.
\end{itemize}
Together, these steps establish Proposition~\ref{prop:graph_seminorm_ho}. It is worth pointing out that we do not try to establish the pointwise estimate $L_{P,\varepsilon}^sf \to \sigma_{\eta}^s\Delta_{P}^sf$ in $L^2(P)$. If we had such an estimate, it would immediately follow that $\dotp{L_{P,\varepsilon}^sf}{f}_{P} \to \sigma_{\eta}^s\dotp{\Delta_P^sf}{f}_{P}$. Unfortunately, we assume only that $f$ has $s$ bounded derivatives, while $\Delta_P^s$ is an order-$2s$ differential operator; thus in general $L_{P,\varepsilon}^sf$ may not approach $\sigma_{\eta}^s\Delta_{P}^sf$ as $\varepsilon \to 0$. Instead we opt for the slightly more complicated approach outlined above, in which we only ever need show that $L_{P,\varepsilon}^jf(x) \to \sigma_{\eta}^j \Delta_P^jf(x)$ for some $j < s/2$. 

\paragraph{Neighborhood graph eigenvalues.}
On the other hand, several recent works \citep{burago2014,garciatrillos18,calder2019} have analyzed the convergence of $\lambda_{k}$ towards $\lambda_{k}(\Delta_P)$. They provide explicit bounds on the relative error $|\lambda_{k} - \lambda_{k}(\Delta_P)|/\lambda_{k}(\Delta_P)$, which show that the relative error is small for sufficiently large $n$ and small $\varepsilon$. Crucially, the guarantees hold simultaneously for all $1 \leq k \leq K$ as long as $\lambda_{K}(\Delta_P) = O(\varepsilon^{-2})$. These results are actually stronger than are necessary to establish Theorems~\ref{thm:laplacian_eigenmaps_estimation_fo}-\ref{thm:laplacian_eigenmaps_testing_ho}---in order to get rate-optimality, we need only show that for the relevant values of $K$, $\lambda_{K}/\lambda_K(P) = \Omega_P(1)$---but unfortunately they all assume $P$ is supported on a manifold without boundary (i.e. they assume Model~\ref{def:model_manifold} rather than Model~\ref{def:model_flat_euclidean}). 

In the case where $\mc{X}$ is assumed to have a boundary, the graph Laplacian $L_{n,\varepsilon}$ is a reasonable approximation of the operator $\Delta_P$ at points $x \in \mc{X}$ for which $B(x,\varepsilon) \subseteq \mc{X}$. In contrast, at points $x$ near the boundary of $\mc{X}$, the graph Laplacian is known to approximate a different operator altogether \citep{belkin2012}.\footnote{This is directly related to the boundary bias of kernel smoothing, since the graph Laplacian can be viewed as a kernel-based estimator of $\Delta_P$.} This renders analysis of $\lambda_k$ substantially more challenging, since its continuum limit is not $\lambda_k(\Delta_P)$.  Rather than establishing the convergence of $\lambda_k$, we will instead use Lemma~2 of \cite{green2021}, whose assumptions match our own, and who give a weaker bound on the ratio $\lambda_k/\lambda_k(\Delta_P)$ that will nevertheless suffice for our purposes. 

\begin{proposition}[Lemma~2 of \cite{green2021}]
	\label{prop:graph_eigenvalue}
	Suppose Model~\ref{def:model_flat_euclidean}. Then there exist constants $c$ and $C$ such that the following statement holds: if $\eta$ satisfies~\ref{asmp:kernel_flat_euclidean} and $C(\log n/n)^{1/d} < \varepsilon < c$, then
	\begin{equation}
	\label{eqn:graph_eigenvalue}
	\lambda_k \geq c \cdot \min\Bigl\{\lambda_k(\Delta_P), \frac{1}{\varepsilon^{2}} \Bigr\} \quad \textrm{for all $1 \leq k \leq n$,}
	\end{equation}
	with probability at least $1 - Cn\exp\{-c n\varepsilon^d\}$. 
\end{proposition}
By our assumptions on $P$, $\lambda_0(\Delta_P) = \lambda_0 = 0$. Furthermore, Weyl's Law~\eqref{eqn:weyl} tells us that under Model~\ref{def:model_flat_euclidean}, $k^{2/d} \lesssim \lambda_{k}(\Delta_P) \lesssim k^{2/d}$ for all $k \in \mathbb{N}, k > 1$. Combining these statements with~\eqref{eqn:graph_eigenvalue}, we conclude that $\lambda_{K} = \Omega_P(K^{2/d})$ so long as $K \lesssim \varepsilon^{-d}$. 

\paragraph{Empirical norm.}
Finally, in order to show that $\varphi$ has small Type II error whenever~$\|f_0\|_P$ is greater than the critical radius given by~\eqref{eqn:sobolev_space_testing_critical_radius}, we require a lower bound on $\|f_0\|_n^2$ in terms of $\|f_0\|_P^2$. In Proposition~\ref{prop:empirical_norm_sobolev} we establish that such a one-sided bound holds whenever $\|f_0\|_P$ is sufficiently large.
\begin{proposition}
	\label{prop:empirical_norm_sobolev}
	Suppose Model~\ref{def:model_flat_euclidean}, and additionally that $f \in H^s(\Xset,M)$ for some $s > d/4$. There exist constants $c$ and $C$ that do not depend on $f_0$ or $n$ such that the following statement holds for any $\delta > 0$:  if
	\begin{equation}
	\label{eqn:empirical_norm_sobolev_1}
	\norm{f}_{P} \geq C M \biggl(\frac{1}{\delta n}\biggr)^{s/d}
	\end{equation}
	then with probability at least $1 - \exp\{-(cn \wedge 1/\delta)\}$,
	\begin{equation}
	\label{eqn:empirical_norm_sobolev}
	\norm{f}_n^2 \geq \frac{1}{2} \|f_0\|_P^2.
	\end{equation}
\end{proposition}
To prove Proposition~\ref{prop:empirical_norm_sobolev}, we use the Gagliardo-Nirenberg interpolation inequality (see e.g. Theorem~(?) of~\citep{evans10}) to control the $4$th moment of $f$ in terms of $\|f\|_P$ and $|f|_{H^s(\mc{X})}$, then invoke a one-sided Bernstein's inequality as in \cite[Section 14.2]{wainwright2019}. Note carefully that the statement~\eqref{eqn:empirical_norm_sobolev} is \emph{not} a uniform guarantee over all $f \in H^s(\mc{X};M)$, as such a statement cannot hold in the sub-critical regime ($2s \leq d$). Fortunately, a pointwise bound---meaning a bound that holds with high probability for a single $f \in H^s(\mc{X})$---is sufficient for our purposes.
% (AG) Ask Ryan for reference to Evans.
% (SB via AG): More discussion about the uniform guarantee point.

Finally, invoking the bounds of Propositions~\ref{prop:graph_seminorm_fo}-\ref{prop:empirical_norm_sobolev} inside the bias-variance tradeoffs~\eqref{eqn:estimation_biasvariance} and~\eqref{eqn:testing_biasvariance} and then choosing $K$ to balance bias and variance (when possible), leads to the conclusions of Theorems~\ref{thm:laplacian_eigenmaps_estimation_fo}-\ref{thm:laplacian_eigenmaps_testing_ho_suboptimal}.

% (AG 8/27/21): Following chunk pasted here temporarily.
On the other hand, it is somewhat remarkable that Laplacian eigenmaps can \emph{ever} take advantage of higher-order smoothness in $f_0$. The sharpest known results show that the graph Laplacian eigenvectors $v_k$ converge to eigenfunctions $\psi_k$ at a rate of \textcolor{red}{(?)}. Naively applying these results, one can show that $\wh{f}$ to $\wt{f}$, but only at a rate far slower than the optimal rates for regression. Of course when the number of eigenvectors $K$ increases with $n$, as is necessary to optimally balance bias and variance of Laplacian eigenmaps, the issue only gets worse. Clearly, as a method for estimation and testing, the rate of convergence of Laplacian eigenmaps is much better than the rate implied by (what is currently known about) the concentration of individual eigenvectors around their continuum limits.

\subsection{Computational considerations}
Recall that when $s = 1$, we have shown that Laplacian eigenmaps is optimal when $\varepsilon \asymp (\log n/n)^{1/d}$ is (up to a constant) as small as possible while still ensuring the graph $G$ is connected. On the other hand, when $s > 1$, we can show Laplacian eigenmaps is optimal only when $\varepsilon = \omega(n^{-c})$ for some $c < 1/d$. For such a choice of $\varepsilon$, the average degree in $G$ will grow polynomially in $n$ as $n \to \infty$, and computing eigenvectors of the Laplacian of a graph will be more computationally intensive than if the graph were sparse \textcolor{red}{(reference)}. Thus Theorems~\ref{thm:laplacian_eigenmaps_estimation_fo} and~\ref{thm:laplacian_eigenmaps_estimation_ho} can be seen as revealing a tradeoff between statistical and computational efficiency; although to be clear, we have no theoretical evidence that Laplacian eigenmaps \emph{fails} to adapt to higher-order smoothness when $\varepsilon \asymp (\log n/n)^{1/d}$---we simply cannot prove that it succeeds. 
\textcolor{red}{(TODO): If we decide it is worth our time to investigate the optimal choice of graph radius empirically, we can add a line here mentioning this.}

Suppose one does choose $\varepsilon$ meaningfully larger than the connectivity threshold, as our theory requires when $s > 1$. We now discuss a procedure to efficiently compute an approximation to the Laplacian eigenmaps estimate, without changing the rate of convergence of the resulting estimator: \emph{edge sparsification}. By now there exist various methods see (e.g., the seminal papers of \citet{spielman2011,spielman2013,spielman2014}, or the overview by \citet{vishnoi2012} and references therein) to efficiently remove many edges from the graph $G$ while only slightly perturbing the spectrum of the Laplacian. Specifically such algorithms take as input a parameter $\sigma \geq 1$, and return a sparser graph $\wt{G}$, $E(\wt{G}) \subseteq E(G)$, with a Laplacian $\wt{L}_{n,\varepsilon}$ satisfying
\begin{equation*}
\frac{1}{\sigma} \cdot u^{\top} \wt{L}_{n,\varepsilon} u \leq u^{\top} L_{n,\varepsilon} u \leq \sigma \cdot u^{\top} \wt{L}_{n,\varepsilon}u \quad \textrm{for all $u \in \Reals^n$.}
\end{equation*}
Let $\wt{f}$ be the Laplacian eigenmaps estimator computed using the eigenvectors of the sparsified graph Laplacian $\wc{L}_{n,\varepsilon}$ . Because $\wt{G}$ is sparser than $G$, it can be (much) faster to compute the eigenvectors of $\wt{L}_{n,\varepsilon}$ than the eigenvectors of $L_{n,\varepsilon}$, and consequently much faster to compute $\wt{f}$ than $\wh{f}$ \textcolor{red}{(reference needed)}. Statistically speaking, letting $\wt{\lambda}_k$ be the $k$th eigenvalue of $\wc{L}_{n,\varepsilon}$, we have that conditional on $X_1,\ldots,X_n$,
\begin{equation*}
\|\wt{f} - f_0\|_n^2 \leq \frac{\dotp{\wt{L}_{n,\varepsilon}^s f_0}{f_0}_n}{\wt{\lambda}_{K + 1}^s} + \frac{5K}{n} \leq \sigma^{2s} \frac{\dotp{\wt{L}_{n,\varepsilon}^s f_0}{f_0}_n}{\wt{\lambda}_{K + 1}^s} + \frac{5K}{n},
\end{equation*}
with probability at least $1 - \exp(-K)$. Consequently $\wt{f}$ has $L^2(P_n)$-error of at most $\sigma^{2s}$ times our upper bound on the error of $\wh{f}$, and for any choice of $\sigma$ that is constant in $n$ the estimator $\wt{f}$ will also be rate-optimal. 

In fact the aforementioned edge sparsification algorithms are overkill for our needs. For one thing, they are designed to work when $\sigma$ is very close to $1$, whereas in order for $\wc{f}$ to be rate-optimal setting $\sigma$ to be any constant greater than $1$, say $\sigma = 2$, is sufficient. Additionally, edge sparsification algorithms are traditionally designed to work in the worst-case, where no assumptions are made on the structure of the graph $G$. But the geometric graphs we consider in this paper exhibit a special structure, in which very roughly speaking no single edge is a bottleneck. As pointed out by~\citet{sadhanala16b}, in this special case there are far simpler and faster methods for sparsification, which at least empirically seem to do the job.

\section{Manifold Adaptivity}
\label{sec:manifold_adaptivity}

In this section we consider the manifold setting, where $(X_1,Y_1),\ldots,(X_n,Y_n)$ are observed according to Model~\ref{def:model_manifold}. A theory has been developed \citep{niyogi2008finding,belkin03,belkin05,niyogi2013,balakrishnan2012minimax,balakrishnan2013cluster} establishing that the neighborhood graph $G$ can ``learn'' the manifold $\Xset$ in various senses, so long as $\Xset$ is locally linear. We build on this work by showing that when $f_0 \in H^s(\mc{X})$ and $P$ is supported on a manifold, Laplacian eigenmaps achieve the sharper minimax estimation and testing rates reviewed in Section~\ref{subsec:minimax_rates_sobolev}.

\subsection{Laplacian eigenmaps error rates under the manifold hypothesis}
Unlike in the flat-Euclidean case, since Model~\ref{def:model_manifold} assumes that $\mc{X}$ is boundaryless it is easy to deal with the first-order $(s = 1)$ and higher-order $(s > 1)$ cases all at once. A more important distinction between the results of this section and those of Section~\ref{sec:minimax_optimal_laplacian_eigenmaps} is that we will establish Laplacian eigenmaps is optimal only when the regression function $f_0 \in H^s(\mc{X};M)$ for $s \leq 3$. Otherwise, this section will proceed in a similar fashion to Section~\ref{sec:higher_order_sobolev_classes}.

\paragraph{Estimation.}
To ensure that $\wh{f}$ is an in-sample minimax rate-optimal estimator, we choose the kernel function $\eta$, graph radius $\varepsilon$ and number of eigenvectors $K$ as in~\ref{asmp:parameters_estimation_ho}, except with ambient dimension $d$ replaced by the intrinsic dimension $m$.

\begin{enumerate}[label=(P\arabic*)]
	\setcounter{enumi}{4}
	\item 
	\label{asmp:kernel_manifold}
	The kernel function $\eta$ is a nonincreasing function supported on a subset of $[0,1]$. Its restriction to $[0,1]$ is Lipschitz, and $\eta(1/2) > 0$. Additionally, it is normalized so that
	\begin{equation*}
	\int_{\Reals^m} \eta(\|z\|) \,dz = 1,
	\end{equation*}
	and we assume \smash{$\int_{\Reals^m} \|x\|^2 \eta(\|x\|) \,dx < \infty$}.
	\item 
	\label{asmp:parameters_estimation_manifold}
	For a constant $C_0$, the graph radius $\varepsilon$ and number of eigenvectors $K$ satisfy
	\begin{equation}
	\label{eqn:radius_estimation_manifold}
	C_0\max\biggl\{\biggl(\frac{\log}{n}\biggr)^{1/m}, n^{-1/(2(s - 1) + m)}\biggr\} \leq \varepsilon \leq \min\{\mathrm{inj}(\mc{X}), K^{-1/m}\},
	\end{equation}
	where we recall that $\mathrm{inj}(\mc{X})$ is a lower bound on the injectivity radius of $\mc{X}$. Additionally,
	\begin{equation*}
	K = \min\Bigl\{\floor{(M^2n)^{m/(2s + m)}} \wedge 1,n \Bigr\}.
	\end{equation*}
\end{enumerate}

\begin{theorem}
	\label{thm:laplacian_eigenmaps_estimation_manifold}
	Suppose Model~\ref{def:model_manifold}, and additionally $f_0 \in H^s(\mc{X},M)$ and $p \in C^{s - 1}(\mc{X})$ for $s \leq 3$. There exist constants $c,C$ and $N$ that do not depend on $f_0$, such that the following statement holds all for all $n$ larger than $N$ and for any $\delta \in (0,1)$: if the Laplacian eigenmaps estimator $\wh{f}$ is computed with a kernel $\eta$ satisfying~\ref{asmp:kernel_manifold}, and parameters $\varepsilon$ and $K$ satisfying~\ref{asmp:parameters_estimation_manifold}, then
	\begin{equation}
	\label{eqn:laplacian_eigenmaps_estimation_manifold}
	\|\wh{f} - f_0\|_n^2 \leq C\Bigl(\frac{1}{\delta}M^2(M^2n)^{-2s/(2s + m)} \wedge 1\Bigr) \vee \frac{1}{n},
	\end{equation}
	with probability at least $1 - \delta - Cn\exp(-cn\varepsilon^m) - \exp(-K)$.
\end{theorem}

\paragraph{Testing.}
Likewise, to construct a minimax optimal test using $\wh{T}$, we choose $\varepsilon$ and $K$ as in~\ref{asmp:parameters_testing_fo}, except with the ambient dimension $d$ replaced by the intrinsic dimension $m$.
\begin{enumerate}[label=(P\arabic*)]
	\setcounter{enumi}{5}
	\item 
	\label{asmp:parameters_testing_manifold}
	The graph radius $\varepsilon$ satisfies~\eqref{eqn:radius_estimation_manifold}, and the number of eigenvectors
	\begin{equation*}
	K = \min\Bigl\{\floor{(M^2n)^{2m/(4s + m)}} \wedge 1,n \Bigr\}.
	\end{equation*}
\end{enumerate}

\begin{theorem}
	\label{thm:laplacian_eigenmaps_testing_manifold}
	Fix $a,b \in (0,1)$. Suppose Model~\ref{def:model_manifold}. Then $\mathbb{E}_0[\varphi] \leq a$, i.e $\varphi$ is a level-$a$ test. Suppose additionally $f_0 \in H^s(\mc{X},M)$, that $p \in C^{s-1}(\mc{X})$, and that $s \leq 3$ and $m \leq 4$. Then there exist constants $c$, $C$ and $N$ that do not depend on $f_0$, such that the following statement holds for all $n$ larger than $N$: if the Laplacian eigenmaps test $\varphi$ is computed with a kernel $\eta$ satisfying~\ref{asmp:kernel_manifold}, and parameters $\varepsilon$ and $K$ satisfying~\ref{asmp:parameters_testing_manifold}, and if $f_0$ satisfies
	\begin{equation}
	\label{eqn:laplacian_eigenmaps_testing_criticalradius_manifold}
	\|f_0\|_P^2 \geq \frac{C}{b}\biggl(\Bigl(M^2(M^2n)^{-4s/(4s + m) } \wedge n^{-1/2}\Bigr)\biggl[\sqrt{\frac{1}{a}} + \frac{1}{b}\biggr] \vee \frac{M^2}{b n^{2s/m}} \biggr) \vee \frac{1}{n},
	\end{equation}
	then $\Ebb_{f_0}[1 - \phi] \leq b$.
\end{theorem}

\begin{itemize}
	\item The proofs of Theorems~\ref{thm:laplacian_eigenmaps_estimation_manifold} and~\ref{thm:laplacian_eigenmaps_testing_manifold} follow very similarly to the full-dimensional setting. The difference is that when $\mc{X}$ is a manifold with intrinsic dimension $m$, we can prove analogous results to Propositions~\ref{prop:graph_seminorm_fo}-\ref{prop:graph_eigenvalue}, but with the ambient dimension $d$ replaced by the intrinsic dimension $m$. 
	\item Unlike in the full-dimensional case, in the manifold setting our upper bounds on the estimation and testing error of Laplacian eigenmaps match the minimax rate only when $s \leq 3$.  When $s \geq 4$, the containment $H^s(\mc{X};1) \subset H^{3}(\mc{X};1)$ (taking the radius of the Sobolev ball $M = 1$ for simplicity) implies that the Laplacian eigenmaps estimator $\wh{f}$ has in-sample mean-squared error of at most $n^{-6/(6 + m)}$, and that the Laplacian eigenmaps test has small Type II error whenever $\|f_0\|_P^2 \gtrsim n^{-12/(12 + d)}$; however, these are slower than the usual minimax rates. 
	
	We now explain this discrepancy, between the full-dimensional and manifold settings. At a high level, thinking of the graph $G$ as an estimate of the manifold $\mc{X}$, we incur some error in this estimate by using Euclidean distance rather than geodesic distance to form the edges of $G$. This is in contrast with the full-dimensional setting, where the Euclidean metric exactly coincides with the geodesic distance for all points $x,z \in \mc{X}$ that are sufficiently close to each other and far from the boundary of $\mc{X}$. This extra error incurred in the manifold setting by using the ``wrong distance'' dominates when $s \geq 4$. 
	
	As this explanation suggests, by building $G$ using the geodesic distance one could avoid this error, and might obtain superior rates of convergence. However this is not an option for us, as we assume $\mc{X}$---and in particular its geodesics---are unknown. Likewise, a classical spectral projection estimator, using eigenfunctions of the manifold Laplace-Beltrami operator, will achieve the minimax rate for all values of $s$ and $m$; but this is undesirable for the same reason---we do not want to assume that $\mc{X}$ is known. It is not clear whether this gap between spectral projection and Laplacian eigenmaps estimators---or more generally, between estimators which assume the manifold is known, and those which do not---is real, or a product of loose upper bounds. 
	
	\item Finally, as in the full-dimensional case, when the intrinsic dimension $m > 4$ we cannot choose the graph radius $\varepsilon$ and number of eigenvectors $K$ to optimally balance bias and variance.  Instead, reasoning as in the proof of Theorem~\ref{thm:laplacian_eigenmaps_testing_ho_suboptimal} shows that when $1 \leq s \leq 3$, the Laplacian eigenmaps test has critical radius as given by~\eqref{eqn:laplacian_eigenmaps_testing_criticalradius_ho_suboptimal}, but with the ambient dimension $d$ replaced by $m$.
\end{itemize}
% (SB via AG): Maybe some comment on the difference between testing and estimation.

\paragraph{Analysis.}
The high-level strategy used to prove Theorems~\ref{thm:laplacian_eigenmaps_estimation_manifold} and~\ref{thm:laplacian_eigenmaps_testing_manifold} is the same as in the flat-Euclidean setting. More specifically, we will use precisely the same bias-variance decompositions~\eqref{eqn:estimation_biasvariance} (for estimation) and~\eqref{eqn:testing_biasvariance} (for testing). The difference will be that our bounds on the graph Sobolev seminorm $\dotp{L_{n,\varepsilon}^sf_0}{f_0}_n$, graph eigenvalue $\lambda_K$, and empirical norm $\|f_0\|_n^2$ will now always depend on the intrinsic dimension $m$, rather than the ambient dimension $d$. The precise results we use are contained in Propositions~\ref{prop:graph_seminorm_manifold}-\ref{prop:empirical_norm_sobolev_manifold}.
\begin{proposition}
	\label{prop:graph_seminorm_manifold} 
	Suppose Model~\ref{def:model_manifold}, and additionally that $f_0 \in H^s(\mc{X};M)$ and $p \in C^{s - 1}(\mc{X})$ for $s = 1,2$ or $3$. Then there exist constants $c_0,C_0$ and $C$ that do not depend on $f_0$, $n$ or $M$ such that the following statement holds for any $\delta \in (0,1)$: if $\eta$ satisfies~\ref{asmp:kernel_manifold} and $C_0n^{-1/(2(s - 1) + m)} < \varepsilon < c_0$, then
	\begin{equation}
	\label{eqn:graph_seminorm_manifold}
	\dotp{L_{n,\varepsilon}^s f}{f}_n \leq \frac{C}{\delta} \|f\|_{H^s(\mc{X})}^2,
	\end{equation}
	with probability at least $1 - 2\delta$.
\end{proposition}

As discussed previously, when $\mc{X}$ is a domain without boundary and $\Delta_P$ is the manifold weighted Laplace-Beltrami operator, appropriate bounds on the graph eigenvalues $\lambda_k$ have already been derived in \citep{burago2014,trillos2019,garciatrillos19}. The precise result we need is a direct consequence of Theorem 2.4 of~\citep{calder2019}.
\begin{proposition}[\textbf{c.f Theorem 2.4 of~\citep{calder2019}}]
	\label{prop:graph_eigenvalue_manifold}
	Suppose Model~\ref{def:model_manifold}. Then there exist constants $c$ and $C$ such that the following statement holds: if $\eta$ satisfies~\ref{asmp:kernel_manifold} and $C(\log n/n)^{1/m} < \varepsilon < c$, then
	\begin{equation}
	\label{eqn:graph_eigenvalue_manifold}
	\lambda_k \geq c \cdot \min\Bigl\{\lambda_k(\Delta_P), \frac{1}{\varepsilon^{2}} \Bigr\} \quad \textrm{for all $1 \leq k \leq n$,}
	\end{equation}
	with probability at least $1 - Cn\exp\{-c n\varepsilon^d\}$. 
\end{proposition}
(For the specific computation used to deduce Proposition~\ref{prop:graph_eigenvalue_manifold} from Theorem 2.4 of~\citep{calder2019}, see~\cite{green2021}.)

Finally, we have the following lower bound on the empirical norm $\|f\|_n$ under the hypotheses of Model~\ref{def:model_manifold}. 
\begin{proposition}
	\label{prop:empirical_norm_sobolev_manifold}
	Suppose Model~\ref{def:model_manifold}, and additionally that $f_0 \in H^s(\Xset,M)$ for some $s > m/4$. There exists a constant $C$ that does not depend on $f_0$ such that the following statement holds for all $\delta > 0$:  if
	\begin{equation}
	\label{eqn:empirical_norm_sobolev_manifold_1}
	\norm{f_0}_{P} \geq \frac{C M}{\delta^{s/m}}n^{-s/m},
	\end{equation}
	then with probability at least $1 - \exp\{-(cn \wedge 1/\delta)\}$,
	\begin{equation}
	\label{eqn:empirical_norm_sobolev_manifold}
	\norm{f_0}_n^2 \geq \frac{1}{2} \|f_0\|_P^2.
	\end{equation}
\end{proposition}
We prove Proposition~\ref{prop:empirical_norm_sobolev_manifold} in a parallel manner to its flat Euclidean counterpart (Proposition~\ref{prop:empirical_norm_sobolev}), by first using a Gagliardo-Nirenberg inequality to upper bound the $L^4(\mc{X})$ norm of a Sobolev function defined on a compact Riemannian manifold, and then applying a one-sided Bernstein's inequality. Finally, combining Propositions~\ref{prop:graph_seminorm_manifold}-\ref{prop:empirical_norm_sobolev_manifold} with the conditional-on-design bias-variance decompositions~\eqref{eqn:estimation_biasvariance} and \eqref{eqn:testing_biasvariance} leads to the conclusions of Theorems~\ref{thm:laplacian_eigenmaps_estimation_manifold} and~\ref{thm:laplacian_eigenmaps_testing_manifold}. 

\section{Out-of-sample error}
\label{sec:out_of_sample}
Sections~\ref{sec:minimax_optimal_laplacian_eigenmaps} and~\ref{sec:manifold_adaptivity} show that $\wh{f}$ is a minimax optimal estimator over Sobolev spaces. However, as mentioned previously we have measured loss \emph{in-sample}---that is, measured in $\Leb^2(P_n)$ norm---whereas \emph{out-of-sample} error---error measured in $L^2(P)$ norm---is the more typical metric in the random design setup.

Of course, the Laplacian eigenmaps estimator is only defined at the observed design points $X_1,\ldots,X_n$, and to measure its error in $L^2(P)$ norm we must first extend it to be defined over all of $\Xset$. We propose a simple method, kernel smoothing, to do the job. The method can applied to any estimator defined at the design points, including Laplacian eigenmaps, and we show that a smoothed version of our original estimator $\wh{f}$ has optimal $L^2(P)$ error. For simplicity, in this section we will stick to the flat Euclidean setting, where $(X_1,Y_1),\ldots,(X_n,Y_n)$ are observed according to Model~\ref{def:model_flat_euclidean}.

\paragraph{Extension by kernel smoothing.}
We now formally define our approach to extension by kernel smoothing. For a kernel function $\psi(\cdot): [0,\infty) \to (-\infty,+\infty)$, bandwidth $h > 0$, and a distribution $Q$, the \emph{Nadaraya-Watson kernel smoother} $T_{Q,h}$ is given by
\begin{equation*}
\bigl(T_{Q,h}f)(x) := 
\begin{dcases*}
\frac{1}{d_{Q,h}(x)} \int_{\Omega} f(z)\psi\biggl(\frac{\|z - x\|}{h}\biggr) \,dQ(z), & \textrm{if $d_{Q,h}(x) > 0$,} \\
0, &\textrm{otherwise,}
\end{dcases*}
\end{equation*}
where $d_{Q,h}(x) := \int_{\Omega} \psi\bigl(\|z - x\|/\varepsilon\bigr) \,dQ(z)$. For convenience, we will write $T_{n,h}f(x) := T_{P_n,h}f(x)$, and $d_{n,h}(x) := n \cdot d_{P_n,h}(x)$.  We extend the Laplacian eigenmaps estimator by passing the kernel smoother $T_{n,h}$ over it, that is we consider the estimator $T_{n,h}\wh{f}$, which is defined at every $x \in \mc{X}$ (indeed, at every $x \in \Rd$). Note that ``extension'' here is a slight abuse of nomenclature, since $T_{n,h}\wh{f}(X_i)$ and $\wh{f}_i$ may not agree in-sample.

\paragraph{Out-of-sample error of kernel smoothed Laplacian eigenmaps.}

In Lemma~\ref{lem:kernel_smoothing_insample}, we consider an arbitrary estimator $\wc{f} \in \Reals^n$. We show that the out-of-sample error $\|T_{n,h}\wc{f} - f_0\|_P^2$ can be upper bounded by three terms--- (a constant times) the in-sample error $\|\wc{f} - f_0\|_n^2$, and variance and bias terms that arise naturally in the analysis of kernel smoothing over noiseless data. We shall assume the following conditions on $\psi$ and $h$.
\begin{enumerate}[label=(K\arabic*)]
	\setcounter{enumi}{2}
	\item
	\label{asmp:kernel}
	The kernel function $\psi$ is supported on a subset of $[0,1]$. Additionally, $\psi$ is Lipschitz continuous on $[0,1]$, and is normalized so that
	\begin{equation*}
	\int_{-\infty}^{\infty} \psi(|z|) \,dz = 1.
	\end{equation*}
\end{enumerate}
\begin{enumerate}[label=(P\arabic*)]
	\setcounter{enumi}{6}
	\item
	\label{asmp:bandwidth}
	For constants $c_0$ and $C_0$, the bandwidth parameter $h$ satisfies
	\begin{equation*}
	C_0\biggl(\frac{\log(1/h)}{n}\biggr)^{1/d} \leq h \leq c_0.
	\end{equation*}
\end{enumerate}
\begin{lemma}
	\label{lem:kernel_smoothing_insample}
	Suppose Model~\ref{def:model_flat_euclidean}, and additionally that $\wc{f} \in L^2(P_n)$, $f_0 \in H^1(\mc{X})$ and $p \in C^1(\mc{X})$. If the kernel smoothing estimator $T_{n,h}\wc{f}$ is computed with a kernel $\psi$ satisfying~\ref{asmp:kernel} and bandwidth $h$ satisfying~\ref{asmp:bandwidth}, it holds that
	\begin{equation}
	\label{eqn:kernel_smoothing_insample}
	\|T_{n,h}\wc{f} - f_0\|_P^2 \leq C\biggl(\|\wc{f} - f_0\|_n^2 + \frac{1}{\delta} \cdot \frac{h^2}{nh^d} |f|_{H^1(\mc{X})}^2 + \frac{1}{\delta}\|T_{P,h}f_0 - f_0\|_P^2\biggr),
	\end{equation}
	with probability at least $1 - \delta - Ch^d\exp\{-Cnh^d\}$. 
\end{lemma}
Notice that the variance term in the above is smaller than the typical variance term for kernel smoothing of noisy data, by a factor of $h^2$. On the other hand the bias term is typical. When $\psi$ is an order-$s$ kernel, a standard analysis shows that the $\|T_{P,h}f_0 - f_0\|_P^2 \lesssim \varepsilon^{2s}$.
\begin{enumerate}[label=(K\arabic*)]
	\setcounter{enumi}{3}
	\item
	\label{asmp:ho_kernel}
	The kernel function $\psi$ is an order-$s$ kernel, meaning that it satisfies
	\begin{equation*}
	\int_{-\infty}^{\infty} \psi(|z|) \,dz = 1, \quad \int_{-\infty}^{\infty} z^j \psi(|z|) \,dz = 0 ~~\textrm{for}~j = 1,\ldots, s + d - 2, \quad \textrm{and}~ \int_{-\infty}^{\infty} z^{s + d - 1} \psi(|z|) \,dz < \infty. 
	\end{equation*}
\end{enumerate}
% (SB via AG): Give an example of a kernel which satisfies these requirements.

Choosing $h \asymp n^{-1/(2(s - 1) + d)}$ balances the kernel smoothing bias and variance terms in~\eqref{eqn:kernel_smoothing_insample}, and implies that
\begin{equation}
\label{eqn:kernel_smoothing_insample2}
\|T_{n,h}\wc{f} - f_0\|_P^2 \leq C\biggl(\|\wc{f} - f_0\|_n^2 + \frac{1}{\delta}n^{-2s/(2(s - 1) + d)}\biggr).
\end{equation}
This analysis tells us that the additional error incurred by passing a kernel smoother over an in-sample estimator $\wc{f}$ is negligible compared to the minimax rate of estimation. Consequently, if $\wc{f}$ converges at the minimax rate in in-sample mean squared error, then $T_{n,h}\wc{f}$ will converge at the minimax rate in $L^2(P)$. It follows immediately from Theorem~\ref{thm:laplacian_eigenmaps_estimation_fo} (when $s = 1$) or Theorem~\ref{thm:laplacian_eigenmaps_estimation_ho} (when $s > 1$), that $T_{n,h}\wh{f}$ achieves the optimal rate of convergence in $L^2(P)$.

\begin{theorem}
	\label{thm:laplacian_eigenmaps_estimation_out_of_sample}
	Suppose Model~\ref{def:model_flat_euclidean}. There exist constants $c$, $C$, and $N$ that do not depend on $f_0$ or $n$ such that each the following statements hold with probability at least $1 - \delta - Cn\exp\{-cn\varepsilon^d\} - Ch^d\exp\{-cnh^d\}$,  for all $n \geq N$ and for any $\delta \in (0,1)$.
	\begin{itemize}
		\item If $f_0 \in H^1(\mc{X};M)$, the Laplacian eigenmaps estimator $\wh{f}$ is computed with parameters $\varepsilon$ and $K$ that satisfy~\ref{asmp:parameters_estimation_fo}, and the out-of-sample extension $T_{n,h}\wh{f}$ is computed with bandwidth $h = n^{-1/d}$ and kernel $\psi$ that satisfies~\ref{asmp:kernel}, then
		\begin{equation*}
		\|T_{n,h}\wh{f} - f_0\|_P^2 \leq \frac{C}{\delta}M^2(M^2n)^{-2s/(2s + d)}.
		\end{equation*}
		\item If $f_0 \in H_0^s(\mc{X};M)$ and $p \in C^{s - 1}(\mc{X})$ for some $s \in \mathbb{N}, s > 1$, and the Laplacian eigenmaps estimator $\wh{f}$ is computed with parameters $\varepsilon$ and $K$ that satisfy~\ref{asmp:parameters_estimation_ho}, and the out-of-sample extension $T_{n,h}\wh{f}$ is computed with bandwidth $h = n^{-1/(2(s - 1) + d)}$ and kernel $\psi$ that satisfies~\ref{asmp:kernel} and~\ref{asmp:ho_kernel}, then
		\begin{equation*}
		\|T_{n,h}\wh{f} - f_0\|_P^2 \leq \frac{C}{\delta}M^2(M^2n)^{-2s/(2s + d)}.
		\end{equation*}
	\end{itemize}
\end{theorem}
Some remarks:
\begin{itemize}
	\item Since $T_{n,h}\wh{f}$ is defined out-of-sample, we can use sample splitting or cross validation methods to tune hyperparameters, which we could not do for the original estimator $\wh{f}$. For instance, we can (i) split the sample into two halves, (ii) use the first half to compute $T_{n,h}\wh{f}$ for various values of $\varepsilon$, $h$, and $K$, (iii) choose the optimal values of these three hyperparameters by minimizing error on the held out set. Practically speaking, cross-validation is one of the most common approaches to choosing hyperparameters. Theoretically, it is known \textcolor{red}{(references)} that choosing hyper-parameters through sample splitting can result in estimators that optimally adapt to the order of regularity $s$. In other words, it leads to estimators that are rate-optimal (up to $\log n$ factors), even when $s$ is unknown. Similar arguments should imply that $T_{n,h}\wh{f}$ is adaptive in this sense when $\varepsilon,h$ and $K$ are chosen by sample splitting. \textcolor{red}{(TODO): Siva would like me to fill in some details?}
	\item There exist other approaches to extending a function $f$ from its $\{f(X_1),\ldots,f(X_n)\}$: for instance, minimum-norm interpolation in some RKHS or Banach space \textcolor{blue}{(Rieger2008, Belkin2018)}, or 1-nearest neighbors regression \textcolor{blue}{(citation)}.  We consider extension by kernel smoothing because it is a simple and statistically optimal procedure that does not require any knowledge of the domain $\mc{X}$ or distribution $P$---as we have argued, this latter property is one of the main selling points of Laplacian eigenmaps as a tool for nonparametric regression. One potential drawback to kernel smoothing is that it can change the value of the estimate at data, meaning $\wh{f} \neq (\wc{f}(X_1),\ldots, \wc{f}(X_n))$. More sophisticated procedures such as Nystr\"{o}m extension of the eigenvectors $v_1,\ldots,v_K$ \textcolor{blue}{(citation)} inherit the generality of kernel smoothing while being genuine interpolators; however, their properties are substantially more difficult to analyze. 
	
	%Finally, the Nystr\"{o}m method extends  eigenvectors $v_k$ to functions $f_k: \mc{X} \to \Reals$ as follows,
	%\begin{equation*}
	%L_{n,\varepsilon}f_k = \lambda_k f_k~~\textrm{on $\mc{X}$, such that}~~f_k(X_i) = %v_k(X_i)~~\textrm{for $i = 1,\ldots,n$.}
	%\end{equation*}
	%Here $L_{n,\varepsilon}$ should be thought of as an operator acting on functions $f \in C(\mc{X})$ as follows,
	%\begin{equation*}
	%L_{n,\varepsilon}f(x) = \sum_{i = 1}^{n} (f(x) - f(X_i))\eta\Bigl(\frac{\|X_i - x\|}{\varepsilon}\Bigr)
	%\end{equation*}
	% The Nystr\"{o}m approach to extension makes intriguing (re-)use of the graph Laplacian $L_{n,\varepsilon}$ and its eigenvectors $v_k$. Similar methods have been analyzed in the context of kernel ridge regression (references), but little is known about its application to graph Laplacians.	
\end{itemize}

\section{Experiments}
\label{sec:experiments}

In this section we empirically demonstrate that Laplacian Eigenmaps is a reasonably good alternative to spectral projection, even when $n$ is only moderately large. In order to compare the two methods, in our experiments we stick to simple settings where we can compute eigenfunctions of $\Delta_P$, and thus the spectral projection estimator. Of course in general, it is not easy to compute these eigenfunctions: hence the appeal of Laplacian Eigenmaps.

\begin{figure*}[tb]
	\includegraphics[width=.245\textwidth]{figures/mse/mse_by_sample_size_1d_1s.pdf}
	\includegraphics[width=.245\textwidth]{figures/mse/mse_by_sample_size_1d_2s.pdf}
	\includegraphics[width=.245\textwidth]{figures/mse/mse_by_sample_size_2d_1s.pdf}
	\includegraphics[width=.245\textwidth]{figures/mse/mse_by_sample_size_2d_2s.pdf}
	\includegraphics[width=.245\textwidth]{figures/out_of_sample_mse/mse_by_sample_size_1d_1s.pdf}
	\includegraphics[width=.245\textwidth]{figures/out_of_sample_mse/mse_by_sample_size_1d_2s.pdf}
	\includegraphics[width=.245\textwidth]{figures/out_of_sample_mse/mse_by_sample_size_2d_1s.pdf}
	\includegraphics[width=.245\textwidth]{figures/out_of_sample_mse/mse_by_sample_size_2d_2s.pdf}
	\caption{Mean squared error (mse) of Laplacian eigenmaps and spectral projection estimators. Top row: in-sample mse of Laplacian eigenmaps (\texttt{LE}) and a spectral projection estimator (\texttt{SP}) as a function of sample size $n$. Bottom row: out-of-sample mse of Laplacian eigenmaps plus kernel smoothing (\texttt{LE+KS}) and a spectral projection estimator. Each plot is on the log-log scale, and the results are averaged over 400 repetitions. All estimators are tuned for optimal average mse. The black line shows the minimax rate (in slope only; the intercept is chosen to match the observed error).}
	\label{fig:fig1}
\end{figure*}

In our first experiment, we compare the mean-squared error of Laplacian eigenmaps to that of its classical spectral projection counterpart. We vary the sample size from $n = 1000$ to $n = 4000$; sample $n$ design points $X_1,\ldots,X_n$ from the uniform distribution on the cube $[-1,1]^d$; and sample responses $Y_i$ according to~\eqref{eqn:model} with regression function $f_0 = M/\lambda_K^{s/2} \cdot \psi_K$ for $K \asymp n^{d/(2s + d)}$; the pre-factor $M/\lambda_K^{s/2}$ is chosen so that $|f_0|_{H^s(\mc{X})}^2 = M^2$. In Figure~\ref{fig:fig1} we show the in-sample mean-squared error of Laplacian eigenmaps and a classical spectral projection estimator as a function of $n$, for different dimensions $d$ and order of smoothness $s$. We see that all estimators have mean-squared error converging to zero at roughly the minimax rate. We also see that the mean-squared error of Laplacian Eigenmaps gets closer to that of spectral projection as $n$ gets larger. The fact that spectral projection outperform Laplacian Eigenmaps 

We also compare the error, over a held-out test set, of Laplacian eigenmaps plus kernel smoothing to the spectral projection estimator. The out-of-sample mean squared error of the two estimators is very similar to the in-sample mean-squared error. This supports our theoretical claim that the additional error incurred by kernel smoothing of Laplacian Eigenmaps is negligible.

In our second experiment, we compare tests using Laplacian eigenmaps and spectral projection test statistics. The setup, in terms of $n$ and $P$, is the same as that of our first experiment. To empirically evaluate the \textcolor{red}{critical radius} $\epsilon_n(\phi,H^1(\mc{X};M))$ of a test $\phi$, we compute $\phi$ for each $f_0 \in \mc{F} \subset H^1(\mc{X};M)$, where $\mc{F}$ is a discrete subset of $H^1(\mc{X};M)$. For each of $b = 1,2,\ldots,100$, we compute $\epsilon_b$, the smallest value of $\epsilon$ such that $R_n(\phi,\mc{F},\epsilon_b) \geq b/100$. Then we take $\bar{\epsilon} = 1/100 \cdot \sum_{b = 1}^{100}\epsilon_b$ to be our empirical measure of worst-case risk. In Figure~\ref{fig:fig2}, we see that the critical radii of both Laplacian eigenmaps and spectral projection tests are quite close to each other, and converge to $0$ at roughly the minimax rate.
\begin{figure*}[tb]
	\includegraphics[width=.245\textwidth]{figures/testing/eigenfunction/critical_radius_by_sample_size_1d_1s.pdf}
	\includegraphics[width=.245\textwidth]{figures/testing/eigenfunction/critical_radius_by_sample_size_1d_2s.pdf}
	\includegraphics[width=.245\textwidth]{figures/testing/eigenfunction/critical_radius_by_sample_size_2d_1s.pdf}
	\includegraphics[width=.245\textwidth]{figures/testing/eigenfunction/critical_radius_by_sample_size_2d_2s.pdf}
	\caption{Worst-case testing risk Laplacian eigenmaps (\texttt{LE}) and spectral projection (\texttt{SP}) tests, as a function of sample size $n$. Plots are on the same scale as Figure~\ref{fig:fig1}, and black line shows the minimax rate. All tests are set to have $.05$ Type I error, and are calibrated by simulation under the null.}
	\label{fig:fig2}
\end{figure*}

These experiments demonstrate that in terms of statistical error, Laplacian eigenmaps methods are reasonable replacements for spectral projection methods. Laplacian eigenmaps depends on two tuning parameters, and in our final experiment we investigate the importance of both, focusing now on estimation. In Figure~\ref{fig:fig3}, we see how the mean-squared error of Laplacian eigenmaps changes as each tuning parameter is varied. As suggested by our theory, properly choosing the number of eigenvectors $K$ is crucial: the mean-squared error curves, as a function of $K$, always have a sharply defined minimum. On the other hand, as a function of the graph radius parameter $\varepsilon$ the mean-squared error curve is much closer to flat. This squares completely with our theory, which requires that the number of eigenvectors $K$ be much more carefully tuned that the graph radius $\varepsilon$.

We also plot the out-of-sample mean squared error of Laplacian eigenmaps plus kernel smoothing, as a function of its various tuning parameters (which include the bandwidth $h$ as well as $\varepsilon$ and $K$.) Here the relationship between theory and empirics is more nuanced. On the one hand, empirically it seems that the optimal choice of bandwidth parameter $h$ is usually smaller than $\varepsilon$, as suggested by our theory. On the other hand, for Laplacian eigenmaps plus kernel smoothing we see that mean-squared error curves as a function of $K$ are often quite close to their minima even when we choose many more eigenvectors than is optimal for Laplacian eigenmaps or spectral projection. This is not reflected in our theory, where we require that $K$ be chosen in the same tight range as was required for Laplacian Eigenmaps to be optimal in-sample. However, it does make intuitive sense: extension by kernel smoothing further attenuates the noise, making the algorithm more forgiving to overfitting during the Laplacian Eigenmaps step. 

\begin{figure*}[tb]
	\includegraphics[width=.245\textwidth]{figures/tuning/eigenfunction/mse_by_number_of_eigenvectors_1d_1s.pdf}
	\includegraphics[width=.245\textwidth]{figures/tuning/eigenfunction/mse_by_radius_1d_1s.pdf} 
	\includegraphics[width=.245\textwidth]{figures/tuning/eigenfunction/mse_by_number_of_eigenvectors_2d_1s.pdf}
	\includegraphics[width=.245\textwidth]{figures/tuning/eigenfunction/mse_by_radius_2d_1s.pdf} 
	\includegraphics[width=.245\textwidth]{figures/tuning/sobolev/mse_by_number_of_eigenvectors_1d_1s.pdf}
	\includegraphics[width=.245\textwidth]{figures/tuning/sobolev/mse_by_radius_1d_1s.pdf}
	\includegraphics[width=.245\textwidth]{figures/tuning/sobolev/mse_by_number_of_eigenvectors_2d_1s.pdf}
	\includegraphics[width=.245\textwidth]{figures/tuning/sobolev/mse_by_radius_2d_1s.pdf}  
	\caption{Mean squared error of Laplacian Eigenmaps (\textcolor{red}{red}), Laplacian Eigenmaps plus kernel smoothing (\textcolor{blue}{blue}), and spectral projection (\textcolor{green}{green}) as a function of tuning parameters. Top row: the same regression function $f_0$ as used in Figure~\ref{fig:fig1}. Bottom row: the regression function $f_0 \propto \sum_{k} 1/\lambda_k^{1/2} \psi_k$. For all experiments, the sample size $n = 1000$, and the results are averaged over $200$ repetitions. In each panel, all tuning parameters except the one being varied are set to their optimal values. For Laplacian Eigenmaps plus kernel smoothing, circular points and a solid line are used to denote the error as a function of the graph radius $\varepsilon$, whereas triangular points and a dashed line are used to denote the error as a function of the bandwidth $h$.}
	\label{fig:fig3}
\end{figure*}

\section{Discussion}
\label{sec:discussion}

\subsection{Comparison with other estimators}
In this paper, we have motivated Laplacian eigenmaps by viewing it as a noisy approximation of a classical spectral projection method, which is its most obvious counterpart. We have shown that Laplacian eigenmaps inherits the optimality properties of its more classical counterpart. We now discuss the relationship between Laplacian eigenmaps and three other approaches to nonparametric regression. The first two---nonparametric least squares and kernel smoothing---are classical, whereas the third---Laplacian smoothing---makes use of the graph Laplacian in a different way than Laplacian eigenmaps.

% (SB via AG): Change the name of this to ``spectral projection'' or something like that.
\paragraph{Nonparametric least squares.}
The standard recommended alternative to spectral projection methods, when the distribution $P$ is considered non-uniform or unknown, is to do least-squares. For example, suppose instead of knowing $\psi_1,\psi_2,\ldots$, we had access only to eigenfunctions $\phi_1,\phi_2,\ldots$ of an unweighted Laplace-Beltrami operator $\Delta$. Then letting $\Phi_K = \mathrm{span}\{\phi_1,\ldots,\phi_K\}$, the least-squares estimator and test statistic
\begin{equation*}
\wt{f}_{\mathrm{LS}} := \argmin_{f \in \Phi_K} \|Y - f\|_n^2,\textrm{and}~~\wt{T}_{\mathrm{LS}} = \|\wt{f}\|_{\nu}^2
\end{equation*}
are still rate-optimal over $H_0^s(\mc{X})$. This holds true for both Model~\ref{def:model_flat_euclidean} and~\ref{def:model_manifold}, and is a consequence of our assumption that $p$ is bounded away from $0$.

However, this is not a totally satisfactory fix. For one thing, the least squares approach just outlined requires that we know the domain $\mc{X}$, in the strong sense that we know the Laplace-Beltrami operator $\Delta$ defined on $\mc{X}$. Domain knowledge is generally precious information, and such strong knowledge of $\mc{X}$ seems particularly unrealistic in the case where $\mc{X}$ is a manifold, and $\Delta$ is the manifold Laplace-Beltrami operator. Additionally, even if we know $\mc{X}$, diagonalizing the Laplace-Beltrami operator $\Delta$ is quite difficult for all but a few special domains, such as the unit cube $\mc{X} = [0,1]^d$ or torus $\mc{X} = \mathbb{T}^d$.

% (SB via AG): Maybe also worth discussing the ill-posed counterpart?

\paragraph{Kernel smoothing.}
It is also natural to ask whether the two stage estimator $T_{n,h}\wh{f}$ defined in Section~\ref{sec:out_of_sample} has any advantage over the simpler approach of directly kernel smoothing the responses, i.e. using the estimator $T_{n,h}Y$ (possibly for a different choice of $h$). In Appendix~\ref{subsec:eigenmaps_beats_kernel_smoothing}, we answer this question in the affirmative, by giving a simple example of a sequence of densities and regression functions $\{(p^{(n)}, f_0^{(n)}: n \in \mathbb{N}\}$ such that $\Ebb\|f_0 - T_{n,h}\wh{f}\|_P^2$ is of a strictly lower order than $\inf_{h'} \Ebb\|f_0 - T_{h',n}Y\|_P^2$. This is possible because Laplacian eigenmaps induces a completely different bias than kernel smoothing. For example, when $f_0$ and $p$ satisfy the so-called \emph{cluster} assumption--- e.g. $f_0$ is piecewise constant in high-density regions (clusters) of $p$--- then the bias of Laplacian eigenmaps can much smaller than that of kernel smoothing (for equivalent levels of variance). 

We emphasize that this does not contradict the well-known fact that kernel smoothing is an optimal method for nonparametric regression over e.g. H\"{o}lder balls. It simply reflects that in the standard nonparametric regression setup---which we adopt in the main part of this paper, and in which $P$ is assumed to be equivalent to Lebesgue measure---the biases of Laplacian eigenmaps and kernel smoothing are equivalent. On the other hand, when $f_0$ and $p$ satisfy some special relationship, such as the cluster assumption, the biases of these two methods can be quite different. There has been some work \textcolor{red}{(Rigollet, Wasserman, Niyogi, El Alaoui)} analyzing semi-supervised learning under various instantiations of the cluster assumption. However, a comprehensive analysis of various methods, including Laplacian eigenmaps and kernel smoothing, in this setting remains outstanding.

\paragraph{Laplacian smoothing.}
As mentioned previously, graph Laplacian smoothing uses the graph Laplacian $L_{n,\varepsilon}$ to form the penalty in a penalized least squares estimator. It can be calculated by one solve of a sparse, diagonally dominant linear system. Thus it should be much faster to compute than Laplacian eigenmaps, in which one must find (many) eigenvectors of $L_{n,\varepsilon}$. On the other hand, Laplacian smoothing is known to be minimax rate-optimal only in very limited regimes (over the Sobolev spaces $H^1(\mc{X})$ for $1 \leq d < 4$.) In contrast, we have shown that Laplacian eigenmaps is minimax rate-optimal for all $d$ (when $s = 1$) . We have also shown that Laplacian eigenmaps can adapt to higher-order smoothness, i.e. it can be minimax rate-optimal when $s > 1$. Thus, the known statistical properties of Laplacian eigenmaps are much stronger than those of Laplacian smoothing. 

\subsection{Future Work}
We view our work can be viewed as a contribution both to the fields of nonparametric regression with series estimators, and to graph-based learning. We end our discussion by mentioning some open work in each of these directions. 

% (SB via AG): Add references, and clarify what is meant by used to.
Much is known about classical spectral projection methods beyond their rate optimality. For instance: such estimators and tests exhibit \emph{sharp optimality}, meaning their risk is within a $(1 + o(1))$ factor of the optimal risk; they can adapt to unknown smoothness of the regression function; they can be used to estimate smooth functionals of the regression function;  finally, they can be used to form confidence sets in $L^2(P)$. It would be interesting to see if Laplacian eigenmaps could replicate the performance of classical methods in any, or all, of these problems.

On the other hand, there are many variants of Laplacian eigenmaps worth considering. For instance, one can change the graph under consideration (e.g. by using the k-nearest neighbors), or the normalization of the graph Laplacian $L_{n,\varepsilon}$ (e.g. by using the symmetric normalized Laplacian). The former is practically useful, because it typically leads to connected graphs while always ensuring a given level of edge sparsity. In the latter, the graph Laplacian converges to a different limiting operator, which possesses different eigenvectors than $\Delta_P$ and thereby induces a different bias. We believe that under the setup we consider here, both methods will continue to be optimal.

% AG: I think this following might just be garbage, but I will leave it in in case it piques Ryan or Siva's interest. 

% Proposition~\ref{prop:graph_seminorm_ho}, and its proof, show that the iterated graph Laplacian $L_{n,\varepsilon}^s$ does a reasonable job of approximating a higher-order differential operator. The conditions we require --- that $\varepsilon$ be sufficiently large, that $f$ be zero-trace, and that $p$ have regularity of order $s - 1$ --- indicate that the iterated graph Laplacian is an imperfect, if adequate, tool for this job. This is no fault of the graph Laplacian. Rather, as pointed out by \textcolor{red}{(Sadhanala et al. 2017)} its reflects that graph Laplacians were designed for a very general circumstance in which the only notions of derivation are first-differentials, divergences, and compositions of the two. Indeed, viewed in this light it is a pleasant surprise that iterated graph Laplacians approximate higher-order differentials at all. When one assumes Euclidean structure, as we do in this paper, there exist many methods (e.g. local linear embeddings, Hessian local linear embedding, local tangent space alignment) which leverage this structure to approximate differential operators in a more sophisticated manner. However, thus far little theoretical investigation has been done into even the pointwise behavior of these approaches.%


\bibliographystyle{plainnat}
\bibliography{../../graph_regression_bibliography} 

\appendix

\noindent 

\section{Proofs in Manifold Setting}
\label{sec:manifold_proofs}

In more detail, recall from our discuss.on in Section~\ref{subsec:analysis} that an essential step in our proofs is relating the iterated non-local operator $L_{P,\varepsilon}^j$ to the weighted Laplace-Beltrami operator $\Delta_P^j$, for $j = \floor{s/2}$. Suppose for the moment $s = 4$. Then we relate $L_{P,\varepsilon}$ to $\Delta_P$ by means of an intermediary nonlocal operator, as described by the following chain of estimates:
	\begin{align*}
	L_{P,\varepsilon} f(x) & = \int_{\mc{X}} \bigl(f(z) - f(x)\bigr)\eta\biggl(\frac{|z - x|}{\varepsilon}\biggr) \,dP(x) \\
	& = \bigl(1 + O(\varepsilon^2)\bigr) \int_{\mc{X}} \bigl(f(z) - f(z)\bigr)\eta\biggl(\frac{d_{\mc{X}}(z,x)}{\varepsilon}\biggr) \,dP(x) \\
	& = \Delta_Pf(x) + O\bigl(\varepsilon^2M\bigr) + O(\varepsilon^3).
	\end{align*}
	Here, $d_{\mc{X}}(\cdot,\cdot)$ is the geodesic distance on $\mc{X}$. For simplicity we have assumed $f \in C^3(\mc{X})$ and $p \in C^2(\mc{X})$, but the main idea will not change if we assume Sobolev smoothness instead. From this, we may deduce that
	\begin{align*}
	L_{P,\varepsilon}^2 f(x) = 
	\end{align*}
	When $s = 1$, this suffices.

\end{document}