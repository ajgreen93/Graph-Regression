\documentclass{article}
\usepackage{amsmath}
\usepackage{amsfonts, amsthm, amssymb}
\usepackage{graphicx}
\usepackage{hyperref}
\hypersetup{
	colorlinks=true,
	linkcolor=blue,
	citecolor=blue
}
\usepackage[parfill]{parskip}
\usepackage{algpseudocode}
\usepackage{algorithm}
\usepackage{enumerate}
\usepackage[shortlabels]{enumitem}
\usepackage{fullpage}
\usepackage{mathtools}
\usepackage{tikz}

\usepackage{natbib}
\renewcommand{\bibname}{REFERENCES}
\renewcommand{\bibsection}{\subsubsection*{\bibname}}

\DeclareFontFamily{U}{mathx}{\hyphenchar\font45}
\DeclareFontShape{U}{mathx}{m}{n}{<-> mathx10}{}
\DeclareSymbolFont{mathx}{U}{mathx}{m}{n}
\DeclareMathAccent{\wb}{0}{mathx}{"73}

\DeclarePairedDelimiterX{\norm}[1]{\lVert}{\rVert}{#1}
\DeclarePairedDelimiterX{\seminorm}[1]{\lvert}{\rvert}{#1}

\newcommand{\eqdist}{\ensuremath{\stackrel{d}{=}}}
\newcommand{\Graph}{\mathcal{G}}
\newcommand{\Reals}{\mathbb{R}}
\newcommand{\Identity}{\mathbb{I}}
\newcommand{\Xsetistiid}{\overset{\text{i.i.d}}{\sim}}
\newcommand{\convprob}{\overset{p}{\to}}
\newcommand{\convdist}{\overset{w}{\to}}
\newcommand{\Expect}[1]{\mathbb{E}\left[ #1 \right]}
\newcommand{\Risk}[2][P]{\mathcal{R}_{#1}\left[ #2 \right]}
\newcommand{\Prob}[1]{\mathbb{P}\left( #1 \right)}
\newcommand{\iset}{\mathbf{i}}
\newcommand{\jset}{\mathbf{j}}
\newcommand{\myexp}[1]{\exp \{ #1 \}}
\newcommand{\abs}[1]{\left \lvert #1 \right \rvert}
\newcommand{\restr}[2]{\ensuremath{\left.#1\right|_{#2}}}
\newcommand{\ext}[1]{\widetilde{#1}}
\newcommand{\set}[1]{\left\{#1\right\}}
\newcommand{\seq}[1]{\set{#1}_{n \in \N}}
\newcommand{\floor}[1]{\left\lfloor #1 \right\rfloor}
\newcommand{\Var}{\mathrm{Var}}
\newcommand{\Cov}{\mathrm{Cov}}
\newcommand{\diam}{\mathrm{diam}}

\newcommand{\emC}{C_n}
\newcommand{\emCpr}{C'_n}
\newcommand{\emCthick}{C^{\sigma}_n}
\newcommand{\emCprthick}{C'^{\sigma}_n}
\newcommand{\emS}{S^{\sigma}_n}
\newcommand{\estC}{\widehat{C}_n}
\newcommand{\hC}{\hat{C^{\sigma}_n}}
\newcommand{\vol}{\text{vol}}
\newcommand{\spansp}{\mathrm{span}~}
\newcommand{\1}{\mathbf{1}}

\newcommand{\Linv}{L^{\Xsetagger}}
\DeclareMathOperator*{\argmin}{argmin}
\DeclareMathOperator*{\argmax}{argmax}

\newcommand{\emF}{\mathbb{F}_n}
\newcommand{\emG}{\mathbb{G}_n}
\newcommand{\emP}{\mathbb{P}_n}
\newcommand{\F}{\mathcal{F}}
\newcommand{\D}{\mathcal{D}}
\newcommand{\R}{\mathcal{R}}
\newcommand{\Rd}{\Reals^d}
\newcommand{\Nbb}{\mathbb{N}}

%%% Vectors
\newcommand{\thetast}{\theta^{\star}}
\newcommand{\betap}{\beta^{(p)}}
\newcommand{\betaq}{\beta^{(q)}}
\newcommand{\vardeltapq}{\varDelta^{(p,q)}}
\newcommand{\lambdavec}{\boldsymbol{\lambda}}


%%% Matrices
\newcommand{\X}{X} % no bold
\newcommand{\Y}{Y} % no bold
\newcommand{\Z}{Z} % no bold
\newcommand{\Lgrid}{L_{\grid}}
\newcommand{\Xsetgrid}{D_{\grid}}
\newcommand{\Linvgrid}{L_{\grid}^{\Xsetagger}}
\newcommand{\Lap}{{\bf L}}
\newcommand{\NLap}{{\bf N}}
\newcommand{\PLap}{{\bf P}}
\newcommand{\Id}{I}

%%% Sets and classes
\newcommand{\Xset}{\mathcal{X}}
\newcommand{\Sset}{\mathcal{S}}
\newcommand{\Hclass}{\mathcal{H}}
\newcommand{\Pclass}{\mathcal{P}}
\newcommand{\Leb}{L}
\newcommand{\mc}[1]{\mathcal{#1}}

%%% Distributions and related quantities
\newcommand{\Pbb}{\mathbb{P}}
\newcommand{\Ebb}{\mathbb{E}}
\newcommand{\Qbb}{\mathbb{Q}}
\newcommand{\Ibb}{\mathbb{I}}

%%% Operators
\newcommand{\Tadj}{T^{\star}}
\newcommand{\Xsetive}{\mathrm{div}}
\newcommand{\Xsetif}{\mathop{}\!\mathrm{d}}
\newcommand{\gradient}{\mathcal{D}}
\newcommand{\Hessian}{\mathcal{D}^2}
\newcommand{\dotp}[2]{\langle #1, #2 \rangle}
\newcommand{\Dotp}[2]{\Bigl\langle #1, #2 \Bigr\rangle}

%%% Misc
\newcommand{\grid}{\mathrm{grid}}
\newcommand{\critr}{R_n}
\newcommand{\Xsetx}{\,dx}
\newcommand{\Xsety}{\,dy}
\newcommand{\Xsetr}{\,dr}
\newcommand{\Xsetxpr}{\,dx'}
\newcommand{\Xsetypr}{\,dy'}
\newcommand{\wt}[1]{\widetilde{#1}}
\newcommand{\wh}[1]{\widehat{#1}}
\newcommand{\ol}[1]{\overline{#1}}
\newcommand{\spec}{\mathrm{spec}}
\newcommand{\LE}{\mathrm{LE}}
\newcommand{\LS}{\mathrm{LS}}
\newcommand{\SM}{\mathrm{SM}}
\newcommand{\OS}{\mathrm{FS}}
\newcommand{\PLS}{\mathrm{PLS}}

%%% Order of magnitude
\newcommand{\soom}{\sim}

% \newcommand{\span}{\textrm{span}}

\newtheoremstyle{alden}
{6pt} % Space above
{6pt} % Space below
{} % Body font
{} % Indent amount
{\bfseries} % Theorem head font
{.} % Punctuation after theorem head
{.5em} % Space after theorem head
{} % Theorem head spec (can be left empty, meaning `normal')

\theoremstyle{alden} 


\newtheoremstyle{aldenthm}
{6pt} % Space above
{6pt} % Space below
{\itshape} % Body font
{} % Indent amount
{\bfseries} % Theorem head font
{.} % Punctuation after theorem head
{.5em} % Space after theorem head
{} % Theorem head spec (can be left empty, meaning `normal')

\theoremstyle{aldenthm}
\newtheorem{theorem}{Theorem}
\newtheorem{conjecture}{Conjecture}
\newtheorem{lemma}{Lemma}
\newtheorem{example}{Example}
\newtheorem{corollary}{Corollary}
\newtheorem{proposition}{Proposition}
\newtheorem{assumption}{Assumption}
\newtheorem{remark}{Remark}


\theoremstyle{definition}
\newtheorem{definition}{Definition}[section]

\theoremstyle{remark}

\begin{document}
\title{Minimax-optimal Laplacian Eigenmaps regression over Sobolev Spaces with Neighborhood Graphs}
\author{Alden Green}
\date{\today}
\maketitle

\section{Introduction}
\label{sec:introduction}

\textbf{(1) Geometric graphs.} In graph-based learning, one observes data $X_1,\ldots,X_n$ sampled independently from an unknown distribution $P$, and forms a geometric graph $G$---with edges $e_{ij}$ corresponding to proximity between samples $X_i$ and $X_j$---over the observed data. Geometric graphs encode information about $P$ in an extremely general manner, and can thus be leveraged to conduct many different fundamental statistical tasks. These include clustering, semi-supervised learning, classification, and regression---both estimation and goodness-of-fit testing. Though much theoretical work has been done on the consistency of graph-based learning methods---and more recently, rates of convergence have been established for some problems---little is known so far about their optimality, even for classic statistical tasks. 

\textbf{(2) What we study: Regression and Laplacian eigenmaps.} In this paper we focus on regression where in addition to the design points $X_1,\ldots,X_n$ one observes real-valued responses $Y_1,\ldots,Y_n$, and learns the regression function $f_0(x) := E[Y|X = x]$. We consider both the estimation and goodness-of-fit testing problems. The methods we will study are based on \emph{Laplacian eigenmaps}, first introduced by \cite{belkin03a}, which projects the response vector $Y = (Y_1,\ldots,Y_n)$ onto the span of the leading eigenvectors of the graph Laplacian $L$. The Laplacian $L$ is a difference operator, acting on functions $f: {\bf X} \to \Reals$ as follows,
\begin{equation}
\label{eqn:graph_laplacian}
Lf(X_i) = \sum_{j = 1}^{n} \bigl(f(X_i) - f(X_j)\bigr)e_{ij}. 
\end{equation}
The Laplacian $L$ can be seen as a discretization of the Laplace-Beltrami operator $\Delta_P$ \textcolor{red}{(...)}. The eigenvectors of $L$ form an orthonormal basis of $L^2(X)$, and are approximations to the eigenvectors of $\Delta_P$. Their corresponding eigenvalues are estimates of~\textcolor{red}{(...)}, and provide a notion of smoothness for each eigenvector---the smaller the eigenvalue, the smoother the eigenvector. 

\textbf{(3) Why we study.} Laplacian eigenmaps is thus an example of a \emph{spectral series} estimator. A spectral series estimator is a special case of an orthogonal series estimator, one of the most classical methods of non-parametric regression. Traditionally, orthogonal series estimators are defined with respect to a fixed set of basis functions, orthogonalized with respect to some pre-determined reference measure $Q$. Spectral series estimators take a particular orthogonal basis, the eigenfunctions of $\Delta_Q$. In contrast, eigenvectors of the graph Laplacian are data-dependent objects, and adapt to the geometry of the design distribution $P$ in a rich manner. 

The theoretical properties of (classical) spectral series estimators are by this point well-understood \textcolor{red}{(references)}. In particular, they are minimax optimal for nonparametric regression over H\"{o}lder and Sobolev spaces. However, in practice these estimators suffer from some serious drawbacks. Finding the eigenfunctions of $\Delta_Q$ is in general non-trivial. The statistical optimality of such estimators also hinges on the reference measure $Q$ being equal to $P$, an unrealistic assumption when $P$ is unknown. If $Q \neq P$, the standard recommendation is to instead do least squares---this can be statistically rate-optimal, but is numerically unstable when dimension $d$ is even moderately large.

\textbf{(4) Our contributions.} Laplacian eigenmaps avoid these drawbacks \textcolor{red}{(...)} On the other hand their statistical properties are not as well understood \textcolor{red}{(...)}. The primary contribution of our paper is to fill this theoretical gap, by answering the following question:

\begin{quote}
	\textcolor{red}{(TODO)}
\end{quote}

Broadly speaking, we show that when the regression function $f_0$ is smooth, in the sense of having bounded derivatives in the Sobolev sense, Laplacian eigenmaps is statistically minimax optimal. This statement holds for different relations between the dimension $d$ and number of derivatives $s$, depending on the problem (estimation or testing). \textcolor{red}{(TODO): Add two tables summarizing your major results here.} 

\textbf{(5) Organization.}

\textcolor{red}{(TODO)}

\textbf{(6) Notation.}

\textcolor{red}{TODO}: Define $\Leb^2(\mc{X}), \Leb^2(P)$ and $\Leb^2(P_n)$. 


For a square, symmetric matrix $A \in \Reals^{n \times n}$, an eigenvector $v$ of $A$ is a \textcolor{red}{unit-norm} solution to the eigenproblem
\begin{equation*}
Av = \lambda v,
\end{equation*}
and $\lambda$ is the corresponding eigenvalue.

\section{Setup, Background, and Overview of Results}
\label{sec:setup_main_results}

In this section, we begin by giving a precise definition of Laplacian eigenmaps. We then review minimax rates for non-parametric regression over Sobolev spaces, allowing us to summarize our main results. 

\subsection{Non-parametric regression with Laplacian Eigenmaps}
\label{sec:regression_laplacian_eigenmaps}

\textbf{(1) Formal setup.}
We will operate in the usual setting of non-parametric regression with random design. We observe independent random samples $(X_1,Y_1),\ldots,(X_n,Y_n)$, where the design points $X_1,\ldots,X_N$ are drawn i.i.d from a distribution $P$ supported on a compact set $\mc{X} \subseteq \Rd$, and responses
\begin{equation*}
Y_i = f_0(X_i) + \epsilon_i,
\end{equation*}
with regression function $f_0: \mc{X} \to \Reals$, and $\epsilon_i \sim N(0,1)$ independent Gaussian noise. For simplicity we will assume throughout that the noise has unit-variance, but all of our results extend in a straightforward manner to the case where the variance is equal to a known positive value. 

\textbf{(2) Laplacian eigenmaps.} Laplacian eigenmaps constructs an estimate of the regression function $f_0$ using the eigenvectors of a neighborhood graph Laplacian operator. For a positive, radially symmetric kernel $\eta: [0,\infty) \to [0,\infty)$, and a bandwidth parameter $\varepsilon > 0$, the \emph{neighborhood graph Laplacian} operator $L_{n,\varepsilon}$ is
\begin{equation}
\label{eqn:neighborhood_graph_laplacian}
L_{n,\varepsilon}u(x) = \frac{1}{n\varepsilon^{d + 2}} \sum_{i = 1}^{n} \bigl(u(x) - u(X_j)\bigr) \eta\biggl(\frac{\|x - X_j\|}{\varepsilon}\biggr)
\end{equation}
The constant scaling $(n\varepsilon^{d + 2})^{-1}$ is purely for convenience in taking limits as $n \to \infty, \varepsilon \to 0$. Also, although~\eqref{eqn:neighborhood_graph_laplacian} makes sense for any $x \in \mc{X}$ and $u: \mc{X} \to \Reals$, we will typically think of $x \in \{X_1,\ldots,X_n\}$ and $u = (u(X_1),\ldots,u(X_n)) \in \Leb^2({\bf X})$. The graph Laplacian is positive semi-definite, and we will index its eigenpairs $(\lambda_1,v_1),\ldots,(\lambda_n,v_n)$ in ascending order of eigenvalue, $0 = \lambda_1 \leq \ldots \leq \lambda_n$. 

For a given $K \in \{1,\ldots,n\}$, the order-$K$ Laplacian eigenmaps estimator simply projects the response vector ${\bf Y}$ onto the first $K$ eigenvectors of $L_{n,\varepsilon}$: letting $V_K \in \Reals^{n \times K}$ be the matrix with columns $v_1,\ldots,v_K$, we have that
\begin{equation}
\label{eqn:laplacian_eigenmaps_estimator}
\wh{f} := \sum_{k = 1}^{K} \dotp{Y}{v_k}_{n} v_k = \frac{1}{n} V_K V_K^{\top} Y.
\end{equation} 
Since the eigenvectors $v_1,\ldots,v_n$ are orthogonalized with respect to the $L^2(P_n)$ inner product, the estimator $\wh{f}$ is the least-squares solution to the linear regression problem with responses $Y_1,\ldots,Y_n$ and features $v_1,\ldots,v_K$. Note that $\wh{f}$ is defined only in-sample, that is, only at the design points $X_1,\ldots,X_n$---we will consider the question of out-of-sample estimation later in Section~\ref{sec:out_of_sample}. 

If $\wh{f}$ is a reasonable estimate of $f_0$, then the test statistic
\begin{equation}
\label{eqn:laplacian_eigenmaps_test}
\wh{T} := \|\wh{f}\|_n^2
\end{equation}
is in turn a reasonable estimate of $\|f_0\|_{P}^2$, and can be used to distinguish whether or not $f_0 = 0$.

\subsection{Sobolev Classes}
\label{sec:sobolev}
Before we give an overview of our main results, we will first pause to review the Sobolev classes and \textcolor{red}{associated} minimax rates.

\textbf{(1) Sobolev norms, semi-norms, and balls.} 
The Sobolev class $H^s(P)$ consists of all those functions $f \in \Leb^2(P)$ which have weak partial derivatives of all orders $j = 1,\ldots,s$, all of which are bounded in $\Leb^2(P)$ norm. The Sobolev norm $\|f\|_{H^s(P)}$ is defined according to
\begin{equation*}
\|f\|_{H^s(P)}^2 := \|f\|_{\Leb^2(P)}^2 + \sum_{j = 1}^{s} \|D^jf\|_{\Leb^2(P)}^2,
\end{equation*}
where $D^jf := \sum_{|\alpha| = j}D^{\alpha}f$ is the sum of all weak partial derivatives of order $j$, and we use the multiindex notation $D^{\alpha}f = \partial^j f/(\partial^{\alpha_1}x^1\ldots \partial^{\alpha_d}x^d)$ for multiindex $\alpha$. The Sobolev ball of radius $M$ consists of all those functions $f \in H^s(P)$ with Sobolev norm no greater than $M$,
\begin{equation*}
H^s(P;M) := \bigl\{f \in H^s(P): \|f\|_{H^s(P)} \leq M\bigr\}
\end{equation*}

\textcolor{red}{(TODO)}: 
\begin{itemize}
	\item Possibly define the iterated Laplace-Beltrami operator. Point out that for $H^{2s}(P)$ functions, the Sobolev semi-norm can be stated in terms of inner products using the iterated Laplace-Beltrami operator.
	\item Introduce the notation $H^s(\mc{X}) = H^s(\nu)$, where $\nu$ is Lebesgue measure.
\end{itemize}

\textbf{(2) Zero-trace and periodic conditions.}
When $s \geq 1$, our results regarding Laplacian eigenmaps will require some boundary assumptions on $f_0$. In particular, we will assume that $f_0$ belongs to the set of zero trace Sobolev functions $H_0^{s}(P)$, meaning \textcolor{red}{roughly} that $f_0 \in H^s(P)$ additionally satisfies,
\begin{equation}
\label{eqn:zero_trace}
D^jf(x) = 0,~~\textrm{for all $x \in \partial \mc{X}$ and $j = 0,\ldots,s - 1$.}
\end{equation}
\textcolor{red}{(TODO)}: This is in spirit right but technically incoherent unless $f \in C^{\infty}(\mc{X})$. Need to use the formal definition of trace.

Let us justify why we impose a boundary condition such as zero-trace. For this part only we concentrate on the special case of $d = 1$ and the domain $\mc{X} = [0,1]$. In this case, there exists another way to define Sobolev space, via a sequence space representation (equiv Fourier transform). Let $\psi_1,\psi_2,\ldots$ denote the trigonometric basis of $\Leb^2([0,1])$,\footnote{To be explicit, $\psi_1(x) := 1$, $\psi_{2k}(x) := \sqrt{2} \cos(2\pi k x)$ and $\psi_{2k + 1}(x) := \sqrt{2} \sin(2\pi k x)$} and let $\Theta \subset \ell^2(\mathbb{N})$ be the Sobolev ellipsoid $\Theta = \{\theta \in \ell^2(\mathbb{N}): \sum_{k = 1}^{\infty} k^{2s} \theta_k^2 \leq M\}$. The function class $\{f = \sum_{k = 1}^{\infty} \theta_k \psi_k: \theta \in \Theta\}$ is \textcolor{red}{equivalent to} the \emph{periodic Sobolev space} $H_{\mathrm{per}}^s([0,1])$, defined as
\begin{equation}
\label{eqn:periodic_sobolev_space}
H_{\mathrm{per}}^s([0,1]) = \Bigl\{f \in H^s([0,1],M): D^jf(0) = D^jf(1)~~\textrm{for $j = 0,\ldots, s - 1$} \Bigr\}
\end{equation}
It is known that when $P$ is uniform, the eigenvectors $v_k$ converge to $\psi_k$ as $n \to \infty, \varepsilon \to 0$, and it is therefore natural that they accurately approximate only periodic functions $f_0 \in H^s([0,1])$. Although the zero-trace boundary condition is somewhat more restrictive than mere periodicity, the point is that some kind of boundary condition is inherently necessary in order to obtain sharp rates for spectral series estimators. For simplicity, we will stick to the specific boundary condition~\eqref{eqn:zero_trace}. 

\subsection{Minimax Rates}
\label{subsec:minimax_rates_sobolev}
We now turn to reviewing the minimax estimation and goodness-of-fit testing rates over Sobolev classes. We will always be interested in measuring loss in mean-squared error, that is in squared $\Leb^2$-norm. In this case, standard non-parametric minimax rates hold under some regularity conditions on $\mc{X}$ and $P$. Specifically, we will assume the following.\footnote{Assumption~\ref{asmp:sobolev_radius} ensures that we get a non-parametric rate. If $M \lesssim n^{-1/2}$ then both the estimation and testing rates become the parametric $n^{-1}$.}
\begin{enumerate}[label=(A\arabic*)]
	\item 
	\label{asmp:domain}
	The domain $\mc{X}$ is an open connected subset of $\Rd$, with Lipschitz boundary.
	\item
	\label{asmp:density}
	The design distribution $P$ admits a density $p$ with respect to Lebesgue measure. The density $p$ is bounded away from $0$ and $\infty$: there exists $\rho \geq 1$ such that
	\begin{equation*}
	0 < \frac{1}{\rho} \leq p(x) \leq \rho < \infty,~~\textrm{for all $x \in \mc{X}$}
	\end{equation*}
	\item
	\label{asmp:sobolev_radius}
	The radius $M$ of the Sobolev class satisfies $M = M(n) \gtrsim n^{-1/2}$.
\end{enumerate}
Under assumptions~\ref{asmp:domain}-\ref{asmp:sobolev_radius}, the minimax estimation rate over $\mc{H} = H^s(\mc{X};M)$ or $\mc{H} = H_0^s(\mc{X};M)$) is (see e.g. \citet{tsybakov2008_book})
\begin{equation}
\label{eqn:minimax_estimation_rate}
\inf_{\wh{f}} \sup_{f_0 \in \mc{H}} \|f_0 - f\|_P^2 \asymp M^{2d/(2s + d)}n^{-2s/(2s + d)};
\end{equation}
here the infimum is over estimators $\wh{f}$. 

In the goodness-of-fit testing problem, we ask for a test function---formally, a Borel measurable function $\phi$ that takes values in $\{0,1\}$--- which can distinguish between the hypotheses
\begin{equation}
\mathbf{H}_0: f_0 = f_0^{\star}, ~~\textrm{versus}~~ \mathbf{H}_a: f_0 \in \mc{H} \setminus \{f_0^{\star}\}.
\end{equation} 
Typically, the null hypothesis $f_0 = f_0^{\star} \in \mc{F}$ reflects the absence of interesting structure, and $\mc{F} \setminus  \{f_0^{\star}\}$ is a set of smooth departures from this null. To fix ideas, as in \citet{ingster2009} we focus on the problem of \emph{signal detection} in Sobolev spaces, where $f_0^{\star} = 0$ and $\mc{H}$ is a Sobolev ball, $\mc{H} = H^s(\Xset,M)$ or $\mc{H} = H_0^s(\mc{X};M)$ is a Sobolev ball; this is without loss of generality.

The minimax critical radius is the smallest value of $\sigma$ such that some level-${\alpha}$ test $\phi$ has power at least $1 - \alpha$ over all $\mc{H}_{\sigma} := \mc{H} \cap \{f: \|f\|_{\Leb^2(P)} \geq \sigma\}$:
\begin{equation*}
\sigma(\mc{H}) := \inf\Biggl\{\sigma > 0: \inf_{\phi} \biggl[ \sup_{f_0 \in \mc{H}_{\sigma}} \Ebb_{f_0}[1 - \phi]\biggr] \leq \alpha\Biggr\}
\end{equation*} 
where in the above the infimum is over all level-$\alpha$ tests $\phi$, and $\Ebb_{f_0}[\cdot]$ is the expectation under the regression function $f_0$. Testing whether a regression function $f_0$ is equal to $0$ is an easier problem than estimating $f_0$, and so the minimax testing critical radius over $\mc{H}$ is much smaller than the minimax estimation rate \citep{ingster2009}:
\begin{equation}
\label{eqn:sobolev_space_testing_critical_radius}
\sigma^2(\mc{H}) \asymp M^{2d/(4s + d)}n^{-4s/(4s + d)}~~\textrm{for $1 \leq d < 4s$.}
\end{equation}
When $4s \geq d$ the functions in $H^s(\Xset)$ are very irregular---formally speaking $H^s(\Xset)$ does not \textcolor{red}{continuously} embed into $\Leb^4(\Xset)$ when $4s \geq d$---and the minimax testing rates in this regime are unknown.

\textcolor{red}{(TODO)}
\begin{itemize}
	\item Is there a better citation than \citet{tsybakov2008_book} for the minimax rate of estimation~\eqref{eqn:minimax_estimation_rate}? Tsybakov considers the univariate case with fixed, equispaced design.
	\item I copy pasted the testing part directly from our AISTATS paper, then made a few necessary changes. Is that bad?
	\item I have stated the rate in~\eqref{eqn:sobolev_space_testing_critical_radius} over $\mc{H}$. In reality,~\citep{ingster2009} make some kind of boundary assumptions and I cannot remember exactly what. Is it ok to state it as is, or do I need to be more precise?
\end{itemize}

\subsection{Spectral series methods for Sobolev Spaces}
Spectral series estimators---and tests using their $\Leb^2$-norm---are particularly designed for regression in Sobolev spaces. Let $f_1,f_2,\ldots$ be the unit norm eigenfunctions of $\Delta_P$. The classical order-$K$ spectral series estimator $\wt{f}$ and test statistic $\wt{T}$ are
\begin{equation*}
\wt{f} = \sum_{k = 1}^{K} \dotp{Y}{f_k}_n f_k~~\textrm{and}~~\wt{T} = \|\wt{f}\|_P^2.
\end{equation*}
The estimator $\wt{f}$ and a test based on $\wt{T}$ are minimax rate-optimal over $H_0^s(\mc{X})$, a fact intimately related to the series representation of periodic Sobolev spaces, e.g.~\eqref{eqn:periodic_sobolev_space}. 

Unfortunately, in practice the estimator $\wt{f}$ is usually unworkable. Typically one does not have direct knowledge of the design distribution $P$, and even if $P$ is known, the eigenfunctions of $\Delta_P$ are rarely easy to compute. As an alternative, suppose instead we had access only to eigenfunctions $f_1',f_2',\ldots$ of the unweighted Laplace-Beltrami operator $\Delta$: then, letting $F_K = \mathrm{span}\{f_1,\ldots,f_K\}$, the least-squares estimator and test statistic
\begin{equation*}
\wt{f}_{\mathrm{LS}} := \argmin_{f \in F_K} \|Y - f\|_n^2,\textrm{and}~~\wt{T}_{\mathrm{LS}} = \|\wt{f}\|_{\nu}^2
\end{equation*}
are still rate-optimal over $H_0^s(P)$. However, this is not a totally satisfactory fix. For one thing, the least squares approach still requires that we know $\Delta$, and is not well suited for the case where $\mc{X}$ is a manifold, and $\Delta$ is the manifold Laplace-Beltrami operator. Additionally, diagonalizing even the unweighted Laplace-Beltrami operator $\Delta$ is quite difficult for all but a few special domains, such as $\mc{X} = [0,1]^d$. The general message is that in order to be minimax optimal, classical spectral series and least-squares estimators require an unrealistic amount of information on the design, and can be implemented only in selective situations. 

In contrast, Laplacian eigenmaps regression uses the eigenvectors of the neighborhood graph Laplacian $L_{n,\varepsilon}$ as features. The neighborhood graph Laplacian can be constructed (basically) without any knowledge of $P$ or $\mc{X}$, and Laplacian eigenmaps is thus a practicable method for regression. Moreover, the eigenvectors of $L_{n,\varepsilon}$ are approximations to the eigenfunctions $f_1,f_2,\ldots$ of $\Delta_P$---as we will discuss momentarily---so that $\wh{f}$ and $\wh{T}$ can in turn be viewed as approximations to the $\wt{f}$ and $\wt{T}$, respectively. As this view suggests, Laplacian eigenmaps naturally incurs some additional error compared to traditional spectral series estimators---this is the price we pay for using an approximation of $\Delta_P$. Fortunately, despite this extra error Laplacian eigenmaps retains the favorable optimality properties of its classical spectral series counterparts while freeing us from needing unrealistic knowledge of $\Delta_P$, as now summarize. 

\subsection{Our contributions}
The following items summarize our major results. In all of them we assume~\ref{asmp:domain} and~\ref{asmp:density} hold, but not that any of $P$, $\mc{X}$, $\Delta_P$ or $\Delta$ are known. Put $r := s - 1$.
\begin{itemize}
	\item \textbf{Minimax optimal estimation.} For any $f_0 \in H^1(\Xset;M)$, with high probability $\norm{\wh{f} - f_0}_{n}^2 \lesssim M^{2d/(2 + d)}n^{-2/(2 + d)}$.
	\item \textbf{Minimax optimal testing.}
	A test constructed using $\wh{T}$ has non-trivial power whenever $f_0 \in H^1(\Xset;M)$ satisfies $\norm{f_0}_{P}^2 \gtrsim M^{2d/(2 + d)}n^{-4/(4 + d)}$ and $d < 4$.
	\item \textbf{Higher-order smoothness conditions.} When the regression function $f_0 \in H_0^s(\Xset;M)$ and additionally the density $p \in C^{r}(\mc{X};M)$, with high probability the error $\norm{\wh{f} - f_0}_{n}^2 \lesssim M^{2d/(2s + d)}n^{-2s/(2s + d)}$. Additionally, a test constructed using $\wh{T}$ has non-trivial power whenever $f_0 \in H_0^s(\Xset;M)$ satisfies $\norm{f_0}_{P}^2 \gtrsim M^{2d/(4s + d)}n^{-4s/(4s + d)}$ and $d < 4$.
	\item \textbf{Manifold adaptivity.}
	If $\mc{X} \subset \Rd$ is a manifold of dimension $m < d$, and if $H^s(\mc{X})$ for $s \leq 2$, then each of the aforementioned rates hold with $d$ replaced by $m$.
\end{itemize}
The above estimation results are stated with respect to in-sample mean squared error, meaning the loss is measured in the empirical squared-norm $\|\cdot\|_n^2$.\footnote{To state the obvious: this is very different than measuring the \emph{training error} $\|Y - \wh{f}\|_n^2$.} This is natural, as Laplacian eigenmaps is defined only at the observed design points $X_1,\ldots,X_n$. That being said, often one would like an upper bound on the prediction risk, or equivalently the out-of-sample loss $\|\cdot\|_P^2$. We therefore also introduce a simple, intuitive way of extending $\wh{f}$ to a function defined over all of $\mc{X}$, and show that this extension is minimax rate-optimal with respect to error measured in $L^2(P)$ norm.

\subsection{Related Work}
\begin{itemize}
	\item \textcolor{red}{(TODO)} Summarize what is known on fixed graphs.
	\item \textcolor{red}{Belkin, Slepcev} show that for a fixed $k$, the eigenvector $v_k$ converges to the eigenfunction $u_k$, in the sense that \textcolor{red}{(...)}. \textcolor{red}{Shi, Burago, Garcia Trillos, Calder, Cheng} build on this, giving finite sample bounds, rates of convergence, and making statements uniform over $1 \leq k \leq k_{\max}$. This justifies our previous discussion, in which we treat Laplacian eigenmaps as an approximation of spectral series methods. However, we emphasize that the minimax optimality of $\wh{f}$ or a test based on $\wh{T}$ does not follow straightforwardly from such results. Instead our analysis proceeds along different lines, which we detail in  Section~\ref{sec:analysis}.
	\item There has been limited analysis of the estimator $\wh{f}$. \cite{zhou2011} consider a similar estimator, but in the semi-supervised setting where the number of unlabeled points grows to infinity. In this case the estimator reduces to a classical spectral series estimator, and there is no error incurred by approximating $\Delta_P$. \cite{lee2016} consider a similar estimator---using eigenvectors of a different normalization of the Laplacian $L$---in both the supervised and semi-supervised setups, but get suboptimal rates for the supervised problem. As far as we know, there has been no analysis of the test statistic $\wh{T}$ in the random design framework which we study.
	\item \textcolor{red}{(TODO)}: Alternatively, \cite{trillos2020} \textcolor{blue}{(Green 2021)} use the graph Laplacian operator to induce a penalty over functions $f: \{\bf X\} \to \Reals$, and study a regularized estimator estimator, which minimizes the penalized loss
	\begin{equation*}
	\|Y - f\|_n^2 + \lambda \dotp{L_{n,\varepsilon}f}{f}_n.
	\end{equation*}
	\textcolor{blue}{(Green 2021)} show that the resulting estimator is (nearly)-minimax optimal, but only for $s = 1$ and $d \leq 4$. In contrast, the Laplacian eigenmaps estimator is optimal for all $s$ and $d$. 
\end{itemize}

\section{Minimax Optimality of Laplacian Eigenmaps}
\label{sec:minimax_optimal_laplacian_eigenmaps}

In this section, we will show that the estimator $\wh{f}$, and a test using the statistic $\wh{T}$, achieve optimal estimation and goodness-of-fit testing rates over Sobolev classes. We will divide our statements based on whether $f_0$ belongs to the first order Sobolev class ($s = 1$) or a higher-order Sobolev class ($s > 1$), since the details of the two settings are somewhat different. Throughout this section, we will always assume~\ref{asmp:domain} and~\ref{asmp:density}.

\subsection{First-order Sobolev classes}
\label{sec:first_order_sobolev_classes}
To begin, we handle the case where $f_0 \in H^1(P)$. We show that $\wh{f}$ and a test based on $\wh{T}$ are minimax optimal, for all values of $d$, and under no additional assumptions on the data generating process, i.e. on either $P$ or $f_0$. 

\paragraph{Estimation.} When the graph radius $\varepsilon$ and number of eigenvectors $K$ are chosen appropriately, the estimator $\wh{f}$ has (in-sample) optimal risk (up to constant factors) $\|\wh{f} - f_0\|_n^2 \lesssim n^{-2/(2 + d)}$. 
\begin{enumerate}[label=(A\arabic*)]
	\setcounter{enumi}{3}
	\item 
	\label{asmp:parameters_estimation_fo} 
	The graph radius $\varepsilon$ and the number of eigenvectors $K$ satisfy the following inequalities:
	\begin{equation}\\
	\label{eqn:radius_fo} 
	C_0 \biggl(\frac{\log n}{n}\biggr)^{1/d} \leq \varepsilon \leq \min\{i_0,K^{-1/d}\},
	\end{equation}
	and 
	\begin{equation}
	\label{eqn:eigenvector_estimation_fo} 
	c_1 (M^2 n)^{2d/(2 + d)}\leq K \leq C_1 (M^2 n)^{2d/(2 + d)}.
	\end{equation}
\end{enumerate}
\begin{theorem}
	\label{thm:laplacian_eigenmaps_estimation_fo}
	Suppose $f_0 \in H^1(P,M)$. If the Laplacian eigenmaps estimator $\wh{f}$ is computed with parameters $\varepsilon$ and $K$ that satisfy~\ref{asmp:parameters_estimation_fo}, then the following statement holds for any $\delta \in (0,1)$: with probability at least $1 - \delta - \textcolor{red}{(???)}$,
	\begin{equation}
	\label{eqn:laplacian_eigenmaps_estimation_fo}
	\|\wh{f} - f_0\|_n^2 \leq \frac{C}{\delta}M^{2d/(2 + d)}n^{-2/(2 + d)}
	\end{equation}
\end{theorem}
Some remarks:
\begin{itemize}
	\item Regarding the restrictions on hyperparameters: the lower bound on $\varepsilon$ is the connectivity threshold, the smallest length scale at which the resulting graph will still, with high probability, be connected. On the other hand, as we will see in Section~\ref{subsec:analysis}, the upper bound on $\varepsilon$ is needed to ensure that the graph eigenvalue $\lambda_K \asymp \lambda_K(\Delta_P) \asymp K^{2/d}$. Finally, the restriction $K \asymp (M^2n)^{2d/(2 + d)}$ is chosen  to optimally trade-off bias and variance. Together, these restrictions suggest that the 
	\item \textbf{(1.1) How to tune hyper-parameters.} We note that the ranges~\eqref{eqn:radius_fo} and~\eqref{eqn:eigenvector_estimation_fo} depend on unknown quantities such as the dimension $d$ and radius of the Sobolev ball $M$. In practice, one typically tunes hyper-parameters by sample-splitting or cross-validation. However, because the estimator $\wh{f}$ is defined only in-sample, such approaches cannot be straightforwardly applied to select the graph radius $\varepsilon$, or number of eigenvectors $K$. We return to this issue upon considering out-of-sample extensions of $\wh{f}$ later, in Section~\ref{sec:out_of_sample}.
	\item \textbf{(1.2) The dependence on failure probability can be improved.} The upper bound on in-sample risk given by Equation~\eqref{eqn:laplacian_eigenmaps_estimation_fo} depends on the pre-factor $C/\delta$, and holds with probability $1 - \delta - \textcolor{red}{(?)}$. When $f_0 \in C^1(\mc{X};M)$, we can improve these dependencies, changing the pre-factor in~\eqref{eqn:laplacian_eigenmaps_estimation_fo} to $C(1 + 1/\delta)$, and showing that the bound holds with probability at least $1 - C\delta^2/n$. 
\end{itemize}

\textbf{(2) Laplacian eigenmaps is an optimal test.} Consider the test $\varphi = \1\{\wh{T} \geq t_{\delta}\}$, where $t_{\delta}$ is the threshold
\begin{equation*}
t_{\delta} := \frac{K}{n} + \frac{2K}{\sqrt{n}}.
\end{equation*}
The choice of $t_{\delta}$ guarantees that $\varphi$ is a level-$\delta$ test. Under appropriate choices of $\varepsilon$ and $K$, the test $\varphi$ has non-negligible power (up to constant factors) against alternatives separated the null by at least $\|f_0\|_{P}^2 \gtrsim M^{2d/(4 + d)}n^{-4/(4 + d)}$, whenever $d < 4$. 

\begin{enumerate}[label=(A\arabic*)]
	\setcounter{enumi}{4}
	\item 
	\label{asmp:parameters_testing_fo}
	The graph radius $\varepsilon$ and the number of eigenvectors $K$ satisfy~\eqref{eqn:radius_fo}. Additionally,
	\begin{equation}
	\label{eqn:eigenvector_testing_fo}
	c_2 (M^2 n)^{2d/(4 + d)}\leq K \leq C_2 (M^2 n)^{2d/(4 + d)}.
	\end{equation}
\end{enumerate}
Let $p(d) = \min\{1,2/d\}$.
\begin{theorem}
	\label{thm:laplacian_eigenmaps_testing_fo}
	Suppose $f_0 \in H^1(\mc{X};M)$. If the Laplacian eigenmaps test $\varphi$ is computed with parameters $\varepsilon$ and $K$ that satisfy~\ref{asmp:parameters_testing_fo}, then the following statement holds for any $\delta \in (0,1)$: if $f_0$ satisfies
	\begin{equation}
	\label{eqn:laplacian_eigenmaps_testing_criticalradius_fo}
	\|f_0\|_P^2 \geq \frac{C}{\delta} \max\biggl\{\frac{1}{\delta}M^{2d/(4 + d)}n^{-4/(4 + d)}, M n^{-p(d)}\biggr\},
	\end{equation}
	then $\varphi$ has Type II error upper bounded by 
	\begin{equation}
	\Pbb_{f_0}(\varphi = 0) \leq \delta + \textcolor{red}{(...)}
	\end{equation}
\end{theorem}
Some remarks:
\begin{itemize}
	\item \textbf{(2.1) When $d \geq 4$, the testing problem is different.} When $d \geq 4$, it is not the case that $H^1(\mc{X})$ continuously embeds into $\Leb^4(\mc{X})$, and as far we know the optimal testing rates over $H^1(\mc{X})$ are not known. On the other hand, if we explicitly assume $f_0 \in \Leb^4(\mc{X})$, then simply taking $T = \|Y\|_n^2$ achieves the optimal rates.
	\item \textbf{(2.2) Both of the remarks after Theorem~\ref{thm:laplacian_eigenmaps_estimation_fo} also apply to testing.}
\end{itemize}

\subsection{Higher-order Sobolev classes}
\label{sec:higher_order_sobolev_classes}
In this section, we now assume that the regression function displays some higher-order regularity, $f_0 \in H_0^s(\mc{X})$. We point out that on a graph there is no natural analogue to higher-order derivatives. For this reason it is not at all obvious that graph-based methods \emph{can} adapt to higher-order smoothness. Remarkably, we show that this is in fact the case. Under appropriate choices of $\varepsilon$ and $K$, the estimator $\wh{f}$ has (in-sample) optimal error (up to constant factors) $\|\wh{f} - f_0\|_n^2 \lesssim n^{-2s/(2s + d)}$. 

\begin{enumerate}[label=(A\arabic*)]
	\setcounter{enumi}{5}
	\item 
	\label{asmp:parameters_estimation_ho}
	The graph radius $\varepsilon$ and number of eigenvectors $K$ satisfy
	\begin{equation*}
	C_0 \cdot \max\biggl\{\biggl(\frac{\log}{n}\biggr)^{1/d}, n^{-1/(2r + d)}\biggr\} \leq \varepsilon \leq \min\{i_0, K^{-1/d}\}
	\end{equation*}
	and
	\begin{equation*}
	K \asymp (M^2n)^{2d/(2s + d)}
	\end{equation*}
\end{enumerate}

\begin{theorem}
	\label{thm:laplacian_eigenmaps_estimation_ho}
	Under appropriate conditions, with probability at least $1 - \delta$,
	\begin{equation}
	\label{eqn:laplacian_eigenmaps_estimation_ho}
	\|\wh{f} - f_0\|_n^2 \lesssim n^{-2s/(2s + d)}
	\end{equation}
\end{theorem}
Theorem~\ref{thm:laplacian_eigenmaps_estimation_ho}, in combination with Theorem~\ref{thm:laplacian_eigenmaps_estimation_fo}, implies that in the flat Euclidean setting Laplacian eigenmaps is a rate-optimal estimator over Sobolev classes, for all scalings of $s$ and $d$. Some other remarks:
\begin{itemize}
	\item \textbf{(1.1) Optimal rates, no RKHS.} It is worth pointing out that we do not require that the regularity of the Sobolev space satisfy $2s > d$, a condition often seen in the literature. In the ``sub-critical'' regime $2s < d$, the Sobolev space $H^s(\mc{X})$ is quite irregular. For instance, it is not a Reproducing Kernel Hilbert Space, nor does it embed into any H\"{o}lder space. As a result, certain versions of the nonparametric regression problem are ill-posed---for instance when loss is measured in $\Leb^{\infty}$ norm, or when the design points $\{X_1,\ldots,X_n\}$ are assumed to be fixed. Likewise, certain estimators are ``off the table'', most notably RKHS-based methods such as smoothing splines. Nevertheless, for random design regression with error measured in $\Leb^2(P)$-norm, spectral series estimators obtain the minimax rates for all values of $s$ and $d$. Theorem~\ref{thm:laplacian_eigenmaps_estimation_ho} shows that the same is true with respect to Laplacian eigenmaps, with error measured in $\Leb^2(P_n)$ norm.
	\item \textbf{(1.2) The zero-trace condition.} The zero-trace condition can likely be weakened. However, requiring some boundary condition on $f_0$ and its first through $(k -1 )$th-order derivatives is unavoidable. \textcolor{red}{Explain why: graphs are implicitly boundaryless objects, and the eigenvectors of $L$ converge to those eigenvectors of $\Delta_P$ which satisfy boundary conditions. Repeat that this is a standard assumption made when analyzing spectral series estimators}.
	\item \textbf{(1.3) The bounded norm and smooth density conditions.} We require that $f_0$ be bounded in the Sobolev norm $\|f_0\|_{H^s(\mc{X})}$---as opposed to the semi-norm $|f_0|_{H^s(\mc{X})}$--- and that $p \in C^{s - 1}(\mc{X})$. Both requirements are related and fundamental, as we shall now explain. As previously mentioned, $L$ converges to a weighted-version of the Laplace operator $\Delta_P$ (also known as the Fokker-Planck) operator. The operator $\Delta_P$ includes both first and second-order derivatives,
	\begin{equation}
	\label{eqn:fokker_planck_1}
	\Delta_Pf= \frac{-1}{p} \mathrm{div}(p^2 \nabla f) = \Delta f - 2\frac{\nabla p^{\top} \nabla f}{p},
	\end{equation}
	and therefore a function $f \in H_0^s(\mc{X})$ has finite energy $\|\Delta_P f\|_{P}^2$ only if the first-order partial derivatives of $f$ and $p$ are bounded. \textcolor{red}{This discussion---and in particular the last sentence---are obviously not quite correct. This is in part because you realize now that the requirements are not in fact fundamental. Instead, the requirement should merely be that $\|\Delta_P^kf\|_{\Leb^2(P)}^2$ is bounded---here $k = s/2$ and we assume $s$ is even---and the upper bounds should be in terms of this semi-norm.}.
	\item \textbf{(1.4) Both of the remarks after Theorem~\ref{thm:laplacian_eigenmaps_estimation_fo} continue to apply.}
\end{itemize}

\textbf{(1.4) Rates for testing with higher-order smoothness assumptions.} For testing, we can get the optimal rate $n^{-4s/(4s + d)}$ when $d \leq 4$. When $4 < d \leq 4s$, we get the rate~\textcolor{red}{(...)} Recall that the rate $n^{-4s/(4s + d)}$ is optimal whenever $d \leq 4s$. Thus when $4s > d$, our upper bounds do not imply that $\varphi$ is an optimal test. We do not know whether this is due to looseness in our proof techniques, or a problem with the test itself, and leave this for future work.

\subsection{Analysis}
\label{subsec:analysis}

\textbf{(1) Bias and variance terms, for estimation and testing}. We analyze the estimation error of $\wh{f}$, and the testing error of $\wh{\varphi}$, by first conditioning on the design points $X$ and deriving \emph{design-dependent} bias and variance terms. For estimation, we have that with high probability, conditional on the design $X$,
\begin{equation*}
\|\wh{f} - f_0\|_n^2 \leq C\frac{\dotp{L^s f_0}{f_0}_n}{\lambda_{\kappa}^s} + \frac{K}{n}
\end{equation*}
For testing, we have~\textcolor{red}{(...)}.

The semi-norm $\dotp{L^s f_0}{f_0}_n$ and eigenvalue $\lambda_{K}$ are random variables that depend the random design points $X_1,\ldots,X_n$. It remains to establish suitable upper and lower bounds on these quantities. 

\textbf{(2) Upper bound on the semi-norm. } Proposition~\ref{prop:graph_seminorm} gives a sufficient upper bound on $\dotp{L^s f_0}{f_0}_n$. Take $\beta := s - 1$. 
\begin{proposition}
	\label{prop:graph_seminorm} 
	Suppose $f \in H_0^s(\mc{X})$, that $p \in C^{\beta}(\mc{X})$, and that $r \gtrsim n^{-1/d}$. Then with probability at least $1 - \delta$, it holds that 
	\begin{equation}
	\label{eqn:graph_seminorm_1}
	\dotp{L^s f}{f}_n \leq \frac{C}{\delta} \|f\|_{H^s} \|p\|_{C^{\beta}} + \textcolor{red}{(?)}
	\end{equation}
	Therefore, if additionally $r \gtrsim n^{-1/(2\beta + d)}$, then 
	\begin{equation}
	\label{eqn:graph_seminorm_2}
	\dotp{L^s f}{f}_n \leq \frac{C}{\delta} \|f\|_{H^s} \|p\|_{C^{\beta}}
	\end{equation}
	When $s = 1$, the results hold for $f \in H^1(\mc{X})$ and any density $p$ that satisfies Assumption~\ref{asmp:density}.
\end{proposition}
The proof of Proposition~\ref{prop:graph_seminorm} involves two steps. In the first step, we show that with probability at least $1 - \delta$, the graph semi-norm $\dotp{L^s f}{f}_n$ is upper bounded by a non-local continuum energy $E_{\varepsilon}^{(s)}(f)$, defined as
\begin{equation*}
E_{\varepsilon}^{(s)}(f) := \dotp{L_{P,\varepsilon}^sf}{f}_{P}
\end{equation*}
where $L_{P,\varepsilon}$ is a non-local approximation to $\Delta_P$, 
\begin{equation}
\label{eqn:nonlocal_laplacian}
L_{P,\varepsilon}f(x) := \frac{1}{\varepsilon^{d + 2}}\int_{\mc{X}}\bigl(f(z) - f(x)\bigr) \eta\biggl(\frac{\|z - x\|}{\varepsilon}\biggr) \,dP(x).
\end{equation}
In the second step, we show that $E_{\varepsilon}^{(s)}(f)$ is in turn upper bounded by a constant times $D_s(f) := \|f\|_{H^s(P)}^2$. Together, these steps establish Proposition~\ref{prop:graph_seminorm}.

When $s = 1$, similar results have been shown in \textcolor{red}{(?)}. On the other hand, much less work has been done in the $s > 1$ case, where the analysis is more complicated. For one thing, we note that when $s > 1$ the graph energy $\dotp{L^s f}{f}_n$ is itself a biased estimate of the non-local energy $E_{\varepsilon}^{(s)}(f)$. This bias term is the second term on the right hand side of~\eqref{eqn:graph_seminorm_1}, and is neglible only when $r \gtrsim n^{-1/(2r + d)}$. On the other hand, in order to give an estimate of $E_{\varepsilon}^{(s)}(f)$ in terms of the Sobolev norm $D_s(f)$, we first show that $L_{P,\varepsilon}^jf(x) \approx \Delta_P^jf(x)$, for $j = s/2$ when $s$ is odd and $j = (s - 1)/2$ when $s$ is even. In words, we show that a $j$th-order iterated difference operator approximates a $2j$-th (or $2j + 1$-th) order differential operator.  In to establish this fact, we require that on the one hand $p \in C^{r}(\mc{X};M)$, and on the other hand $f$ be zero-trace.

\textbf{(3) Lower bound on the eigenvalue.} On the other hand, recent work \textcolor{red}{(Burago14, Shi16, Garcia Trillos 18)} has analyzed the convergence of $\lambda_{k}$ towards $\lambda_{k}(\Delta_P)$. They provide explicit bounds on the relative error $|\lambda_{k} - \lambda_{k}(\Delta_P)|/\lambda_{k}(\Delta_P)$, showing that the relative error is small for sufficiently large $n$, small $\varepsilon$, and any $1 \leq k \leq n$ that is not too large relative to $n$. These results are actually stronger than are necessary to establish Theorems~\ref{thm:laplacian_eigenmaps_estimation_fo}-\ref{thm:laplacian_eigenmaps_estimation_ho}---in order to get rate-optimality, we need only show that $\lambda_{K}/\lambda_K(P)$ is $\Omega_P(1)$---but unfortunately they all assume $P$ is supported on a manifold without boundary. We will instead use the results of \textcolor{red}{(Green 2021)}, whose assumptions match our own, and who give a weaker bound on $\lambda_k/\lambda_k(\Delta_P)$ that will nevertheless suffice for our purposes. 

\begin{proposition}
	\label{prop:graph_eigenvalue}
	\textcolor{red}{(TODO)}
	With probability at least $1 - \textcolor{red}{(?)}$,
	\begin{equation*}
	\lambda_k \geq c \cdot \min\Bigl\{\lambda_k(\Delta_P), \frac{1}{r^{2}} \Bigr\}
	\end{equation*}
\end{proposition}
By Weyl's Law, $c k^{2/d} \leq \lambda_{k}(\Delta_P) \leq Ck^{2/d}$, and as a result $\lambda_{K} \geq C\lambda_{K}(\Delta_P)$ as long as $r \geq \kappa^{-1/d}$. 

\textbf{(1.2) Traps we avoid.} It is tempting to use recent results regarding spectral convergence to analyze the estimator $\wh{f}$ and test $\wh{T}$. This suffers from an accumulation of error problem. 

It is also key that we directly analyze the semi-norm $\dotp{L^s f}{f}_n$, rather than invoking the pointwise convergence of $L^{s}f \to \Delta_P^{s}f$ to obtain a bound on $\dotp{L^s f}{f}_n$. Recall that we assume $f$ has bounded derivatives up to (and including) order $s$. This suffices to gain control of the semi-norm $\dotp{L^s f}{f}_n$, because $\dotp{L^s f}{f}_n$ approximates the $\Leb^2$ norm of an order $s$ differential operator, and so we have exactly as many derivatives as we need. On the other hand, $L^s$ approximates an order $2s$ differential operator, and $\Delta_P^sf$ is only a well-defined limiting object if we assume $f$ has order $2s$ derivatives. 

\textcolor{red}{(TODO)}
\begin{itemize}
	\item Possibly turn the first part of the discussion after Proposition~\ref{prop:graph_seminorm} into a small proof.
\end{itemize}


\section{Manifold Adaptivity}
\label{sec:manifold_adaptivity}

The optimal rates for nonparametric regression suffer from a bona fide curse of dimensionality, and are thus difficult to reconcile with the practical success of many methods for nonparametric regression. One explanation that resolves this seeming paradox is the \emph{manifold hypothesis}, where the design points $X$ are assumed to lie on or near a manifold $\mc{X}$ of intrinsic dimension $m \ll d$. Under the manifold hypothesis nonparametric regression is demonstrably easier, and in particular it is known \citep{bickel2007,ariascastro2018} that in this setting the optimal rates over $H^s(\Xset)$ scale like $n^{-2s/(2s + m)}$ (for estimation) and $n^{-4s/(4s + m)}$ (for testing). 

On the other hand, a theory has been developed~\citep{belkin03,belkin05,niyogi2013} establishing the the neighborhood graph $G_{n,r}$ can ``learn'' the manifold $\Xset$ in various senses, so long as $\Xset$ is locally linear. In this section, we contribute to this line of work by showing that under the manifold hypothesis, Laplacian eigenmaps achieve the sharper minimax estimation and testing rates. First, we give some relevant background on Riemmanian manifolds, and Sobolev classes defined upon them.

\subsection{Riemmanian manifolds}
Throughout this section, we will assume that the support $\mc{X}$ of the distribution $P$ is a manifold of dimension $m$ embedded in $\Rd$. We give to $\mc{X}$ the Riemannian structure induced by the ambient space $\Rd$, and denote the resulting volume form by $d\mu$.  We will also assume some regularity conditions on $\mc{X}$ and $P$. Recall that the injectivity radius of $\mc{X}$ is the maximum value of $\delta$ such that the exponential map $\exp_x: B_m(0,\delta) \subset T_x(\mc{X}) \to B_{\mc{X}}(x,\delta) \subset \mc{X}$ is a diffeomorphism for all $x \in \mc{X}$.
\begin{enumerate}[label=(A\arabic*)]
	\setcounter{enumi}{4}
	\item 
	\label{asmp:domain_manifold} The manifold $\mc{X}$ is closed, connected, smooth and boundaryless. Additionally, the injectivity radius of $\mc{X}$ is at least $i_0 > 0$.
	\item 
	\label{asmp:density_manifold} The distribution $P$ admits a density $p$ with respect to the volume form $d\mu$. The density is bounded away from $0$ and $\infty$,
	\begin{equation*}
	0 < \frac{1}{\rho} < p(x) \leq \rho < \infty
	\end{equation*}
	for all $x \in \mc{X}$.
\end{enumerate}
The definition of the Sobolev space $H^s(P)$, and the normed ball $H^s(P;M)$ are the same, mutatis mutandi, as when $\mc{X}$ is full dimensional. 

\subsection{Error rates under the manifold hypothesis}
We now show that Laplacian eigenmaps achieves the faster minimax rates under the manifold hypothesis, when the regression function $f_0 \in H^1(P;M)$ or $f_0 \in H^2(P;M)$. This section will proceed in a similar fashion to~\ref{sec:higher_order_sobolev_classes}, except with the ambient dimension $d$ replaced by intrinsic dimension $m$. 

\paragraph{Estimation.}
To obtain a minimax optimal estimator, we choose the graph radius $\varepsilon$ and number of eigenvectors $K$ as in~\ref{asmp:parameters_estimation_ho}, except with ambient dimension $d$ replaced by the intrinsic dimension $m$.

\begin{enumerate}[label=(A\arabic*)]
	\setcounter{enumi}{5}
	\item 
	\label{asmp:parameters_estimation_manifold}
	The graph radius $\varepsilon$ and number of eigenvectors $K$ satisfy
	\begin{equation}
	\label{eqn:radius_estimation_manifold}
	C_0 \cdot \max\biggl\{\biggl(\frac{\log}{n}\biggr)^{1/m}, n^{-1/(2r + m)}\biggr\} \leq \varepsilon \leq \min\{i_0, K^{-1/m}\}
	\end{equation}
	and
	\begin{equation*}
	K \asymp (M^2n)^{2m/(2s + m)}
	\end{equation*}
\end{enumerate}

\begin{theorem}
	\label{thm:laplacian_eigenmaps_estimation_manifold}
	Suppose $f_0 \in H^s(\mc{X},M)$, for $s \leq 4$. Under appropriate conditions, with probability at least $1 - \delta - \textcolor{red}{(?)}$,
	\begin{equation}
	\label{eqn:laplacian_eigenmaps_estimation_manifold}
	\|\wh{f} - f_0\|_n^2 \lesssim M^{2m/(2s + m)}n^{-2s/(2s + m)}
	\end{equation}
\end{theorem}

\paragraph{Testing.}
To construct a minimax optimal test using $\wh{T}$, we choose $\varepsilon$ and $K$ as in~\ref{asmp:parameters_testing_fo}, except with the ambient dimension $d$ replaced by the intrinsic dimension $m$.
\begin{enumerate}[label=(A\arabic*)]
	\setcounter{enumi}{5}
	\item 
	\label{asmp:parameters_testing_manifold}
	The graph radius $\varepsilon$ and number of eigenvectors $K$ satisfy~\eqref{eqn:radius_estimation_manifold}, for $r = 0$. Additionally, there exist constants $c_2$ and $C_2$ such that
	\begin{equation*}
	c_2 (M^2n)^{2m/(4s + m)} \leq K \leq C_2 (M^2n)^{2m/(4s + m)}
	\end{equation*}
\end{enumerate}

\begin{theorem}
	\label{thm:laplacian_eigenmaps_testing_manifold}
	Suppose $f_0 \in H^1(\mc{X};M)$. If the Laplacian eigenmaps test $\varphi$ is computed with parameters $\varepsilon$ and $K$ that satisfy~\ref{asmp:parameters_testing_manifold}, then the following statement holds for any $\delta \in (0,1)$: if $f_0$ satisfies
	\begin{equation}
	\label{eqn:laplacian_eigenmaps_testing_criticalradius_manifold}
	\|f_0\|_P^2 \geq \frac{C}{\delta} \max\biggl\{\frac{1}{\delta}M^{2m/(4 + m)}n^{-4/(4 + m)}, M n^{-p(m)}\biggr\},
	\end{equation}
	then $\varphi$ has Type II error upper bounded by 
	\begin{equation}
	\Pbb_{f_0}(\varphi = 0) \leq \delta + \textcolor{red}{(...)}
	\end{equation}
\end{theorem}

\begin{itemize}
	\item \textbf{Proof strategy.} The proofs of Theorems~\ref{thm:laplacian_eigenmaps_estimation_manifold} and~\ref{thm:laplacian_eigenmaps_testing_manifold} follow very similarly to the full-dimensional setting. The difference is that when $\mc{X}$ is a manifold with intrinsic dimension $m$, we can prove analogous results to Propositions~\ref{prop:graph_seminorm} and~\ref{prop:graph_eigenvalue}, but with the ambient dimension $d$ replaced by the intrinsic dimension $m$. 
	\item \textbf{Restriction on number of derivatives.} Unlike in the full-dimensional case, our upper bound on the estimation error of $\wh{f}$ matches the minimax rate $n^{-2s/(2s + m)}$ only when $s \leq 4$. At a high level, thinking of the graph $G$ as an estimate of the manifold $\mc{X}$, we incur some error by using Euclidean distance rather than geodesic distance to form the edges of $G$. By contrast, in the full-dimensional setting the Euclidean metric exactly coincides with the geodesic distance for all points $x,z \in \mc{X}$ that are sufficiently close to each other and far from the boundary of $\mc{X}$). This extra error incurred in the manifold setting dominates when $s > 4$. We give a more detailed explanation of this phenomenon in Appendix~\ref{sec:manifold_proofs}.
	\item \textbf{Comparison with traditional spectral series estimators.} Traditional spectral series estimators achieve the minimax rate for all values of $s$ and $m$. On the other hand, as we have already discussed, their default implementation assumes a priori knowledge of the domain $\mc{X}$, which is arguably a particularly unrealistic assumption when $\mc{X}$ is assumed to be a manifold. It is not clear whether this gap between spectral series and Laplacian eigenmaps estimators---or more generally, between estimators which assume strong knowledge of the manifold, and those which do not---is real, or a product of loose upper bounds. 
\end{itemize}

\section{Out-of-sample error}
\label{sec:out_of_sample}
Sections~\ref{sec:minimax_optimal_laplacian_eigenmaps} and~\ref{sec:manifold_adaptivity} show that Laplacian eigenmaps is an optimal method for nonparametric regression. As explained previously, we have measured estimation error in $\|\cdot\|_n$ norm---in words, the average squared loss incurred over the $n$ observed design points--- whereas error measured in $L^2(P)$ norm is the more typical metric in the random design setup.

Of course, the Laplacian eigenmaps estimator is only defined at the observed design points $X_1,\ldots,X_n$, and to measure its error in $L^2(P)$ norm we must first extend it to the rest of $\Xset$. We propose a simple method, essentially kernel smoothing, to do the job, which can be applied to any estimator defined at the design points, including Laplacian eigenmaps. We show that the smoothed version of $\wh{f}$ has optimal $L^2(P)$ error. 

\subsection{Extension using Kernel Smoothing}

\subsection{Out-of-sample error of Kernel Smoothed Laplacian Eigenmaps}

\begin{theorem}
	\label{thm:kernel_smoothing}
	\textcolor{red}{(TODO)}
\end{theorem}

Some remarks:
\begin{itemize}
	\item \textbf{Tuning parameters by sample-splitting.}
	\item \textbf{Other extensions.} Comment that there does not exist any canonical way to extend Laplacian based estimators such as $\wh{\theta}$ to an ambient domain. Point out that there exist various proposals to extend eigenvectors in the literature, such as $1$-nearest-neighbor regression (piecewise constant extrapolation over Voronoi cells), convolution with a specialized kernel, or Nystr\"{o}m extension. Explain that each of these is designed for a highly specialized purpose, such as suitability for mathematical analysis or computational efficiency, and in any case are suggested only for eigenvectors rather than a regression estimate. State explicitly that we choose kernel smoothing for its simplicity, ubiquity, and strong theoretical properties under our assumptions, but that considerations of extensions for specifically tailored to Laplacian based estimators could be of interest.
\end{itemize}


\subsection{Why not just do kernel smoothing?}
On the other hand, the allusion to kernel smoothing raises a very natural question: why should we prefer to first compute $\wh{\theta}$ and then smooth it out, as opposed to simply performing kernel smoothing over the original responses $Y$?

We now answer this question, by \textcolor{red}{(...)}.

\section{Experiments}
\label{sec:experiments}

\textcolor{red}{(TODO)}: Come up with a set of experiments that might be interesting to run. Run them by Ryan and Siva, so that you can prioritize.

\section{Discussion}
\label{sec:discussion}

\textbf{(1) Future items:}
\begin{itemize}
	\item \textbf{()}
\end{itemize}


\bibliographystyle{plainnat}
\bibliography{../../graph_regression_bibliography} 

\appendix

\noindent 

\section{Graph-dependent error bounds}
\label{sec:fixed_graph_error_bounds}
In this section, we adopt the fixed design perspective; or equivalently, condition on $X_i = x_i$ for $i = 1,\ldots,n$. Let $G = \bigl([n],W\bigr)$ be a fixed graph on $\{1,\ldots,n\}$ with Laplacian matrix $L = \sum_{k = 1}^{n}\lambda_k v_k v_k^{\top}$; the eigenvectors have unit empirical norm, $\|v_k\|_n^2 = 1$. The randomness thus all comes from the responses 
\begin{equation}
\label{eqn:fixed_graph_regression_model}
Y_i = f_{0}(x_i) + w_i
\end{equation}
where the noise variables $w_i$ are independent $N(0,1)$. In the rest of this section, we will mildly abuse notation and write $f_0 = (f_0(x_1),\ldots,f_0(x_n)) \in \Reals^n$. We will also write ${\bf Y} = (Y_1,\ldots,Y_n)$.

\subsection{Upper bound on Estimation Error of Laplacian Eigenmaps}

\begin{lemma}
	\label{lem:fixed_graph_estimation}
	For any integer $s > 0$, and any integer $0 \leq K \leq n$, the Laplacian eigenmaps estimator $\wh{f}$ of~\eqref{eqn:laplacian_eigenmaps_estimator} satisfies
	\begin{equation}
	\label{eqn:fixed_graph_estimation}
	\|\wh{f} - f_0\|_n^2 \leq \frac{\dotp{L^sf_0}{f_0}_n}{\lambda_{K + 1}^s} + \frac{5K}{n};
	\end{equation}
	this is guaranteed if $K = 0$, and otherwise holds with probability at least $1 - \exp(-K)$ if $1 \leq K \leq n$. 
\end{lemma}
\paragraph{Proof (of Lemma~\ref{lem:fixed_graph_estimation}).}
	By the triangle inequality,
	\begin{equation}
	\label{pf:fixed_graph_estimation_1}
	\|\wh{f} - f_0\|_n^2 \leq 2\Bigl(\|\mathbb{E}\wh{f} - f_0\|_n^2 + \|\wh{f} - \mathbb{E}\wh{f}\|_n^2\Bigr).
	\end{equation}
	The first term in~\eqref{pf:fixed_graph_estimation_1} (approximation error) is non-random, since the design is fixed. The expectation $\mathbb{E}\wh{f} = \sum_{k = 1}^{K} \dotp{v_k}{f_0}_n v_k$, so that
	\begin{equation*}
	\|\mathbb{E}\wh{f} - f_0\|_n^2 = \Bigl\|\sum_{k = K + 1}^{n} \dotp{v_k}{f_0}_n v_k\Bigr\|_n^2 = \sum_{k = K + 1}^n \dotp{v_k}{f_0}_n^2.
	\end{equation*}
	In the above, the last equality relies on the fact that $v_k$ are orthonormal in $L^2(P_n)$. Using the fact that the eigenvalues are in increasing order, we obtain
	\begin{equation*}
	\sum_{k = K + 1}^n \dotp{v_k}{f_0}_n^2 \leq \frac{1}{\lambda_{K + 1}^s} \sum_{k = K + 1}^n \lambda_k^s \dotp{v_k}{f_0}_n^2 \leq \frac{\dotp{L^sf_0}{f_0}_n}{\lambda_{K + 1}^s}.
	\end{equation*}
	
	If $K = 0$, $\wh{f} = \Ebb{\wh{f}} = 0$, and the second term in~\eqref{pf:fixed_graph_estimation_1} is $0$. Otherwise the second   in~\eqref{pf:fixed_graph_estimation_1} (estimation error) is random. Observe that $\dotp{v_k}{\varepsilon}_n \overset{d}{=} Z_k/\sqrt{n}$, where $(Z_1,\ldots,Z_n) \sim N(0,I_{n \times n})$. Again using the orthonormality of the eigenvectors $v_k$, we have
	\begin{equation*}
	\|\wh{f} - \mathbb{E}\wh{f}\|_n^2 = \sum_{k = 1}^{K} \dotp{v_k}{\varepsilon}_n^2 \overset{d}{=} \frac{1}{n}\sum_{k = 1}^{K} Z_k^2.
	\end{equation*}
	Thus $\|\wh{f} - \mathbb{E}\wh{f}\|_n^2$ is equal to $1/n$ times a $\chi^2$ distribution with $K$ degrees of freedom. Consequently, it follows from a result of \citep{laurent00} that
	\begin{equation*}
	\Pbb\biggl(\|\wh{f} - \mathbb{E}\wh{f}\|_n^2 \geq \frac{K}{n} + 2\frac{\sqrt{K}}{n}\sqrt{t} + \frac{2t}{n}\biggr) \leq \exp(-t).
	\end{equation*}
	Setting $t = K$ completes the proof of the lemma.

\subsection{Upper bound on Testing Error of Laplacian Eigenmaps}

Let $\wh{T} = \sum_{k = 1}^{K} \dotp{{\bf Y}}{v_k}_n^2$, and let $\varphi = \1\{\wh{T} \geq t_a\}$. In the following Lemma, we upper bound the Type I and Type II error of the test $\varphi$.

\begin{lemma}
	\label{lem:fixed_graph_testing}
	Suppose we observe $(Y_1,x_1),\ldots,(Y_n,x_n)$ according to~\eqref{eqn:fixed_graph_regression_model}.
	\begin{itemize}
		\item If $f_0 = 0$, then $\Ebb_0[\varphi] \leq a$.
		\item Suppose $f_0 \neq 0$ satisfies
		\begin{equation}
		\label{eqn:fixed_graph_testing_critical_radius}
		\|f_0\|_n^2 \geq \frac{\dotp{L^sf_0}{f_0}_n}{\lambda_{K + 1}^s} + \frac{\sqrt{2K}}{n}\biggl[2\sqrt{\frac{1}{a}} + \sqrt{\frac{2}{b}} + \frac{32}{bn}\biggr],
		\end{equation}
		for some $s \in \mathbb{N}\setminus \{0\}$. Then $\Ebb_{f_0}[1 - \phi] \leq b$.
	\end{itemize}
\end{lemma}
\paragraph{Proof (of Lemma~\ref{lem:fixed_graph_testing}).}
We first compute the expectation and variance of $\wh{T}$, then apply Chebyshev's inequality to upper bound the Type I and Type II error.

\underline{\emph{Expectation}.}
Recall that $\wh{T} = \sum_{k = 1}^{K} \dotp{Y}{v_k}_n^2$. Expanding the square gives
\begin{equation*}
\Ebb[\wh{T}] = \sum_{k = 1}^{K} \Ebb[\dotp{Y}{v_k}_n^2] = \sum_{k = 1}^{K} \dotp{f_0}{v_k}_n^2 + \Ebb[2\dotp{f_0}{v_k}_n\dotp{\varepsilon}{v_k}_n + \dotp{\varepsilon}{v_k}_n^2] = \frac{K}{n} + \sum_{k = 1}^{K} \dotp{f_0}{v_k}_n^2.
\end{equation*}
Thus $\Ebb[\wh{T}] - t_a = \sum_{k = 1}^{K} \dotp{f_0}{v_k}_n^2 - \sqrt{2K}/n \cdot \sqrt{1/a}$. Furthermore, it is a consequence of~\eqref{eqn:fixed_graph_testing_critical_radius} that 
\begin{equation}
\label{pf:fixed_graph_testing_1}
\sum_{k = 1}^{K} \dotp{f_0}{v_k}_n^2 - \frac{\sqrt{2K}}{n}\sqrt{1/a} \geq \|f_0\|_n^2 - \frac{\dotp{L^sf_0}{f_0}_n}{\lambda_{K + 1}^s} - \frac{\sqrt{2K}}{n}\sqrt{1/a} \geq \frac{\sqrt{2K}}{n}\biggl[\sqrt{\frac{1}{a}} + \sqrt{\frac{2}{b}} + \frac{32}{bn}\biggr].
\end{equation} 

\underline{\emph{Variance}.}
Recall from the proof of Lemma~\ref{lem:fixed_graph_estimation} that $\dotp{\varepsilon}{v_k}_n \overset{d}{=} Z_k/\sqrt{n}$ for $(Z_1,\ldots,Z_n) \sim N(0,I_{n \times n})$. Expanding the square, and recalling that $\Cov[Z,Z^2] = 0$ for Gaussian random variables, we have that
\begin{equation*}
\Var\bigl[\dotp{{\bf Y}}{v_k}_n^2\bigr] = \Var\biggl[\frac{2}{n}\dotp{f_0}{v_k}_nZ_k + \frac{2}{n^2}Z_k^2\biggr] = \frac{4\dotp{f_0}{v_k}_n^2}{n} + \frac{2}{n^2}.
\end{equation*}
Moreover, since $\Cov[Z_k^2,Z_{\ell}^2] = 0$ for each $k = 1,\ldots,K$, we see that
\begin{equation*}
\Var\bigl[\wh{T}\bigr] = \sum_{k = 1}^{K} \Var\bigl[\dotp{{\bf Y}}{v_k}_n^2\bigr] = \frac{2K}{n^2} + \sum_{k = 1}^{K}\frac{4\dotp{f_0}{v_k}_n^2}{n}.
\end{equation*}

\underline{\emph{Bounds on Type I and Type II error}.}
The upper bound on Type I error follows immediately from Chebyshev's inequality. 

The upper bound on Type II error also follows from Chebyshev's inequality. We observe that~\eqref{eqn:fixed_graph_testing_critical_radius} implies $\Ebb_{f_0}[\wh{T}] = t_a$, and apply Chebyshev's inequality to deduce
\begin{equation*}
\Pbb_{f_0}\bigl(\wh{T} < t_a\bigr) \leq \Pbb_{f_0}\Bigl(|\wh{T} - \Ebb_{f_0}[\wh{T}]|^2 > |\Ebb_{f_0}[\wh{T}] - t_a|^2\Bigr) \leq \frac{\Var\bigl[\wh{T}\bigr]}{\bigl[\Ebb_{f_0}[\wh{T}] - t_a\bigr]^2} = \frac{2K/n^2 + 4/n\sum_{k = 1}^{K}\dotp{f_0}{v_k}_n^2}{\bigl[\Ebb_{f_0}[\wh{T}] - t_a\bigr]^2}.
\end{equation*}
Thus we have upper bounded the Type II error by the sum of two terms, each of which are no more than $1/(2b)$, as we now show. For the first term, after noting that~\eqref{pf:fixed_graph_testing_1} implies $\Ebb_{f_0}[\wh{T}] - t_a \geq \sqrt{2K}/n \cdot \sqrt{2/b}$, the upper bound follows:
\begin{equation*}
\frac{2K/n^2}{\bigl[\Ebb_{f_0}[\wh{T}] - t_a\bigr]^2} \leq \frac{b}{2}.
\end{equation*}
On the other hand, for the second term we use~\eqref{pf:fixed_graph_testing_1} in two ways: first to conclude that $\Ebb_{f_0}[\wh{T}] - t_a \geq 1/2 \cdot \sum_{k = 1}^{K}\dotp{f_0}{v_k}_n^2$, and second to obtain
\begin{equation*}
\frac{4\sum_{k = 1}^{K}\dotp{f_0}{v_k}_n^2}{n\bigl[\Ebb_{f_0}[\wh{T}] - t_a\bigr]^2} \leq \frac{4\sum_{k = 1}^{K}\dotp{f_0}{v_k}_n^2}{n\bigl(\sum_{k = 1}^{K}\dotp{f_0}{v_k}_n^2/2\bigr)^2} \leq \frac{16}{n\sum_{k = 1}^{K}\dotp{f_0}{v_k}_n^2} \leq \frac{b}{2}.
\end{equation*}

\section{Graph Sobolev semi-norm, flat Euclidean domain}
\label{sec:graph_quadratic_form_euclidean}
In this section we prove Proposition~\ref{prop:graph_seminorm_ho}. The proposition will follow from several intermediate results.
\begin{enumerate}
	\item~In Section~\ref{subsec:decomposition_graph_seminorm}, we show that
	\begin{equation}
	\label{pf:graph_seminorm_ho_1}
	\dotp{L_{n,\varepsilon}^sf}{f}_n \leq \frac{1}{\delta} \dotp{L_{P,\varepsilon}^sf}{f}_{P} + \frac{C\varepsilon^2}{\delta n\varepsilon^{2 + d}}M^2.
	\end{equation}
	with probability at least $1 - 2\delta$. 
	
	We term the first term on the right hand side the \emph{non-local Sobolev semi-norm}, as it is a kernelized approximation to the Sobolev semi-norm $\dotp{\Delta_P^sf}{f}_{P}$. The second term on the right hand side is a pure bias term, which as we will see is negligible compared to the non-local Sobolev semi-norm as long as $\varepsilon \ll n^{-1/(2(s -1 + d))}$. 
	\item~In Section~\ref{subsec:approximation_error_nonlocal_laplacian}, we show that when $x$ is sufficiently in the interior of $\mc{X}$, then $L_{P,\varepsilon}^kf(x)$ is a good approximation to $\Delta_P^kf(x)$, as long as $f \in H^{s}(\mc{X})$ and $p \in C^{s - 1}(\mc{X})$ for some $s \geq 2k + 1$. 
	\item~In Section~\ref{subsec:boundary_behavior_nonlocal_laplacian}, we show that when $x$ is sufficiently near the boundary of $\mc{X}$, then $L_{P,\varepsilon}^kf(x)$ is close to $0$, as long as $f \in H_0^{s}(\mc{X})$ for some $s > 2k$.
	\item~In Section~\ref{subsec:estimate_nonlocal_seminorm}, we use the results of the preceding two sections to show that if $f \in H_0^s(\mc{X};M)$ and $p \in C^{s - 1}(\mc{X})$, there exists a constant $C$ which does not depend on $f$ such that
	\begin{equation}
	\label{pf:graph_seminorm_ho_2}
	\dotp{L_{P,\varepsilon}^sf}{f}_{P} \leq CM^2.
	\end{equation}
\end{enumerate}
Finally, in Section~\ref{subsec:integrals} we provide some assorted estimates used in Sections~\ref{subsec:decomposition_graph_seminorm}. 

\paragraph{Proof (of Proposition~\ref{prop:graph_seminorm_ho}).}
Proposition~\ref{prop:graph_seminorm_ho} follows immediately from~\eqref{pf:graph_seminorm_ho_1} and~\eqref{pf:graph_seminorm_ho_2}. \qed

One note regarding notation: suppose a function $g \in H^{\ell}(U)$, where $\ell \in \mathbb{N}$ and $U$ is an open set. Let $V$ be another open set, compactly contained within $U$. Then we will use the notation $g \in H^{\ell}(V)$ to mean that the restriction $\restr{g}{V}$ of $g$ to $V$ belongs to $H^{\ell}(V)$.

\subsection{Decomposition of graph Sobolev semi-norm}
\label{subsec:decomposition_graph_seminorm}

In Lemma~\ref{lem:graph_seminorm_bias}, we decompose the graph Sobolev semi-norm (a V-statistic) into an unbiased estimate of the non-local Sobolev semi-norm (a U-statistic), and a pure bias term. We establish that the pure bias term will be small (in expectation) relative to the U-statistic whenever $\varepsilon$ is sufficiently small.
\begin{lemma}
	\label{lem:graph_seminorm_bias}
	For any $f \in L^2(\mc{X})$, the graph Sobolev semi-norm satisfies
	\begin{equation}
	\label{eqn:graph_seminorm_bias_1}
	\dotp{L_{n,\varepsilon}^sf}{f}_{n} = U_{n,\varepsilon}^{(s)}(f) + B_{n,\varepsilon}^{(s)}(f),
	\end{equation}
	such that $\mathbb{E}[U_{n,\varepsilon}^{(s)}(f)] = (n - s - 1)!/n! \cdot \dotp{L_{P,\varepsilon}^sf}{f}_P$. If additionally $f \in H^1(\mc{X};M)$ and $\varepsilon \geq n^{-1/d}$, then the bias term $B_{n,\varepsilon}^{(s)}(f)$ satisfies
	\begin{equation}
	\label{eqn:graph_seminorm_bias_2}
	\mathbb{E}\bigl[|B_{n,\varepsilon}^{(s)}(f)|\bigr] \leq \frac{C\varepsilon^2}{\delta n\varepsilon^{2 + d}}M^2.
	\end{equation}
\end{lemma}
Then~\ref{pf:graph_seminorm_ho_1} follows immediately from Lemma~\ref{lem:graph_seminorm_bias}, by Markov's inequality.
\paragraph{Proof (of Lemma~\ref{lem:graph_seminorm_bias}).}
We begin by introducing some notation. We will use bold notation $\bj = (j_1,\ldots,j_s)$ for a vector of indices where $j_i \in [n]$ for each $i$. We write $[n]^s$ for the collection of all such vectors, and $(n)^s$ for the subset of such vectors with no repeated indices. Finally, we write $D_if$ for a kernelized difference operator,
\begin{equation*}
D_if(x) := \bigl(f(x) - f(X_i)\bigr) \eta\biggl(\frac{\|X_i - x\|}{\varepsilon}\biggr),
\end{equation*}
and we let $D_{\bj}f(x) := \bigl(D_{j_1}\circ \cdots \circ D_{j_s}f\bigr)(x)$.

With this notation in hand, it is easy to represent $\dotp{L_{n,\varepsilon}^sf}{f}_{n}$ as the sum of a U-statistic and a bias term,
\begin{align*}
\dotp{L_{n,\varepsilon}^sf}{f}_{n} & = \frac{1}{n} \sum_{i = 1}^{n} L_{n,\varepsilon}^sf(X_i) \cdot f(X_i) \\
& = \underbrace{\frac{1}{n^{s + 1}\varepsilon^{s(d + 2)}} \sum_{{i\bf j} \in (n)^{s + 1}}D_{\bj}f(X_i) \cdot f(X_i)}_{=:U_{n,\varepsilon}^{(s)}(f)} + \underbrace{\frac{1}{n^{s + 1}\varepsilon^{s(d + 2)}} \sum_{\substack{i\bj \in \\ [n]^{s + 1}\setminus (n)^{s + 1}}} D_{\bj}f(X_i) \cdot f(X_i)}_{=:B_{n,\varepsilon}^{(s)}(f)}
\end{align*}
When the indices of $i\bj$ are all distinct, it follows straightforwardly from the law of iterated expectation that
\begin{equation*}
\mathbb{E}[ D_{\bj}f(X_i) \cdot f(X_i)] = \varepsilon^{s(d + 2)}\mathbb{E}[L_{P,\varepsilon}^sf(X_i) \cdot f(X_i)] = \dotp{L_{P,\varepsilon}^sf}{f}_{P}, 
\end{equation*}
which in turn implies $\mathbb{E}[U_{n,\varepsilon}^{(s)}(f)] = (n - s - 1)!/n! \cdot \dotp{L_{P,\varepsilon}^sf}{f}_P$. 

It remains to show~\eqref{eqn:graph_seminorm_bias_2}. By adding and subtracting $f(X_{\bj_1})$, we obtain by symmetry that
\begin{equation*}
\sum_{\substack{i\bj \in \\ [n]^{s + 1}\setminus (n)^{s + 1}}} D_{\bj}f(X_i) \cdot f(X_i) = \frac{1}{2} \cdot \sum_{\substack{i\bj \in \\ [n]^{s + 1}\setminus (n)^{s + 1}}} D_{\bj}f(X_i) \cdot \bigl(f(X_i) - f(X_{\bj_1})\bigr),
\end{equation*}
and consequently
\begin{equation*}
\Ebb\Bigl[\sum_{\substack{i\bj \in \\ [n]^{s + 1}\setminus (n)^{s + 1}}} D_{\bj}f(X_i) \cdot f(X_i)\Bigr] \leq \frac{1}{2} \cdot \sum_{\substack{i\bj \in \\ [n]^{s + 1}\setminus (n)^{s + 1}}} \Ebb\Bigl[\bigl|D_{\bj}f(X_i)\bigr| \cdot \bigl|f(X_i) - f(X_{\bj_1})\bigr|\Bigr].
\end{equation*}
In Lemma~\ref{lem:graph_seminorm_bias2}, we show that if $f \in H^1(\mc{X};M)$, then for any $i\bj \in [n]^{s + 1}$ which contains a total of $k + 1$ distinct indices, 
\begin{equation*}
\Ebb\Bigl[\bigl|D_{\bj}f(X_i)\bigr| \cdot \bigl|f(X_i) - f(X_{\bj_1})\bigr|\Bigr] \leq C_1 \varepsilon^{2 + kd} M^2.
\end{equation*}
This shows us that the expectation of $|B_{n,\varepsilon}^s(f)|$ can bounded from above by the sum over several different terms, as follows:
\begin{align*}
\Ebb\Bigl[|B_{n,\varepsilon}^s(f)|\Bigr] & \leq C_1\frac{\varepsilon^2}{n\varepsilon^{2s}}M^2 \sum_{\substack{i\bj \in \\ [n]^{s + 1}\setminus (n)^{s + 1}}} \frac{1}{(n\varepsilon^d)^s}  \varepsilon^{(|i\bj| - 1)d} \\
& \leq C_1\frac{\varepsilon^2}{n\varepsilon^{2s}}M^2  \sum_{k = 1}^{s - 1} \frac{(n\varepsilon^d)^k}{(n\varepsilon^d)^s}n.
\end{align*}
Finally, we note that by assumption $n\varepsilon^d \geq 1$, so that in the above sum the factor of $(n\varepsilon^d)^k$ is largest when $k = s- 1$. We conclude that
\begin{equation*}
\Ebb\Bigl[|B_{n,\varepsilon}^s(f)|\Bigr] \leq C_1 (s - 1) \frac{\varepsilon^2}{n\varepsilon^{2s + d}}M^2,
\end{equation*}
which is the desired result.

\subsection{Approximation error of non-local Laplacian}
\label{subsec:approximation_error_nonlocal_laplacian}

In this section, we establish the convergence $L_{P,\varepsilon}^kf \to \sigma_{\eta}^k\Delta_P^kf$ as $\varepsilon \to 0$. More precisely, we give an upper bound on the squared difference between $L_{P,\varepsilon}^kf$ and  $\sigma_{\eta}^k\Delta_P^kf$ as a function of $\varepsilon$. The bound holds for all $x \in \mc{X}_{k\varepsilon}$, and $f \in H^{s}(\mc{X})$, as long as $s \geq 2k + 1$.  \textcolor{red}{(TODO): Make sure that $\sigma_{\eta}$ is defined somewhere.}
\begin{lemma}
	\label{lem:approximation_error_nonlocal_laplacian}
	Assume Model~\ref{def:model_flat_euclidean}. Let $s \in \mathbb{N} \setminus \{0,1\}$, suppose that $f \in H^s(\mc{X};M)$, and if $s > 1$ suppose that $p \in C^{s - 1}(\mc{X})$. Let $L_{P,\varepsilon}$ be define with respect to a kernel $\eta$ that satisfies~\ref{asmp:kernel_flat_euclidean}. Then there exist constants $C_1$ and $C_2$ that do not depend on $f$, such that each of the following statements hold.
	\begin{itemize}
		\item If $s$ is odd and $k = (s - 1)/2$, then
		\begin{equation}
		\label{eqn:approximation_error_nonlocal_laplacian_1}
		\|L_{P,\varepsilon}^kf - \Delta_P^kf\|_{L^2(\mc{X}_{k\varepsilon})} \leq C_1 M \varepsilon
		\end{equation}
		\item If $s$ is even and $k = (s - 2)/2$, then
		\begin{equation}
		\label{eqn:approximation_error_nonlocal_laplacian_2}
		\|L_{P,\varepsilon}^kf - \Delta_P^kf\|_{L^2(\mc{X}_{k\varepsilon})} \leq C_2 M \varepsilon^2.
		\end{equation}
	\end{itemize}
\end{lemma}
We remark that when $k = 1$ and $f \in C^3(\mc{X})$ or $C^4(\mc{X})$, statements of this kind are well known \textcolor{red}{(references)}, and indeed stronger results---with $L^{\infty}(\mc{X})$ norm replacing $L^2(\mc{X})$ norm---hold. When dealing with the iterated Laplacian, and functions $f$ which are regular only in the Sobolev sense, the proof is somewhat more lengthy, but the spirit of the result is largely the same.
 
\paragraph{Proof (of Lemma~\ref{lem:approximation_error_nonlocal_laplacian}).}
Throughout this proof, we shall assume that $f$ and $p$ are smooth functions, meaning they belong to $C^{\infty}(\mc{X})$. This is without loss of generality, since $C^{\infty}(\mc{X})$ is dense in both $H^s(\mc{X})$ and $C^{s - 1}(\mc{X})$, and since both sides of the inequalities~\eqref{eqn:approximation_error_nonlocal_laplacian_1} and~\eqref{eqn:approximation_error_nonlocal_laplacian_2} are continuous with respect to $\|\cdot\|_{H^s(\mc{X})}$ and $\|\cdot\|_{C^{s - 1}(\mc{X})}$ norms.

We will actually prove a more general set of statements than contained in Lemma~\ref{lem:approximation_error_nonlocal_laplacian}, more general in the sense that they give estimates for all $k$, rather than simply the particular choices of $k$ given above. In particular, we will prove that the following two statements hold for any $s \in \mathbb{N}$ and any $k \in \mathbb{N} \setminus \{0\}$. 
\begin{itemize}
	\item If $k \geq s/2$, then for every $x \in \mc{X}_{k\varepsilon}$, 
	\begin{equation}
	\label{pf:approximation_error_nonlocal_laplacian_0}
	L_{P,\varepsilon}^kf(x) = g_s(x) \varepsilon^{s - 2k}
	\end{equation}
	for a function $g_s$ that satisfies
	\begin{equation}
	\label{pf:approximation_error_nonlocal_laplacian_0.5}
	\|g_s\|_{L^2(\mc{X}_{k\varepsilon})} \leq C \|p\|_{C^{q}(\mc{X})}^k M 
	\end{equation}
	where $q = 1$ if $s =0$ or $s = 1$, and otherwise $q = s - 1$. 
	\item If $k < s/2$, then for every $x \in \mc{X}_{k\varepsilon}$,
	\begin{equation}
	\label{pf:approximation_error_nonlocal_laplacian_1}
	L_{P,\varepsilon}^kf(x) = \sigma_{\eta}^k \cdot \Delta_{P}^kf(x) + \sum_{j = 1}^{\floor{(s - 1)/2} - k} g_{2(j + k)}(x)\varepsilon^{2j} + g_{s}(x) \varepsilon^{s - 2k}.
	\end{equation}
	for functions $g_j$ that satisfy
	\begin{equation}
	\label{pf:approximation_error_nonlocal_laplacian_1.5}
	\|g_j\|_{H^{s - j}(\mc{X}_{k\varepsilon})} \leq C \|p\|_{C^{s - 1}(\mc{X})}^k M.
	\end{equation}
\end{itemize}
In the statement above, recall that $H^0(\mc{X}_{k\varepsilon}) = L^2(\mc{X}_{k\varepsilon})$. Additionally, note that we may speak of the pointwise behavior of derivatives of $f$ because we have assumed that $f$ is a smooth function. Observe that~\eqref{eqn:approximation_error_nonlocal_laplacian_1} follows upon taking $k = \floor{(s - 1)/2}$ in~\eqref{pf:approximation_error_nonlocal_laplacian_1}, whence we have
\begin{equation*}
\bigl(L_{P,\varepsilon}^kf(x) - \sigma_{\eta}^k \Delta_{P}^kf(x)\bigr)^2 = \varepsilon^2 \bigl(g_s(x)\bigr)^2
\end{equation*}
for some $g_s \in \Leb^2(\mc{X}_{k\varepsilon},C \cdot M \cdot \|p\|_{C^{s - 1}(\mc{X})})$, and integrating over $\mc{X}_{k\varepsilon}$ gives the desired result. \eqref{eqn:approximation_error_nonlocal_laplacian_2} follows from~\eqref{pf:approximation_error_nonlocal_laplacian_1} in an identical fashion. 

It thus remains establish~\eqref{pf:approximation_error_nonlocal_laplacian_1}, and~\eqref{pf:approximation_error_nonlocal_laplacian_0} which is an important part of proving~\eqref{pf:approximation_error_nonlocal_laplacian_1}. We will do so by induction on $k$. Note that throughout, we will let $g_j$ refer to functions which may change from line to line, but which always satisfy~\eqref{pf:approximation_error_nonlocal_laplacian_1.5}. 

\underline{\textit{Proof of~\eqref{pf:approximation_error_nonlocal_laplacian_0} and~\eqref{pf:approximation_error_nonlocal_laplacian_1}, base case.}}

We begin with the base case, where $k = 1$. Again, we point out that although desired result is known when $s = 3$ or $s = 4$, and $f$ is regular in the H\"{o}lder sense, we require estimates for all $s \in \mathbb{N}$ when $f$ is regular in the Sobolev sense.

When $s = 0$, the inequality~\eqref{pf:approximation_error_nonlocal_laplacian_0} is implied by Lemma~\ref{lem:l2estimate_nonlocal_laplacian}.  When $s \geq 1$, we proceed using Taylor expansion. For any $x \in \mc{X}_{\varepsilon}$, we have that $B(x,\varepsilon) \subseteq \mc{X}$. Thus for any $x' \in B(x,\varepsilon)$, we may take an order $s$ Taylor expansion of $f$ around $x' = x$, and an order $q$ Taylor expansion of $p$ around $x' = x$, where $q = 1$ if $s = 1$, and otherwise $q = s - 1$. (See Section~\ref{subsec:taylor_expansion} for a review of the notation we use for Taylor expansions, as well as some properties that we make use of shortly.) This allows us to express $L_{P,\varepsilon}f(x)$ as the sum of three terms,
\begin{align*}
L_{P,\varepsilon}f(x) & = \frac{1}{\varepsilon^{d + 2}}\sum_{j_1 = 1}^{s - 1} \sum_{j_2 = 0}^{q - 1}\frac{1}{j_1!j_2!}  \int_{\mc{X}} \bigl(d_x^{j_1}f\bigr)(x' - x) \bigl(d_x^{j_2}p\bigr)(x' - x) \eta\biggl(\frac{\|x' - x\|}{\varepsilon}\biggr) \,dx' \quad + \\
& \quad \frac{1}{\varepsilon^{d + 2}}\sum_{j = 1}^{s - 1} \frac{1}{j!} \int_{\mc{X}} \bigl(d_x^jf\bigr)(x' - x)  r_{x'}^{q}(x;p) \eta\biggl(\frac{\|x' - x\|}{\varepsilon}\biggr) \,dx' \quad  + \\
& \quad \frac{1}{\varepsilon^{d + 2}} \int_{\mc{X}} r_{x'}^j(x;f) \eta\biggl(\frac{\|x' - x\|}{\varepsilon}\biggr) \,dP(x').
\end{align*}
Here we have adopted the convention that $\sum_{j = 1}^{0} = 0$. 

Changing variables to $z = (x' - x)/\varepsilon$, we can rewrite the above expression as 
\begin{align*}
L_{P,\varepsilon}f(x) & = \frac{1}{\varepsilon^{2}}\sum_{j_1 = 1}^{s - 1} \sum_{j_2 = 0}^{q - 1}\frac{\varepsilon^{j_1 + j_2}}{j_1!j_2!}  \int d_x^{j_1}f(z) d_x^{j_2}p(z) \eta\bigl(\|z\|\bigr) \,dz \quad + \\
& \quad \frac{1}{\varepsilon^{2}} \sum_{j = 1}^{s - 1} \frac{\varepsilon^j}{j!} \int d_x^jf(z)  r_{zh + x}^{q}(x;p) \eta\bigl(\|z\|\bigr) \,dz \quad  + \\
& \quad \frac{1}{\varepsilon^{2}} \int r_{zh + x}^j(x;f) \eta\bigl(\|z\|\bigr) p(zh + x)\,dz \\
& := G_1(x) + G_2(x) + G_3(x).
\end{align*}
We now separately consider each of $G_1(x),G_2(x)$ and $G_3(x)$. We will establish that if $s = 1$ or $s = 2$, then $G_1(x) = 0$, and otherwise if $s \geq 3$ that
\begin{equation*}
G_1(x) = \sigma_{\eta}\Delta_Pf(x) + \sum_{j = 1}^{\floor{(s - 1)/2} - 1}g_{2(j + 1)}(x)\varepsilon^{2j} + g_{s}(x)\varepsilon^{s - 2}.
\end{equation*}
On the other hand, we will establish that if $s = 1$ then $G_2(x) = 0$, and otherwise for $s \geq 2$
\begin{equation}
\label{pf:approximation_error_nonlocal_laplacian_2}
\|G_2\|_{\Leb^2(\mc{X}_{\varepsilon})} \leq C \varepsilon^{s - 2} M \|p\|_{C^{s - 1}(\mc{X})};
\end{equation}
this same estimate will hold for $G_3$ for all $s \geq 1$. Together these will imply~\eqref{pf:approximation_error_nonlocal_laplacian_0} and~\eqref{pf:approximation_error_nonlocal_laplacian_1}. 

\emph{Estimate on $G_1(x)$.}
If $s = 1$, then $s - 1 = 0$, and so $G_1(x) = 0$. We may therefore suppose $s \geq 2$. Recall that
\begin{equation}
G_1(x) = \sum_{j_1 = 1}^{s - 1} \sum_{j_2 = 0}^{q - 1} \frac{\varepsilon^{j_1 + j_2 - 2}}{j_1!j_2!}  \underbrace{\int_{B(0,1)} d_x^{j_1}f(z) d_x^{j_2}p(z) \eta(\|z\|) \,dz}_{:= g_{j_1,j_2}(x)} \label{pf:approximation_error_nonlocal_laplacian_3}
\end{equation}
The nature of $g_{j_1,j_2}(x)$ depends on the sum $j_1 + j_2$. Since $d_x^{j_1}f d_x^{j_2}$ is an order $j_1 + j_2$ (multivariate) monomial, we have (see Section~\ref{subsec:taylor_expansion}) that whenever $j_1 + j_2$ is odd,
\begin{equation*}
g_{j_1,j_2}(x) = \int_{\mc{X}} d_x^{j_1}f(z) d_x^{j_2}p(z) \eta(\|z\|) \,dz = 0.
\end{equation*}
In particular this is the case when $j_1 = 1$ and $j_2 = 0$. Thus when $s = 2$,  $G_1(x) = g_{1,0}(x) = 0$. On the other hand if $s \geq 3$, then the lowest order terms in~\eqref{pf:approximation_error_nonlocal_laplacian_3} are those where $j_1 + j_2 = 2$, so that either $j_1 = 1$ and $j_2 = 1$, or $j_1 = 2$ and $j_2 = 0$. We have that
\begin{align*}
g_{1,1}(x) + \frac{1}{2}g_{2,0}(x) & = \int_{\mc{X}} d_x^{1}f(z) d_x^{1}p(z) \eta(\|z\|) \,dz + \frac{p(x)}{2} \int_{\mc{X}} d_x^{2}f(z) \eta(\|z\|) \,dz \\
& = \sum_{i_1 = 1}^{d} \sum_{i_2 = 1}^{d} D^{e_{i_1}}f(x) D^{e_{i_2}}p(x) \int_{\mc{X}} z^{e_{i_1} + e_{i_2}} \eta(\|z\|) \,dz + \frac{p(x)}{2} \sum_{i_1 = 1}^{d} \sum_{i_2 = 1}^{d}D^{e_{i_2}+e_{i_2}}f(x)\int_{\mc{X}} z^{e_{i_1} + e_{i_2}} \eta(\|z\|) \,dz\\
& = \sum_{i = 1}^{d} D^{e_{i}}f(x) D^{e_{i}}p(x) \int_{\mc{X}} z^2 \eta(\|z\|) \,dz + \frac{p(x)}{2} \sum_{i = 1}^{d} D^{2e_{i}}f(x)\int_{\mc{X}} z^2 \eta(\|z\|) \,dz\\ 
& = \sigma_{\eta}\Delta_Pf(x),
\end{align*}
which is the leading term order term. Now it remains only to deal with the higher-order terms, where $j_1 + j_2 > 2$, and where it suffices to show that each function $g_{j_1,j_2}$ satisfies~\eqref{pf:approximation_error_nonlocal_laplacian_1.5} for $j = \min\{j_1 + j_2 - 2,s - 2\}$. It is helpful to write $g_{j_1,j_2}$ using multi-index notation, 
\begin{align*}
g_{j_1,j_2}(x) = \sum_{|\alpha_1| = j_1} \sum_{|\alpha_2| = j_2} D^{\alpha_1}f(x) D^{\alpha_2}p(x) \int_{B(0,1)} z^{\alpha_1 + \alpha_2} \eta(\|z\|) \,dz,
\end{align*}
where we note that $|\int_{B(0,1)} z^{\alpha_1 + \alpha_2} \eta(\|z\|) \,dz| < \infty$ for all $\alpha_1, \alpha_2$, by the assumption that $\eta$ is Lipschitz on its support. Finally, by H\"{o}lder's inequality we have that
\begin{align*}
\|D^{\alpha_1}f D^{\alpha_2}p\|_{H^{s - (j + 2)}(\mc{X})} & \leq \|D^{\alpha_1}f\|_{H^{s - (j + 2)}(\mc{X})} \|D^{\alpha_2}p\|_{C^{s - (j + 2)}(\mc{X})} \\
& \leq \|D^{\alpha_1}f\|_{H^{s - j_1}(\mc{X})} \|D^{\alpha_2}p\|_{C^{s - (j_2 + 1)}(\mc{X})} \\
& \leq M \cdot \|p\|_{C^{s - 1}(\mc{X})},
\end{align*}
and summing over all $|\alpha_1| = j_1$ and $|\alpha_2| = j_2$ establishes that $g_{j_1,j_2}$ satisfies~\eqref{pf:approximation_error_nonlocal_laplacian_1.5}.

\emph{Estimate on $G_2(x)$.}
Note immediately that $G_2(x) = 0$ if $s = 1$. Otherwise if $s \geq 2$, then $q = s - 1$. Recalling that $|r_{x + z\varepsilon}^{s - 1}(x; p)| \leq C\varepsilon^{s - 1}\|p\|_{C^{s - 1}(\mc{X})}$ for any $z \in B(0,1)$, and that $d_x^jf(\cdot)$ is a $j$-homogeneous function, we have that
\begin{align}
|G_2(x)| & \leq \sum_{j = 1}^{s - 1} \frac{\varepsilon^{j - 2}}{j!}\int_{B(0,1)} \Bigl|\bigl(d_x^{j}f\bigr)(z)\Bigr| \cdot |r_{x + z\varepsilon}^{s - 1}(x;p)| \cdot \eta(\|z\|) \,dz \nonumber \\
& \leq C\varepsilon^{s - 2}\|p\|_{C^{s - 1}(\mc{X})} \sum_{j = 1}^{s - 1} \frac{1}{j!} \int_{B(0,1)} \Bigl|\bigl(d_x^{j}f\bigr)(z)\Bigr| \cdot \eta(\|z\|) \,dz \label{pf:approximation_error_nonlocal_laplacian_4}.
\end{align}
Furthermore, for each $j = 1,\ldots,s - 1$ convolution of $d_x^jf$ with $\eta$ only decreases the $\Leb^2(\mc{X}_{\varepsilon})$ norm, meaning
\begin{equation}
\label{pf:approximation_error_nonlocal_laplacian_5}
\begin{aligned}
\int_{\mc{X}_{\varepsilon}} \biggl(\int_{B(0,1)} \Bigl|\bigl(d_x^{j}f\bigr)(z)\Bigr| \cdot \eta(\|z\|) \,dz\biggr)^2 \,dx & \leq \int_{\mc{X}_{\varepsilon}} \biggl(\int_{B(0,1)} \Bigl|\bigl(d_x^jf\bigr)(z)\Bigr|^2 \eta(\|z\|)\,dz \biggr) \cdot \biggl(\int_{B(0,1)} \eta(\|z\|) \,dz \biggr) \,dx \\
& \leq \int_{B(0,1)} \int_{\mc{X}_{\varepsilon}} \Bigl[\bigl(d^jf\bigr)(x)\Bigr]^2 \eta(\|z\|) \,dx  \,dz \\
& \leq \|d^jf\|_{\Leb^2(\mc{X_{\varepsilon}})}^2.
\end{aligned}
\end{equation}
In the above, we have used both that $|d_x^jf(z)| \leq |d^jf(x)|$ for all $z \in B(0,1)$, and that the kernel is normalized so that $\int \eta(\|z\|) \,dz = 1$. 
Combining this with~\eqref{pf:approximation_error_nonlocal_laplacian_4}, we conclude that
\begin{align*}
\int_{\mc{X}_{\varepsilon}} |G_2(x)|^2 \,dx & \leq C \Bigl(\varepsilon^{s - 2}\|p\|_{C^{s - 1}(\mc{X})}\Bigr)^2 \sum_{j = 1}^{s - 1} \int_{\mc{X}_{\varepsilon}}\biggl(\frac{1}{j!} \int_{B(0,1)} \Bigl|\bigl(d_x^{j}f\bigr)(z)\Bigr| \cdot \Bigl|\eta(\|z\|)\Bigr| \,dz\biggr)^2 \,dx \\
& \leq C \Bigl(\varepsilon^{s - 2}\|p\|_{C^{s - 1}(\mc{X})}\Bigr)^2 \sum_{j = 1}^{s - 1} \|d^ju\|_{\Leb^2(\mc{X_{\varepsilon}})}^2,
\end{align*}
establishing the desired estimate.

\emph{Estimate on $G_3(x)$.}
Applying the Cauchy-Schwarz inequality, we deduce a pointwise upper bound on $|G_3(x)|^2$,
\begin{align*}
|G_3(x)|^2 & \leq \biggl(\frac{p_{\max}}{\varepsilon^2}\biggr)^2 \cdot \biggl(\int_{B(0,1)} \bigl|r_{x + \varepsilon z}^s(x;u)\bigr|^2 \eta(\|z\|)\,dz\biggr) \cdot \biggl(\int_{B(0,1)} \eta(\|z\|) \,dz\biggr) \\
& \leq \biggl(\frac{p_{\max}}{\varepsilon^2}\biggr)^2 \int_{B(0,1)} \bigl|r_{x + \varepsilon z}^s(x;u)\bigr|^2 \eta(\|z\|) \,dz.
\end{align*}
Applying this pointwise over all $x \in \mc{X}_{\varepsilon}$ and integrating, we obtain
\begin{align*}
\int_{\mc{X}_{\varepsilon}} |G_3(x)|^2 \,dx & \leq \biggl(\frac{p_{\max}}{\varepsilon^2}\biggr)^2 \int_{\mc{X}_{\varepsilon}} \int_{B(0,1)} \bigl|r_{x + \varepsilon z}^s(x;f)\bigr|^2 \eta(\|z\|) \,dz \,dx \\
& = \biggl(\frac{p_{\max}}{\varepsilon^2}\biggr)^2 \int_{B(0,1)} \int_{\mc{X}_{\varepsilon}} \bigl|r_{x + \varepsilon z}^s(x;f)\bigr|^2 \eta(\|z\|) \,dx \,dz \\
& \leq \biggl(\frac{p_{\max}\varepsilon^s}{\varepsilon^2}\biggr)^2  \|d^sf\|_{L^2(\mc{X}_{\varepsilon})}^2,
\end{align*}
with the last inequality following from~\eqref{eqn:sobolev_remainder_term}. Noting that $p_{\max} = \|p\|_{C^0(\mc{X})} \leq \|p\|_{C^{s - 1}(\mc{X})}$, we see that this is a sufficient bound on $\|G_3\|_{\Leb^2(\mc{X}_{\varepsilon})}$.

\underline{\textit{Proof of~\eqref{pf:approximation_error_nonlocal_laplacian_0} and~\eqref{pf:approximation_error_nonlocal_laplacian_1}, induction step.}}
We now assume that~\eqref{pf:approximation_error_nonlocal_laplacian_0} and~\eqref{pf:approximation_error_nonlocal_laplacian_1} hold for all order up to some $k$, and show that they then hold for order $k + 1$ as well. The proof is relatively straightforward, once we introduce a bit of notation. Namely, for any $\ell,j \in \mathbb{N}$ such that $1 \leq j \leq \ell \leq$, we will use $g_j^{\ell}$ to refer to a function satisfying
\begin{equation}
\label{pf:approximation_error_nonlocal_laplacian_6}
\|g_j^{\ell}\|_{H^{\ell - j}(\mc{X}_{(k + 1)\varepsilon})} \leq C \|p\|_{C^{q}(\mc{X})}^{k + 1} M.
\end{equation}
Note that $g_j^{\ell}(x) = g_{(s - \ell) + j}(x)$, so that $g_j^{s}(x) = g_j(x)$. As before, the functions $g_j^{\ell}$ may change from line to line, but will always satisfy~\eqref{pf:approximation_error_nonlocal_laplacian_6}. We immediately illustrate the purpose of this notation. Suppose $g \in H^{\ell}(\mc{X}_{k\varepsilon}; C \|p\|_{C^{q}(\mc{X})}^k M)$ for some $\ell \leq s$. If $\ell \leq 2$, then by the inductive hypothesis, it follows that for any $x \in \mc{X}_{(k + 1)\varepsilon}$
\begin{equation}
\label{pf:approximation_error_nonlocal_laplacian_7}
L_{P,\varepsilon}g(x) = g_{\ell}^{\ell}(x) \varepsilon^{\ell - 2}.
\end{equation} 
On the other hand if $2 < \ell \leq s$, then by the inductive hypothesis, it follows that for any $x \in \mc{X}_{(k + 1)\varepsilon}$,
\begin{equation}
\label{pf:approximation_error_nonlocal_laplacian_8}
L_{P,\varepsilon}g(x) = \sigma_{\eta} \Delta_Pg(x) + \sum_{j = 1}^{\floor{(\ell - 1)/2} - 1} g_{2j + 2}^{\ell}(x) \varepsilon^{2j} + g_{\ell}^{\ell}(x) \varepsilon^{\ell - 2}.
\end{equation}

\emph{Proof of \eqref{pf:approximation_error_nonlocal_laplacian_0}.} If $s \leq 2(k + 1)$, then by the inductive hypothesis it follows that for all $x \in \mc{X}_{k\varepsilon}$, we have $L_{P,\varepsilon}^kf(x) = g_{s}(x) \cdot \varepsilon^{s - 2k}$, for some $g_s \in L^2(\mc{X}_{k\varepsilon}, C\|p\|_{C^{s - 1}(\mc{X})}^k M)$. Note that we may know more about $L_P^kf(x)$ than simply that it is bounded in $L^2$-norm, but a bound in $L^2$-norm suffices. In particular, from such a bound along with~\eqref{pf:approximation_error_nonlocal_laplacian_7} we deduce that for any $x \in \mc{X}_{(k + 1)\varepsilon}$,
\begin{equation}
\label{pf:approximation_error_nonlocal_laplacian_8.5}
L_{P,\varepsilon}^{k + 1}f(x) = \bigl(L_{P,\varepsilon} \circ L_{P,\varepsilon}^k f)(x)= L_{P,\varepsilon} g_s(x)\varepsilon^{s - 2k} = g_{s}^{s}(x) \varepsilon^{s - 2(k + 1)},
\end{equation}
establishing~\eqref{pf:approximation_error_nonlocal_laplacian_0}. 

\emph{Proof of \eqref{pf:approximation_error_nonlocal_laplacian_1}.} If $s > 2(k + 1)$, then by the inductive hypothesis we have that for all $x \in \mc{X}_{k\varepsilon}$, 
\begin{equation*}
L_{P,\varepsilon}^kf(x) = \sigma_{\eta}^k \Delta_P^kf(x) + \sum_{j = 1}^{\floor{(s - 1)/2} - k} g_{2(j + k)}(x) \varepsilon^{2j} + g_s(x) \varepsilon^{s - 2k}.
\end{equation*}
Thus for any $x \in \mc{X}_{(k + 1)\varepsilon}$, 
\begin{equation*}
L_{P,\varepsilon}^{k + 1}f(x) = \bigl(L_{P,\varepsilon} \circ L_{P,\varepsilon}^k f\bigr)(x) = \sigma_{\eta}^k L_{P,\varepsilon}\Delta_P^kf(x) + \sum_{j = 1}^{\floor{(s - 1)/2} - k} L_{P,\varepsilon}g_{2(j + k)}(x) \varepsilon^{2j} + L_{P,\varepsilon}g_s(x) \varepsilon^{s - 2k}
\end{equation*}
There are three terms on the right hand side of this equality, and we now analyze each separately.
\begin{enumerate}
	\item Noting that $\Delta_P^kf \in H^{s - 2k}(\mc{X}; C\|p\|_{C^{s - 1}(\mc{X})}^kM)$, we use~\eqref{pf:approximation_error_nonlocal_laplacian_8} to derive that
	\begin{align}
	L_{P,\varepsilon}\Delta_P^kf(x) & = \sigma_{\eta} \Delta_P^{k + 1}f(x) + \sum_{j = 1}^{(s - 2k - 1)/2 - } g_{2j + 2}^{s - 2k}(x)\varepsilon^{2j} + g_{s - 2k}^{s - 2k}(x) \varepsilon^{s - 2k - 2} \nonumber \\
	& = \sigma_{\eta} \Delta_P^{k + 1}f(x) + \sum_{j = 1}^{(s - 1)/2 - (k + 1)} g_{2(k + 1 + j)}(x)\varepsilon^{2j} + g_{s}(x) \varepsilon^{s - 2(k + 1)}, \label{pf:approximation_error_nonlocal_laplacian_9}
	\end{align}
	where in the second equality we have simply used the fact $g_j^{\ell}(x) = g_{(s - \ell) + j}(x)$ to rewrite the equation.
	\item Suppose $j < \floor{(s - 1)/2} - k$. Then we use~\eqref{pf:approximation_error_nonlocal_laplacian_8} to derive that
	\begin{align*}
	L_{P,\varepsilon}g_{2(j + k)}(x) & = \sigma_{\eta}\Delta_P g_{2(j + k)}(x) + \sum_{i = 1}^{\floor{(s - 2j - 2k - 1)/2} - 1} g_{2(i + 1)}^{s - 2(j + k)}(x)\varepsilon^{2i} + g_{s - 2(j + k)}^{s - 2(j + k)}(x) \varepsilon^{s - 2(j + k + 1)} \\
	& = g_{2(j + k + 1)}(x) + \sum_{i = 1}^{\floor{(s - 1)/2} - (j + k + 1)} g_{2(i + j + k + 1)}(x)\varepsilon^{2i} + g_{s}(x) \varepsilon^{s - 2(j + k + 1)},
	\end{align*}
	where in the second equality we have again used $g_j^{\ell}(x) = g_{(s - \ell) + j}(x)$, and also written $\sigma_{\eta} \Delta_Pf = g_{2}^{s - 2(j + k)} = g_{2(j + k + 1)}$, since the particular dependence on the Laplacian $\Delta_P$ will not matter. From here, multiplying by $\varepsilon^{2j}$, we conclude that
	\begin{align}
	\varepsilon^{2j} L_{P,\varepsilon}g_{2(j + k)}(x) & = g_{2(j + k + 1)}(x) \varepsilon^{2j} + \sum_{i = 1}^{\floor{(s - 1)/2} - (j + k + 1)} g_{2(i + j + k + 1)}(x)\varepsilon^{2(i + j)} + g_{s}(x) \varepsilon^{s - 2(k + 1)} \nonumber \\ 
	& = g_{2(j + k + 1)}(x) \varepsilon^{2j} + \sum_{m = 1}^{\floor{(s - 1)/2} - (k + 1)}  g_{2(m + k + 1)}(x)\varepsilon^{2m} + g_{s}(x) \varepsilon^{s - 2(k + 1)} \label{pf:approximation_error_nonlocal_laplacian_10},
	\end{align}
	with the second equality following upon changing variables to $m = i + j$. 
	
	On the other hand if $j = \floor{(s - 1)/2} - k$, then the calculation is much simpler,
	\begin{equation}
	\label{pf:approximation_error_nonlocal_laplacian_11}
	\varepsilon^{2j} L_{P,\varepsilon}g_{2(j + k)}(x) = g_{s - 2(j + k)}^{s - 2(j + k)}(x) \varepsilon^{2j} \varepsilon^{s - 2(j + k) - 2} = g_s(x) \varepsilon^{s - 2(k + 1)}.
	\end{equation}
	\item Finally, it follows immediately from~\eqref{pf:approximation_error_nonlocal_laplacian_8} that
	\begin{equation}
	\label{pf:approximation_error_nonlocal_laplacian_12}
	L_{P,\varepsilon}g_s(x) \varepsilon^{s - 2k} = g_{s}(x) \varepsilon^{s - 2(k + 1)}.
	\end{equation}
\end{enumerate}
Plugging~\eqref{pf:approximation_error_nonlocal_laplacian_9}-\eqref{pf:approximation_error_nonlocal_laplacian_12} back into~\eqref{pf:approximation_error_nonlocal_laplacian_8.5} proves the claim.

\subsection{Boundary behavior of non-local Laplacian}
\label{subsec:boundary_behavior_nonlocal_laplacian}

In Lemma~\ref{lem:approximation_error_nonlocal_laplacian_boundary}, we establish that if $f$ is Sobolev smooth of order $s > 2k$ and zero-trace, then near the boundary of $\mc{X}$ the non-local Laplacian $L_{P,\varepsilon}^kf$ is close to $0$ in the $\Leb^2$-sense.
\begin{lemma}
	\label{lem:approximation_error_nonlocal_laplacian_boundary}
	Assume Model~\ref{def:model_flat_euclidean}. Let $s,k \in \mathbb{N}$. Suppose that $f \in H_0^{s}(\mc{X};M)$. Then there exist numbers $c,C > 0$ that do not depend on $M$, such that for all $\varepsilon < c$, 
	\begin{equation*}
	\|L_{P,\varepsilon}^kf\|_{L^2(\partial_{k\varepsilon}\mc{X})}^2 \leq C \varepsilon^{2(s - 2k)}M^2.
	\end{equation*}
\end{lemma}

\paragraph{Proof (of Lemma~\ref{lem:approximation_error_nonlocal_laplacian_boundary})}
Applying Lemma~\ref{lem:l2estimate_nonlocal_laplacian}, we have that
\begin{equation*}
\|L_{P,\varepsilon}^kf\|_{L^2(\partial_{k\varepsilon}(\mc{X}))}^2 \leq \frac{(Cp_{\max})^{2}}{\varepsilon^4} \|L_{P,\varepsilon}^{k - 1}f\|_{L^2(\partial_{k\varepsilon}(\mc{X}))}^2 \leq \cdots \leq \frac{(Cp_{\max})^{2}}{\varepsilon^{4k}} \|f\|_{L^2(\partial_{k\varepsilon}(\mc{X}))}^2   
\end{equation*}
Thus it remains to show that for all $\varepsilon < c$,
\begin{equation}
\label{pf:approximation_error_nonlocal_laplacian_boundary_0}
\|f\|_{L^2(\partial_{k\varepsilon}(\mc{X}))}^2 = \int_{\partial_{k\varepsilon}(\mc{X})} \bigl(f(x)\bigr)^2 \,dx \leq C_1 \varepsilon^{2s} \|f\|_{H^s(\mc{X})}^2.
\end{equation}
We will build to~\eqref{pf:approximation_error_nonlocal_laplacian_boundary_0} by a series of intermediate steps, following the same rough structure as the proof of Theorem 18.1 in \citet{leoni2017}. For simplicity, we will take $k = 1$; the exact same proof applies to the general case upon assuming $\varepsilon < c/k$.

\underline{\textit{Step 1: Local Patch.}}
To begin, we assume that for some $c_0 > 0$ and a Lipschitz mapping $\phi: \Reals^{d - 1} \to [-c_0,c_0]$, we have that $f \in C_c^{\infty}(U_{\phi}(c_0))$, where 
\begin{equation*}
U_{\phi}(c_0) = \Bigl\{y \in Q(0,c_0): \phi(y_{-d}) \leq y_d\Bigr\}, 
\end{equation*}
and here $Q(0,c_0)$ is the $d$-dimensional cube of side length $c_0$, centered at $0$. We will show that for all $0 < \varepsilon < c_0$, and for the tubular neighborhood $V_{\phi}(\varepsilon) = \{y \in Q(0,c_0): \phi(y_{-d}) \leq y_d \leq \phi(y_{-d}) + \varepsilon\}$, we have that
\begin{equation*}
\int_{V_{\phi}(\varepsilon)} |f(x)|^2 \,dx \leq C\varepsilon^{2s} \|f\|_{H^s(U_{\phi}(c_0))}^2.
\end{equation*}
For a given $y = (y',y_d) \in V_{\phi}(\varepsilon)$, let $y_0 = (y',\phi(y'))$. Taking the Taylor expansion of $f(y)$ around $y = y_0$ because $u$ is compactly supported in $V_{\phi}$ it follows that,
\begin{align*}
f(y) & = f(y_0) + \sum_{j = 1}^{s - 1} \frac{1}{j!} D^{je_d}f(y_0) \bigl(y_d - \phi(y')\bigr)^j + \frac{1}{(s - 1)!}\int_{\phi(y')}^{y_d} (1 - t)^{s - 1} D^{se_d}f(y',z) \bigl(y_d - z\bigr)^{s - 1} \,dz \Longrightarrow\\
|f(y)| & \leq C\varepsilon^{s - 1}\int_{\phi(y')}^{y_d} \bigl|D^{se_d}f(y',z)\bigr| \,dz. 
\end{align*}
Consequently, by squaring both sides and applying Cauchy-Schwarz, we have that
\begin{equation*}
|f(y)|^2 \leq C\varepsilon^{2(s - 1)} \biggl(\int_{\phi(y')}^{y_d} \bigl|D^{se_d}f(y',z)\bigr| \,dz\biggr)^2 \leq C\varepsilon^{2s - 1} \int_{\phi(y')}^{y_d} \bigl|D^{se_d}f(y',z)\bigr|^2 \,dz.
\end{equation*}
Applying this bound for each $y \in V_{\phi}(\varepsilon)$, and then integrating, we obtain
\begin{align}
\int_{V_{\phi}(\varepsilon)} |f(y)|^2 \,dy & \leq \int_{Q_{d - 1}(c_0)} \int_{\phi(y')}^{\phi(y') + \varepsilon} |f(y',y_d)|^2 \,dy_d \,dy' \nonumber \\
& \leq C\varepsilon^{2s - 1}\int_{Q_{d - 1}(c_0)}  \int_{\phi(y')}^{\phi(y') + \varepsilon} \int_{\phi(y')}^{y_d} \bigl|D^{se_d}f(y',z)\bigr|^2 \,dz \,dy_d \,dy' \label{pf:approximation_error_nonlocal_laplacian_boundary_1}
\end{align}
where we have written $Q_{d - 1}(0,c_0)$ for the $d - 1$ dimensional cube of side length $c_0$, centered at $0$. Exchanging the order of the inner two integrals then gives
\begin{align*}
\int_{\phi(y')}^{\phi(y') + \varepsilon} \int_{\phi(y')}^{y_d} \bigl|D^{se_d}f(y',z)\bigr|^2 \,dz \,dy_d & = \int_{\phi(y')}^{\phi(y') + \varepsilon} \int_{z}^{\varepsilon} \bigl|D^{se_d}f(y',z)\bigr|^2 \,dy_d \,dz \\
& \leq C \varepsilon \int_{\phi(y')}^{\phi(y') + \varepsilon} \bigl|D^{se_d}f(y',z)\bigr|^2 \,dz \\
& \leq C \varepsilon \int_{\phi(y')}^{c_0} \bigl|D^{se_d}f(y',z)\bigr|^2 \,dz.
\end{align*}
Finally, plugging back into~\eqref{pf:approximation_error_nonlocal_laplacian_boundary_1}, we conclude that
\begin{equation*}
\int_{V_{\phi}(\varepsilon)} |f(y)|^2 \,dy \leq C \varepsilon^{2s} \int_{Q_{d - 1}(0,c_0)} \int_{\phi(y')}^{c_0} \bigl|D^{se_d}f(y',z)\bigr|^2 \,dz \,dy' \leq C \varepsilon^{2s} |u|_{H^s(U_{\phi}(c_0))}^2.
\end{equation*}

\underline{\textit{Step 2: Rigid motion of local patch.}} Now, suppose that at a point $x_0 \in \partial \mc{X}$, there exists a rigid motion $T: \Rd \to \Rd$ for which $T(x_0) = 0$, and a number $C_0$ such that for all $\varepsilon \cdot C_0 \leq c_0$, 
\begin{equation*}
T\bigl(Q_{T}(x_0,c_0) \cap \partial_{\varepsilon}\mc{X}\bigr) \subseteq V_{\phi}\bigl(C_0\varepsilon\bigr) \quad\textrm{and}\quad T\bigl(Q_T(x_0,c_0) \cap \mc{X}\bigr) = U_{\phi}(c_0).
\end{equation*}
Here $Q_{T}(x_0,c_0))$ is a (not necessarily coordinate-axis-aligned) cube of side length $c_0)$, centered at $x_0$. Define $v(y) := f(T^{-1}(y))$ for $y \in U_{\phi}(c_0)$. If $u \in C_c^{\infty}(\mc{X})$, then $v \in C_c^{\infty}(U_{\phi}(c_0))$, and moreover $\|v\|_{H^s(U_{\phi}(c_0))}^2 = \|f\|_{H^s(Q_{T}(x_0,c_0) \cap \mc{X})}^2$. Therefore, using the upper bound that we derived in Step 1,
\begin{equation*}
\int_{V_{\phi}(C_0 \cdot \varepsilon)} |v(y)|^2 \,dy \leq C \varepsilon^{2s} \|v\|_{H^s(U_{\phi}(c_0))}^2,
\end{equation*}
we conclude that
\begin{align*}
\int_{Q_{T}(x_0,c_0) \cap \partial_{\varepsilon}\mc{X}} |f(x)|^2 \,dx & = \int_{T(Q_T(x_0,c_0)) \cap \partial_{\varepsilon}\mc{X})} |v(y)|^2 \,dy \\
& \leq \int_{V_{\phi}(C_0 \cdot \varepsilon)} |v(y)|^2 \,dy \\
& \leq C \varepsilon^{2s} \|v\|_{H^s(U_{\phi}(c_0))}^2 = C \varepsilon^{2s} \|f\|_{H^s(Q_{T}(x_0,c_0)) \cap \mc{X})}^2 \leq C \varepsilon^{2s} \|f\|_{H^s(\mc{X})}^2.
\end{align*}

\underline{\textit{Step 3: Lipschitz domain}}.
Finally, we deal with the case where $\mc{X}$ is assumed to be an open, bounded subset of $\Rd$, with Lipschitz boundary. In this case, at every $x_0 \in \partial \mc{X}$, there exists a rigid motion $T_{x_0}: \Rd \to \Rd$ such that $T_{x_0}(x_0) = 0$, a number $c_0(x_0)$, a Lipschitz function $\phi_{x_0}:\Reals^{d - 1} \to [-c_0,c_0]$, and a number $C_0(x_0)$, such that for all $\varepsilon \cdot C_0(x_0) \leq c_0(x_0)$,
\begin{equation*}
T\bigl(Q_{T}(x_0,c_0(x_0)) \cap \partial_{\varepsilon}\mc{X}\bigr) \subseteq V_{\phi}\bigl(C_0(x_0) \cdot \varepsilon\bigr) \quad\textrm{and}\quad T\bigl(Q_T(x_0,c_0(x_0)) \cap \mc{X}\bigr) = U_{\phi}(c_0(x_0)).
\end{equation*}
Therefore for every $x_0 \in \partial \mc{X}$, it follows from the previous step that
\begin{equation*}
\int_{Q_{T_{x_0}}(x_0,c_0(x_0)) \cap \partial_{\varepsilon}\mc{X}} |f(x)|^2 \,dx \leq C(x_0) \varepsilon^{2s} \|f\|_{H^s(\mc{X})}^2,
\end{equation*}
where on the right hand side $C(x_0)$ is a constant that may depend on $x_0$, but not on $u$ or $\varepsilon$.

We conclude by taking a collection of cubes that covers $\partial_{\varepsilon}\mc{X}$ for all $\epsilon$ sufficiently small. First, we note that by a compactness argument there exists a finite subset of the collection of cubes $\{Q_{T_{x_0}}(x_0,c_0(x_0)/2): x_0 \in \partial\mc{X} \}$ which covers $\partial \mc{X}$, say $Q_{T_{x_1}}(x_1,c_0(x_1)/2),\ldots, Q_{T_{x_N}}(x_N,c_0(x_N)/2)$. Then, for any $\varepsilon \leq \min_{i = 1,\ldots,N} c_0(x_i)/2$, it follows from the triangle inequality that
\begin{equation*}
\partial_{\varepsilon}\mc{X} \subseteq \bigcup_{i = 1}^{N} Q_{T_{x_i}}(x_i, c_0(x_i)).
\end{equation*}
As a result,
\begin{equation*}
\int_{\partial_{\varepsilon}\mc{X}} |f(x)|^2 \leq \sum_{i = 1}^{N} \int_{Q_{T_{x_i}}(x_i, c_0(x_i)) \cap \partial_{\varepsilon}(\mc{X})} |f(x)|^2 \leq  \varepsilon^{2s} \|f\|_{H^s(\mc{X})}^2 \sum_{i = 1}^{N}C_0(x_i),
\end{equation*}
which proves the claim of~\eqref{pf:approximation_error_nonlocal_laplacian_boundary_0}.

\subsection{Estimate of non-local Sobolev seminorm}
\label{subsec:estimate_nonlocal_seminorm}

Now, we use the results of the preceding two sections to prove~\eqref{pf:graph_seminorm_ho_2}. We will divide our analysis in two cases, depending on whether $s$ is odd or even, but before we do this we state some facts that will be applicable to both cases. First, we  recall that $L_{P,\varepsilon}$ is self-adjoint in $L^2(P)$, meaning $\dotp{L_{P,\varepsilon}f}{g}_{P} = \dotp{f}{L_{P,\varepsilon}g}_{P}$ for all $f, g \in L^2(P)$. We also recall the definition of the Dirichlet energy $E_{P,\varepsilon}(f;\mc{X})$,
\begin{equation}
\label{eqn:dirichlet_energy}
\dotp{L_{P,\varepsilon}f}{f}_{P} = \frac{1}{\varepsilon^{d + 2}}\int_{\mc{X}} \int_{\mc{X}} \bigl(f(x) - f(x')\bigr)^2 \eta\biggl(\frac{\|x' - x\|}{\varepsilon}\biggr) \,dP(x') \,dP(x) =: E_{P,\varepsilon}(f;\mc{X}).
\end{equation}
Finally, we recall a result of \textcolor{blue}{(Green 21)}: there exist constants $c_0$ and $C_0$ which do not depend on $M$, such that for all $\varepsilon < c_0$ and for any $f \in H^1(\mc{X};M)$,
\begin{equation}
\label{pf:estimate_nonlocal_seminorm_1}
E_{P,\varepsilon}(f;\mc{X}) \leq C_0 M^2.
\end{equation}
\paragraph{Case 1: $s$ odd.}
Suppose $s$ is odd, so that $s \geq 3$. Taking $k = (s - 1)/2$, we use the self-adjointness of $L_{P,\varepsilon}$ to relate the non-local semi-norm $\dotp{L_{P,\varepsilon}^sf}{f}_{P}$ to a non-local Dirichlet energy,
\begin{equation*}
\dotp{L_{P,\varepsilon}^sf}{f}_P = \dotp{L_{P,\varepsilon}^{k + 1}f}{L_{P,\varepsilon}^{k}f}_P = E_{P,\varepsilon}(L_{P,\varepsilon}^{k}f;\mc{X}).
\end{equation*}
We now separate this energy into integrals over $\mc{X}_{k\varepsilon}$ and $\partial_{k\varepsilon}(\mc{X})$,
\begin{align}
E_{P,\varepsilon}(L_{P,\varepsilon}^{k}f;\mc{X}) & = \frac{1}{\varepsilon^{d + 2}}\Biggl\{\int_{\mc{X}_{k\varepsilon}} \int_{\mc{X}_{k\varepsilon}} \bigl(L_{P,\varepsilon}^kf(x) - L_{P,\varepsilon}^kf(x')\bigr)^2 \eta\biggl(\frac{\|x' - x\|}{\varepsilon}\biggr) \,dP(x') \,dP(x) \nonumber \\
& \quad + \int_{\partial_{k\varepsilon}\mc{X}} \int_{\partial_{k\varepsilon}\mc{X}} \bigl(L_{P,\varepsilon}^kf(x) - L_{P,\varepsilon}^kf(x')\bigr)^2 \eta\biggl(\frac{\|x' - x\|}{\varepsilon}\biggr) \,dP(x') \,dP(x)\Biggr\} \nonumber \\
& := E_{P,\varepsilon}(L_{P,\varepsilon}^{k}f;\mc{X}_{k\varepsilon}) + E_{P,\varepsilon}(L_{P,\varepsilon}^{k}f;\partial_{k\varepsilon}\mc{X}) \label{pf:estimate_nonlocal_seminorm_1.5}
\end{align}
and upper bound each energy separately. For the first term, we add and substract $\sigma_{\eta}^k\Delta_P^kf(x)$ and $\sigma_{\eta}^k\Delta_P^kf(x')$ within the integrand, then use the triangle inequality and \textcolor{red}{the symmetry between $x$ and $x'$} to deduce that
\begin{equation}
\label{pf:estimate_nonlocal_seminorm_2}
E_{P,\varepsilon}(L_{P,\varepsilon}^{k}f;\mc{X}_{k\varepsilon}) \leq 3 \sigma_{\eta}^{2k} E_{P,\varepsilon}(\Delta_P^kf;\mc{X}_{k\varepsilon}) + \frac{2}{\varepsilon^{d + 2}}\int_{\mc{X}_{k\varepsilon}} \int_{\mc{X}_{k\varepsilon}} \bigl(L_{P,\varepsilon}^kf(x) - \sigma_{\eta}^k \Delta_P^kf(x)\bigr)^2 \eta\biggl(\frac{\|x' - x\|}{\varepsilon}\biggr) \,dP(x') \,dP(x).
\end{equation}
Noticing that $\Delta_P^kf \in H^1(\mc{X};\|p\|_{C^{s - 1}(\mc{X})}^kM)$, we use~\eqref{pf:estimate_nonlocal_seminorm_1} to conclude that $E_{P,\varepsilon}(\Delta_P^kf;\mc{X}_{k\varepsilon}) \leq C_0M^2$. On the other hand, it follows from Assumption~\ref{asmp:kernel_flat_euclidean} and~\eqref{eqn:approximation_error_nonlocal_laplacian_1} that
\begin{align*}
\frac{2}{\varepsilon^{d + 2}}\int_{\mc{X}_{k\varepsilon}} \int_{\mc{X}_{k\varepsilon}} \bigl(L_{P,\varepsilon}^kf(x) - \sigma_{\eta}^k \Delta_P^kf(x)\bigr)^2 \eta\biggl(\frac{\|x' - x\|}{\varepsilon}\biggr) \,dP(x') \,dP(x) & \leq \frac{2p_{\max}}{\varepsilon^{2}}\int_{\mc{X}_{k\varepsilon}} \bigl(L_{P,\varepsilon}^kf(x) - \sigma_{\eta}^k \Delta_P^kf(x)\bigr)^2 \,dP(x) \\
& \leq C_1M^2.
\end{align*}
Plugging these two bounds into~\eqref{pf:estimate_nonlocal_seminorm_2} gives the desired upper bound on $E_{P,\varepsilon}(L_{P,\varepsilon}^{k};\mc{X}_{k\varepsilon})$. 

For the second term in~\eqref{pf:estimate_nonlocal_seminorm_1.5}, we apply Lemmas~\ref{lem:dirichlet_estimate_nonlocal_laplacian} and~\ref{lem:approximation_error_nonlocal_laplacian_boundary} and conclude that,
\begin{equation*}
E_{P,\varepsilon}(L_{P,\varepsilon}^{k}f;\partial_{k\varepsilon}\mc{X}) \leq \frac{4p_{\max}^2}{\varepsilon^2}\|L_{P,\varepsilon}^kf\|_{L^2(\partial_{k\varepsilon}\mc{X})} \leq C M^2.
\end{equation*}
\paragraph{Case 2: $s$ even.}
If $s \in \mathbb{N}$ is even, $s \geq 2$, then letting $k = (s - 2)/2$, the  self-adjointness of $L_{P,\varepsilon}$ implies
\begin{equation*}
\dotp{L_{P,\varepsilon}^sf}{f}_P = \|L_{P,\varepsilon}^{k + 1}f\|_P^2.
\end{equation*}
As in the first case, we divide the integral up into the interior region $\mc{X}_{k\varepsilon}$ and the boundary region $\partial_{k\varepsilon}\mc{X}$,
\begin{equation}
\label{pf:estimate_nonlocal_seminorm_3}
\|L_{P,\varepsilon}^{k + 1}f\|_P^2 \leq p_{\max} \|L_{P,\varepsilon}^{k + 1}f\|_{\Leb^2(\mc{X})}^2 \leq p_{\max}\biggl\{\int_{\mc{X}_{k\varepsilon}} \bigl(L_{P,\varepsilon}^{k + 1}f(x)\bigr)^2 \,dP(x) + \int_{\partial_{k\varepsilon}\mc{X}} \bigl(L_{P,\varepsilon}^{k + 1}f(x)\bigr)^2 \,dP(x)\biggr\},
\end{equation}
and upper bound each term separately. For the first term, adding and subtracting $\sigma_{\eta}^k \Delta_P^kf(x)$ gives
\begin{align*}
\int_{\mc{X}_{k\varepsilon}} \bigl(L_{P,\varepsilon}^{k + 1}f(x)\bigr)^2 \,dP(x) & \leq 2\int_{\mc{X}_{k\varepsilon}} \bigl(L_{P,\varepsilon}\Delta_P^kf(x)\bigr)^2 \,dP(x)  + 2 \int_{\mc{X}_{k\varepsilon}} \Bigl(L_{P,\varepsilon}\bigl(L_{P,\varepsilon}^kf- \sigma_{\eta}\Delta_P^kf\bigr)(x)\Bigr)^2 \,dP(x) \\
& \overset{(i)}{\leq} CM^2  + 2 \int_{\mc{X}_{k\varepsilon}} \Bigl(L_{P,\varepsilon}\bigl(L_{P,\varepsilon}^kf- \sigma_{\eta}\Delta_P^kf\bigr)(x)\Bigr)^2 \,dP(x) \\
& \overset{(ii)}{\leq} CM^2  + \frac{Cp_{\max}^2}{\varepsilon^{2}} \|L_{P,\varepsilon}^kf- \sigma_{\eta}\Delta_P^kf\|_{L^2(\mc{X}_{k\varepsilon})}^2 \\
& \overset{(iii)}{\leq} CM^2,
\end{align*}
with $(i)$ following from~\eqref{pf:approximation_error_nonlocal_laplacian_0} since $\Delta_P^kf \in H^2(\mc{X};M\|p\|_{C^{s - 1}(\mc{X})}^l)$, $(ii)$ following from Lemma~\ref{lem:l2estimate_nonlocal_laplacian}, and $(iii)$ following from~\eqref{eqn:approximation_error_nonlocal_laplacian_2}.

Then Lemma~\ref{lem:approximation_error_nonlocal_laplacian_boundary} shows that the second term in~\eqref{pf:estimate_nonlocal_seminorm_3} satisfies
\begin{equation*}
\int_{\partial_{k\varepsilon}\mc{X}} \bigl(L_{P,\varepsilon}^{k + 1}f(x)\bigr)^2 \,dP(x) \leq CM^2.
\end{equation*}

\subsection{Assorted integrals}
\label{subsec:integrals}

\textcolor{red}{(TODO): The language ``...for a Borel set $U \subseteq \mc{X}$...'' should be checked.}

\begin{lemma}
	\label{lem:l2estimate_nonlocal_laplacian}
	Assume Model~\ref{def:model_flat_euclidean}. Suppose $f \in L^2(U;M)$ for a Borel set $U \subseteq \mc{X}$, and let $L_{P,\varepsilon}$ be defined with respect to a kernel $\eta$ that satisfies~\ref{asmp:kernel_flat_euclidean}. Then there exists a constant $C$ which does not depend on $f$ or $M$ such that
	\begin{equation}
	\label{eqn:l2estimate_nonlocal_laplacian}
	\|L_{P,\varepsilon}f\|_{\Leb^2(U)} \leq \frac{2 p_{\max}}{\varepsilon^2} \|f\|_{L^2(U)}
	\end{equation}
\end{lemma}

\begin{lemma}
	\label{lem:dirichlet_estimate_nonlocal_laplacian}
	Assume Model~\ref{def:model_flat_euclidean}. Suppose $f \in L^2(U;M)$ for a Borel set $U \subseteq \mc{X}$, and let $L_{P,\varepsilon}$ be defined with respect to a kernel $\eta$ that satisfies~\ref{asmp:kernel_flat_euclidean}. Then there exists a constant $C$ which does not depend on $f$ or $M$ such that
	\begin{equation}
	\label{eqn:dirichlet_estimate_nonlocal_laplacian}
	E_{P,\varepsilon}(f;U) \leq \frac{4 p_{\max}^2}{\varepsilon^2} \|f\|_{L^2(U)}^2
	\end{equation}
\end{lemma}

\begin{lemma}
	\label{lem:graph_seminorm_bias2}
	Assume Model~\ref{def:model_flat_euclidean}. Suppose $f \in H^1(\mc{X};M)$, and let $D_if$ be defined with respect to a kernel $\eta$ that satisfies~\ref{asmp:kernel_flat_euclidean}. Then there exists a constant $C$ which does not depend on $f$ or $M$, such that for any $i \in [n]$ and $\bj \in [n]^s$,
	\begin{equation*}
	\Ebb\Bigl[|D_{\bj}f(X_i)| \cdot |f(X_i) - f(X_{\bj_1})|\Bigr] \leq C \varepsilon^{2 + dk}M^2,
	\end{equation*}
	where $k + 1$ is the number of distinct indices in $i\bj$. 
\end{lemma}

\paragraph{Proof (of Lemma~\ref{lem:l2estimate_nonlocal_laplacian}).}
We fix a version of $f \in \Leb^2(U)$, so that we may speak of its pointwise values.

At a given point $x \in U$, we can upper bound $|L_{P,\varepsilon}f(x)|^2$ using the Cauchy-Schwarz inequality as follows,
\begin{align*}
|L_{P,\varepsilon}f(x)|^2 & \leq \biggl(\frac{p_{\max}}{\varepsilon^{2 + d}}\biggr)^2 \Biggl(\int_U \bigl(|f(x')| + |f(x)|\bigr)^2 \eta\biggl(\frac{\|x' - x\|}{\varepsilon}\biggr) \,dx'\Biggr)^2 \\
& \leq \biggl(\frac{p_{\max}}{\varepsilon^{2 + d}}\biggr)^2 \Biggl(\int_U \bigl(|f(x')| + |f(x)|\bigr)^2 \eta\biggl(\frac{\|x' - x\|}{\varepsilon}\biggr) \,dx' \cdot \int \eta\biggl(\frac{\|x' - x\|}{\varepsilon}\biggr) \,dx' \Biggr) \\
& = \frac{p_{\max}^2}{\varepsilon^{4 + d}} \int_U \bigl(|f(x')| + |f(x)|\bigr)^2 \eta\biggl(\frac{\|x' - x\|}{\varepsilon}\biggr) \,dx'.
\end{align*}
The equality follows by the assumption $\int_{\Rd} \eta(\|z\|)\,dx = 1$ in~\ref{asmp:kernel_flat_euclidean}. Integrating over all $x \in U$, it follows from the triangle inequality that
\begin{align}
\|L_{P,\varepsilon}\|_{\Leb^2(U)}^2 & \leq \frac{2 p_{\max}^2}{\varepsilon^{4 + d}} \int_{U} \int_{U} \bigl(|f(x')|^2 + |f(x)|^2\bigr) \eta\biggl(\frac{\|x' - x\|}{\varepsilon}\biggr) \,dx' \,dx \nonumber \\
& \leq \frac{2 p_{\max}^2}{\varepsilon^{4 + d}} \int_{U} \int_{U} \bigl(|f(x')|^2 + |f(x)|^2\bigr) \eta\biggl(\frac{\|x' - x\|}{\varepsilon}\biggr) \,dx' \,dx. \label{pf:l2estimate_nonlocal_laplacian_1}
\end{align}
Finally, using Fubini's Theorem we determine that
\begin{equation}
\int_{U} \int_{U} \bigl(|f(x')|^2 + |f(x)|^2\bigr) \eta\biggl(\frac{\|x' - x\|}{\varepsilon}\biggr) \,dx' \,dx = 2 \int_{U} \int_{U} |f(x)|^2 \eta\biggl(\frac{\|x' - x\|}{\varepsilon}\biggr) \,dx \leq 2 \varepsilon^d \int_{U} |f(x)|^2 \,dx = 2\varepsilon^d \|f\|_{\Leb^2(U)}^2,
\label{pf:l2estimate_nonlocal_laplacian_2}
\end{equation}
and by combining~\eqref{pf:l2estimate_nonlocal_laplacian_1} and~\eqref{pf:l2estimate_nonlocal_laplacian_2} we conclude that
\begin{equation*}
\|L_{P,\varepsilon}\|_{\Leb^2(U)}^2 \leq \frac{4p_{\max}^2}{\varepsilon^4} \|f\|_{\Leb^2(U)}^2. 
\end{equation*}

\paragraph{Proof (of Lemma~\ref{lem:dirichlet_estimate_nonlocal_laplacian}).}
We have
\begin{equation*}
E_{P,\varepsilon}(f) = \frac{1}{\varepsilon^{2 + d}} \int_{U} \int_{U} \bigl(f(x) - f(x')\bigr)^2 \eta\biggl(\frac{\|x' - x\|}{\varepsilon}\biggr) \,dP(x') \,dP(x) \leq \frac{2p_{\max}^2}{\varepsilon^{2 + d}} \int_{U} \int_{U} \bigl(|f(x)|^2 + |f(x')|^2\bigr) \eta\biggl(\frac{\|x' - x\|}{\varepsilon}\biggr) \,dx' \,dx,
\end{equation*}
and the claim follows from~\eqref{pf:l2estimate_nonlocal_laplacian_2}.

\paragraph{Proof (of Lemma~\ref{lem:graph_seminorm_bias2}).}
Let $G_{n,\varepsilon}[X_{i\bj}]$ be the subgraph induced by vertices $X_i, X_{\bj_1},\ldots,X_{\bj_s}$. We make two observations. First, in order for $|D_{\bj}f(X_i)| \cdot |f(X_i)  - f(X_j)|$ to be non-zero, it must be the case that the subgraph $G_{n,\varepsilon}[X_{i\bj}]$ is connected. Second, noting that for any indices $i$ and $j$,
\begin{equation*}
|D_{ij}f(x)| \leq \Bigl(|D_jf(X_i)| + |D_jf(x)|\Bigr)\|\eta\|_{\infty}, 
\end{equation*}
a straightforward inductive argument implies that 
\begin{equation*}
|D_{\bj}f(X_i)| \leq s\|\eta\|_{\infty}^s \sum_{j \in i\bj} |D_{\bj_s}f(X_j)|.
\end{equation*}
Combining these two observations, we reduce the task to upper bounding the product of two (first-order) differences,
\begin{align*}
\mathbb{E}\Bigl[ |D_{\bj}f(X_i)| |f(X_i) - f(X_{\bj_1})| \Bigr] & = \mathbb{E}\Bigl[ |D_{\bj}f(X_i)| |f(X_i) - f(X_{\bj_1})| \cdot \1\bigl\{G_{n,\varepsilon}[X_{i\bj}]~~\textrm{is connected.} \bigr\} \Bigr] \\
& \leq s\|\eta\|_{\infty}^s \sum_{j \in i\bj} \mathbb{E}\Bigl[ |D_{\bj_s}f(X_j)| \cdot |f(X_i) - f(X_{\bj_1})| \cdot \1\bigl\{G_{n,\varepsilon}[X_{i\bj}]~~\textrm{is connected.} \bigr\}   \Bigr] \\
& \leq s\|\eta\|_{\infty}^s \sum_{j \in i\bj} \mathbb{E}\Bigl[ |f(X_j) - f(X_{\bj_s})| \cdot |f(X_i) - f(X_{\bj_1})| \cdot \1\bigl\{G_{n,\varepsilon}[X_{i\bj}]~~\textrm{is connected.} \bigr\}   \Bigr] 
\end{align*}
Next, from the Cauchy-Schwarz inequality we have that for any $j \in \bj$,
\begin{align*}
& \mathbb{E}\Bigl[ |f(X_j) - f(X_{\bj_s})| \cdot |f(X_i) - f(X_{\bj_1})| \cdot \1\bigl\{G_{n,\varepsilon}[X_{i\bj}]~~\textrm{is connected.} \bigr\}   \Bigr] \\
& \quad \leq \sqrt{\mathbb{E}\Bigl[ |f(X_j) - f(X_{\bj_s})|^2 \cdot \1\bigl\{G_{n,\varepsilon}[X_{i\bj}]~~\textrm{is connected.} \bigr\} \Bigr]} \cdot \sqrt{\mathbb{E}\Bigl[ |f(X_j) - f(X_{\bj_s})|^2 \cdot \1\bigl\{G_{n,\varepsilon}[X_{i\bj}]~~\textrm{is connected.} \bigr\} \Bigr]} \\
& \quad = \mathbb{E}\Bigl[ |f(X_j) - f(X_{i})|^2 \cdot \1\bigl\{G_{n,\varepsilon}[X_{i\bj}]~~\textrm{is connected.} \bigr\} \Bigr],
\end{align*}
with the equality following since each $X_i$ are identically distributed. Marginalizing out the contribution of all indices in $\bj$ not equal to $i$ or $j$ gives
\begin{align}
\mathbb{E}\Bigl[ |f(X_j) - f(X_{i})|^2 \cdot \1\bigl\{G_{n,\varepsilon}[X_{i\bj}]~~\textrm{is connected.} \bigr\} \Bigr] & \leq \bigl((s + 1)p_{\max}\nu_d\varepsilon^d\bigr)^{|i\bj\setminus\{j \cup i\}|} \cdot \mathbb{E}\Bigl[ |f(X_j) - f(X_{i})|^2 \1\{\|X_i - X_j\| \leq \varepsilon\}\Bigr] \nonumber \\
& \leq \bigl((s + 1)p_{\max}\nu_d\varepsilon^d\bigr)^{|i\bj\setminus\{j \cup i\}|} \cdot p_{\max}^2 \nu_d \varepsilon^{2 + d} M^2 \label{pf:graph_seminorm_bias2_1}
\end{align}
with the second inequality following from the proof of Lemma~1 in \textcolor{blue}{(Green 2021).} Finally, we notice that $|i\bj\setminus \{i \cup j\}| + 1 = k$, so that~\eqref{pf:graph_seminorm_bias2_1} gives the desired result.

\section{Graph Sobolev semi-norm, manifold domain}
\label{sec:graph_quadratic_form_manifold}
In this section we prove Proposition~\ref{prop:graph_seminorm_manifold}. Note that when $s = 1$, the upper bound~\eqref{eqn:graph_seminorm_manifold} follows immediately from Lemma~\ref{lem:dirichlet_energy_sobolev} and Markov's inequality. 

On the other hand when $s = 2$ or $s = 3$, we prove Proposition~\ref{prop:graph_seminorm_manifold} by first establishing some intermediate results, many of which are analogous to results we have already shown in the flat Euclidean case. Indeed, in some ways the proof will be simpler in the manifold setting than in the flat Euclidean case: there is no boundary, and we do not need to analyze the iterated nonlocal Laplacian $L_{P,\varepsilon}^j$ for $j > 1$. 

That being said, as mentioned in our main text, in the manifold setting there is some extra error induced by using Euclidean rather than geodesic distance. We upper bound this error by comparing $L_{P,\varepsilon}$ to an alternative nonlocal Laplacian $\wt{L}_{P,\varepsilon}$, which is defined with respect to geodesic distance. Precisely, let $d_{\mc{X}}(x,x')$ denote the geodesic distance between $x,x' \in \mc{X}$, and define
\begin{equation*}
\wt{L}_{P,\varepsilon}f(x) := \int_{\mc{X}} \bigl(f(x') - f(x)\bigr) \eta \biggl(\frac{d_{\mc{X}}(x',x)}{\varepsilon}\biggr) p(x') \,dx'.
\end{equation*}

We show the following results, each of which hold under the same assumptions as Proposition~\ref{prop:graph_seminorm_manifold}.
\begin{itemize}
	\item In Section~\ref{subsec:manifold_decomposition_graph_seminorm} we show that the graph Sobolev seminorm $\dotp{L_{n,\varepsilon}^sf}{f}_n$ is upper bounded by the sum of a nonlocal seminorm and a pure bias term: specifically, with probability at least $1 - 2\delta$,
	\begin{equation}
	\label{pf:graph_seminorm_manifold_1}
	\dotp{L_{n,\varepsilon}^sf}{f}_n \leq \frac{\dotp{L_{P,\varepsilon}^sf}{f}_P}{\delta} + C_1\frac{\varepsilon^2}{n\varepsilon^{2s + m}}M^2.
	\end{equation}
	This upper bound is essentially the same as~\eqref{pf:graph_seminorm_ho_1}, but with the intrinsic dimension $m$ taking the place of the ambient dimension $d$. The pure bias term will be of at most constant order when $\varepsilon \gtrsim n^{-1/(2(s-1) + m)}$. 
	\item In Section~\ref{subsec:error_euclidean_distance}, we show that the error incurred by using the ``wrong'' metric is negligible. Precisely, we find that
	\begin{equation}
	\label{eqn:nonlocal_laplacian_geodesic_error}
	\|L_{P,\varepsilon}f - \wt{L}_{P,\varepsilon}f\|_{L^2(\mc{X})}^2 \leq C_2 \varepsilon^2 |f|_{H^1(\mc{X})}^2.
	\end{equation}
	\item In Section~\ref{subsec:manifold_approximation_error_nonlocal_laplacian}, we analyze the approximation error of $\wt{L}_{P,\varepsilon}$. We show that when $f \in H^2(\mc{X})$ and $p \in C^1(\mc{X})$, 
	\begin{equation}
	\label{eqn:nonlocal_laplacian_approximation_error_manifold_l2}
	\|\wt{L}_{P,\varepsilon}f\|_{\Leb^2(\mc{X})}^2 \leq C_3 \|f\|_{H^2(\mc{X})}^2,
	\end{equation}
	whereas if $f \in H^3(\mc{X})$ and $p \in C^2(\mc{X})$, 
	\begin{equation}
	\label{eqn:nonlocal_laplacian_approximation_error_manifold_sobolev}
	\|\wt{L}_{P,\varepsilon}f - \sigma_{\eta}\Delta_Pf\|_{\Leb^2(\mc{X})}^2 \leq C_3 \varepsilon^2 \|f\|_{H^3(\mc{X})}^2.
	\end{equation}
	\item In Section~\ref{subsec:manifold_estimate_nonlocal_seminorm}, we use the results of the preceding two sections to show that if $f \in H^s(\mc{X})$ and $p \in C^{s - 1}(\mc{X})$, then 
	\begin{equation}
	\label{eqn:manifold_nonlocal_seminorm}
	\dotp{L_{P,\varepsilon}^sf}{f}_P \leq C_4\|f\|_{H^s(\mc{X})}^2.
	\end{equation}
	\item In Section~\ref{subsec:manifold_integrals} we state some technical results used in the previous sections.
\end{itemize}
We point out that when $f$ is H\"{o}lder smooth, results analogous to~\eqref{eqn:nonlocal_laplacian_approximation_error_manifold_sobolev} have been established in \citet{calder2019}. When $f$ is Sobolev smooth, our analysis (which relies heavily on Taylor expansions) is largely similar, except that the remainder term in the relevant Taylor expansion will be bounded in $L^2(\mc{X})$ norm rather than $L^{\infty}(\mc{X})$ norm. This is analogous to the situation in the flat Euclidean model.

\paragraph{Proof (of Proposition~\ref{prop:graph_seminorm_manifold}).} Follows immediately from~\eqref{pf:graph_seminorm_manifold_1} and~\eqref{eqn:manifold_nonlocal_seminorm}. \qed

\subsection{Decomposition of graph Sobolev seminorm}
\label{subsec:manifold_decomposition_graph_seminorm}
The proof of~\eqref{pf:graph_seminorm_manifold_1} is identical to the proof of~\eqref{pf:graph_seminorm_ho_1}, except substituting the intrinsic dimension $m$ for ambient dimension $d$, and using Lemma~\ref{lem:manifold_graph_seminorm_bias2} rather than Lemma~\ref{lem:graph_seminorm_bias2}.

\subsection{Error due to Euclidean Distance}
\label{subsec:error_euclidean_distance}

In this section, we prove~\eqref{eqn:nonlocal_laplacian_geodesic_error}. By applying Cauchy-Schwarz we obtain an upper bound on $|L_{P,\varepsilon}f(x) - \wt{L}_{P,\varepsilon}f(x)|^2$:
\begin{align}
\bigl[L_{P,\varepsilon}f(x) - \wt{L}_{P,\varepsilon}f(x)\bigr]^2 & \leq \frac{p_{\max}^2}{\varepsilon^{2(2 + m)}} \int_{\mc{X}} \bigl[f(x') - f(x)\bigr]^2 \biggl|\eta\biggl(\frac{\|x' - x\|}{\varepsilon}\biggr) - \eta\biggl(\frac{d_{\mc{X}}(x',x)}{\varepsilon}\biggr)\biggr| \,d\mu(x') \nonumber \\
& \quad \cdot \int_{\mc{X}} \biggl|\eta\biggl(\frac{\|x' - x\|}{\varepsilon}\biggr) - \eta\biggl(\frac{d_{\mc{X}}(x',x)}{\varepsilon}\biggr)\biggr| \,d\mu(x') \nonumber \\
& = \frac{1}{\varepsilon^{2(2 + m)}} A_1(x) \cdot A_2(x) \label{pf:error_euclidean_distance_0}
\end{align} 
Thus we have upper bounded $|L_{P,\varepsilon}f(x) - \wt{L}_{P,\varepsilon}f(x)|^2$ by the product of two terms, each of which we now suitably bound. 

To do so, we will use the following estimates, from Proposition 4 of \cite{trillos2019}: letting $R$ denote the reach of $\mc{X}$, for all $\|x' - x\| \leq R/2$,
\begin{equation}
\label{eqn:distance_error}
\|x' - x\| \leq d_{\mc{X}}(x',x) \leq \|x' - x\| + \frac{8}{R^2} \|x' - x\|^3.
\end{equation}
\paragraph{Upper bound on $A_1(x)$.}
From here forward we will assume $\varepsilon < R/2$. Consequently $\eta(\|x' - x\|/\varepsilon) \geq \eta(d_{\mc{X}}(x',x)/\varepsilon)$. Furthermore, letting $L_{\eta}$ denote the Lipschitz constant of $\eta$, and setting $\wt{\varepsilon} := (1 + 27\varepsilon^2/R^2)\varepsilon$ we have that
\begin{equation*}
\biggl|\eta\biggl(\frac{\|x' - x\|}{\varepsilon}\biggr) - \eta\biggl(\frac{d_{\mc{X}}(x',x)}{\varepsilon}\biggr)\biggr| \leq \frac{L_{\eta} 8 \varepsilon^2}{R^2} \cdot \1\bigl\{d_{\mc{X}}(x',x) \leq \varepsilon\bigr\} + \|\eta\|_{\infty} \cdot 1\{\varepsilon < d_{\mc{X}}(x',x) \leq \wt{\varepsilon}\}.
\end{equation*}
Thus,
\begin{equation*}
A_1(x) \leq \frac{8L_{\eta}\varepsilon^2}{R^2}\int_{\mc{X}}\bigl[f(x') - f(x)\bigr]^2 \1\{\|x' - x\| \leq \varepsilon\} \,d\mu(x') + \|\eta\|_{\infty} \int_{\mc{X}}\bigl[f(x') - f(x)\bigr]^2 \1\bigl\{\varepsilon < d_{\mc{X}}(x',x) \leq \wt{\varepsilon}\bigr\} \,d\mu(x') \\
\end{equation*}
Integrating over $\mc{X}$, we conclude from Lemma~\ref{lem:dirichlet_energy_remainder} and Lemma~3.3 of \citep{burago2014} and  that
\begin{equation*}
\int_{\mc{X}} A_1(x) \,d\mu(x) \leq \frac{8L_{\eta}\nu_m\varepsilon^2}{R^2(m + 2)} \Bigl(1 + CmKR^2\Bigr) \varepsilon^{m + 2} |f|_{H^1(\mc{X})}^2 + C\|\eta\|_{\infty}\varepsilon^{m + 4} |f|_{H^1(\mc{X})}^2 =: C_5 \varepsilon^{m + 4}|f|_{H^1(\mc{X})}.
\end{equation*}

\paragraph{Upper bound on $A_2(x)$.}
Integrating over $x' \in \mc{X}$, we see that
\begin{align}
\int_{\mc{X}} \biggl|\eta\biggl(\frac{\|x' - x\|}{\varepsilon}\biggr) - \eta\biggl(\frac{d_{\mc{X}}(x',x)}{\varepsilon}\biggr)\biggr| \,d\mu(x') & \leq \frac{8L_{\eta}\varepsilon^2}{R^2} \int_{\mc{X}} \1\bigl\{d_{\mc{X}}(x',x)\bigr\} \,d\mu(x') + p_{\max}\|\eta\|_{\infty} \int_{\mc{X}} \1\bigl\{ \varepsilon < d_{\mc{X}}(x',x) \leq \wt{\varepsilon} \bigr\} \,d\mu(x') \nonumber \\
& = \frac{8L_{\eta}\varepsilon^2}{R^2} \cdot \mu\bigl(B(x,\varepsilon)\bigr) +  p_{\max}\|\eta\|_{\infty} \Bigl[\mu\bigl(B(x,\wt{\varepsilon})\bigr) - \mu\bigl(B(x,\varepsilon)\bigr) \Bigr]. \label{pf:error_euclidean_distance_1}
\end{align}
Equation (1.36) in \cite{trillos2019} states that
\begin{equation*}
\bigl|\mu(B_\mc{X}(x,\varepsilon))  - \omega_m \varepsilon^m\bigr|  \leq CmK\varepsilon^{m + 2},
\end{equation*}
where $K$ is an upper bound on the \textcolor{red}{sectional curvature} of $\mc{X}$. Plugging this back into~\eqref{pf:error_euclidean_distance_1}, we conclude that
\begin{align*}
\int_{\mc{X}} \biggl|\eta\biggl(\frac{\|x' - x\|}{\varepsilon}\biggr) - \eta\biggl(\frac{d_{\mc{X}}(x',x)}{\varepsilon}\biggr)\biggr| \,d\mu(x') & \leq  \frac{8L_{\eta}\varepsilon^2}{R^2}\Bigl[\omega_m\varepsilon^{m} + CmK\varepsilon^{m + 2}\Bigr] + \|\eta\|_{\infty} \Bigl[\omega_m(\wt{\varepsilon}^m - {\varepsilon}^m) + 2CmK\varepsilon^{m + 2}\Bigr] \\
& \leq \frac{8L_{\eta}\varepsilon^2}{R^2}\Bigl[\omega_m\varepsilon^{m} + R^2CmK\varepsilon^{m}\Bigr] + \|\eta\|_{\infty} \varepsilon^{m + 2}\Bigl[\frac{27\omega_m}{R^2} + 2CmK\Bigr] \\
& =: C_6 \varepsilon^{m + 2}.
\end{align*}

\paragraph{Putting together the pieces.}
Plugging our upper bounds on $A_1(x)$ and $A_2(x)$ back into~\eqref{pf:error_euclidean_distance_0}, we deduce that
\begin{align*}
\|\wt{L}_{P,\varepsilon}f - L_{P,\varepsilon}f\|_{L^2(\mc{X})}^2 & \leq \frac{1}{\varepsilon^{2(2 + m)}} \int_{\mc{X}} A_1(x) \cdot A_2(x) \,d\mu(x) \\
& \leq \frac{C_6}{\varepsilon^{(2 + m)}} \int_{\mc{X}} A_1(x) \,d\mu(x) \\
& \leq C_5 C_6 \varepsilon^2 |f|_{H^1(\mc{X})}^2,
\end{align*}
thus proving the claimed result.

\subsection{Approximation Error of non-local Laplacian}
\label{subsec:manifold_approximation_error_nonlocal_laplacian}
Fix $x \in \mc{X}$. We begin with a pointwise estimate of $\wt{L}_{P,\varepsilon}f$, facilitated by expressing $w(v) = f(\exp_x(v))$ and $q(v) = p(\exp_x(v))$ in normal coordinates, as in \citep{calder2019}. Letting $J_x(v)$ be the Jacobian of the exponential map $\exp_x: B(0,\varepsilon) \subseteq T_x(\mc{X}) \to B_{\mc{X}}(x,\varepsilon)$ evaluated at $v \in T_x(\mc{X})$, we have
\begin{align*}
\wt{L}_{P,\varepsilon}f(x) & = \frac{1}{\varepsilon^{m + 2}} \int_{\mc{X}} \bigl(f(x') - f(x)\bigr) \eta\biggl(\frac{d_{\mc{X}}(x',x)}{\varepsilon}\biggr) \,dP(x') \\
& = \frac{1}{\varepsilon^{m + 2}} \int_{B(0,\varepsilon) \subset T_x(\mc{X})} \bigl(w(v) - w(0)\bigr) \eta\biggl(\frac{\|v\|}{\varepsilon}\biggr) J_x(v) q(v) \,dv \\
& = \frac{1}{\varepsilon^{2}}\biggl\{\int_{B(0,1)} \bigl(w(\varepsilon v) - w(0)\bigr) \eta(\|v\|) q(\varepsilon v) \,dv + \int_{B(0,1)} \bigl(w(\varepsilon v) - w(0)\bigr) \eta(\|v\|) q(\varepsilon v) \bigl(J_x(\varepsilon v) - 1\bigr) \,dv \biggr\} \\
& = A_1(x) + A_2(x)
\end{align*}
Note that $w$ and $q$ have the same smoothness properties as $f$ and $p$. Moreover, arguing exactly as we did in the flat Euclidean case, we can show that when $f \in H^2(\mc{X})$ and $p \in C^1(\mc{X})$, then
\begin{equation*}
\|A_1\|_{\Leb^2(\mc{X})}^2 \leq C \|f\|_{H^2(\mc{X})}^2
\end{equation*}
whereas if $f \in H^3(\mc{X})$ and $p \in C^2(\mc{X})$ then 
\begin{equation*}
\|A_1 - \sigma_{\eta} \Delta_Pf\|_{\Leb^2(\mc{X})}^2 \leq C\|f\|_{H^3(\mc{X})}^2 \varepsilon^2.
\end{equation*}
\textcolor{red}{(TODO): This is correct, but feels unacceptably vague. On the other hand, I don't want to have to redo the entire proof from the flat Euclidean section, since all the steps will be basically identical. Ask Ryan + Siva for guidance.}

Therefore it remains only to upper bound $A_2$ in $\Leb^2(\mc{X})$ norm. To do so, we recall (1.34) of \cite{trillos2019}: for any $\varepsilon < i_0$ and all $x \in \mc{X}$, the Jacobian $J_x(v)$ satisfies the upper bound
\begin{equation*}
|J_x(v) - 1| \leq CmK\varepsilon^2, \quad  \textrm{for all} ~~ v \in B(0,\varepsilon) \subseteq T_x(\mc{X}).
\end{equation*}
Combining this estimate with the Cauchy-Schwarz inequality, we conclude that
\begin{align*}
\|A_2\|_{\Leb^2(\mc{X})}^2 & \leq Cm^2K^2 \biggl[\int_{B(0,1)} \bigl(w(\varepsilon v) - w(0)\bigr)^2 \eta(\|v\|) q(\varepsilon v) \,dv\biggr] \cdot \biggl[\int_{B(0,1)} \eta(\|v\|) q(\varepsilon v) \,dv\biggr] \\
& \leq Cm^2K^2 \sigma_{\eta} (1 + L_q\varepsilon)  \int_{B(0,1)} \bigl(w(\varepsilon v) - w(0)\bigr)^2 \eta(\|v\|) q(\varepsilon v) \,dv \\
& \leq Cm^2K^2 \sigma_{\eta}^2 (1 + L_q\varepsilon) p_{\max}  \varepsilon^2 |f|_{H^1(\mc{X})}^2,
\end{align*}
with the final inequality following from (3.2) of~\cite{burago2014}. Combining our estimates on $A_1$ and $A_2$ yields the claim.

\subsection{Estimate of non-local Sobolev seminorm}
\label{subsec:manifold_estimate_nonlocal_seminorm}
In this subsection we establish that the upper bound~\eqref{eqn:manifold_nonlocal_seminorm} holds when $f \in H^s(\mc{X})$ and $p \in C^{s - 1}(\mc{X})$. We first consider $s = 2$, and then $s = 3$.

\paragraph{Case 1: $s = 2$.}
When $s = 2$, the triangle inequality implies that
\begin{equation*}
\dotp{L_{P,\varepsilon}^sf}{f}_P \leq 2p_{\max}\Bigl(\|L_{P,\varepsilon}f - \wt{L}_{P,\varepsilon}\|_{L^2(\mc{X})}^2 + \|\wt{L}_{P,\varepsilon}f\|_{L^2(\mc{X})}^2\Bigr)
\end{equation*}
The first term on the right hand side is upper bounded in~\eqref{eqn:nonlocal_laplacian_geodesic_error}, and the second term is upper bounded in~\eqref{eqn:nonlocal_laplacian_approximation_error_manifold_l2}. Together these estimates imply the claim.

\paragraph{Case 2: $s = 3$.}
When $s = 3$, the triangle inequality implies that
\begin{equation*}
\dotp{L_{P,\varepsilon}^sf}{f}_P = E_{P,\varepsilon}(L_{P,\varepsilon}f;\mc{X}) \leq 3\Bigl( E_{P,\varepsilon}(L_{P,\varepsilon}f - \wt{L}_{P,\varepsilon}f;\mc{X}) +  E_{P,\varepsilon}(\wt{L}_{P,\varepsilon}f - \sigma_{\eta} \Delta_Pf;\mc{X}) + \sigma_{\eta}^2 E_{P,\varepsilon}(\Delta_Pf;\mc{X})\Bigr)
\end{equation*}
We now upper bound each of the three terms on the right hand side of the above inequality. First, we note that by Lemma~\ref{lem:dirichlet_energy_l2} and~\eqref{eqn:nonlocal_laplacian_geodesic_error}, 
\begin{equation*}
 E_{P,\varepsilon}(L_{P,\varepsilon}f - \wt{L}_{P,\varepsilon}f;\mc{X}) \leq  \frac{C}{\varepsilon^2}\|L_{P,\varepsilon}f - \wt{L}_{P,\varepsilon}f\|_{\Leb^2(\mc{X})}^2 \leq C |f|_{H^1(\mc{X})}^2.
\end{equation*}
An equivalent upper bound on $E_{P,\varepsilon}(\wt{L}_{P,\varepsilon}f - \sigma_{\eta} \Delta_Pf;\mc{X})$ follows from Lemma~\ref{lem:dirichlet_energy_l2} and~\eqref{eqn:nonlocal_laplacian_approximation_error_manifold_sobolev}. Finally, we notice that $f \in H^3(\mc{X})$ and $p \in C^2(\mc{X})$ implies $\Delta_Pf \in H^1(\mc{X})$, and furthermore $|\Delta_Pf|_{H^1(\mc{X})} \leq \|p\|_{C^2(\mc{X})} \cdot \|f\|_{H^3(\mc{X})}$. We conclude from Lemma~\ref{lem:dirichlet_energy_sobolev} that
\begin{equation*}
E_{P,\varepsilon}(\Delta_Pf;\mc{X}) \leq C |\Delta_Pf|_{H^1(\mc{X})}^2 \leq C \|f\|_{H^3(\mc{X})}^2,
\end{equation*}
where in the final inequality we have absorbed $\|p\|_{C^2(\mc{X})}$ into the constant $C$. Together, these upper bounds prove the claim.

\subsection{Integrals}
\label{subsec:manifold_integrals}
Recall the Dirichlet energy $E_{P,\varepsilon}(f;\mc{X}) = \dotp{L_{P,\varepsilon}f}{f}_P$, defined in~\eqref{eqn:dirichlet_energy}. Now we establish some estimates on $E_{P,\varepsilon}(f;\mc{X})$ under Model~\ref{def:model_manifold}, and under various assumptions regarding the regularity of $f$.
\begin{lemma}
	\label{lem:dirichlet_energy_l2}
	Suppose Model~\ref{def:model_manifold}, and additionally that $f \in L^2(\mc{X})$. Then there exists a constant $C$ such that
	\begin{equation}
	\label{eqn:dirichlet_energy_l2}
	E_{P,\varepsilon}(f;\mc{X}) \leq \frac{C}{\varepsilon^2} \|f\|_{L^2(\mc{X})}^2.
	\end{equation}
\end{lemma}
\begin{lemma}
	\label{lem:dirichlet_energy_sobolev}
	Suppose Model~\ref{def:model_manifold}, and additionally that $f \in H^1(\mc{X})$. Then there exist constants $c$ and $C$ which do not depend on $f$ such that for any $0 < \varepsilon < c$,
	\begin{equation}
	\label{eqn:dirichlet_energy_sobolev}
	E_{P,\varepsilon}(f;\mc{X}) \leq C |f|_{H^1(\mc{X})}^2.
	\end{equation}
\end{lemma}

We use Lemma~\ref{lem:dirichlet_energy_remainder} to help upper bound the error incurred by using $\|\cdot\|$ rather than $d_{\mc{X}}(\cdot,\cdot)$. Recall the notation $\wt{\varepsilon} = (1 + 27\varepsilon^2/R^2)\varepsilon$, where $R$ is the reach of $\mc{X}$.
\begin{lemma}
	\label{lem:dirichlet_energy_remainder}
	Suppose Model~\ref{def:model_manifold}, and additionally that $f \in H^1(\mc{X})$. There exist constants $c$ and $C$ such that for any $\varepsilon < c$,
	\begin{equation}
	\label{eqn:dirichlet_energy_remainder}
	\int_{\mc{X}} \int_{\mc{X}} \bigl(f(x') - f(x)\bigr)^2 \1\{\varepsilon < d_{\mc{X}}(x',x) \leq \wt{\varepsilon}\} \,d\mu(x') \,d\mu(x) \leq C \varepsilon^{4 + m} \|f\|_{H^1(\mc{X})}^2
	\end{equation}
\end{lemma}

Finally, we use Lemma~\ref{lem:manifold_graph_seminorm_bias2} to show that the pure bias component of $\dotp{L_n^sf,f}_n$ is small in expectation. This is analogous to Lemma~\ref{lem:graph_seminorm_bias2}, except assuming Model~\ref{def:model_manifold} rather than Model~\ref{def:model_flat_euclidean}.
\begin{lemma}
	\label{lem:manifold_graph_seminorm_bias2}
	Assume Model~\ref{def:model_manifold}. Suppose $f \in H^1(\mc{X})$, and let $D_if$ be defined with respect to a kernel $\eta$ that satisfies~\ref{asmp:kernel_manifold}. Then there exists a constant $C$ which does not depend on $f$ or $n$, such that for any $i \in [n]$ and $\bj \in [n]^s$,
	\begin{equation*}
	\Ebb\Bigl[|D_{\bj}f(X_i)| \cdot |f(X_i) - f(X_{\bj_1})|\Bigr] \leq C \varepsilon^{2 + mk} \cdot \|f\|_{H^1(\mc{X})}^2,
	\end{equation*}
	where $k + 1$ is the number of distinct indices in $i\bj$. 
\end{lemma}

\paragraph{Proof (of Lemmas~\ref{lem:dirichlet_energy_l2} and~\ref{lem:dirichlet_energy_sobolev}).}
Define the non-local energy $\wt{E}_{P,\varepsilon}$ with respect to geodesic distance,
\begin{equation*}
\wt{E}_{P,\varepsilon}(f;{\mc{X}}) := \dotp{\wt{L}_{P,\varepsilon}f}{f}_{P} = \int_{\mc{X}} \int_{\mc{X}} \bigl(f(x') - f(x)\bigr)^2 \eta\biggl(\frac{d_{\mc{X}}(x',x)}{\varepsilon}\biggr) \,dP(x') \,dP(x).
\end{equation*}
From the lower bound in~\eqref{eqn:distance_error}, it follows that $E_{P,\varepsilon}(f;X) \leq \wt{E}_{P,\varepsilon}(f;{\mc{X}})$, and from the upper bounds $p(x) \leq p_{\max}$ and $\eta(|x|) \leq \|\eta\|_{\infty} \cdot \1\{x \in [-1,1]\}$ we further have
\begin{equation*}
\wt{E}_{P,\varepsilon}(f;{\mc{X}}) \leq p_{\max}^2 \|\eta\|_{\infty} \cdot \int_{\mc{X}} \int_{B_{\mc{X}}(\varepsilon)} \bigl(f(x') - f(x)\bigr)^2 \,d\mu(x') \,d\mu(x).
\end{equation*}
The estimates~\eqref{eqn:dirichlet_energy_l2} and~\eqref{eqn:dirichlet_energy_sobolev} then respectively follow from (3.1) and Lemma~3.3 of \cite{burago2014}.

\paragraph{Proof (of Lemma~\ref{lem:dirichlet_energy_remainder}).}
Following exactly the steps of the proof of Lemma~3.3 of \citet{burago2014}, but replacing all references to a ball of radius $r$ by references to the set difference between balls of radius $\wt{\varepsilon}$ and $\varepsilon$, we obtain that
\begin{equation*}
\int_{\mc{X}} \int_{\mc{X}} \bigl(f(x') - f(x)\bigr)^2 \1\{\varepsilon < d_{\mc{X}}(x',x) \leq \wt{\varepsilon}\} \,d\mu(x') \,d\mu(x) \leq (1 + CmK\varepsilon^2) \cdot \int_{\mc{X}} \int_{B_{m}(0,\wt{\varepsilon})} |d_x^{1}f(v)|^2 \,dv \,d\mu(x).
\end{equation*}
From (2.7) of~\citet{burago2014}, we further have
\begin{equation*}
\int_{\mc{X}} \int_{B_{m}(0,\wt{\varepsilon})} |d_x^{1}f(v)|^2 \,dv \,d\mu(x)  = \frac{\nu_m}{2 + m} (\wt{\varepsilon}^{2 + m} - \varepsilon^{2 + m}) \int_{\mc{X}} |d_x^1f|^2 \,d\mu(x) = 27\frac{\nu_m}{(2 + m)R^2} \varepsilon^{4 + m} \|d^1f\|_{\Leb^2(\mc{X})}^2. 
\end{equation*}
Recalling that $\|d^1f\|_{\Leb^2(\mc{X})}^2 \leq \|f\|_{H^1(\mc{X})}^2$, we see that this implies the claim of Lemma~\ref{lem:dirichlet_energy_remainder}.

\paragraph{Proof (of Lemma~\ref{lem:manifold_graph_seminorm_bias2}).}
The proof of Lemma~\ref{lem:manifold_graph_seminorm_bias2} is identical to the proof of Lemma~\ref{lem:graph_seminorm_bias2}, upon substituting the ambient dimension $m$ for the intrinsic dimension $d$, and using Lemma~\ref{lem:dirichlet_energy_sobolev} rather than Lemma~\ref{lem:dirichlet_estimate_nonlocal_laplacian} to establish~\eqref{pf:graph_seminorm_bias2_1}.

\section{Lower bound on empirical norm}
\label{sec:empirical_norm}
In this Section we prove Proposition~\ref{prop:empirical_norm_sobolev} (in Section~\ref{subsec:empirical_norm_sobolev}). We also prove an analogous result when $\mc{X}$ is a manifold as in Model~\ref{def:model_manifold} (in Section~\ref{subsec:empirical_norm_sobolev_manifold}).

\subsection{Proof of Proposition~\ref{prop:empirical_norm_sobolev}}
\label{subsec:empirical_norm_sobolev}
In this section we establish Proposition~\ref{prop:empirical_norm_sobolev}. As mentioned, the proof of this Proposition follows from the Gagliardo-Nirenberg interpolation inequality, and a one-sided Bernstein's inequality (Lemma~\ref{lem:one_sided_bernstein}). 

\begin{lemma}[Gagliardo-Nirenberg interpolation inequality]
	\label{lem:gagliardo_nirenberg}
	Suppose Model~\ref{def:model_flat_euclidean}, and that $f \in H^s(\mc{X})$ for some $s \geq d/4$. Then there exist constants $C_1$ and $C_2$ that do not depend on $f$, such that
	\begin{equation}
	\label{eqn:gagliardo_nirenberg}
	\|f\|_{\Leb^4(\mc{X})} \leq C_1 |f|_{H^s(\mc{X})}^{d/4s} \|f\|_{\Leb^2(\mc{X})}^{1 - d/(4s)} + C_2 \|f\|_{\Leb^2(\mc{X})}
	\end{equation}
\end{lemma}


\paragraph{Proof (of Proposition~\ref{prop:empirical_norm_sobolev}).}
Rearranging~\eqref{eqn:gagliardo_nirenberg} and raising both sides to the $4$th power, we see that
\begin{equation*}
\frac{\Ebb[f^4(X)]}{\|f\|_P^4} \leq C \biggl(\frac{\|f\|_{\Leb^4(\mc{X})}}{\|f\|_{\Leb^2(\mc{X})}}\biggr)^4 \leq C_1\biggl(\frac{|f|_{H^s(\mc{X})}}{\|f\|_{\Leb^2(\mc{X})}}\biggr)^{d/s} + C_2,
\end{equation*}
here the constants $C_1,C_2$ are not the same as in~\eqref{eqn:gagliardo_nirenberg}. Therefore taking the constant $C$ in assumption~\eqref{eqn:empirical_norm_sobolev_1} to be sufficiently large relative to $C_1$ and $C_2$, we have that
\begin{equation*}
C_1\biggl(\frac{|f|_{H^s(\mc{X})}}{\|f\|_{\Leb^2(\mc{X})}}\biggr)^{d/s} \leq \frac{\delta n}{64},
\end{equation*} 
and consequently 
\begin{equation*}
\frac{\Ebb[f^4(X)]}{\|f\|_P^4} \leq \frac{\delta n}{8} + 8C_2^3.
\end{equation*}
The claim then follows from Lemma~\ref{lem:one_sided_bernstein}, upon taking $c = 1/(64C_2^3)$ in the statement of Proposition~\ref{prop:empirical_norm_sobolev}.

\subsection{Proof of Proposition~\ref{prop:empirical_norm_sobolev_manifold}}
\label{subsec:empirical_norm_sobolev_manifold}
The proof of Proposition~\ref{prop:empirical_norm_sobolev_manifold} follows exactly the same steps as the proof of Proposition~\ref{prop:empirical_norm_sobolev}, upon replacing Lemma~\ref{lem:gagliardo_nirenberg} by Lemma~\ref{lem:gagliardo_nirenberg_manifold}.
\begin{lemma}[(c.f Theorem~3.70 of~\citet{aubin2012})]
	\label{lem:gagliardo_nirenberg_manifold}
	Suppose Model~\ref{def:model_manifold}, and that $f \in H^s(\mc{X})$ for some $s \geq m/4$. Then there exist constants $C_1$ and $C_2$ that do not depend on $f$, such that
	\begin{equation}
	\label{eqn:gagliardo_nirenberg_manifold}
	\|f\|_{\Leb^4(\mc{X})} \leq C_1 |f|_{H^s(\mc{X})}^{m/4s} \|f\|_{\Leb^2(\mc{X})}^{1 - m/(4s)} + C_2 \|f\|_{\Leb^2(\mc{X})}.
	\end{equation}
\end{lemma}
	
\section{Proof of Main Results}
\label{sec:proofs_main_results}

\subsection{Estimation Results}

\paragraph{Proof of Theorem~\ref{thm:laplacian_eigenmaps_estimation_fo}.}
We condition on the event that the design points $X_1,\ldots,X_n$ satisfy
\begin{equation}
\label{pf:laplacian_eigenmaps_estimation_fo_1}
\dotp{L_{n,\varepsilon}f_0}{f_0}_n \leq \frac{C}{\delta}M^2 \quad \textrm{and} \quad \lambda_k \geq \min\{\lambda_k(\Delta_P), \varepsilon^{-2}\}~~\textrm{for all $2 \leq k \leq n$.}
\end{equation}
Note that by Propositions~\ref{prop:graph_seminorm_fo} and~\ref{prop:graph_eigenvalue}, these statements are both satisfied with probability at least $1 - \delta - Cn\exp\{-cn\varepsilon^d\}$. 

Conditional on~\eqref{pf:laplacian_eigenmaps_estimation_fo_1}, we have from Lemma~\ref{lem:fixed_graph_estimation} that for any $0 \leq K \leq n$,
\begin{equation*}
\|\wh{f} - f_0\|_n^2 \leq C\biggl\{\frac{M^2}{\delta (\lambda_{K + 1}(\Delta_P) \wedge \varepsilon^{-2})} + \frac{K}{n}\biggr\},
\end{equation*}
either deterministically (when $K = 0$), or with probability at least $1 - \exp(-K)$ (when $K \geq 1$). Further, from the bounds $\varepsilon \leq c_0 K^{-1/d}$ (Assumption~\ref{asmp:parameters_estimation_fo}) and $\lambda_{K + 1}(\Delta_P) \geq c (K + 1)^{2/d}$ (Weyl's Law) we can simply the above expression to the following,
\begin{equation}
\label{pf:laplacian_eigenmaps_estimation_fo_2}
\|\wh{f} - f_0\|_n^2 \leq C\biggl\{\frac{M^2}{\delta}(K + 1)^{-2/d} + \frac{K}{n}\biggr\}.
\end{equation}
We now upper bound the right hand side of~\eqref{pf:laplacian_eigenmaps_estimation_fo_2}, based on the value of $K$ chosen in~\ref{asmp:parameters_estimation_fo}.  When possible we choose $K = \floor{M^2n}^{d/(2 + d)}$ to balance bias and variance, in which case~\eqref{pf:laplacian_eigenmaps_estimation_fo_2} implies
\begin{equation*}
\|\wh{f} - f_0\|_n^2 \leq \frac{C}{\delta} M^2 (M^2n)^{-2/(2 + d)}.
\end{equation*}
If $M^2 < n^{-1}$, then we take $K = 1$, and from~\eqref{pf:laplacian_eigenmaps_estimation_fo_2} we get
\begin{equation*}
\|\wh{f} - f_0\|_n^2 \leq \frac{C}{n\delta}.
\end{equation*}
Finally if $M > n^{1/d}$, we take $K = n$. In this case, we note that $\wh{f}(X_i) = Y_i$ for all $i = 1,\ldots,n$, and it immediately follows that
\begin{equation*}
\|\wh{f} - f_0\|_n^2 = \frac{1}{n}\sum_{i = 1}^{n} w_i^2 \leq 5,
\end{equation*}
with probability at least $1 - \exp(-n)$. Combining these three separate cases yields the conclusion of Theorem~\ref{thm:laplacian_eigenmaps_estimation_fo}. 


\paragraph{Proof of Theorem~\ref{thm:laplacian_eigenmaps_estimation_ho}.}
Follows identically to the proof of Theorem~\ref{thm:laplacian_eigenmaps_estimation_fo}, except substituting $L_{n,\varepsilon}^s$ for $L_{n,\varepsilon}$, $\lambda_k^s$ for $\lambda_k$, and using Proposition~\ref{prop:graph_seminorm_ho} rather than Proposition~\ref{prop:graph_seminorm_fo} and Assumption~\ref{asmp:parameters_estimation_ho} rather than Assumption~\ref{asmp:parameters_estimation_fo}.

\paragraph{Proof of Theorem~\ref{thm:laplacian_eigenmaps_estimation_manifold}.}
Follows identically to the proof of Theorem~\ref{thm:laplacian_eigenmaps_estimation_fo}, substituting $L_{n,\varepsilon}^s$ for $L_{n,\varepsilon}$, $\lambda_k^s$ for $\lambda_k$, and using Proposition~\ref{prop:graph_seminorm_manifold} rather than Proposition~\ref{prop:graph_seminorm_fo}, Proposition~\textcolor{red}{(?)} rather than Proposition~\ref{prop:graph_eigenvalue}, and Assumption~\ref{asmp:parameters_estimation_manifold} rather than Assumption~\ref{asmp:parameters_testing_fo}.

\subsection{Testing Results}

\paragraph{Proof of Theorem~\ref{thm:laplacian_eigenmaps_testing_fo}.}
We have already upper bounded the Type I error of $\varphi$ in Lemma~\ref{lem:fixed_graph_testing}, and it remains to upper bound the Type II error. To do so, we condition on the event that the design points $X_1,\ldots,X_n$ satisfy,
\begin{equation}
\label{pf:laplacian_eigenmaps_testing_fo_1}
\dotp{L_{n,\varepsilon}f_0}{f_0}_n \leq \frac{C}{\delta}M^2,\quad \textrm{and} \quad \lambda_k \geq \min\{\lambda_k(\Delta_P), \varepsilon^{-2}\}~~\textrm{for all $2 \leq k \leq n$,}
\end{equation}
as well as that
\begin{equation}
\label{pf:laplacian_eigenmaps_testing_fo_2}
\|f_0\|_n^2 \geq \frac{1}{2}\|f_0\|_P^2.
\end{equation}
Note that by Propositions~\ref{prop:graph_seminorm_fo} and~\ref{prop:graph_eigenvalue}, both statements in~\eqref{pf:laplacian_eigenmaps_testing_fo_1} are satisfied with probability at least $1 - \delta - Cn\exp\{-cn\varepsilon^d\}$. Additionally, by Proposition~\ref{prop:empirical_norm_sobolev} and the assumption in~\eqref{eqn:laplacian_eigenmaps_testing_criticalradius_fo} that $\|f_0\|_P^2 \geq CM^2/(bn^{2/d})$, the one-sided inequality~\eqref{pf:laplacian_eigenmaps_testing_fo_2} follows with probability at least $1 - \exp\{-(cn \wedge 1/b)\}$. Setting $\delta = b/3$ and taking $n \geq N$ to be sufficiently large, the bottom line is that both~\eqref{pf:laplacian_eigenmaps_testing_fo_1} and~\eqref{pf:laplacian_eigenmaps_testing_fo_2} are together satisfied with probability at least $1 - b/2$.

Now, to complete the proof of Theorem~\ref{thm:laplacian_eigenmaps_testing_fo}, we would like to invoke Lemma~\ref{lem:fixed_graph_testing}, and conclude that conditional on $X_1,\ldots,X_n$ satisfying~\eqref{pf:laplacian_eigenmaps_testing_fo_1} and~\eqref{pf:laplacian_eigenmaps_testing_fo_2}, our test $\varphi$ will equal $1$ with probability at least $1 - b/2$. To use Lemma~\ref{lem:fixed_graph_testing}, we will need to establish that~\eqref{eqn:fixed_graph_testing_critical_radius} is satisfied, which we now show. 

On the one hand, we have that the right hand side of~\eqref{eqn:fixed_graph_testing_critical_radius} is upper bounded, 
\begin{align*}
\frac{\dotp{L_{n,\varepsilon}f_0}{f_0}_n}{\lambda_{K + 1}} + \frac{\sqrt{2K}}{n}\biggl[2\sqrt{\frac{1}{a}} + \sqrt{\frac{2}{b}} + \frac{32}{bn}\biggr] & \leq C\biggl(\frac{M^2}{b \min\{\lambda_{K+1}(\Delta_P), \varepsilon^{-2}\}} + \frac{\sqrt{2K}}{n}\biggl[\sqrt{\frac{1}{a}} + \frac{1}{b}\biggr]\biggr) \\
& \leq C\biggl(\frac{M^2}{b}K^{-2/d} + \frac{\sqrt{2K}}{n}\biggl[\sqrt{\frac{1}{a}} + \frac{1}{b}\biggr]\biggr)
\end{align*}
with the second inequality following by the assumption $\varepsilon \leq K^{-1/d}$ and Weyl's Law. On the other hand, we have that $\|f_0\|_n^2 \geq \|f_0\|_P^2/2$. Consequently, to prove Theorem~\ref{thm:laplacian_eigenmaps_testing_fo}, it remains only to verify that
\begin{equation}
\label{pf:laplacian_eigenmaps_testing_fo_3}
\|f_0\|_P^2 \geq C\biggl(\frac{M^2}{b}K^{-2/d} + \frac{\sqrt{2K}}{n}\biggl[\sqrt{\frac{1}{a}} + \frac{1}{b}\biggr]\biggr).
\end{equation}
As in the estimation case, we can further upper bound the right hand side of~\eqref{pf:laplacian_eigenmaps_testing_fo_3}, depending on the value of $K$ chosen in~\ref{asmp:parameters_testing_fo}. The classical case is $K = (M^2n)^{d/(2 + d)}$, in which case~\eqref{pf:laplacian_eigenmaps_testing_fo_3} is satisfied as long as
\begin{equation*}
\|f_0\|_P^2 \geq CM^2(M^2n)^{-4/(4 + d)}\biggl[\sqrt{\frac{1}{a}} + \frac{1}{b}\biggr]
\end{equation*}
If $M^2 < n^{-1}$, then we take $K = 1$, and~\eqref{pf:laplacian_eigenmaps_testing_fo_3} is satisfied whenever
\begin{equation*}
\|f_0\|_P^2 \geq \frac{C}{n}\biggl[\sqrt{\frac{1}{a}} + \frac{1}{b}\biggr].
\end{equation*}
Finally if $M > n^{1/d}$, we take $K = n$, and~\eqref{pf:laplacian_eigenmaps_testing_fo_3} is satisfied if
\begin{equation*}
\|f_0\|_P^2 \geq C\biggl(\frac{M^2}{n^{2/d}b} + n^{-1/2}\biggl[\sqrt{\frac{1}{a}} + \frac{1}{b}\biggr]\biggr).
\end{equation*}
We conclude by observing that~\eqref{eqn:laplacian_eigenmaps_testing_criticalradius_fo} implies each of these three inequalities, and thus implies~\eqref{pf:laplacian_eigenmaps_testing_fo_3}.

\paragraph{Proof of Theorem~\ref{thm:laplacian_eigenmaps_testing_ho}.}
Follows identically to the proof of Theorem~\ref{thm:laplacian_eigenmaps_estimation_fo}, except substituting $L_{n,\varepsilon}^s$ for $L_{n,\varepsilon}$, $\lambda_k^s$ for $\lambda_k$, and using Proposition~\ref{prop:graph_seminorm_ho} rather than Proposition~\ref{prop:graph_seminorm_fo} and Assumption~\ref{asmp:parameters_testing_ho} rather than Assumption~\ref{asmp:parameters_testing_fo}.

\paragraph{Proof of Theorem~\ref{thm:laplacian_eigenmaps_testing_manifold}.}
Follows identically to the proof of Theorem~\ref{thm:laplacian_eigenmaps_estimation_fo}, except substituting $L_{n,\varepsilon}^s$ for $L_{n,\varepsilon}$, $\lambda_k^s$ for $\lambda_k$, and using Proposition~\ref{prop:graph_seminorm_manifold} rather than Proposition~\ref{prop:graph_seminorm_fo}, Proposition~\ref{prop:graph_eigenvalue_manifold} rather than Proposition~\ref{prop:graph_eigenvalue}, Proposition~\ref{prop:empirical_norm_sobolev_manifold} rather than Proposition~\ref{prop:empirical_norm_sobolev}, and Assumption~\ref{asmp:parameters_testing_manifold} rather than Assumption~\ref{asmp:parameters_testing_fo}.

\paragraph{Proof of Theorem~\ref{thm:laplacian_eigenmaps_testing_ho_suboptimal}.}
Note that our choices of $K$ and $\varepsilon$ ensure  that~\eqref{pf:laplacian_eigenmaps_testing_fo_1} (with $L_{n,\varepsilon}^s$ replacing $L_{n,\varepsilon}$) and~\eqref{pf:laplacian_eigenmaps_testing_fo_2} are satisfied with probability at least $1 - b/2$. Proceeding as in the proof of Theorem~\ref{thm:laplacian_eigenmaps_testing_fo}, we upper bound the right hand side of~\eqref{eqn:fixed_graph_testing_critical_radius},
\begin{align*}
\frac{\dotp{L_{n,\varepsilon}f_0}{f_0}_n}{\lambda_{K + 1}} + \frac{\sqrt{2K}}{n}\biggl[2\sqrt{\frac{1}{a}} + \sqrt{\frac{2}{b}} + \frac{32}{bn}\biggr] & \leq C\biggl(\frac{M^2}{b \min\{\lambda_{K+1}(\Delta_P), \varepsilon^{-2}\}} + \frac{\sqrt{2K}}{n}\biggl[\sqrt{\frac{1}{a}} + \frac{1}{b}\biggr]\biggr) \\
& \leq C\biggl(\frac{M^2}{b}\varepsilon^2 + \frac{\sqrt{2K}}{n}\biggl[\sqrt{\frac{1}{a}} + \frac{1}{b}\biggr]\biggr).
\end{align*}
Unlike in the proof of Theorem~\ref{thm:laplacian_eigenmaps_testing_fo}, we note that in this case $\varepsilon^2 \leq C\lambda_K(\Delta_P)$ rather than vice versa. From here, proceeding as in the proof of Theorem~\ref{thm:laplacian_eigenmaps_testing_fo} gives the claimed result.

\section{Analysis of kernel smoothing}
\label{subsec:kernel_smoothing}
In this section we prove Lemma~\ref{lem:kernel_smoothing_insample} (in Section~\ref{subsec:pf_kernel_smoothing_insample}) and Lemma~\ref{lem:kernel_smoothing_bias} (in Section~\ref{subsec:pf_kernel_smoothing_bias}). In Section~\ref{subsec:eigenmaps_beats_kernel_smoothing}, we give a sequence of design densities and regression functions $\{p^{(n)}(x), f_0^{(n)}(x)\}_{n \in \mathbb{N}}$ for which the estimator $T_{n,h}\wc{f}$ (extension of Laplacian eigenmaps by kernel smoothing) strictly outperforms directly kernel smoothing the responses, in the sense that
\begin{equation*}
\lim_{n \to \infty} \sup_{h'} \frac{\Ebb \|T_{n,h}\wh{f} - f_0\|_P^2}{\Ebb \|T_{h',n}Y - f_0\|_P^2 } = 0.
\end{equation*}
We begin with some preliminary estimates in Section~\ref{subsec:kernel_smoothing_preliminaries}, which will ease the subsequent analysis.
\subsection{Some preliminary estimates}
\label{subsec:kernel_smoothing_preliminaries}
In certain parts the analysis of this section will overlap with Section~\ref{sec:graph_quadratic_form_euclidean}, where we upper bounded the non-local graph-Sobolev seminorm of a function $f$ in terms of the Sobolev norm of $f$. To see why this should be, note that for an function $f$ and point $x \in \mc{X}$, we have
\begin{equation*}
T_{P,h}f(x) - f(x) = \frac{1}{d_{Q,h}(x)} \int \bigl(f(x') - f(x)\bigr) \psi\biggl(\frac{\|x' - x\|}{h}\biggr) \,dQ(x') = \frac{h^{d + 2}}{d_{P,h}(x)} L_{P,h}f(x).
\end{equation*}
This expression reflects the known fact that the bias operator of kernel smoothing is equal to the non-local Laplacian, up to a rescaling by the population degree functional $d_{P,h}(x)$.  In the second equality, we are using the notation $L_{P,h}f(x)$ exactly as defined in~\eqref{eqn:nonlocal_laplacian}, but with the kernel $\psi$ instead of $\eta$. Note that $\psi$ satisfies all the same assumptions as $\eta$, except that of positivity; when $\psi$ is a higher-order kernel it may take negative values. 

Now we provide a lower bound on $d_{P,h}(x)$ that holds uniformly over all $x \in \mc{X}$. Recall that by assumption the density $p$ is Lipschitz. Letting $L_p$ denote the Lipschitz constant of $p$, we have that
\begin{align*}
d_{P,h}(x) & = \int \psi\biggl(\frac{\|x' - x\|}{h}\biggr) p(x') \,dx' \\
& = h^d \int \psi(\|z\|) p(hz + x) \1\Bigl\{ hz + x \in \mc{X} \Bigr\} \,dz \\
& \geq h^d p(x) \int \psi(\|z\|) \1\Bigl\{ hz + x  \in \mc{X}\Bigr\} \,dz - L_p h^{d + 1} \|\psi\|_{\infty} \nu_d.
\end{align*}
Since by assumption $\mc{X}$ has Lipschitz boundary, setting $c_0$ to be a sufficiently small constant in~\ref{asmp:bandwidth}, we can further deduce that $\int \psi(\|z\|) \1\{hz + x \in \mc{X}\} \,dz \geq 1/3$, and consequently that
\begin{equation}
\label{eqn:degree_lower_bound}
d_{P,h}(x) \geq \frac{p(x)}{3} h^d \geq \frac{p_{\min}}{3}h^d \quad \textrm{for all $x \in \mc{X}$.}
\end{equation}

\subsection{Proof of Lemma~\ref{lem:kernel_smoothing_insample}}
\label{subsec:pf_kernel_smoothing_insample}

To begin with, we apply the triangle inequality to upper bound $\|T_{n,h}\wc{f} - f_0\|_P$ by the sum of two terms,
\begin{equation}
\label{pf:kernel_smoothing_insample_1}
\|T_{n,h}\wc{f} - f_0\|_P \leq \|T_{n,h}(\wc{f} - f_0)\|_P + \|T_{n,h}f_0 - f_0\|_P.
\end{equation}
We proceed by separately upper bounding each term on the right hand side of~\eqref{pf:kernel_smoothing_insample_1}. We will show that
\begin{equation}
\label{pf:kernel_smoothing_insample_2}
\|T_{n,h}(\wc{f} - f_0)\|_P^2 \leq C \|\wc{f} - f_0\|_n^2
\end{equation}
and that 
\begin{equation}
\label{pf:kernel_smoothing_insample_3}
\|T_{n,h}f_0 - f_0\|_P^2 \leq \frac{C}{\delta} \cdot \frac{h^2}{nh^d} |f|_{H^1(\mc{X})}^2 + \frac{C}{\delta}\|T_{h,P}f_0 - f_0\|_P^2,
\end{equation}
each with probability at least $1 - C\exp(-cnh^d)$. Together these will imply the claim. 

\underline{\emph{Proof of~\eqref{pf:kernel_smoothing_insample_2}}.}
Fix $x \in \mc{X}$. By the Cauchy-Schwarz inequality we have
\begin{align*}
\Bigl[T_{n,h}\bigl(\wc{f} - f_0\bigr)(x)\Bigr]^2 & = \Biggl[\frac{1}{d_{n,h}(x)^2}\int \psi\biggl(\frac{\|x' - x\|}{h}\biggr) \cdot \bigl(\wc{f}(x') - f_0(x')\bigr) \,dP_n(x')\Biggr]^2 \\
& \leq \Biggl[\frac{1}{d_{n,h(x)}^2} \int \biggl|\psi\biggl(\frac{\|x' - x\|}{h}\biggr)\biggr| \,dP_n(x')\Biggr] \cdot \Biggl[\int \biggl|\psi\biggl(\frac{\|x' - x\|}{h}\biggr)\biggr| \cdot \bigl(\wc{f}(x') - f_0(x')\bigr)^2 \,dP_n(x')\Biggr] \\
& = \frac{d_{n,h}^{+}(x)}{|d_{n,h}(x)|} \cdot \frac{1}{|d_{n,h}(x)|} \Biggl[\int \biggl|\psi\biggl(\frac{\|x' - x\|}{h}\biggr)\biggr| \cdot \bigl(\wc{f}(x') - f_0(x')\bigr)^2 \,dP_n(x')\Biggr].
\end{align*}
In the last line all we have done is written $d_{n,h}^{+}(x)$ for the degree functional computed with respect to the kernel $|\psi|$, recalling that $\psi$ may take negative values so $d_{n,h}^{+}(x)$ may not be equal to $d_{n,h}(x)$.

Now we integrate over $x \in \mc{X}$ to get 
\begin{align}
\|T_{n,h}(\wc{f} - f_0)\|_P^2 & = \int \Bigl[T_{n,h}\bigl(\wc{f} - f_0\bigr)(x)\Bigr]^2 \,dP(x) \nonumber \\
& \leq \int \int \frac{d_{n,h}^{+}(x)}{|d_{n,h}(x)|} \cdot \frac{1}{|d_{n,h}(x)|} \biggl|\psi\biggl(\frac{\|x' - x\|}{h}\biggr)\biggr| \cdot \bigl(\wc{f}(x') - f_0(x')\bigr)^2 \,dP_n(x') \,dP(x) \nonumber \\
& \leq \sup_{x \in \mc{X}} \frac{d_{n,h}^{+}(x)}{|d_{n,h}(x)|} \cdot \int \int  \frac{1}{|d_{n,h}(x)|} \biggl|\psi\biggl(\frac{\|x' - x\|}{h}\biggr)\biggr| \cdot \bigl(\wc{f}(x') - f_0(x')\bigr)^2 \,dP(x) \,dP_n(x') \nonumber \\
& \leq \sup_{x \in \mc{X}} \frac{d_{n,h}^{+}(x) d_{P,h}^+(x)}{|d_{n,h}(x)|^2} \cdot \|\wc{f} - f_0\|_n^2. \label{pf:kernel_smoothing_insample_4}
\end{align}
Thus we have reduced the problem to showing that the various degree functionals $d_{n,h}^{+}, d_{P,h}^{+}$ and $d_{n,h}$ all put similar weight on a given point $x$. We use~\eqref{eqn:uniform_bound_empirical_degree_2}, which gives a uniform multiplicative bound on deviations of the empirical degree around its mean, to conclude that with probability at least $1 - C\exp\{-cnh^d\}$,
\begin{equation*}
d_{n,h}(x) \geq \frac{1}{2}d_{P,h}(x)~~ \textrm{and} ~~ d_{n,h}^{+}(x) \leq \frac{3}{2} d_{P,h}^{+}(x) \quad\textrm{for all $x \in \mc{X}$.}
\end{equation*}
Therefore,
\begin{equation*}
\sup_{x \in \mc{X}} \frac{d_{n,h}^{+}(x) d_{P,h}^+(x)}{|d_{n,h}(x)|^2} \leq 6 \cdot \sup_{x \in \mc{X}} \frac{|d_{P,h}^{+}(x)|^2}{|d_{P,h}(x)|^2} \leq 36 \biggl(\frac{\|\psi\|_{\infty} p_{\max} \nu_d}{p_{\min}}\biggr)^2
\end{equation*}
with the second inequality following from~\eqref{eqn:degree_lower_bound}. Plugging this back into~\eqref{pf:kernel_smoothing_insample_4} gives the claim.

\underline{\emph{Proof of~\eqref{pf:kernel_smoothing_insample_3}}.}
At a given point $x \in \mc{X}$, we have
\begin{align*}
T_{n,h}f_0(x) - f_0(x) & = \frac{1}{d_{n,h}(x)} \sum_{i = 1}^{n} \bigl(f_0(X_i) - f_0(x)\bigr) \psi\biggl(\frac{\|X_i - x\|}{h}\biggr) \\
& = \frac{d_{P,h}(x)}{d_{n,h}(x)} \cdot \frac{1}{nd_{P,h}(x)} \sum_{i = 1}^{n} \bigl(f_0(X_i) - f_0(x)\bigr) \psi\biggl(\frac{\|X_i - x\|}{h}\biggr).
\end{align*}
Thus,
\begin{equation}
\label{pf:kernel_smoothing_insample_5}
\Bigl[T_{n,h}f_0(x) - f_0(x)\Bigr]^2 = \biggl[\frac{d_{P,h}(x)}{d_{n,h}(x)}\biggr]^2 \cdot \biggl[\underbrace{\frac{1}{nd_{P,h}(x)}\sum_{i = 1}^{n} \bigl(f_0(X_i) - f_0(x)\bigr) \psi\biggl(\frac{\|X_i - x\|}{h}\biggr)}_{:= \wt{L}_{n,h}f_0(x)}\biggr]^2
\end{equation}
In the proof of~\eqref{pf:kernel_smoothing_insample_2} we have already given an upper bound on the ratio of population to empirical degree, which implies that
\begin{equation*}
\sup_{x \in \mc{X}} \biggl[\frac{d_{P,h}(x)}{d_{n,h}(x)}\biggr]^2 \leq 4,
\end{equation*}
with probability at least $1 - C\exp\{-cnh^d\}$. On the other hand, we note that the second term in the product in~\eqref{pf:kernel_smoothing_insample_5} has expectation
\begin{equation*}
\Ebb\Bigl[\wt{L}_{n,h}f_0(x)\Bigr] = T_{P,h}f_0(x) - f_0(x),
\end{equation*} 
and variance 
\begin{equation*}
\Var\Bigl[\wt{L}_{n,h}f_0(x)\Bigr] \leq \frac{1}{n(d_{P,h}(x))^2} \Ebb\biggl[(f_0(X) - f_0(x))^2 \cdot \biggl|\psi\biggl(\frac{\|X - x\|}{h}\biggr)\biggr|^2\biggr].
\end{equation*}
Integrating with respect to $P$ gives
\begin{align*}
\Ebb\biggl[\int \Bigl(\wt{L}_{n,h}f_0(x)\Bigr)^2 \,dP(x)\biggr] & = \int \Ebb\Bigl[\Bigl(\wt{L}_{n,h}f_0(x)\Bigr)^2\Bigr] \,dP(x) \\
& \leq \| T_{P,h}f_0 - f_0\|_P^2 + \frac{1}{n}\int \int \frac{1}{\bigl(d_{P,h}(x)\bigr)^2} \bigl(f_0(x') - f_0(x)\bigr)^2 \cdot \biggl|\psi\biggl(\frac{\|x' - x\|}{h}\biggr)\biggr|^2 \,dP(x') \,dP(x) \\
& \leq \| T_{P,h}f_0 - f_0\|_P^2 + \frac{3h^2}{p_{\min}n} \wt{E}_{P,h}(f_0;\psi^2).
\end{align*}
In the final inequality we have used the lower bound on $d_{P,h}(x)$ from~\eqref{eqn:degree_lower_bound}, and written $E_{P,h}(f_0;\psi^2)$ for the non-local Dirichlet energy defined with respect to the kernel $\psi^2$. 

Putting the pieces together, we conclude that
\begin{align*}
\|T_{n,h}f_0(x) - f_0(x)\|_P^2 & = \int \bigl(T_{n,h}f_0(x) - f_0(x)\bigr)^2 \,dP(x) \\
& \leq \sup_{x \in \mc{X}} \biggl[\frac{d_{P,h}(x)}{d_{n,h}(x)}\biggr]^2 \cdot \int \Bigl(\wt{L}_{n,h}f_0(x)\Bigr)^2 \,dP(x) \\
& \overset{(i)}{\leq} 4\frac{\| T_{P,h}f_0 - f_0\|_P^2}{\delta} + \frac{12h^2}{\delta p_{\min}nh^d} {E}_{P,h}(f_0;\psi^2) \\
& \overset{(ii)}{\leq}  4\frac{\| T_{P,h}f_0 - f_0\|_P^2}{\delta} + \frac{Ch^2}{\delta p_{\min}nh^d} |f_0|_{H^1(\mc{X})}^2,
\end{align*}
with probability at least $1 - \delta - C\exp(-cnh^d)$. In $(i)$ we have used Markov's inequality, and in $(ii)$ we have applied the estimate~\eqref{pf:estimate_nonlocal_seminorm_1} to the non-local Dirichlet energy ${E}_{P,h}(f_0;\psi^2)$. This establishes~\eqref{pf:kernel_smoothing_insample_3}.

\subsection{Kernel smoothing bias}
\label{subsec:pf_kernel_smoothing_bias}
Lemma~\ref{lem:kernel_smoothing_bias} gives the necessary upper bounds on the bias of kernel smoothing.
\begin{lemma}
	\label{lem:kernel_smoothing_bias}
	Suppose Model~\ref{def:model_flat_euclidean}, and that the kernel smoothing operator $T_{P,h}$ is computed with respect to a kernel $\eta$ that satisfies~\ref{asmp:kernel}.
	\begin{itemize}
		\item If $f_0 \in H^1(\mc{X})$, then there exists a constant $C$ which does not depend on $f_0$ such that
		\begin{equation*}
		\|T_{P,h}f_0 - f_0\|_P^2 \leq C h^{2} |f|_{H^1(\mc{X})}^2.
		\end{equation*}
		\item If $f_0 \in H_0^{s}(\mc{X})$, $p \in C^{s - 1}(\mc{X})$, and $\eta$ satisfies~\ref{asmp:ho_kernel}, then there exists a constant $C$ which does not depend on $f_0$ such that
		\begin{equation*}
		\|T_{P,h}f_0 - f_0\|_P^2 \leq C h^{2s} |f|_{H^s(\mc{X})}^2.
		\end{equation*}
	\end{itemize}
\end{lemma}
We separately prove the first-order ($s = 1$) and higher-order ($s > 1$) parts of Lemma~\ref{lem:kernel_smoothing_bias}. In both cases, the proof will rely heavily on results already established regarding the non-local Laplacian $L_{P,h}$ and non-local Dirichlet energy $E_{P,h}$, which we recall are given for a kernel function $\mc{K}$ by
\begin{equation*}
L_{P,h}f(x) = \frac{1}{h^{d + 2}} \int \bigl(f(x') - f(x)\bigr)\mc{K}\biggl(\frac{\|x' - x\|}{h}\biggr)\,dP(x'),
\end{equation*}
and $E_{P,h}(f;\mc{K}) = \dotp{L_{P,h}f}{f}_{P}$, respectively. 

\paragraph{Proof of Lemma~\ref{lem:kernel_smoothing_bias}, $s = 1$.}
Using the conclusions from Section~\ref{subsec:kernel_smoothing_preliminaries}, we have that
\begin{equation}
\label{pf:kernel_smoothing_bias_1}
\|T_{P,h}f - f\|_P^2 \leq \frac{9h^4}{p_{\min}^2} \int \bigl[L_{P,h}f(x)\bigr]^2 \,dP(x).
\end{equation}
By the Cauchy-Schwarz inequality, we have that
\begin{align*}
\int \bigl[L_{P,h}f(x)\bigr]^2 \,dP(x) & = \frac{1}{h^{2d + 4}}\int \biggl[\int \bigl(f(x') - f(x)\bigr)\psi\biggl(\frac{\|x' - x\|}{h}\biggr) \,dP(x')\biggr]^2 \,dP(x) \\
& \leq \frac{C}{h^{d + 4}} \int \int \bigl(f(x') - f(x)\bigr)^2 \cdot \biggl|\psi\biggl(\frac{\|x' - x\|}{h}\biggr)\biggr| \,dP(x') \,dP(x) \\
& = \frac{C}{h^{2}} E_{P,h}(f;|\psi|).
\end{align*}
Applying the estimate~\eqref{pf:estimate_nonlocal_seminorm_1} to the non-local Dirichlet energy ${E}_{P,h}(f;|\psi|)$ and plugging back into~\eqref{pf:kernel_smoothing_bias_1} gives the claimed result.

\paragraph{Proof of Lemma~\ref{lem:kernel_smoothing_bias}, $s > 1$.}
Proceeding from~\eqref{pf:kernel_smoothing_bias_1}, we separate the integral into the portion sufficiently in the interior of $\mc{X}$ and that near the boundary, obtaining
\begin{equation}
\label{pf:kernel_smoothing_bias_2}
\|T_{P,h}f - f\|_P^2 \leq\frac{9p_{\max}h^4}{p_{\min}^2}\Bigl(\|L_{P,h}f\|_{L^2(\mc{X}_h)}^2 + \|L_{P,h}f\|_{L^2(\partial_{h}(\mc{X}))}^2\Bigr).
\end{equation}
In Lemma~\ref{lem:approximation_error_nonlocal_laplacian_boundary}, we established a sufficient upper bound on the second term,
\begin{equation*}
\|L_{P,h}f\|_{L^2(\partial_{h}(\mc{X}))}^2 \leq Ch^{2(s - 2)} \|f\|_{H^s(\mc{X})}^2.
\end{equation*}
Thus it remains to upper bound the first term. Here we recall that at a given $x \in \mc{X}_h$, we can write
\begin{align*}
L_{P,h}f(x) & = \frac{1}{h^{2}}\sum_{j_1 = 1}^{s - 1} \sum_{j_2 = 0}^{q - 1}\frac{h^{j_1 + j_2}}{j_1!j_2!}  \int d_x^{j_1}f(z) d_x^{j_2}p(z) \psi\bigl(\|z\|\bigr) \,dz \quad + \\
& \quad \frac{1}{h^{2}} \sum_{j = 1}^{s - 1} \frac{h^j}{j!} \int d_x^jf(z)  r_{zh + x}^{q}(x;p) \psi\bigl(\|z\|\bigr) \,dz \quad  + \\
& \quad \frac{1}{h^{2}} \int r_{zh + x}^j(x;f) \psi\bigl(\|z\|\bigr) p(zh + x)\,dz \\
& = G_1(x) + G_2(x) + G_3(x).
\end{align*}
(Here $q = s - 1$.) 

We have already given sufficient upper bounds on $\|G_j\|_{L^2(\mc{X}_{h})}$ for $j = 2,3$ in~\eqref{pf:approximation_error_nonlocal_laplacian_2}.  Thus it remains only to upper bound $\|G_1\|_{L^2(\mc{X}_h)}$. Recall the expansion of $G_1$ from~\eqref{pf:approximation_error_nonlocal_laplacian_3},
\begin{equation*}
G_1(x) = \sum_{j_1 = 1}^{s - 1} \sum_{j_2 = 0}^{q - 1} \frac{h^{j_1 + j_2 - 2}}{j_1!j_2!}  \underbrace{\int_{B(0,1)} d_x^{j_1}f(z) d_x^{j_2}p(z) \eta(\|z\|) \,dz}_{:= g_{j_1,j_2}(x)}.
\end{equation*}
Noting that $d_x^{j_1} \cdot d_x^{j_2}$ is a degree-$(j_1 + j_2)$ multivariate polynomial, and recalling that $\psi$ is an order-$s$ kernel, we have that 
\begin{equation*}
\int g_{j_1,j_2}(z) \psi\bigl(\|z\|\bigr) \,dz = 0,\quad \textrm{for all $j_1,j_2$ such that $j_1 + j_2 < s$.}
\end{equation*}
Otherwise, derivations similar to those used in the proof of Lemma~\ref{lem:approximation_error_nonlocal_laplacian} imply that
\begin{equation*}
\|g_{j_1,j_2}\|_{L^2(\mc{X}_h)} \leq C \|f\|_{H^s(\mc{X})} \|p\|_{C^{s - 1}(\mc{X})},\quad \textrm{for all $j_1,j_2$ such that $j_1 + j_2 \geq s$,}
\end{equation*}
from which it follows that
\begin{equation*}
\|G_1\|_{L^2(\mc{X}_h)}^2 \leq C  h^{2(s - 2)} \|f\|_{H^s(\mc{X})} \|p\|_{C^{s - 1}(\mc{X})}.
\end{equation*}
Together these upper bounds on $\|G_j\|_{L^2(\mc{X}_j)}$ for $j = 1,2,3$ imply that
\begin{equation*}
\|L_{P,h}f\|_{\mc{X}_h}^2 \leq Ch^{2(s - 2)} \|f\|_{H^s(\mc{X})}^2,
\end{equation*}
and plugging this back into~\eqref{pf:kernel_smoothing_bias_2} yields the claim.

\subsection{Comparison of Laplacian eigenmaps to kernel smoothing}
\label{subsec:eigenmaps_beats_kernel_smoothing}

In this section, we give a sequence of densities $p^{(n)}$ and regression functions $f_0^{(n)}$ for which the $L^2(P)$ error of the two-stage estimator $T_{n,h}\wh{f}$ is stochastically dominated by the $L^2(P)$ error of applying kernel smoothing directly to the responses. We begin by precisely setting up the problem and reviewing the two estimators we will compare. Then we give and prove our formal claim.

\paragraph{Setup.}
As usual, we assume the design points $(X_1,\ldots,X_n)$ are sampled independently from a distribution $P$, and the responses
\begin{equation*}
Y_i = f_0(X_i) + \varepsilon_i,
\end{equation*}
where the noise terms $\varepsilon_i \sim N(0,1)$ are sampled independently, and are also independent of the design points.

Unlike in our main text, we shall assume a specific (parametric) form for $p = p^{(n)}$ and $f_0 = f_0^{(n)}$, that changes as a function of $n$. More specifically, using the notation
\begin{equation*}
r := \frac{(\log n)^2}{n}, \quad Q_1 := [0,1/2 - r], \quad Q_2 := [1/2 + r,1],
\end{equation*}
we take the domain $\mc{X}^{(n)} = Q_1 \cup Q_2$, and 
\begin{equation}
\label{def:model_cluster_assumption}
p^{(n)}(x) := \frac{1}{1 - 2r}\1\bigl\{x \in Q_1 \cup Q_2\bigr\}, \quad f_0^{(n)}(x) := \theta \cdot \Bigl(\1\bigl\{x \in Q_1\bigr\} - \1\bigl\{x \in Q_2\bigr\}\Bigr),
\end{equation}
where $\theta \in \Reals$. Note that as $n \to \infty$, the density $p^{(n)}$ is converging (in, say, $L^2([0,1])$ norm) to uniform distribution on $[0,1]$, and the regression function $f_0^{(n)}$ to a step function. 

Our goal is to estimate $f_0$ using an estimator $f$, in such a way that the resulting $L^2(P)$ error $\|f - f_0\|_P^2$ is small. We shall consider two estimators: Laplacian eigenmaps + extension by kernel smoothing, and direct kernel smoothing of the responses. In the notation of our main text, these are respectively $T_{n,h}\wh{f}$ and $T_{n,h'}{\bf Y}$. For simplicity, in both cases we shall consider only the boxcar kernel,
\begin{equation}
\label{asmp:boxcar_kernel}
\eta(z) = \psi(z) = \1\{z \leq 1\}.
\end{equation}

\paragraph{Laplacian eigenmaps + kernel smoothing beats direct kernel smoothing.}
Now we are in a position to state our result. Let $P^{(n)}$ be the distribution with density $p^{(n)}$, i.e. the uniform probability distribution on $Q_1 \cup Q_2$. 
\begin{proposition}
	\label{prop:eigenmaps_beats_kernel_smoothing}
	Suppose $(X_1,Y_1),\ldots(X_n,Y_n)$ are sampled according to~\eqref{def:model_cluster_assumption}. 
	\begin{itemize}
		\item Compute the Laplacian eigenmaps estimator $\wh{f}$ using a kernel $\eta$ which satisfies~\eqref{asmp:boxcar_kernel}, number of eigenvectors $K = 2$, and radius $\varepsilon = ((\log n)^2/n)/2$. Compute the kernel smoothed extension $T_{n,h}\wh{f}$ using a kernel $\psi$ which satisfies~\eqref{asmp:boxcar_kernel}, and bandwidth $h = ((\log n)^2/n)/2$. Then,
		\begin{equation}
		\label{eqn:eigenmaps_beats_kernel_smoothing_1}
		\Ebb \|T_{n,h}\wh{f} - f_0^{(n)}\|_{L^2(P^{(n)})}^2 \leq \frac{8 (\log n)^2 + 6 (\log n)^2 \theta^2}{n}.
		\end{equation}
		\item Compute the kernel smoothing estimator $T_{n,h}{\bf Y}$, using a kernel $\psi$ which satisfies~\eqref{asmp:boxcar_kernel}. Then
		\begin{equation}
		\label{eqn:eigenmaps_beats_kernel_smoothing_2}
		\inf_{h' > 0} \Ebb\Bigl[\|T_{n,h'}{\bf Y} - f_0^{(n)}\|_{L^2(P^{(n)})}^2\Bigr] \geq \min\biggl\{ \frac{1}{16(\log n)^2}, \frac{\theta}{\sqrt{32 n}} \biggr\}.
		\end{equation}
	\end{itemize}
\end{proposition}
The statements~\eqref{eqn:eigenmaps_beats_kernel_smoothing_1} and~\eqref{eqn:eigenmaps_beats_kernel_smoothing_2} are finite sample and hold for any $\theta$. Taking a sequence $(\theta_n)_{n \in \mathbb{N}}$ such that $\theta = o(\sqrt{n})$ and $\theta = \omega((\log n)^2/\sqrt{n})$ implies that 
\begin{equation*}
\lim_{n \to \infty} \sup_{h'}  \frac{\Ebb \|T_{n,h}\wh{f} - f_0^{(n)}\|_{P^{(n)}}^2}{	\Ebb \|T_{n,h'}{\bf Y} - f_0^{(n)}\|_{P^{(n)}}^2 } = 0.
\end{equation*}

\paragraph{Proof of Proposition~\ref{prop:eigenmaps_beats_kernel_smoothing}.}
First we prove~\eqref{eqn:eigenmaps_beats_kernel_smoothing_1}, then~\eqref{eqn:eigenmaps_beats_kernel_smoothing_2}.

\underline{\emph{Proof of~\eqref{eqn:eigenmaps_beats_kernel_smoothing_1}}.}
We begin by showing that with high probability the eigenvectors $v_1,v_2$ respect the cluster structure of $p$. Formally, letting $u_1 = (\1\{X_i \in Q_1\})_{i \in [n]}$, and likewise $u_2 = (\1\{X_i \in Q_2\})_{i \in [n]}$, we establish that
\begin{equation}
\label{pf:eigenmaps_beats_kernel_smoothing_1}
\Pbb\Bigl(\mathrm{span}\{v_1,v_2\} = \mathrm{span}\{u_1,u_2\}\Bigr) \geq 1 - \frac{1}{n}.
\end{equation}
Note that~\eqref{pf:eigenmaps_beats_kernel_smoothing_1} is equivalent to the statement that the neighborhood graph $G_{n,\varepsilon}$ consists of exactly two connected components: one consisting of all design points $X_i \in Q_1$, and the other consisting of all design points $X_i \in Q_2$. Of course, since $h < r$ and $\eta$ is compactly supported on $[0,1]$, it follows that no $X_i \in Q_1$ and $X_j \in Q_2$ can be connected. On the other hand, using an elementary concentration argument (stated in Lemma~\ref{lem:balls_in_bins}) and the triangle inequality, we deduce that with probability at least $1 - 1/n$ there exists a path in $G_{n,\varepsilon}$ between each $X_i, X_j \in Q_1$, and likewise between each $X_i,X_j \in Q_2$. This establishes~\eqref{pf:eigenmaps_beats_kernel_smoothing_1}.

Now we condition on the ``good'' event $\mc{E}$ that the design points $X_1,\ldots,X_n$ satisfy~\eqref{eqn:balls_in_bins}, and therefore that $\mathrm{span}\{v_1,v_2\} = \mathrm{span}\{u_1,u_2\}$. Consider the empirical mean $\wb{Y}_Q := \frac{1}{\sharp\{Q \cup {\bf X}\}} \sum_{i: X_i \in Q} Y_i$. Since $\mathrm{span}\{v_1,v_2\} = \mathrm{span}\{u_1,u_2\}$, the estimator $\wh{f}$ will be piecewise constant on $Q_1$ and $Q_2$, and in fact we have that
\begin{equation*}
\wh{f} = \wb{Y}_{Q_1}u_1 + \wb{Y}_{Q_2}u_2.
\end{equation*}
Moreover, by~\eqref{eqn:balls_in_bins} every $x \in Q_1$ is within distance $h$ of at least one $X_i \in {\bf X} \cup Q_1$ and likewise every $x \in Q_2$ is within distance $h$ of at least one $X_j \in {\bf X} \cup Q_2$. Consequently, the kernel smoothening of $\wh{f}$ also exhibits a piecewise constant structure,
\begin{equation*}
T_{n,h}\wh{f}(x) = \wb{Y}_{Q_1}\1\{x \in Q_1\} + \wb{Y}_{Q_2}\1\{x \in Q_2\},\quad \textrm{for all}~~x \in Q_1 \cup Q_2.
\end{equation*}
Letting
\begin{equation*}
\|T_{n,h}\wh{f} - f_0^{(n)}\|_{L^2(P^{(n)})}^2 = \frac{1}{2}\Bigl((\wb{Y}_{Q_1} - \theta)^2 + (\wb{Y}_{Q_2} + \theta)^2\Bigr),
\end{equation*}
it follows from Lemma~\ref{lem:chi_square_bound} that conditional on $\mc{E}$,
\begin{equation}
\label{pf:eigenmaps_beats_kernel_smoothing_0}
\|T_{n,h}\wh{f} - f_0^{(n)}\|_{L^2(P^{(n)})}^2 \leq \frac{4(\log n)^2}{n}.
\end{equation}

Now we derive a crude upper bound on $\|T_{n,h}\wh{f} - f_0^{(n)}\|_{L^2(P^{(n)})}$ that will suffice to control the error conditional on $\mc{E}^c$. Note that if $d_{n,h}(x) = 0$ then
$T_{n,h}\wh{f}(x) = 0$, and $\bigl(T_{n,h}\wh{f}(x) - f_0(x)\bigr)^2 = [f_0(x)]^2$. On the other hand, if $d_{n,h}(x) \geq 1$, then by Jensen's inequality it follows that
\begin{align*}
\bigl(T_{n,h}\wh{f}(x) - f_0(x)\bigr)^2 & \leq \biggl(\frac{1}{d_{n,h}(x)} \sum_{i = 1}^{n} (\wh{f}_i - f_0(x)) \cdot \1\{\|X_i - x\| \leq h \}\biggr)^2 \\
& \leq \frac{1}{d_{n,h}(x)} \sum_{i = 1}^{n} (\wh{f}_i - f_0(x))^2 \cdot \1\{\|X_i - x\| \leq h \} \\
& \leq 2[f_0(x)]^2 + \frac{2}{d_{n,h}(x)} \sum_{i = 1}^{n} (\wh{f}_i)^2 \cdot \1\{\|X_i - x\| \leq h\},
\end{align*}
and integrating over all $x \in \mc{X}^{(n)}$ gives
\begin{align*}
\|T_{n,h}\wh{f} - f_0\|_{\Leb^2(P^{(n)})}^2 & \leq 2 \|f_0\|_{\Leb^2(P^{(n)})}^2 + \sum_{i = 1}^{n} (\wh{f}_i)^2 \biggl(\int \frac{\1\{\|X_i - x\| \leq h \}}{d_{n,h}(x)} \,dP(x)\biggr) \\
& \leq 2 \|f_0\|_{\Leb^2(P^{(n)})}^2 + 2h\sum_{i = 1}^{n} (\wh{f}_i)^2 \\
& = 2 \|f_0\|_{\Leb^2(P^{(n)})}^2 + 2 (\log n)^2 \|\wh{f}\|_n^2.
\end{align*}
We can further upper bound the empirical norm of $\|\wh{f}\|_n^2$ using the Cauchy-Schwarz inequality:
\begin{equation*}
\|\wh{f}\|_n^2 \leq \frac{2}{n}\sum_{i = 1}^{n}\dotp{Y}{v_1}_n^2v_{1,i}^2 + \dotp{{\bf Y}}{v_2}_n^2v_{2,i}^2 \leq 2\bigl(\dotp{{\bf Y}}{v_1}_n^2 + \dotp{{\bf Y}}{v_2}_n^2\bigr) \leq 4 \|{\bf Y}\|_n^2. 
\end{equation*}
Noting that $\Ebb[\|{\bf Y}\|_n^2|{\bf X}] = \|f_0\|_n^2 + 1/n$, and that $\max_x |f_0(x)| = \theta$, we conclude that
\begin{equation*}
\Ebb\Bigl[\|T_{n,h}\wh{f} - f_0\|_{\Leb^2(P^{(n)})}^2 \cdot \1\{\mc{E}^c\}\Bigr] \leq \Ebb\Bigl[\Bigl(2 \|f_0\|_{\Leb^2(P^{(n)})}^2 + 4 (\log n)^2 \|{\bf Y}\|_n^2\Bigr) \cdot \1\{\mc{E}^c\} \Bigr] \leq \frac{2 \theta^2}{n} + \frac{4 (\log n)^2(\theta^2 + 1)}{n}.
\end{equation*}
Combining this with~\eqref{pf:eigenmaps_beats_kernel_smoothing_0} implies~\eqref{eqn:eigenmaps_beats_kernel_smoothing_1}.

\underline{\emph{Proof of~\eqref{eqn:eigenmaps_beats_kernel_smoothing_2}}.}
Fix $x \in \mc{X}^{(n)}$. A standard argument using the law of iterated expectation implies the following lower bound on the pointwise risk in terms of squared-bias and variance-like quantities,
\begin{equation*}
\Ebb\Bigl[\Bigl(T_{n,h'}{\bf Y}(x) - f_0(x)\Bigr)^2\Bigr] \geq \Ebb\biggl[\Bigl(f_0(X) - f_0(x)\Bigr)^2|X \in B(x,h')\biggr] + \Ebb\biggl[\frac{1}{d_{n,h'}(x)}\biggr].
\end{equation*}
The variance term can be lower bounded quite simply for any $x \in \mc{X}^{(n)}$; noting that $\sup_{x} p^{(n)}(x) < 2$ and $\nu(B(x,h') \cap \mc{X}^{(n)}) \leq 2h'$, it follows by Jensen's inequality that 
\begin{equation*}
\Ebb\biggl[\frac{1}{d_{n,h'}(x)}\biggr] \geq \frac{1}{\Ebb[d_{n,h'}(x)]} \geq \frac{1}{4nh'}.
\end{equation*}
On the other hand the squared bias term is quite large for $x$ close to $1/2$. Precisely, if $h' \geq 4r$ then a simple calculation implies
\begin{equation*}
\Ebb[(f_0(X) - f_0(x))^2|X \in B(x,h')] \geq \frac{\theta^2}{8} \quad \textrm{for all}~x \in [(1 - h'/2)_{+}, 1/2 - r].
\end{equation*}
Combining these lower bounds on variance and squared bias terms and integrating over $x' \in \mc{X}$, we deduce the following: if $h' \leq 4r$, then
\begin{equation*}
\Ebb\Bigl[\|T_{n,h'}{\bf Y} - f_0^{(n)}\|_{L^2(P^{(n)})}^2\Bigr] \geq \frac{1}{16(\log n)^2},
\end{equation*}
whereas if $h' > 4r$  then
\begin{equation*}\Ebb\Bigl[\|T_{n,h'}{\bf Y} - f_0^{(n)}\|_{L^2(P^{(n)})}^2\Bigr] \geq \frac{1}{4nh'} + \frac{\theta^2h'}{32}.
\end{equation*}
In the latter case, setting the derivative equal to $0$ shows that the right hand side is always at least $\theta/\sqrt{32 n}$, and taking the minimum over the two cases then yields~\eqref{eqn:eigenmaps_beats_kernel_smoothing_2}.

\paragraph{A Useful Lemma.}
Let $m = 4n/\log^2n$. Furthermore, for $i = 0,\ldots,m - 1$, let
\begin{equation*}
Q_{i1} = [i/m,(i + 1)/m] \cdot (1/2 - r), \quad Q_{i2} = 1/2 + [i/m,(i + 1)/m] \cdot (1/2 - r).
\end{equation*}
\begin{lemma}
	\label{lem:balls_in_bins}
	Suppose $(X_1,Y_1),\ldots(X_n,Y_n)$ are sampled according to~\eqref{def:model_cluster_assumption}. For any $\log(n) > 16$, we have that 
	\begin{equation}
	\label{eqn:balls_in_bins}
	\Pbb\Bigl(\sharp\{Q_{ij} \cup {\bf X}\} > 0 ~~\textrm{for all $i = 1,\ldots,m - 1$ and $j = 1,2$} \Bigr) \geq 1 - \frac{1}{n}.
	\end{equation}
\end{lemma}
Let $\varepsilon = h = \log^2n/2n$. Note that by construction,~\eqref{eqn:balls_in_bins} implies that any points $x$ and $x'$ in adjacent intervals $Q_{ij}$ and $Q_{i'j}$ must be connected in $G_{n,\varepsilon}$. It also implies that $d_{n,h}(x) > 0$ for every $x \in Q_1 \cup Q_2$.

\paragraph{Proof (of Lemma~\ref{lem:balls_in_bins}).}
For each $Q_{ij}$, we have that 
\begin{equation*}
\Pbb\bigl(\sharp\{Q_{ij} \cup {\bf X}\} = 0\bigr) = (1 - 1/(2m))^{n} \leq \exp\{-n/2m\}.
\end{equation*}
Thus by a union bound,
\begin{equation*}
\Pbb\Bigl(\sharp\{Q_{ij} \cup {\bf X}\} = 0 ~~\textrm{for any $i = 1,\ldots,m - 1$ and $j = 1,2$} \Bigr) \leq 2m\exp\{-n/2m\} = \frac{8n}{\log^2n}\exp\{-\log^2n/8\},
\end{equation*}
and the claim follows from some basic algebra. 

\section{Miscellaneous}
Here we give assorted helpful Lemmas used at various points in the above proofs. We also review notation and relevant facts regarding Taylor expansion.

\subsection{Concentration Inequalities}
Lemma~\ref{lem:chi_square_bound} controls the deviation of a chi-squared random variable. It is from~\cite{laurent00}.
\begin{lemma}
	\label{lem:chi_square_bound}
	Let $\xi_1,\ldots,\xi_N$ be independent $N(0,1)$ random variables, and let $U := \sum_{k = 1}^{N} a_k(\xi_k^2 - 1)$.  Then for any $t > 0$,
	\begin{equation*}
	\Pbb\Bigl[U \geq 2 \norm{a}_2 \sqrt{t} + 2 \norm{a}_{\infty}t\Bigr] \leq \exp(-t).
	\end{equation*}
	In particular if $a_k = 1$ for each $k = 1,\ldots,N$, then
	\begin{equation*}
	\Pbb\Bigl[U\geq 2\sqrt{N t} + 2t\Bigr] \leq \exp(-t).
	\end{equation*}
\end{lemma}

Lemma~\ref{lem:one_sided_bernstein} is an immediate consequence of the one-sided Bernstein's inequality (14.23) in \cite{wainwright2019}.
\begin{lemma}[One-sided Bernstein's inequality]
	\label{lem:one_sided_bernstein}
	Let $X, X_1,\ldots,X_n \sim P$, and $f$ satisfy $\Ebb[f^4(X)] < \infty$. Then
	\begin{equation*}
	\|f\|_n^2 \geq \frac{1}{2}\|f\|_P^2,
	\end{equation*}
	with probability at least $1 - \exp\bigl(-n/8 \cdot \|f\|_P^4 /\Ebb[f^4(X)]\bigr)$.
\end{lemma}

In the proof of Lemma~\ref{lem:kernel_smoothing_insample}, we require uniform control of the empirical degree functional $d_{n,h}(x)$ over all $x \in \mc{X}$. Such a result is available to us because the kernel $\psi$ is Lipschitz on its support, so that the class of functions $\{\psi((x - \cdot)/h): h \in \Rd\}$ has finite VC dimension. The precise estimate we use is due to \textcolor{blue}{(Gine + Guillou 2002)}. 
\begin{lemma}[Uniform bound for empirical degree.]
	\label{lem:uniform_bound_empirical_degree}
	Suppose Model~\ref{def:model_flat_euclidean}. For a kernel $\psi$ satisfying~\ref{asmp:kernel} and bandwidth $h$ satisfying~\ref{asmp:bandwidth}, there exist constants $c, C, c_1$ and $C_1$ which do not depend on $h$ or $n$ such that
	\begin{equation*}
	\Pbb\biggl(\sup_{x \in \Rd} n \cdot \Bigl|d_{n,h}(x) - d_{P,h}(x)\Bigr| > t\biggr) \leq C\exp\biggl(-c \frac{t^2}{nh^d}\biggr),
	\end{equation*}
	for any $t \in \Reals$ satisfying
	\begin{equation}
	\label{eqn:uniform_bound_empirical_degree_1}
	C_1\sqrt{nh^d \log(1/h)} \leq t \leq c_1 nh^d.
	\end{equation}
\end{lemma}
Now we translate Lemma~\ref{lem:uniform_bound_empirical_degree} into a multiplicative bound, which will be more useful for our purposes. 
Recall the lower bound on $d_{P,h}(x)$ given in~\eqref{eqn:degree_lower_bound}. By setting $C_0$ to be a sufficiently large constant in~\ref{asmp:bandwidth}, we can ensure that choosing $t = n h^d p_{\min}/6$ satisfies both the inequalities~\eqref{eqn:uniform_bound_empirical_degree_1}. For this choice of $t$ it follows from~\eqref{eqn:degree_lower_bound} and Lemma~\ref{lem:uniform_bound_empirical_degree} that
\begin{equation}
\label{eqn:uniform_bound_empirical_degree_2}
\sup_{x \in \Rd} \biggl|\frac{d_{n,h}(x) - d_{P,h}(x)}{d_{P,h}(x)}\biggr| \leq \sup_{x \in \Rd} \biggl|\frac{d_{n,h}(x) - d_{P,h}(x)}{2t/n}\biggr| \leq \frac{1}{2}
\end{equation}
with probability at least $1 - C\exp(-cnh^d)$. This is the form of the result we use in the proof of Lemma~\ref{lem:kernel_smoothing_insample}.

\subsection{Taylor expansion}
\label{subsec:taylor_expansion}
We begin with some notation that allows us to concisely derivatives. For a given $z \in \Rd$ and $s$-times differentiable function $f: \mc{X} \to \Reals$, we denote $\bigl(d_x^sf\bigr)(z) := \sum_{\abs{\alpha} = s} D^{\alpha}f(x) z^{\alpha}$. We also write $d^sf := \sum_{\abs{\alpha} = j} D^{\alpha}f$. We point out that in the first-order case $d_x^1f$ is the differential of $f$ at $x \in \mc{X}$, while $d^1f$ is the divergence of $f$.

Let $u$ be a function which is $s$ times continuously differentiable at all $x \in \mc{X}$, for $k \in \mathbb{N}\setminus\{0\}$. Suppose that for some $h > 0$, $x \in \mc{X}_{h}$ and $x' \in B(x,h)$. We write the order-$s$ Taylor expansion of $u(x')$ around $x' = x$ as
\begin{equation*}
u(x') = u(x) + \sum_{j = 1}^{s - 1} \frac{1}{j!}\bigl(d_x^{j}u\bigr)(x' - x) + r_{x'}^{s}(x;u)
\end{equation*}
For notational convenience we have adopted the convention that $\sum_{j = 1}^{0} a_j = 0$.  Thus $\bigl(d_x^{j}f\bigr)(z)$ is a degree-$j$ polynomial---and so a $j$-homogeneous function---in $z$, meaning for any $t \in \Reals$,
\begin{equation*}
\bigl(d_x^{j}f\bigr)(tz) = t^{j} \cdot \bigl(d_x^{j}f\bigr)(z).
\end{equation*}
The remainder term $r_{x'}$ is given by
\begin{equation*}
r_{x'}^s(x;f) = \frac{1}{(j - 1)!} \int_{0}^{1}(1 - t)^{j - 1} \bigl(d_{x + t(x' - x)}^{s}f\bigr)(x' - x) \,dt,
\end{equation*}
where we point out that the integral makes sense because $x + t(x' - x) \in B(x,h) \subseteq \mc{X}$. We now give estimates on the remainder term in both sup-norm and $L^2(\mc{X}_{h})$ norm, each of which hold for any $z \in B(0,1)$. In sup-norm, we have that 
\begin{equation*}
\sup_{x \in \mc{X}_{h}}|r_{x + hz}^j(x;f)| \leq C h^j \|f\|_{C^j(\mc{X})},
\end{equation*}
whereas in $L^2(\mc{X}_{h})$ norm we have,
\begin{equation}
\label{eqn:sobolev_remainder_term}
\int_{\mc{X}_{h}} \bigl|r_{x + thz}^j(x;f)\bigr|^2 \,dx \leq h^{2j} \int_{\mc{X}_{h}} \int_{0}^{1} |d_{x + thz}^jf(z)|^2 \,dt \,dx \leq h^{2j} \|d^jf\|_{\Leb^2(\mc{X})}^2.
\end{equation}
In the last inequality 

Finally, we recall some facts regarding the interaction between smoothing kernels and polynomials.  Let $q_j(z)$ be an arbitrary degree-$j$ (multivariate) polynomial. If $\eta$ is a radially symmetric kernel and $j$ is odd, then by symmetry it follows that
\begin{equation*}
\int_{B(0,1)} q_j(z) \eta(\|z\|) \,dz = 0.
\end{equation*}
On the other hand, if $\psi$ is an order-$s$ kernel for some $s > j$, then by converting to polar coordinates we can verify that
\begin{equation*}
\int_{B(0,1)} q_j(z) \eta(\|z\|) \,dz = 0.
\end{equation*}

\textcolor{red}{(TODO): Check with Ryan and Siva that (i) they believe these facts, and (ii) these facts do not require additional justification.}

\subsection{Upper bound on $L^2(P_n)$ risk}
\label{subsec:upper_bound_l2pn_risk}
\textcolor{red}{(TODO)}

\end{document}