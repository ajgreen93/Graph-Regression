\documentclass[twoside]{article}
\usepackage{aistats2021}
\usepackage{amsmath, amsfonts, amsthm, amssymb}
\usepackage{algpseudocode}
\usepackage{algorithm}
\usepackage{enumerate}
\usepackage[shortlabels]{enumitem}
\usepackage{mathtools}
\usepackage{tikz}
\usepackage{verbatim}

\usepackage{natbib}
\renewcommand{\bibname}{REFERENCES}
\renewcommand{\bibsection}{\subsubsection*{\bibname}}

\DeclareFontFamily{U}{mathx}{\hyphenchar\font45}
\DeclareFontShape{U}{mathx}{m}{n}{<-> mathx10}{}
\DeclareSymbolFont{mathx}{U}{mathx}{m}{n}
\DeclareMathAccent{\wb}{0}{mathx}{"73}

\DeclarePairedDelimiterX{\norm}[1]{\lVert}{\rVert}{#1}
\DeclarePairedDelimiterX{\seminorm}[1]{\lvert}{\rvert}{#1}

% Make a widecheck symbol (thanks, Stack Exchange!)
\DeclareFontFamily{U}{mathx}{\hyphenchar\font45}
\DeclareFontShape{U}{mathx}{m}{n}{
	<5> <6> <7> <8> <9> <10>
	<10.95> <12> <14.4> <17.28> <20.74> <24.88>
	mathx10
}{}
\DeclareSymbolFont{mathx}{U}{mathx}{m}{n}
\DeclareFontSubstitution{U}{mathx}{m}{n}
\DeclareMathAccent{\widecheck}{0}{mathx}{"71}
% widecheck made

\newcommand{\eqdist}{\ensuremath{\stackrel{d}{=}}}
\newcommand{\Graph}{\mathcal{G}}
\newcommand{\Reals}{\mathbb{R}}
\newcommand{\iid}{\overset{\text{i.i.d}}{\sim}}
\newcommand{\convprob}{\overset{p}{\to}}
\newcommand{\convdist}{\overset{w}{\to}}
\newcommand{\Expect}[1]{\mathbb{E}\left[ #1 \right]}
\newcommand{\Risk}[2][P]{\mathcal{R}_{#1}\left[ #2 \right]}
\newcommand{\Prob}[1]{\mathbb{P}\left( #1 \right)}
\newcommand{\iset}{\mathbf{i}}
\newcommand{\jset}{\mathbf{j}}
\newcommand{\myexp}[1]{\exp \{ #1 \}}
\newcommand{\abs}[1]{\left \lvert #1 \right \rvert}
\newcommand{\restr}[2]{\ensuremath{\left.#1\right|_{#2}}}
\newcommand{\ext}[1]{\widetilde{#1}}
\newcommand{\set}[1]{\left\{#1\right\}}
\newcommand{\seq}[1]{\set{#1}_{n \in \N}}
\newcommand{\floor}[1]{\left\lfloor #1 \right\rfloor}
\newcommand{\Var}{\mathrm{Var}}
\newcommand{\Cov}{\mathrm{Cov}}
\newcommand{\diam}{\mathrm{diam}}

\newcommand{\emC}{C_n}
\newcommand{\emCpr}{C'_n}
\newcommand{\emCthick}{C^{\sigma}_n}
\newcommand{\emCprthick}{C'^{\sigma}_n}
\newcommand{\emS}{S^{\sigma}_n}
\newcommand{\estC}{\widehat{C}_n}
\newcommand{\hC}{\hat{C^{\sigma}_n}}
\newcommand{\vol}{\text{vol}}
\newcommand{\spansp}{\mathrm{span}~}
\newcommand{\1}{\mathbf{1}}

\newcommand{\Linv}{L^{\dagger}}
\DeclareMathOperator*{\argmin}{argmin}
\DeclareMathOperator*{\argmax}{argmax}

\newcommand{\emF}{\mathbb{F}_n}
\newcommand{\emG}{\mathbb{G}_n}
\newcommand{\emP}{\mathbb{P}_n}
\newcommand{\F}{\mathcal{F}}
\newcommand{\D}{\mathcal{D}}
\newcommand{\R}{\mathcal{R}}
\newcommand{\Rd}{\Reals^d}
\newcommand{\Nbb}{\mathbb{N}}

%%% Vectors
\newcommand{\thetast}{\theta^{\star}}
\newcommand{\betap}{\beta^{(p)}}
\newcommand{\betaq}{\beta^{(q)}}
\newcommand{\vardeltapq}{\varDelta^{(p,q)}}
\newcommand{\lambdavec}{\boldsymbol{\lambda}}

%%% Matrices
\newcommand{\X}{X} % no bold
\newcommand{\Y}{Y} % no bold
\newcommand{\Z}{Z} % no bold
\newcommand{\Lgrid}{L_{\grid}}
\newcommand{\Dgrid}{D_{\grid}}
\newcommand{\Linvgrid}{L_{\grid}^{\dagger}}
\newcommand{\Lap}{L}
\newcommand{\NLap}{{\bf N}}
\newcommand{\PLap}{{\bf P}}
\newcommand{\Id}{I}

%%% Sets and classes
\newcommand{\Xset}{\mathcal{X}}
\newcommand{\Vset}{\mathcal{V}}
\newcommand{\Sset}{\mathcal{S}}
\newcommand{\Hclass}{\mathcal{H}}
\newcommand{\Pclass}{\mathcal{P}}
\newcommand{\Leb}{L}
\newcommand{\mc}[1]{\mathcal{#1}}

%%% Distributions and related quantities
\newcommand{\Pbb}{\mathbb{P}}
\newcommand{\Ebb}{\mathbb{E}}
\newcommand{\Qbb}{\mathbb{Q}}
\newcommand{\Ibb}{\mathbb{I}}

%%% Operators
\newcommand{\Tadj}{T^{\star}}
\newcommand{\dive}{\mathrm{div}}
\newcommand{\dif}{\mathop{}\!\mathrm{d}}
\newcommand{\gradient}{\mathcal{D}}
\newcommand{\Hessian}{\mathcal{D}^2}
\newcommand{\dotp}[2]{\langle #1, #2 \rangle}
\newcommand{\Dotp}[2]{\Bigl\langle #1, #2 \Bigr\rangle}

%%% Misc
\newcommand{\grid}{\mathrm{grid}}
\newcommand{\critr}{R_n}
\newcommand{\dx}{\,dx}
\newcommand{\dy}{\,dy}
\newcommand{\dr}{\,dr}
\newcommand{\dxpr}{\,dx'}
\newcommand{\dypr}{\,dy'}
\newcommand{\wt}[1]{\widetilde{#1}}
\newcommand{\wh}[1]{\widehat{#1}}
\newcommand{\ol}[1]{\overline{#1}}
\newcommand{\spec}{\mathrm{spec}}
\newcommand{\LE}{\mathrm{LE}}
\newcommand{\LS}{\mathrm{LS}}
\newcommand{\SM}{\mathrm{SM}}
\newcommand{\OS}{\mathrm{OS}}
\newcommand{\PLS}{\mathrm{PLS}}

%%% Order of magnitude
\newcommand{\soom}{\sim}

%%% Theorem environments
\newtheorem{theorem}{Theorem}
\newtheorem{conjecture}{Conjecture}
\newtheorem{lemma}{Lemma}
\newtheorem{example}{Example}
\newtheorem{corollary}{Corollary}
\newtheorem{proposition}{Proposition}
\newtheorem{assumption}{Assumption}
\newtheorem{remark}{Remark}


\theoremstyle{definition}
\newtheorem{definition}{Definition}[section]

\theoremstyle{remark}

\begin{document}

% If your paper is accepted and the title of your paper is very long,
% the style will print as headings an error message. Use the following
% command to supply a shorter title of your paper so that it can be
% used as headings.
%
%\runningtitle{I use this title instead because the last one was very long}

% If your paper is accepted and the number of authors is large, the
% style will print as headings an error message. Use the following
% command to supply a shorter version of the authors names so that
% they can be used as headings (for example, use only the surnames)
%
%\runningauthor{Surname 1, Surname 2, Surname 3, ...., Surname n}

% Supplementary material: To improve readability, you must use a single-column format for the supplementary material.
\onecolumn
\aistatstitle{Instructions for Paper Submissions to AISTATS 2021: \\
Supplementary Materials}

\section{FORMATTING INSTRUCTIONS}

To prepare a supplementary pdf file, we ask the authors to use \texttt{aistats2021.sty} as a style file and to follow the same formatting instructions as in the main paper.
The only difference is that the supplementary material must be in a \emph{single-column} format.
You can use \texttt{supplement.tex} in our starter pack as a starting point, or append the supplementary content to the main paper and split the final PDF into two separate files.

Note that reviewers are under no obligation to examine your supplementary material.

\section{MISSING PROOFS}

The supplementary materials may contain detailed proofs of the results that are missing in the main paper.

\subsection{Proof of Lemma 3}

\textit{In this section, we present the detailed proof of Lemma 3 and then [ ... ]}

\section{ADDITIONAL EXPERIMENTS}

If you have additional experimental results, you may include them in the supplementary materials.

\subsection{The Effect of Regularization Parameter}

\textit{Our algorithm depends on the regularization parameter $\lambda$. Figure 1 below illustrates the effect of this parameter on the performance of our algorithm. As we can see, [ ... ]}

\vfill

\end{document}
