\documentclass{article}
\usepackage{amsmath, amsfonts, amsthm, amssymb}
\usepackage{graphicx}
\usepackage{hyperref}
\hypersetup{
	colorlinks=true,
	linkcolor=blue,
	citecolor=blue
}
\usepackage[parfill]{parskip}
\usepackage{algpseudocode}
\usepackage{algorithm}
\usepackage{enumerate}
\usepackage[shortlabels]{enumitem}
\usepackage{mathtools}
\usepackage{tikz}
\usepackage{verbatim}
\usepackage{fullpage}

\usepackage{natbib}
\renewcommand{\bibname}{REFERENCES}
\renewcommand{\bibsection}{\subsubsection*{\bibname}}

\DeclareFontFamily{U}{mathx}{\hyphenchar\font45}
\DeclareFontShape{U}{mathx}{m}{n}{<-> mathx10}{}
\DeclareSymbolFont{mathx}{U}{mathx}{m}{n}
\DeclareMathAccent{\wb}{0}{mathx}{"73}

\DeclarePairedDelimiterX{\norm}[1]{\lVert}{\rVert}{#1}
\DeclarePairedDelimiterX{\seminorm}[1]{\lvert}{\rvert}{#1}

% Make a widecheck symbol (thanks, Stack Exchange!)
\DeclareFontFamily{U}{mathx}{\hyphenchar\font45}
\DeclareFontShape{U}{mathx}{m}{n}{
	<5> <6> <7> <8> <9> <10>
	<10.95> <12> <14.4> <17.28> <20.74> <24.88>
	mathx10
}{}
\DeclareSymbolFont{mathx}{U}{mathx}{m}{n}
\DeclareFontSubstitution{U}{mathx}{m}{n}
\DeclareMathAccent{\widecheck}{0}{mathx}{"71}
% widecheck made

\newcommand{\eqdist}{\ensuremath{\stackrel{d}{=}}}
\newcommand{\Graph}{\mathcal{G}}
\newcommand{\Reals}{\mathbb{R}}
\newcommand{\iid}{\overset{\text{i.i.d}}{\sim}}
\newcommand{\convprob}{\overset{p}{\to}}
\newcommand{\convdist}{\overset{w}{\to}}
\newcommand{\Expect}[1]{\mathbb{E}\left[ #1 \right]}
\newcommand{\Risk}[2][P]{\mathcal{R}_{#1}\left[ #2 \right]}
\newcommand{\Prob}[1]{\mathbb{P}\left( #1 \right)}
\newcommand{\iset}{\mathbf{i}}
\newcommand{\jset}{\mathbf{j}}
\newcommand{\myexp}[1]{\exp \{ #1 \}}
\newcommand{\abs}[1]{\left \lvert #1 \right \rvert}
\newcommand{\restr}[2]{\ensuremath{\left.#1\right|_{#2}}}
\newcommand{\ext}[1]{\widetilde{#1}}
\newcommand{\set}[1]{\left\{#1\right\}}
\newcommand{\seq}[1]{\set{#1}_{n \in \N}}
\newcommand{\floor}[1]{\left\lfloor #1 \right\rfloor}
\newcommand{\Var}{\mathrm{Var}}
\newcommand{\Cov}{\mathrm{Cov}}
\newcommand{\diam}{\mathrm{diam}}

\newcommand{\emC}{C_n}
\newcommand{\emCpr}{C'_n}
\newcommand{\emCthick}{C^{\sigma}_n}
\newcommand{\emCprthick}{C'^{\sigma}_n}
\newcommand{\emS}{S^{\sigma}_n}
\newcommand{\estC}{\widehat{C}_n}
\newcommand{\hC}{\hat{C^{\sigma}_n}}
\newcommand{\vol}{\text{vol}}
\newcommand{\spansp}{\mathrm{span}~}
\newcommand{\1}{\mathbf{1}}

\newcommand{\Linv}{L^{\dagger}}
\DeclareMathOperator*{\argmin}{argmin}
\DeclareMathOperator*{\argmax}{argmax}

\newcommand{\emF}{\mathbb{F}_n}
\newcommand{\emG}{\mathbb{G}_n}
\newcommand{\emP}{\mathbb{P}_n}
\newcommand{\F}{\mathcal{F}}
\newcommand{\D}{\mathcal{D}}
\newcommand{\R}{\mathcal{R}}
\newcommand{\Rd}{\Reals^d}
\newcommand{\Nbb}{\mathbb{N}}

%%% Vectors
\newcommand{\thetast}{\theta^{\star}}
\newcommand{\betap}{\beta^{(p)}}
\newcommand{\betaq}{\beta^{(q)}}
\newcommand{\vardeltapq}{\varDelta^{(p,q)}}
\newcommand{\lambdavec}{\boldsymbol{\lambda}}

%%% Matrices
\newcommand{\X}{X} % no bold
\newcommand{\Y}{Y} % no bold
\newcommand{\Z}{Z} % no bold
\newcommand{\Lgrid}{L_{\grid}}
\newcommand{\Dgrid}{D_{\grid}}
\newcommand{\Linvgrid}{L_{\grid}^{\dagger}}
\newcommand{\Lap}{L}
\newcommand{\NLap}{{\bf N}}
\newcommand{\PLap}{{\bf P}}
\newcommand{\Id}{I}

%%% Sets and classes
\newcommand{\Xset}{\mathcal{X}}
\newcommand{\Vset}{\mathcal{V}}
\newcommand{\Sset}{\mathcal{S}}
\newcommand{\Hclass}{\mathcal{H}}
\newcommand{\Pclass}{\mathcal{P}}
\newcommand{\Leb}{L}
\newcommand{\mc}[1]{\mathcal{#1}}

%%% Distributions and related quantities
\newcommand{\Pbb}{\mathbb{P}}
\newcommand{\Ebb}{\mathbb{E}}
\newcommand{\Qbb}{\mathbb{Q}}
\newcommand{\Ibb}{\mathbb{I}}

%%% Operators
\newcommand{\Tadj}{T^{\star}}
\newcommand{\dive}{\mathrm{div}}
\newcommand{\dif}{\mathop{}\!\mathrm{d}}
\newcommand{\gradient}{\mathcal{D}}
\newcommand{\Hessian}{\mathcal{D}^2}
\newcommand{\dotp}[2]{\langle #1, #2 \rangle}
\newcommand{\Dotp}[2]{\Bigl\langle #1, #2 \Bigr\rangle}

%%% Misc
\newcommand{\grid}{\mathrm{grid}}
\newcommand{\critr}{R_n}
\newcommand{\dx}{\,dx}
\newcommand{\dy}{\,dy}
\newcommand{\dr}{\,dr}
\newcommand{\dxpr}{\,dx'}
\newcommand{\dypr}{\,dy'}
\newcommand{\wt}[1]{\widetilde{#1}}
\newcommand{\wh}[1]{\widehat{#1}}
\newcommand{\ol}[1]{\overline{#1}}
\newcommand{\spec}{\mathrm{spec}}
\newcommand{\LE}{\mathrm{LE}}
\newcommand{\LS}{\mathrm{LS}}
\newcommand{\SM}{\mathrm{SM}}
\newcommand{\OS}{\mathrm{OS}}
\newcommand{\PLS}{\mathrm{PLS}}

%%% Theorem environments
\newtheorem{theorem}{Theorem}
\newtheorem{conjecture}{Conjecture}
\newtheorem{lemma}{Lemma}
\newtheorem{example}{Example}
\newtheorem{corollary}{Corollary}
\newtheorem{proposition}{Proposition}
\newtheorem{assumption}{Assumption}
\newtheorem{remark}{Remark}

\theoremstyle{definition}
\newtheorem{definition}{Definition}[section]

\theoremstyle{remark}

\begin{document}
\title{Note on: Local Weyl's Law}
\author{Alden Green}
\date{\today}
\maketitle

Let $\Delta = -\sum_{i = 1}^{d} \frac{\partial^2}{\partial x_i^2}$ be the continuum Laplacian operator in $\Rd$. Let $\mc{X}$ be a connected open bounded set in $\Rd$, with $C^{1,1}$ boundary. The eigenvector equation with Neumann boundary conditions is
\begin{equation}
\label{eqn:eigenvector}
\Delta \psi = \lambda \psi,\quad \frac{\partial}{\partial {\bf n}}\psi = 0~~\textrm{on $\partial \mc{X}$.}
\end{equation}
The operator $\Delta$ has a discrete spectrum, by which we mean that there exist enumerable pairs $(\lambda,\psi)$ which satisfy~\eqref{eqn:eigenvector}. We enumerate them $(\lambda_1,\psi_1),(\lambda_2,\psi_2),\ldots$ in increasing order of eigenvalue, $\lambda_1 \leq \lambda_2 \leq \ldots$. 

Note that $(\psi_k)_{k}$ form an orthonormal basis of $L^2(\mc{X})$. Thus the spectral function $E_{\lambda}u := \sum_{k: \lambda_k \leq \lambda} \dotp{u}{\psi_k} \psi_k$ is the projection of a given $u \in L^2(\mc{X})$ onto the ``bottom'' part of the spectrum. The following estimate on the spectral function is due to H\"{o}rmander, Theorem 17.5.3:
\begin{equation}
\label{eqn:local_weyl_law}
\sup_{x \in \mc{X}} |E_{\lambda}(x)|^2 \leq C \lambda^{d/2} \quad \textrm{for all $K \in \mathbb{N}$.}
\end{equation}
I prove~\eqref{eqn:local_weyl_law}, following the proof of H\"{o}rmander but giving some extra exposition where I didn't initially understand his reasoning. First, I make some useful notes about the result.
\begin{itemize}
	\item The statement~\eqref{eqn:local_weyl_law} also holds when $\Delta$ is replaced by a symmetric second order positive differential operator,
	\begin{equation*}
	\mc{L} :=  -\sum_{i,j = 1}^{d} g^{ij}\frac{\partial^2}{\partial x_ix_j} + \sum_{j = 1}^{d} b^j \frac{\partial}{\partial x_j} + c
	\end{equation*}
	with coefficients $g$ and $b$ in $C^{\infty}(\mc{X})$. This includes the case of the weighted Laplace-Beltrami operator $\Delta_P$ when $P$ admits a smooth density $p$ over $\mc{X}$.
	\item Let $N(\lambda) = \#\{k: \lambda_k \leq \lambda\}$. Using the trace trick, we have that
	\begin{equation*}
	N(\lambda) = \sharp\int_{\mc{X}} |E_\lambda(x)|^2 \,dx \leq C\lambda^{d/2}
	\end{equation*}
	implying that $\lambda_K \gtrsim K^{2/d}$. This is one-half of the Weyl's law asymptotic scaling $\lambda_K \asymp K^{2/d}$; in fact it is only half needed to establish 
	\item 
\end{itemize}


and show that it implies the asymptotic estimate $\lambda_K \gtrsim K^{2/d}$, which is one-half of the Weyl's Law estimate. Note that the other half of Weyl's Law, $\lambda_K \lesssim K^{2/d}$, along with~\eqref{eqn:local_weyl_law}, implies that 
\begin{equation}
\label{eqn:local_weyl_law2}
\sup_{x \in \mc{X}} |E_{\lambda_K}(x)|^2 \leq C K \quad \textrm{for all $K \in \mathbb{N}$.}
\end{equation}
~\eqref{eqn:local_weyl_law2} is how we state the result in our Laplacian eigenmaps paper.

\section{Proof of~\eqref{eqn:local_weyl_law}}



\end{document}
