\documentclass{article}
\usepackage{amsmath}
\usepackage{amsfonts, amsthm, amssymb}
\usepackage{graphicx}
\usepackage{hyperref}
\hypersetup{
	colorlinks=true,
	linkcolor=blue,
	citecolor=blue
}
\usepackage[parfill]{parskip}
\usepackage{algpseudocode}
\usepackage{algorithm}
\usepackage{enumerate}
\usepackage[shortlabels]{enumitem}
\usepackage{fullpage}
\usepackage{mathtools}
\usepackage{tikz}

\usepackage{natbib}
\renewcommand{\bibname}{REFERENCES}
\renewcommand{\bibsection}{\subsubsection*{\bibname}}

\DeclareFontFamily{U}{mathx}{\hyphenchar\font45}
\DeclareFontShape{U}{mathx}{m}{n}{<-> mathx10}{}
\DeclareSymbolFont{mathx}{U}{mathx}{m}{n}
\DeclareMathAccent{\wb}{0}{mathx}{"73}

\DeclarePairedDelimiterX{\norm}[1]{\lVert}{\rVert}{#1}
\DeclarePairedDelimiterX{\seminorm}[1]{\lvert}{\rvert}{#1}

\newcommand{\eqdist}{\ensuremath{\stackrel{d}{=}}}
\newcommand{\Graph}{\mathcal{G}}
\newcommand{\Reals}{\mathbb{R}}
\newcommand{\Identity}{\mathbb{I}}
\newcommand{\Xsetistiid}{\overset{\text{i.i.d}}{\sim}}
\newcommand{\convprob}{\overset{p}{\to}}
\newcommand{\convdist}{\overset{w}{\to}}
\newcommand{\Expect}[1]{\mathbb{E}\left[ #1 \right]}
\newcommand{\Risk}[2][P]{\mathcal{R}_{#1}\left[ #2 \right]}
\newcommand{\Prob}[1]{\mathbb{P}\left( #1 \right)}
\newcommand{\iset}{\mathbf{i}}
\newcommand{\jset}{\mathbf{j}}
\newcommand{\myexp}[1]{\exp \{ #1 \}}
\newcommand{\abs}[1]{\left \lvert #1 \right \rvert}
\newcommand{\restr}[2]{\ensuremath{\left.#1\right|_{#2}}}
\newcommand{\ext}[1]{\widetilde{#1}}
\newcommand{\set}[1]{\left\{#1\right\}}
\newcommand{\seq}[1]{\set{#1}_{n \in \N}}
\newcommand{\floor}[1]{\left\lfloor #1 \right\rfloor}
\newcommand{\Var}{\mathrm{Var}}
\newcommand{\Cov}{\mathrm{Cov}}
\newcommand{\diam}{\mathrm{diam}}

\newcommand{\emC}{C_n}
\newcommand{\emCpr}{C'_n}
\newcommand{\emCthick}{C^{\sigma}_n}
\newcommand{\emCprthick}{C'^{\sigma}_n}
\newcommand{\emS}{S^{\sigma}_n}
\newcommand{\estC}{\widehat{C}_n}
\newcommand{\hC}{\hat{C^{\sigma}_n}}
\newcommand{\vol}{\text{vol}}
\newcommand{\spansp}{\mathrm{span}~}
\newcommand{\1}{\mathbf{1}}

\newcommand{\Linv}{L^{\Xsetagger}}
\DeclareMathOperator*{\argmin}{argmin}
\DeclareMathOperator*{\argmax}{argmax}

\newcommand{\emF}{\mathbb{F}_n}
\newcommand{\emG}{\mathbb{G}_n}
\newcommand{\emP}{\mathbb{P}_n}
\newcommand{\F}{\mathcal{F}}
\newcommand{\D}{\mathcal{D}}
\newcommand{\R}{\mathcal{R}}
\newcommand{\Rd}{\Reals^d}
\newcommand{\Nbb}{\mathbb{N}}

%%% Vectors
\newcommand{\thetast}{\theta^{\star}}
\newcommand{\betap}{\beta^{(p)}}
\newcommand{\betaq}{\beta^{(q)}}
\newcommand{\vardeltapq}{\varDelta^{(p,q)}}
\newcommand{\lambdavec}{\boldsymbol{\lambda}}


%%% Matrices
\newcommand{\X}{X} % no bold
\newcommand{\Y}{Y} % no bold
\newcommand{\Z}{Z} % no bold
\newcommand{\Lgrid}{L_{\grid}}
\newcommand{\Xsetgrid}{D_{\grid}}
\newcommand{\Linvgrid}{L_{\grid}^{\Xsetagger}}
\newcommand{\Lap}{{\bf L}}
\newcommand{\NLap}{{\bf N}}
\newcommand{\PLap}{{\bf P}}
\newcommand{\Id}{I}

%%% Sets and classes
\newcommand{\Xset}{\mathcal{X}}
\newcommand{\Sset}{\mathcal{S}}
\newcommand{\Hclass}{\mathcal{H}}
\newcommand{\Pclass}{\mathcal{P}}
\newcommand{\Leb}{L}
\newcommand{\mc}[1]{\mathcal{#1}}

%%% Distributions and related quantities
\newcommand{\Pbb}{\mathbb{P}}
\newcommand{\Ebb}{\mathbb{E}}
\newcommand{\Qbb}{\mathbb{Q}}
\newcommand{\Ibb}{\mathbb{I}}

%%% Operators
\newcommand{\Tadj}{T^{\star}}
\newcommand{\Xsetive}{\mathrm{div}}
\newcommand{\Xsetif}{\mathop{}\!\mathrm{d}}
\newcommand{\gradient}{\mathcal{D}}
\newcommand{\Hessian}{\mathcal{D}^2}
\newcommand{\dotp}[2]{\langle #1, #2 \rangle}
\newcommand{\Dotp}[2]{\Bigl\langle #1, #2 \Bigr\rangle}

%%% Integrals
\def\Xint#1{\mathchoice
	{\XXint\displaystyle\textstyle{#1}}%
	{\XXint\textstyle\scriptstyle{#1}}%
	{\XXint\scriptstyle\scriptscriptstyle{#1}}%
	{\XXint\scriptscriptstyle\scriptscriptstyle{#1}}%
	\!\int}
\def\XXint#1#2#3{{\setbox0=\hbox{$#1{#2#3}{\int}$}
		\vcenter{\hbox{$#2#3$}}\kern-.5\wd0}}
\def\ddashint{\Xint=}
\def\dashint{\Xint-}

%%% Misc
\newcommand{\grid}{\mathrm{grid}}
\newcommand{\critr}{R_n}
\newcommand{\Xsetx}{\,dx}
\newcommand{\Xsety}{\,dy}
\newcommand{\Xsetr}{\,dr}
\newcommand{\Xsetxpr}{\,dx'}
\newcommand{\Xsetypr}{\,dy'}
\newcommand{\wt}[1]{\widetilde{#1}}
\newcommand{\wh}[1]{\widehat{#1}}
\newcommand{\ol}[1]{\overline{#1}}
\newcommand{\spec}{\mathrm{spec}}
\newcommand{\LE}{\mathrm{LE}}
\newcommand{\LS}{\mathrm{LS}}
\newcommand{\SM}{\mathrm{SM}}
\newcommand{\OS}{\mathrm{FS}}
\newcommand{\PLS}{\mathrm{PLS}}

%%% Order of magnitude
\newcommand{\soom}{\sim}

% \newcommand{\span}{\textrm{span}}

\newtheoremstyle{alden}
{6pt} % Space above
{6pt} % Space below
{} % Body font
{} % Indent amount
{\bfseries} % Theorem head font
{.} % Punctuation after theorem head
{.5em} % Space after theorem head
{} % Theorem head spec (can be left empty, meaning `normal')

\theoremstyle{alden} 


\newtheoremstyle{aldenthm}
{6pt} % Space above
{6pt} % Space below
{\itshape} % Body font
{} % Indent amount
{\bfseries} % Theorem head font
{.} % Punctuation after theorem head
{.5em} % Space after theorem head
{} % Theorem head spec (can be left empty, meaning `normal')

\theoremstyle{aldenthm}
\newtheorem{theorem}{Theorem}
\newtheorem{conjecture}{Conjecture}
\newtheorem{lemma}{Lemma}
\newtheorem{example}{Example}
\newtheorem{corollary}{Corollary}
\newtheorem{proposition}{Proposition}
\newtheorem{assumption}{Assumption}
\newtheorem{remark}{Remark}


\theoremstyle{definition}
\newtheorem{definition}{Definition}[section]

\theoremstyle{remark}

\begin{document}
\title{Note on: Avoiding Pitfalls when Thinking about Sobolev Spaces}
\author{Alden Green}
\date{\today}
\maketitle

The purpose of this note is to record some errors we've made when thinking about Sobolev spaces, and continuum estimators defined using their norms (thin-plate splines). The goal is to avoid repeating these errors in the future.

\begin{itemize}
	\item \textbf{Lesson 1: Pick your representative.} Elements $u$ of Sobolev spaces $W^{k,p}(\Omega)$ are equivalence classes. Functions $f$ and $g$ are equivalent if they disagree only a set of Lebesgue measure-zero. In order to speak of pointwise evaluation of some $u \in W^{k,p}(\Omega)$, we must pick first pick a representative $f^{\star} \in u$. 
	\begin{itemize}
		\item When $pk > d$, by the Sobolev Embedding Theorem every $u \in W^{k,p}(\Omega)$ has a continuous representative, and the convention is to take $f^{\star}$ to be that continuous representative.
		\item When $pk \leq d$, $W^{k,p}(\Omega)$ no longer embeds into $C(\Omega)$. In this case, we choose $f^{\star}$ to be the \emph{precise representative} of $u$, defined for $x \in \Omega$ as
		\begin{equation*}
		f^{\star}(x) :=
		\begin{dcases}
		\lim_{r \to 0} \biggl\{\frac{1}{\nu(B(x,r))} \int_{B(x,r)} u(x) \,dx\biggr\},& ~~\textrm{if the limit exists} \\
		0,& ~~\textrm{otherwise.}
		\end{dcases}
		\end{equation*}
		Note that by the Lebesgue Differentation Theorem, the set $E$ of Lebesgue points of any $u \in W^{k,p}(\Omega) \subset L^1(\Omega)$ satisfies $\nu(\Omega\setminus E) = 0$. Thus $f^{\star}$ is indeed a representative of $u$. Moreover when $2k > d$, $f^{\star}$ is continuous. 
	\end{itemize}
	Once we have picked representatives, we can make sense of the objective function
	\begin{equation*}
	R(u) = \|Y - u\|_n^2 + \lambda |u|_{W^{k,p}(\Omega)}^2.
	\end{equation*}
	Henceforth when we say ``function $f \in W^{k,p}(\Omega)$'', we are referring to the precise representative $f^{\star}$ of an equivalence class $u$.
	\item \textbf{Lesson 2: Be precise about why the smoothing spline problem is well-posed when $2k > d$.} Let $p = 2$. When $2k > d$, the Hilbert-Sobolev space $H^k(\Omega) = W^{k,2}(\Omega)$ can be continuously embedded into $C^{j,\gamma}(\Omega)$ with $\gamma = \floor{d/2} + 1 - d/2$ and $j = k - \gamma - d/2$ (Here for convenience we have assumed $d$ is odd.) For the moment set $d = 1$ and $\Omega = [0,1]$. When $d = 1$, observe that $\gamma = 1/2$ and $j = k - 1$. Therefore any function $f \in H^k(\Omega)$ is absolutely continuous, with continuous derivatives $f',f'',\ldots,f^{(k- 1)}$, and with $k$th weak derivative bounded in $L^2([0,1])$ norm. These continuity properties ensure that the \emph{smoothing spline} optimization problem,
	\begin{equation}
	\label{eqn:smoothing_spline}
	\min_{H^k([0,1])} \|Y - u\|_n^2 + \lambda |u|_{W^{k,p}(\Omega)}^2,
	\end{equation}
	is well-defined. By this we mean that there exists a unique function $\wt{u} \in H^k([0,1])$ which achieves the minimum in~\eqref{eqn:smoothing_spline}. Much more is known about the properties of $\wt{u}$, but that is not important for this discussion.
	\item \textbf{Lesson 3: Be precise about why the smoothing spline problem is ill-posed when $2k \leq d$.} Let's return to the general multivariate setting, where $d$ is arbitrary and $\Omega$ is a regular domain. Suppose $2k \leq d$. Now we know problem~\eqref{eqn:smoothing_spline} is ill-posed. My previous explanation of this facet would have been ``it is ill-posed because you can put spikes at the data'' but that is not the case--we have already ruled out changes on a set of measure zero by picking a precise representative. The correct explanation is that $R(u)$ is not a continuous function of $u$ with respect to the topology induced by $H^k(\Omega)$ norm. For instance, take a sequence of bump function $g_h = \textcolor{red}{(?)}$. Clearly $g_h(x) = 0$ for any $x \in \Omega\setminus B(0,h)$. On the one hand, a standard calculation will show that $g_h \to 0$ as $h \to \infty$, where crucially convergence is meant in the Sobolev norm. On the other hand, $g_h(0) = 1$ for all $h$, and so $R(g_h) \not\to R(0)$. For this reason~\eqref{eqn:smoothing_spline} does not have a minimizer in $H^k(\Omega)$. 
	
	NB: This is a particular example of a general mistake I often make, which is forgetting that convergence of a sequence in Sobolev norm does not imply pointwise convergence, either of the function $f$ (when $2k \leq d$) or of the $j$th derivative of the function (when $2(k - j) \leq d$.).
	\item \textbf{Lesson 4: Practical implementation.} In our AISTATS paper,  I make the following comment:
	\begin{quotation}
		...any practical implementation of~\eqref{eqn:smoothing_spline} will be inconsistent [when $d \geq 2$ and $k = 1$.]
	\end{quotation}
	This comment is too imprecise to be interpretable. Let's give it some meaning. Numerically, one could try to solve~\eqref{eqn:smoothing_spline} by discretizing the space $\Omega$ and solving a discrete version of~\eqref{eqn:smoothing_spline}. \textcolor{red}{This discrete problem} will always be well-posed, meaning it will always admit a unique solution. Unfortunately, if one takes the discretization of $\Omega$ to be \textcolor{red}{sufficiently fine} (compared to the number of samples $n$) the resulting estimator $\check{u}$ will be an inconsistent estimator of $f_0$.
	\item \textbf{Lesson 5: Be more imaginative.} I tend to think of Sobolev spaces as containing continuous and continuously differentiable functions, up to local deviations (spikes). This intuition is only appropriate when $2k > d$. For instance, the fractional Sobolev space $H^{1/2}([0,1])$ contains step-functions. When $d \geq 2$, weak differentiability does not imply continuity. 
\end{itemize}

\textcolor{red}{(TODO):}
\begin{itemize}
	\item Note that when $2k > d$, and $d > 1$, the $(k-1)$st partial derivatives of $f \in H^k(\Omega)$ may not be classically defined, and yet~\eqref{eqn:smoothing_spline} is still well-defined. Find what continuity properties suffice for~\eqref{eqn:smoothing_spline} to be well-defined.
	\item Give an example of a \textcolor{red}{reasonable function} $f \in H^1([-1,1]^2)$ which is discontinuous. A brief Google search finds examples which seem straight out of the mind of Georg Cantor.
\end{itemize}

\end{document}
