\documentclass{article}
\usepackage{amsmath}
\usepackage{amsfonts, amsthm, amssymb}
\usepackage{bm}
\usepackage{graphicx}
\usepackage[colorlinks]{hyperref}
\usepackage[parfill]{parskip}
\usepackage{algpseudocode}
\usepackage{algorithm}
\usepackage{enumerate}

\usepackage{natbib}
\renewcommand{\bibname}{REFERENCES}
\renewcommand{\bibsection}{\subsubsection*{\bibname}}

\makeatletter
\newcommand{\leqnomode}{\tagsleft@true}
\newcommand{\reqnomode}{\tagsleft@false}
\makeatother

\newcommand{\eqdist}{\ensuremath{\stackrel{d}{=}}}
\newcommand{\Graph}{\mathcal{G}}
\newcommand{\Reals}{\mathbb{R}}
\newcommand{\Identity}{\mathbb{I}}
\newcommand{\distiid}{\overset{\text{i.i.d}}{\sim}}
\newcommand{\convprob}{\overset{p}{\to}}
\newcommand{\convdist}{\overset{w}{\to}}
\newcommand{\Expect}[1]{\mathbb{E}\left[ #1 \right]}
\newcommand{\Risk}[2][P]{\mathcal{R}_{#1}\left[ #2 \right]}
\newcommand{\Var}[1]{\mathrm{Var}\left( #1 \right)}
\newcommand{\Prob}[1]{\mathbb{P}\left( #1 \right)}
\newcommand{\iset}{\mathbf{i}}
\newcommand{\jset}{\mathbf{j}}
\newcommand{\myexp}[1]{\exp \{ #1 \}}
\newcommand{\norm}[1]{\left\lVert#1\right\rVert}
\newcommand{\dotp}[2]{\langle #1 , #2 \rangle}
\newcommand{\abs}[1]{\left \lvert #1 \right \rvert}
\newcommand{\restr}[2]{\ensuremath{\left.#1\right|_{#2}}}
\newcommand{\defeq}{\overset{\mathrm{def}}{=}}
\newcommand{\convweak}{\overset{w}{\rightharpoonup}}
\newcommand{\dive}{\mathrm{div}}

\newcommand{\emC}{C_n}
\newcommand{\emCpr}{C'_n}
\newcommand{\emCthick}{C^{\sigma}_n}
\newcommand{\emCprthick}{C'^{\sigma}_n}
\newcommand{\emS}{S^{\sigma}_n}
\newcommand{\estC}{\widehat{C}_n}
\newcommand{\hC}{\hat{C^{\sigma}_n}}
\newcommand{\vol}{\text{vol}}
\newcommand{\Bal}{\textrm{Bal}}
\newcommand{\Cut}{\textrm{Cut}}
\newcommand{\Ind}{\textrm{Ind}}
\newcommand{\set}[1]{\left\{#1\right\}}
\newcommand{\seq}[1]{\set{#1}_{n \in \N}}
\newcommand{\Perp}{\perp \! \! \! \perp}
\newcommand{\Naturals}{\mathbb{N}}


\newcommand{\Linv}{L^{\dagger}}
\newcommand{\tr}{\text{tr}}
\newcommand{\h}{\textbf{h}}
% \newcommand{\l}{\ell}
\newcommand{\x}{\textbf{x}}
\newcommand{\y}{\textbf{y}}
\newcommand{\bl}{\bm{\ell}}
\newcommand{\bnu}{\bm{\nu}}
\newcommand{\Lx}{\mathcal{L}_X}
\newcommand{\Ly}{\mathcal{L}_Y}
\DeclareMathOperator*{\argmin}{argmin}
\DeclareMathOperator*{\argmax}{argmax}


\newcommand{\emG}{\mathbb{G}_n}
\newcommand{\A}{\mathcal{A}}
\newcommand{\F}{\mathcal{F}}
\newcommand{\G}{\mathcal{G}}
\newcommand{\X}{\mathcal{X}}
\newcommand{\Rd}{\Reals^d}
\newcommand{\N}{\mathbb{N}}
\newcommand{\E}{\mathcal{E}}

%%% Matrix related notation
\newcommand{\Xbf}{\mathbf{X}}
\newcommand{\Ybf}{\mathbf{Y}}
\newcommand{\Zbf}{\mathbf{Z}}
\newcommand{\Abf}{\mathbf{A}}
\newcommand{\Dbf}{\mathbf{D}}
\newcommand{\Wbf}{\mathbf{W}}
\newcommand{\Lbf}{\mathbf{L}}
\newcommand{\Ibf}{\mathbf{I}}
\newcommand{\Bbf}{\mathbf{B}}

%%% Vector related notation
\newcommand{\lbf}{\bm{\ell}}
\newcommand{\fbf}{\mathbf{f}}

%%% Set related notation
\newcommand{\Dset}{\mathcal{D}}
\newcommand{\Aset}{\mathcal{A}}
\newcommand{\Wset}{\mathcal{W}}
\newcommand{\Hset}{\mathcal{H}}

%%% Distribution related notation
\newcommand{\Pbb}{\mathbb{P}}
\newcommand{\Qbb}{\mathbb{Q}}
% \newcommand{\Pr}{\mathrm{Pr}}}

%%% Functionals
\newcommand{\1}{\mathbf{1}}
\newtheoremstyle{alden}
{6pt} % Space above
{6pt} % Space below
{} % Body font
{} % Indent amount
{\bfseries} % Theorem head font
{.} % Punctuation after theorem head
{.5em} % Space after theorem head
{} % Theorem head spec (can be left empty, meaning `normal')

\theoremstyle{alden} 
\newtheorem{definition}{Definition}[section]

\newtheoremstyle{aldenthm}
{6pt} % Space above
{6pt} % Space below
{\itshape} % Body font
{} % Indent amount
{\bfseries} % Theorem head font
{.} % Punctuation after theorem head
{.5em} % Space after theorem head
{} % Theorem head spec (can be left empty, meaning `normal')

\theoremstyle{aldenthm}
\newtheorem{theorem}{Theorem}
\newtheorem{conjecture}{Conjecture}
\newtheorem{lemma}{Lemma}
\newtheorem{example}{Example}
\newtheorem{corollary}{Corollary}
\newtheorem{proposition}{Proposition}
\newtheorem{assumption}{Assumption}

\theoremstyle{remark}
\newtheorem{remark}{Remark}


\begin{document}
	
\title{Notes for Week of 1/28/19 - 2/1/18}
\author{Alden Green}
\date{\today}
\maketitle

\textcolor{red}{(WARNING: NOTATION IS NOT WELL-DEFINED. NO GUARANTEE OF CORRECTNESS. THESE ARE INTENDED ONLY AS RECORDS. ALL RELEVANT INFORMATION SHOULD BE TRANSFERRED TO A BETTER-WRITTEN DOCUMENT.)}

We work with distributions $\Pbb$ and $\Qbb$ with support over $\Dset \subset \Reals$. Recall that our \emph{Laplacian smooth} test statistic can be written as
\begin{equation*}
T_2(\lbf; G_{n,r}) = \sup_{\fbf \in \Reals^n: \norm{\Bbf \fbf}_2^2 \leq C_{n,r}} ~ \frac{1}{n} \sum_{k = 1}^{n} \ell_k f_k
\end{equation*}
for $G_{n,r}$ the neighborhood graph of radius $r$ over data,, and $\Bbf \in \Reals^{m \times n}$ corresponding incidence matrix.

Consider the set of functions
\begin{equation*}
\mathcal{H}^{1,2}(\Dset,\nu) := \set{f: \Dset \to \Reals \vert ~f(0) = 0,~ \text{and $f$ is absolutely continuous with $f' \in L^2(\Dset)$} }
\end{equation*}
where $f'$ is the derivative of $f$, and $\nu$ is the Lebesgue measure over $\Reals$. Equip $\Hset^{1,2}(\Dset,\nu)$ with the inner product
\begin{equation*}
\dotp{f}{g}_{\Hset, \nu}^{1,2} := \int_{\Dset} f'(z) g'(z) dz,
\end{equation*} 
and denote the corresponding norm $\norm{\cdot}_{\Hset, \nu}^{1,2}$

Let us consider the test statistic
\begin{equation*}
T_2(\lbf; \Wset^{1,2}) = \sup_{f \in \Wset^{1,2}(\Dset, \nu)} \frac{1}{n} \sum_{k=1}^{n} \ell_k f(z_k)
\end{equation*}
where
\begin{equation*}
\Wset^{1,2}(\Dset, \nu) = \set{f \in \Hset^{1,2}(\Dset,\nu): \norm{f}_{\Hset} \leq 1}
\end{equation*}

\paragraph{Hilbert-Sobolev Space.}
The Hilbert space $\Hset^{1,2}(\Dset,\nu)$ equipped with inner product $\dotp{f}{g}_{\Hset, \nu}^{1,2}$ can be shown to be an RKHS with associated kernel
\begin{equation*}
k(x,z) = \min\{x,z\}
\end{equation*}

As a result, results for two-sample testing in the RKHS setup can be brought to bear.

\textcolor{red}{Summarize results of Gretton.}

and eigenfunction / eigenvalue pairs given by
\begin{equation*}
\phi_j(t) = \frac{\sin(2j - 1)\pi t}{2}, ~~ \mu_j = \left(\frac{2}{(2j - 1)\pi}\right)^2
\end{equation*}
for $j = 1,2,\ldots$.

\end{document}