\documentclass{article}
\usepackage{amsmath}
\usepackage{amsfonts, amsthm, amssymb}
\usepackage{graphicx}
\usepackage[colorlinks]{hyperref}
\usepackage[parfill]{parskip}
\usepackage{algpseudocode}
\usepackage{algorithm}
\usepackage{enumerate}
\usepackage[shortlabels]{enumitem}

\usepackage{natbib}
\renewcommand{\bibname}{REFERENCES}
\renewcommand{\bibsection}{\subsubsection*{\bibname}}

\newcommand{\eqdist}{\ensuremath{\stackrel{d}{=}}}
\newcommand{\Graph}{\mathcal{G}}
\newcommand{\Reals}{\mathbb{R}}
\newcommand{\Identity}{\mathbb{I}}
\newcommand{\distiid}{\overset{\text{i.i.d}}{\sim}}
\newcommand{\convprob}{\overset{p}{\to}}
\newcommand{\convdist}{\overset{w}{\to}}
\newcommand{\Expect}[1]{\mathbb{E}\left[ #1 \right]}
\newcommand{\Risk}[2][P]{\mathcal{R}_{#1}\left[ #2 \right]}
\newcommand{\Prob}[1]{\mathbb{P}\left( #1 \right)}
\newcommand{\iset}{\mathbf{i}}
\newcommand{\jset}{\mathbf{j}}
\newcommand{\myexp}[1]{\exp \{ #1 \}}
\newcommand{\norm}[1]{\left\lVert#1\right\rVert}
\newcommand{\abs}[1]{\left \lvert #1 \right \rvert}
\newcommand{\restr}[2]{\ensuremath{\left.#1\right|_{#2}}}
\newcommand{\ext}[1]{\widetilde{#1}}
\newcommand{\set}[1]{\left\{#1\right\}}
\newcommand{\seq}[1]{\set{#1}_{n \in \N}}
\newcommand{\dotp}[2]{\langle #1, #2 \rangle}
\newcommand{\floor}[1]{\left\lfloor #1 \right\rfloor}
\newcommand{\Var}{\mathrm{Var}}
\newcommand{\diam}{\mathrm{diam}}

\newcommand{\emC}{C_n}
\newcommand{\emCpr}{C'_n}
\newcommand{\emCthick}{C^{\sigma}_n}
\newcommand{\emCprthick}{C'^{\sigma}_n}
\newcommand{\emS}{S^{\sigma}_n}
\newcommand{\estC}{\widehat{C}_n}
\newcommand{\hC}{\hat{C^{\sigma}_n}}
\newcommand{\vol}{\text{vol}}
\newcommand{\spansp}{\mathrm{span}~}
\newcommand{\1}{\mathbb{I}}

\newcommand{\Linv}{L^{\dagger}}
\DeclareMathOperator*{\argmin}{argmin}
\DeclareMathOperator*{\argmax}{argmax}

\newcommand{\emF}{\mathbb{F}_n}
\newcommand{\emG}{\mathbb{G}_n}
\newcommand{\emP}{\mathbb{P}_n}
\newcommand{\F}{\mathcal{F}}
\newcommand{\D}{\mathcal{D}}
\newcommand{\R}{\mathcal{R}}
\newcommand{\Rd}{\Reals^d}

%%% Matrices
\newcommand{\Xbf}{\mathbf{X}}
\newcommand{\Ybf}{\mathbf{Y}}
\newcommand{\Zbf}{\mathbf{Z}}

%%% Sets and classes
\newcommand{\Xset}{\mathcal{X}}
\newcommand{\Sset}{\mathcal{S}}
\newcommand{\Hclass}{\mathcal{H}}
\newcommand{\Pclass}{\mathcal{P}}

%%% Distributions and related quantities
\newcommand{\Pbb}{\mathbb{P}}
\newcommand{\Ebb}{\mathbb{E}}
\newcommand{\Qbb}{\mathbb{Q}}

%%% Operators
\newcommand{\Tadj}{T^{\star}}
\newcommand{\dive}{\mathrm{div}}

% \newcommand{\span}{\textrm{span}}

\newtheoremstyle{alden}
{6pt} % Space above
{6pt} % Space below
{} % Body font
{} % Indent amount
{\bfseries} % Theorem head font
{.} % Punctuation after theorem head
{.5em} % Space after theorem head
{} % Theorem head spec (can be left empty, meaning `normal')

\theoremstyle{alden} 


\newtheoremstyle{aldenthm}
{6pt} % Space above
{6pt} % Space below
{\itshape} % Body font
{} % Indent amount
{\bfseries} % Theorem head font
{.} % Punctuation after theorem head
{.5em} % Space after theorem head
{} % Theorem head spec (can be left empty, meaning `normal')

\theoremstyle{aldenthm}
\newtheorem{theorem}{Theorem}
\newtheorem{conjecture}{Conjecture}
\newtheorem{lemma}{Lemma}
\newtheorem{example}{Example}
\newtheorem{corollary}{Corollary}
\newtheorem{proposition}{Proposition}
\newtheorem{assumption}{Assumption}
\newtheorem{remark}{Remark}


\theoremstyle{definition}
\newtheorem{definition}{Definition}[section]

\theoremstyle{remark}

\begin{document}
\title{Notes for Week 2/8/19 - 2/15/19}
\author{Alden Green}
\date{\today}
\maketitle

Consider $\Xbf = (X_1, \ldots, X_n) \sim \Pbb$ and $\Ybf = (Y_1, \ldots, Y_m) \sim \Qbb$ independently generated random variables. We will assume $\Pbb$ and $\Qbb$ are supported on $\D \subset \Reals^d$ and are absolutely continuous with density functions $f$ and $g$, respectively. Our statistical goal is hypothesis testing: that is, we wish to construct a test function $\phi$ which differentiates between
\begin{equation*}
\mathbb{H}_0: f = g \text{ and } \mathbb{H}_1: f \neq g.
\end{equation*}

For a given function class $\Hclass$, some $\epsilon > 0$, and test function $\phi$ a Borel measurable function of the data with range $\set{0,1}$, we evaluate the quality of the test using \emph{worst-case risk}
\begin{equation*}
R_{\epsilon}^{m,n}(\phi; \Hclass) = \sup_{f \in \Hclass} \Ebb_{f,f}^{(m,n)}(\phi) + \sup_{ \substack{f,g \in \Hclass \\ \delta(f,g) \geq \epsilon } } \Ebb_{f,g}^{m,n}(1 - \phi)
\end{equation*} 

\paragraph{Total variation test.}

Let $\Zbf = (X_1, \ldots, X_n, Y_1, \ldots, Y_m)$, and let $t = n + m$. As in \citep{padilla2018}, define the \emph{$K$-NN graph} $G_K = (V,E_K)$ to have vertex set $V = \set{1, \ldots, t}$ and edge set $E_K$ which contains the pair $(i,j)$ if and only if $x_i$ is among the $K$-nearest neighbors (with respect to Euclidean distance) of $x_j$, or vice versa. Let $D_K$ denote the incidence matrix of $G_K$. 

Define the \emph{kNN-total variation} test statistic to be
\begin{equation}
\label{eqn: knn_tv_test_statistic}
T_{TV} = \sup_{\substack{\theta \in \Reals^{t}: \\ \norm{D_{k} \theta}_1 \leq C_{n,k}} } \left(\sum_{i = 1}^{n} \theta_i - \sum_{j = n+1}^{t} \theta_j\right)
\end{equation}

Hereafter, take $\D = [0,1]^d$, and consider
\begin{equation*}
\Hclass_{lip}(L) = \set{f: [0,1]^d \to \Reals^{+}: \int_{\D}f = 1, f~ \text{$L$-piecewise Lipschitz, bounded above and below} }
\end{equation*}
where
\begin{definition}[Piecewise Lipschitz]
	\label{def: piecewise_lipschitz}
	A function $f$ is $L$-\emph{piecewise lipschitz} over $[0,1]^d$ if there exists a set $\Sset \subset [0,1]^d$ such that
	\begin{enumerate}[(a)]
		\item $\nu(\Sset) = 0$
		\item There exist $C_{\Sset}, \epsilon_0$ such that $\mu \Bigl( \bigl(\Sset_{\epsilon} \cup (\partial \D)_{\epsilon} \bigr) \cap [0,1]^d \Bigr) \leq C_{\Sset} \epsilon$ for all $0 < \epsilon \leq \epsilon_0$.
		\item For any $z, z'$ in the same connected component of $[0,1]^d \setminus \bigl(\Sset_{\epsilon} \cup (\partial \D)_{\epsilon}\bigr)$,
		\begin{equation*}
		\abs{g(z) - g(z')}_2 \leq L\norm{z - z'}_2
		\end{equation*}
	\end{enumerate}
\end{definition}
and
\begin{definition}[Bounded above and below]
	\label{def: bounded}
	A function $f: \D \to \Reals$ is \emph{bounded above and below} if there exists $p_{\min}, p_{\max}$ such that
	\begin{equation*}
	0 < p_{\min} < f(x) < p_{\max} < \infty \tag{$\forall x \in \D$}
	\end{equation*}
\end{definition}

\begin{conjecture}
	For $\tau = ???$ and $K \asymp \log^{1 + 2r}(n)$ for some $r \geq 0$, the test $\phi_{TV} = \set{T_{TV} \geq \tau}$ has worst-case risk
	\begin{equation*}
	R_{\epsilon}^{(m,m)}(\Hclass_{lip}(L)) \leq c_1/a^2
	\end{equation*}
	whenever $\epsilon \geq c_2 a \log^{\alpha}m m^{-1/d}$ where $\alpha = 3r + 5/2 + (2r + 1)/d$ and $c_1$ and $c_2$ are constants which depend only on $(d,L)$.
\end{conjecture}
\begin{proof}
	Write
	\begin{equation*}
	\sum_{i = 1}^{m} \theta_i - \sum_{j = m + 1}^{2m} \theta_j = \dotp{\theta}{1_X - 1_Y}
	\end{equation*}
	where
	\begin{equation*}
	1_X = 
	\begin{cases}
	1,~ i = 1, \ldots,m \\
	0,~ \text{otherwise}
	\end{cases}
	1_Y = 
	\begin{cases}
	1,~ i = j + 1, \ldots,2m \\
	0,~ \text{otherwise}
	\end{cases}
	\end{equation*}
	Let
	\begin{equation*}
	\widehat{\theta} \in \argmax_{\theta \in \Reals^{2m}} \set{\dotp{\theta}{1_X - 1_Y}: \norm{D_K \theta}_1 \leq C_{n,k}}
	\end{equation*}
	satisfy $T_{TV} = \dotp{\widehat{\theta}}{1_X - 1_Y}$. For $i = 1, \ldots, 2m$, introduce $\theta^{\star}$ defined by
	\begin{equation*}
	(\theta^{\star})_i = \left(\frac{f(z_i) - g(z_i)}{f(z_i) + g(z_i)} \right)	\end{equation*}
	
	We have
	\begin{equation*}
	T_{TV} = \dotp{\widehat{\theta} - \theta^{\star}}{1_X - 1_Y} + \dotp{\theta^{\star}}{1_X - 1_Y}
	\end{equation*}
	
	We bound the first term, and begin by employing a basic inequality argument. Let
	\begin{align*}
	\widetilde{\theta} = \frac{\theta^{\star}}{\norm{D\theta^{\star}}_1}.
	\end{align*}
	Since $\widetilde{\theta}$ is feasible and $\widehat{\theta}$ optimal for \eqref{eqn: knn_tv_test_statistic}, we have
	\begin{equation*}
	\dotp{\widehat{\theta}}{1_X - 1_Y} \geq \dotp{\widetilde{\theta}}{1_X - 1_Y}
	\end{equation*}
	which, letting $w = (1_X - 1_Y) - \theta^{\star}$, means
	\begin{equation}
	\label{eqn: basic_inequality_1}
	\dotp{\widehat{\theta}}{\theta^{\star}} \geq \dotp{\widetilde{\theta}}{\theta^{\star}} - \dotp{\widehat{\theta} - \widetilde{\theta}}{w}
	\end{equation}
	
	As a result, we have
	\begin{align*}
	\abs{\dotp{\widehat{\theta} - \theta^{\star}}{1_X - 1_Y}} & \leq \abs{\dotp{\widehat{\theta} - \theta^{\star}}{\theta^{\star}} } + \abs{\dotp{\widehat{\theta} - \theta^{\star}}{w} } \\
	& \overset{(i)}{\leq} \abs{\dotp{\widetilde{\theta} - \theta^{\star}}{\theta^{\star}} } + \abs{\dotp{\widehat{\theta} - \widetilde{\theta}}{w} } + \abs{\dotp{\widehat{\theta} - \theta^{\star}}{w} } \\
	& \leq \abs{\dotp{\widetilde{\theta} - \theta^{\star}}{\theta^{\star}} } + 2\abs{\dotp{\widehat{\theta} - \widetilde{\theta}}{w} } + \abs{\dotp{\widetilde{\theta} - \theta^{\star}}{w}}
	\end{align*}
	where $(i)$ follows from \eqref{eqn: basic_inequality_1}.
\end{proof}

\clearpage

\bibliography{../../graph_testing_bibliography}
\bibliographystyle{plain}

\end{document}