\documentclass{article}
\usepackage{amsmath, amsfonts, amsthm, amssymb}
\usepackage{graphicx}
\usepackage{hyperref}
\hypersetup{
	colorlinks=true,
	linkcolor=blue,
	citecolor=blue
}
\usepackage[parfill]{parskip}
\usepackage{algpseudocode}
\usepackage{algorithm}
\usepackage{enumerate}
\usepackage[shortlabels]{enumitem}
\usepackage{mathtools}
\usepackage{tikz}
\usepackage{verbatim}
\usepackage{fullpage}

\usepackage{natbib}
\renewcommand{\bibname}{REFERENCES}
\renewcommand{\bibsection}{\subsubsection*{\bibname}}

\DeclareFontFamily{U}{mathx}{\hyphenchar\font45}
\DeclareFontShape{U}{mathx}{m}{n}{<-> mathx10}{}
\DeclareSymbolFont{mathx}{U}{mathx}{m}{n}
\DeclareMathAccent{\wb}{0}{mathx}{"73}

\DeclarePairedDelimiterX{\norm}[1]{\lVert}{\rVert}{#1}
\DeclarePairedDelimiterX{\seminorm}[1]{\lvert}{\rvert}{#1}

% Make a widecheck symbol (thanks, Stack Exchange!)
\DeclareFontFamily{U}{mathx}{\hyphenchar\font45}
\DeclareFontShape{U}{mathx}{m}{n}{
	<5> <6> <7> <8> <9> <10>
	<10.95> <12> <14.4> <17.28> <20.74> <24.88>
	mathx10
}{}
\DeclareSymbolFont{mathx}{U}{mathx}{m}{n}
\DeclareFontSubstitution{U}{mathx}{m}{n}
\DeclareMathAccent{\widecheck}{0}{mathx}{"71}
% widecheck made

\newcommand{\eqdist}{\ensuremath{\stackrel{d}{=}}}
\newcommand{\Graph}{\mathcal{G}}
\newcommand{\Reals}{\mathbb{R}}
\newcommand{\iid}{\overset{\text{i.i.d}}{\sim}}
\newcommand{\convprob}{\overset{p}{\to}}
\newcommand{\convdist}{\overset{w}{\to}}
\newcommand{\Expect}[1]{\mathbb{E}\left[ #1 \right]}
\newcommand{\Risk}[2][P]{\mathcal{R}_{#1}\left[ #2 \right]}
\newcommand{\Prob}[1]{\mathbb{P}\left( #1 \right)}
\newcommand{\iset}{\mathbf{i}}
\newcommand{\jset}{\mathbf{j}}
\newcommand{\myexp}[1]{\exp \{ #1 \}}
\newcommand{\abs}[1]{\left \lvert #1 \right \rvert}
\newcommand{\restr}[2]{\ensuremath{\left.#1\right|_{#2}}}
\newcommand{\ext}[1]{\widetilde{#1}}
\newcommand{\set}[1]{\left\{#1\right\}}
\newcommand{\seq}[1]{\set{#1}_{n \in \N}}
\newcommand{\floor}[1]{\left\lfloor #1 \right\rfloor}
\newcommand{\Var}{\mathrm{Var}}
\newcommand{\Cov}{\mathrm{Cov}}
\newcommand{\diam}{\mathrm{diam}}

\newcommand{\emC}{C_n}
\newcommand{\emCpr}{C'_n}
\newcommand{\emCthick}{C^{\sigma}_n}
\newcommand{\emCprthick}{C'^{\sigma}_n}
\newcommand{\emS}{S^{\sigma}_n}
\newcommand{\estC}{\widehat{C}_n}
\newcommand{\hC}{\hat{C^{\sigma}_n}}
\newcommand{\vol}{\text{vol}}
\newcommand{\spansp}{\mathrm{span}~}
\newcommand{\1}{\mathbf{1}}

\newcommand{\Linv}{L^{\dagger}}
\DeclareMathOperator*{\argmin}{argmin}
\DeclareMathOperator*{\argmax}{argmax}

\newcommand{\emF}{\mathbb{F}_n}
\newcommand{\emG}{\mathbb{G}_n}
\newcommand{\emP}{\mathbb{P}_n}
\newcommand{\F}{\mathcal{F}}
\newcommand{\D}{\mathcal{D}}
\newcommand{\R}{\mathcal{R}}
\newcommand{\Rd}{\Reals^d}
\newcommand{\Nbb}{\mathbb{N}}

%%% Vectors
\newcommand{\thetast}{\theta^{\star}}
\newcommand{\betap}{\beta^{(p)}}
\newcommand{\betaq}{\beta^{(q)}}
\newcommand{\vardeltapq}{\varDelta^{(p,q)}}
\newcommand{\lambdavec}{\boldsymbol{\lambda}}

%%% Matrices
\newcommand{\X}{X} % no bold
\newcommand{\Y}{Y} % no bold
\newcommand{\Z}{Z} % no bold
\newcommand{\Lgrid}{L_{\grid}}
\newcommand{\Dgrid}{D_{\grid}}
\newcommand{\Linvgrid}{L_{\grid}^{\dagger}}
\newcommand{\Lap}{L}
\newcommand{\NLap}{{\bf N}}
\newcommand{\PLap}{{\bf P}}
\newcommand{\Id}{I}

%%% Sets and classes
\newcommand{\Xset}{\mathcal{X}}
\newcommand{\Vset}{\mathcal{V}}
\newcommand{\Sset}{\mathcal{S}}
\newcommand{\Hclass}{\mathcal{H}}
\newcommand{\Pclass}{\mathcal{P}}
\newcommand{\Leb}{L}
\newcommand{\mc}[1]{\mathcal{#1}}

%%% Distributions and related quantities
\newcommand{\Pbb}{\mathbb{P}}
\newcommand{\Ebb}{\mathbb{E}}
\newcommand{\Qbb}{\mathbb{Q}}
\newcommand{\Ibb}{\mathbb{I}}

%%% Operators
\newcommand{\Tadj}{T^{\star}}
\newcommand{\dive}{\mathrm{div}}
\newcommand{\dif}{\mathop{}\!\mathrm{d}}
\newcommand{\gradient}{\mathcal{D}}
\newcommand{\Hessian}{\mathcal{D}^2}
\newcommand{\dotp}[2]{\langle #1, #2 \rangle}
\newcommand{\Dotp}[2]{\Bigl\langle #1, #2 \Bigr\rangle}

%%% Misc
\newcommand{\grid}{\mathrm{grid}}
\newcommand{\critr}{R_n}
\newcommand{\dx}{\,dx}
\newcommand{\dy}{\,dy}
\newcommand{\dr}{\,dr}
\newcommand{\dxpr}{\,dx'}
\newcommand{\dypr}{\,dy'}
\newcommand{\wt}[1]{\widetilde{#1}}
\newcommand{\wh}[1]{\widehat{#1}}
\newcommand{\ol}[1]{\overline{#1}}
\newcommand{\spec}{\mathrm{spec}}
\newcommand{\LE}{\mathrm{LE}}
\newcommand{\LS}{\mathrm{LS}}
\newcommand{\SM}{\mathrm{SM}}
\newcommand{\OS}{\mathrm{OS}}
\newcommand{\PLS}{\mathrm{PLS}}

%%% Theorem environments
\newtheorem{theorem}{Theorem}
\newtheorem{conjecture}{Conjecture}
\newtheorem{lemma}{Lemma}
\newtheorem{example}{Example}
\newtheorem{corollary}{Corollary}
\newtheorem{proposition}{Proposition}
\newtheorem{assumption}{Assumption}
\newtheorem{remark}{Remark}

\theoremstyle{definition}
\newtheorem{definition}{Definition}[section]

\theoremstyle{remark}

\begin{document}
\title{Note on: Out-of-Sample Error for Laplacian Eigenmaps}
\author{Alden Green}
\date{\today}
\maketitle

The graph Laplacian eigenmaps estimator $\wh{\theta}_{\LE}$ is a least-squares estimator, where a response vector $Y = (Y_1,\ldots,Y_n)$ is projected onto the span of the first $K$ eigenvectors of a graph Laplacian matrix. It is defined only in-sample, whereas out-of-sample error is more interesting in many applications. There is no commonly agreed upon way in the literature to extend $\wh{\theta}_{\LE}$, or indeed any type of Laplacian estimator. In this note, I discuss three methods for doing so: least-squares regression of $\wh{\theta}_{\LE}$ onto an a priori defined set of features, Nadaraya-Watson kernel smoothing, treating $\wh{\theta}_{\LE}$ as the response, and Nystr\"{o}m extension of Laplacian eigenvectors. These methods are roughly ordered from least to most preferable, but also (as these things go) easiest to hardest to analyze. I begin in Section~\ref{sec:least_squares_arbitrary_design_and_features} by reviewing known bounds on the out-of-sample error of a (generic) least-squares estimator; these bounds will be useful for the analysis of both the first and third methods.

\section{Out-of-sample error for least-squares estimators: arbitrary design and features}
\label{sec:least_squares_arbitrary_design_and_features}
This section largely summarizes results from a paper by Daniel Hsu on random design regression. We start with some general setup that will apply to everything that follows.

Suppose we observe $(X_1,\wh{\theta}_1),\ldots,(X_n,\wh{\theta}_n)$, where $X_1,\ldots,X_n \in \Reals^d$ and $\wh{\theta}_{1:n} = (\wh{\theta}_1,\ldots,\wh{\theta}_n) \in \Reals^n$. Our goal is to use this observed data to construct an estimator $\wh{f}$ which is on average close to $f^{\ast}: \Omega \to \Reals$ on the domain $\Omega$ (for instance, $\Omega = [0,1]^d$.) Specifically, we will consider the following least-squares estimator,
\begin{equation*}
\wh{\beta} = \argmin \| \wh{\theta}_{1:n} - \Phi_n^{\top} \beta \|_n^2 = (\Phi_n^\top \Phi_n)^{-1}\Phi_n^{\top} \wh{\theta}_{1:n},~~ \wh{f} = \sum_{k = 1}^{\kappa} \wh{\beta}_k \varphi_k
\end{equation*}
where $\varphi_1,\ldots,\varphi_{K}$ are each elements of $\Leb^2(\Omega)$, and $\Phi_n \in \Reals^{n \times K}$ has entries $(\Phi_{n})_{ik} = \varphi_k(X_i)$. If $\wh{\theta}_{1:n}$ lies in the span of the columns of $\Phi_n$, then $\wh{f}$ is a interpolant of $\wh{\theta}_{1:n}$; this applies to our third method, Nystr\"{o}m extension of Laplacian eigenvectors, but not to our first method. We will measure error in the $\Leb^2(P)$ norm $\norm{\wh{f} - f^{\ast}}_P^2 = \int_{\Omega} (\wh{f}(x) - f^{\ast}(x))^2 \,dP(x)$, where $P$ is a probability measure supported on domain $\Omega$. For this section, we assume no relationship between $P$ and $X_1,\ldots,X_n$.

We begin by decomposing this norm into several modular terms, which capture the different types of error intrinsic to the problem. To this end, let
\begin{align*}
\wt{\beta} & = \argmin_{\beta \in \Reals^K} \|f^{\ast}_{1:n} - \Phi_n^{\top} \beta \|_n^2 = (\Phi_n^\top \Phi_n)^{-1}\Phi_n^{\top} {f}^{\ast}_{1:n},&& \wt{f}_K = \sum_{k = 1}^{K} \wt{\beta}_k \varphi_k \\ 
\beta^{\star} & = \argmin_{\beta \in \Reals^K} \|f^{\ast} - \sum_{k = 1}^{K} \beta_k \varphi_k \|_P^2,&& f_K^{\ast} = \sum_{k = 1}^{K} \beta_k^{\ast} \varphi_k
\end{align*}
and let $f_{\perp}^{\ast} = f^{\ast} - f_{K}^{\ast}$. We decompose $\norm{\wh{f} - f^{\ast}}_P^2$ as follows:
\begin{equation*}
\norm{\wh{f} - f^{\ast}}_P^2 \leq 3\Bigl(\underbrace{\norm{\wh{f} - \wt{f}_K}_P^2}_{\textrm{estimation error}} + \underbrace{\norm{\wt{f}_K - f_K^{\ast}}_P^2}_{\textrm{sampling error}} + \underbrace{\norm{f_K^{\ast} - f^{\ast}}_P^2}_{\textrm{truncation error}}\Bigr)
\end{equation*}
This leaves us with three separate kinds of error, and we now upper bound the first two.

\paragraph{Notation.} Let $e_{1:n} = \wh{\theta}_{1:n} - {f}^{\ast}_{1:n}$. Let~$\Sigma \in \Reals^{K \times K}$ be the second moment matrix of $\varphi_1,\ldots,\varphi_K$ with $(k,\ell)$-th entry $\Sigma_{k\ell} = \Ebb_{X \sim P}\bigl[\varphi_k(X) \varphi_{\ell}(X)\bigr]$. Likewise let $\wh{\Sigma} = \Phi_n^{\top} \Phi_n/n$ be the empirical second moment matrix. For a matrix $A \in \Reals^{K \times K}$, let $\|A\|$ denote the operator norm of $A$.  For a probability distribution $Q$ on $\Omega$, let $d_{\min}(Q) = \inf_{x \in \Omega} \int K(\|z - x\|/r) \,dQ(z)$ be the minimum degree of $Q$.

\paragraph{Estimation error.}
Noting that
\begin{equation*}
~~\wh{\beta} - \wt{\beta} = \frac{1}{n}\wh{\Sigma}^{-1} \Phi_n^{\top} e_{1:n},
\end{equation*}
the estimation error satisfies the inequality,
\begin{align*}
\norm{\wh{f} - \wt{f}_K}_P^2 & = (\wh{\beta} - \wt{\beta})^{\top} \Sigma (\wh{\beta} - \wt{\beta}) \\
& = \frac{1}{n^2} e_{1:n}^{\top} \Phi_n \wh{\Sigma}^{-1} \Sigma \wh{\Sigma}^{-1} \Phi_n^{\top} e_{1:n} \\
& \leq \|\Sigma^{1/2} \wh{\Sigma}^{-1} \Sigma^{1/2}\| \cdot \|e_{1:n}\|_n^2.
\end{align*}
with the inequality following since $\|AB\| = \|BA\|$. 

\paragraph{Sampling error.}
Multiplying $\wt{\beta} - \beta^{\star}$ on the left by $\Sigma^{1/2}$, we have
\begin{equation*}
\Sigma^{1/2}(\wt{\beta} - \beta^{\ast}) =  \Sigma^{1/2} \wh{\Sigma}^{-1} \bigl( \frac{1}{n} \Phi_n^{\top} f_{1:n}^{\ast} - \frac{1}{n} \Phi_n^{\top} \Phi_n \beta^{\ast} \bigr) = \Sigma^{1/2} \wh{\Sigma}^{-1} \frac{1}{n} \Phi_n^{\top}  \bigl(\underbrace{f_{1:n}^{\ast} - \Phi_n \beta^{\ast}}_{ = (f_{\perp}^{\ast})_{1:n}} \bigr) = \frac{1}{n}\Sigma^{1/2} \wh{\Sigma}^{-1} \Phi_n^{\top} (f_{\perp}^{\ast})_{1:n}.
\end{equation*}
Therefore
\begin{equation*}
\|\wt{f}_K - f_K^{\ast}\|_P^2  = \|\wt{\beta} - \beta^{\ast}\|_{\Sigma}^2 \leq \|\Sigma^{1/2} \wh{\Sigma}^{-1} \Sigma^{1/2}\| \cdot \Bigl\|f_{\perp}^{\ast}\|_n^2
\end{equation*}

The results of this section are summarized by Lemma~\ref{lem:l2p_error_fixed_design}. 
\begin{lemma}
	\label{lem:l2p_error_fixed_design}
	Suppose $\wh{\Sigma} \succeq 0$. Then the $\Leb^2(P)$ error $\|\wh{f} - f^{\ast}\|_P^2$ is upper bounded as follows:
	\begin{equation}
	\label{eqn:l2p_error_fixed_design}
	\|\wh{f} - f^{\ast}\|_P^2 \leq 3\Bigl(\|\Sigma^{1/2} \wh{\Sigma}^{-1} \Sigma^{1/2}\| \cdot \|\wh{\theta}_{1:n} - {f}^{\ast}_{1:n}\|_n^2 + \|\Sigma^{1/2} \wh{\Sigma}^{-1} \Sigma^{1/2}\| \cdot \|f_{\perp}^{\ast}\|_n^2 + \|f_{\perp}^{\ast}\|_P^2\Bigr).
	\end{equation}
	If $\wh{\Sigma} = I$ then
	\begin{equation}
	\label{eqn:l2p_error_fixed_design_identity}
	\|\wh{f} - f^{\ast}\|_P^2 \leq 3\Bigl(\|\Sigma\| \cdot \|\wh{\theta}_{1:n} - {f}^{\ast}_{1:n}\|_n^2 + \|\Sigma\| \cdot \|f_{\perp}^{\ast}\|_n^2 + \|f_{\perp}^{\ast}\|_P^2\Bigr).
	\end{equation}
\end{lemma}
(Basically) no assumptions so far have been made on $X_1,\ldots,X_n$ or $\varphi_1,\ldots,\varphi_K$. In the standard setup $X_1,\ldots,X_n$ are randomly sampled from $P$, and $\varphi_1,\ldots,\varphi_K$ are a priori determined. Then matrix Bernstein's inequality gives an upper bound on $\|\Sigma^{1/2} \wh{\Sigma}^{-1} \Sigma^{1/2}\|$, standard Bernstein's inequality gives an upper bound on $\|f_{\perp}^{\ast}\|_n^2 - \|f_{\perp}^{\ast}\|_P^2$, and we obtain from~\eqref{eqn:l2p_error_fixed_design} an upper bound that depends on $\|e_{1:n}\|_n^2$ and $\|f_{\perp}^{\ast}\|_P^2$.  

If $\varphi_1,\ldots,\varphi_{K}$ are chosen in a data-dependent manner, then the analysis is plainly quite different. Each of $\|\Sigma\|$, $\|f_{\perp}^{\ast}\|_n^2$ and $\|f_{\perp}^{\ast}\|_P^2$ are random variables which depend on $\varphi_1,\ldots,\varphi_K$. We also do not have concentration of $\|f_{\perp}^{\ast}\|_n^2$ around $\|f_{\perp}^{\ast}\|_P^2$. 

\section{Kernel smoothing}
Suppose we observe $(X_1,\wh{\theta}_1),\ldots,(X_n,\wh{\theta}_n)$, where $X_1,\ldots,X_n \in \Reals^d$ are sampled i.i.d from an unknown distribution $P$ on domain $\Omega$, and $\wh{\theta}_{1:n} = (\wh{\theta}_1,\ldots,\wh{\theta}_n) \in \Reals^n$. Our goal is to construct an estimator $\wh{f}$ such that the error $\|\wh{f} - f^{\ast}\|_P^2$ is as small as possible relative to $\|\wh{\theta}_{1:n} - f^{\ast}_{1:n}\|_n^2$.

For a kernel function $\eta(\cdot): [0,\infty) \to (-\infty,+\infty)$, bandwidth $\varepsilon > 0$, and a distribution $Q$, the \emph{Nadaraya-Watson kernel smoother} $T_{\varepsilon,Q}$ is given by
\begin{equation*}
\bigl(T_{\varepsilon,Q}f)(x) := \frac{1}{d_Q(x)} \int_{\Omega} f(z)\eta\biggl(\frac{\|z - x\|}{\varepsilon}\biggr) \,dQ(z)
\end{equation*}
where $d_Q(x) = \int_{\Omega} \eta\bigl(\|z - x\|/\varepsilon\bigr) \,dQ(z)$. We denote $T_{\varepsilon,P_n}$ by $T_{\varepsilon,n}$, and $d_{P_n}$ by $d_n$.

The estimated function $\wh{f}$ we will consider is a kernel smoother passed over $\wh{\theta}$,
\begin{equation}
\label{eqn:kernel_smoother_laplacian_eigenmaps}
\wh{f}(x) := T_{\varepsilon,n}\wh{\theta}_{1:n}(x).
\end{equation}
In order to relate the out-of-sample error $\|\wh{f} - f^{\ast}\|_P$ to the in-sample error $\|\wh{\theta}_{1:n} - f^{\ast}_{1:n}\|_n$, we will assume the following.
\begin{enumerate}[label=(A\arabic*)]
	\item
	\label{asmp:kernel}
	The kernel function $\eta$ is compactly supported on a subset of $[0,1]$. Additionally, $\eta$ is Lipschitz on its support. Finally, for convenience we will assume that $\eta$ is normalized, 
	\begin{equation*}
	\int_{0}^{\infty} \eta(s) \,ds = 1.
	\end{equation*}
	\item 
	\label{asmp:domain} There exists $\alpha > 0$ such that for every $x \in \mc{X}$, $d_{\nu}(x) \geq \alpha \cdot \varepsilon^d$.
	\item 
	\label{asmp:density}The distribution $P$ admits a density $p$ with respect to Lebesgue measure. The density $p \in C^1(\mc{X})$, and is bounded away from $0$ and $\infty$ for all $x \in \mc{X}$,
	\begin{equation*}
	0 < p_{\min} \leq p(x) < p_{\max} < \infty.
	\end{equation*}
\end{enumerate}
Under Assumption~\ref{asmp:kernel}, we automatically have that $d_\nu(x) = \varepsilon^d$ for all $x \in \mc{X}_{\varepsilon}$. Assumption~\ref{asmp:domain} is thus a condition on the boundary $\partial \Omega$, and is implied for all sufficiently small $\varepsilon$ if the boundary is Lipschitz. Denoting $K(\cdot) = (\eta(\cdot))^2$, we note that Assumption~\ref{asmp:kernel} implies that $\eta$ is square-integrable, $\int_{0}^{\infty} K(s) \,ds < \infty$.
\begin{lemma}
	\label{lem:kernel_smoothing_laplacian_eigenmaps}
	Suppose $f^{\ast} \in H^1(\mc{X})$ and $p \in C^1(\mc{X})$. Then with probability at least $1 - \delta - C\varepsilon^d\exp\{-Cn\varepsilon^d\}$, it holds that
	\begin{equation*}
	\|\wh{f} - f^{\ast}\|_P^2 \leq C\biggl(\|\wh{\theta}_{1:n} - f^{\ast}_{1:n}\|_n^2 + \frac{1}{\delta} \cdot \frac{\varepsilon^2}{n\varepsilon^d} |f|_{H^1(\mc{X})}^2 + \frac{1}{\delta}\|T_{\varepsilon,P}f^{\ast} - f^{\ast}\|_P^2\biggr)
	\end{equation*}
\end{lemma}
Notice that the variance term in the above is smaller than the typical variance term for kernel smoothing of noisy data (on the order of $1/(n\varepsilon^d)$) by a factor of $\varepsilon^2$. The bias term, on the other hand, is standard (at least when $f^{\ast} \in C_0^s(\Omega)$). In particular the following estimate of the bias can be obtained. 
\begin{lemma}
	\label{lem:kernel_smoothing_bias}
	Suppose $f^{\ast} \in H_0^{s}(\Omega)$, $p \in C^{r}(\Omega)$ for $r = s - 1$, and that $\eta$ is an order-$s$ kernel. It follows that
	\begin{equation*}
	\|T_{\varepsilon,P}f^{\ast} - f^{\ast}\|_P^2 \leq C \varepsilon^{2s}.
	\end{equation*}
\end{lemma}
\begin{proof}[Proof (when dimension $d = 1$).]
	\textcolor{red}{(TODO)}
\end{proof}

Thus balancing bias and variance, using a smaller bandwidth ($\varepsilon \asymp n^{-1/(2r + d)}$) than is typical for kernel smoothing with noisy data, leads to an asymptotically negligible error ($n^{-2s/(2r + d)}$) compared to the minimax rate for estimating Sobolev functions. 

\begin{proof}[Proof of Lemma~\ref{lem:kernel_smoothing_laplacian_eigenmaps}]
	By the triangle inequality, 
	\begin{equation*}
	\|\wh{f} - f^{\ast}\|_P \leq \|\wh{f} - T_{\varepsilon,n}f^{\ast}_{1:n}\|_P + \|T_{\varepsilon,n}f^{\ast}_{1:n} - f^{\ast}\|_P.
	\end{equation*}
	We analyze each of the two terms in the above expression separately. We will show that
	\begin{equation}
	\label{pf:kernel_smoothing_laplacian_eigenmaps_0}
	\|\wh{f} - T_{\varepsilon,n}f^{\ast}_{1:n}\|_P \leq \frac{1}{\alpha}\bigl(1 + o_P(1)\bigr) \cdot \|\wh{\theta}_{1:n} - f^{\ast}_{1:n}\|_n
	\end{equation}
	and 
	\begin{equation}
	\label{pf:kernel_smoothing_laplacian_eigenmaps_0.5}
	\|T_{\varepsilon,n}f^{\ast}_{1:n} - f^{\ast}\|_P^2 \leq \frac{C}{\delta}\frac{\varepsilon^2}{n\varepsilon^d} |f|_{H^1(\mc{X})}^2 + \frac{1}{\delta}\|T_{\varepsilon,P}f^{\ast} - f^{\ast}\|_P^2,
	\end{equation}
	the latter inequality holding with probability $1 - \delta$.
	
	\textit{Proof of \eqref{pf:kernel_smoothing_laplacian_eigenmaps_0}.} We will begin by assuming that $\eta$ is positive, $\eta(z) \geq 0$ for all $z$, and return to prove the claim in its full generality at the end.
	
	The left hand side of~\eqref{pf:kernel_smoothing_laplacian_eigenmaps_0} is equal to $\|T_{\varepsilon,n}(\wh{\theta}_{1:n} - f^{\ast}_{1:n})\|_P$, and~\eqref{pf:kernel_smoothing_laplacian_eigenmaps_0} is implied by showing that the operator $T_{\varepsilon,n}: L^2(X) \to L^2(\Omega)$ is (up to a constant) a contraction. We will show that for any $u \in L^2(X)$,
	\begin{equation}
	\label{pf:kernel_smoothing_laplacian_eigenmaps_1}
	\|T_{\varepsilon,n}u\|_P^2 \leq \frac{1}{\alpha}\bigl(1 + o_P(1)\bigr) \|u\|_n^2.
	\end{equation}
	We begin with a pointwise estimate, which follows from Jensen's inequality:
	\begin{equation*}
	\Bigl(\bigl(T_{n,\varepsilon}u\bigr)(x)\Bigr)^2 \leq \frac{1}{d_n(x) n} \sum_{i = 1}^{n} \bigl(u(X_i)\bigr)^2 \eta\biggl(\frac{\|X_i - x\|}{\varepsilon}\biggr).
	\end{equation*}
	(Note that it is here that we have used the positivity of $\eta(\cdot)$.) Integrating with respect to $P$ and exchanging sum with integral yields an upper bound on the $L^2(P)$ norm of $T_{n,\varepsilon}u$,
	\begin{equation}
	\label{pf:kernel_smoothing_laplacian_eigenmaps_2}
	\|T_{n,\varepsilon}u\|_P^2 \leq \sum_{i = 1}^{n} u_i^2 \Bigl(\int_{\Omega} \frac{\eta(\|X_i - x\|/\varepsilon)}{n d_n(x)} p(x) \,dx \Bigr)
	\end{equation}
	The degree estimate $d_n(x)$ is quite close to $p(x)$. As we show in Lemma~\ref{lem:measure_of_ball_estimate}, if $\varepsilon = \omega((\log n/n)^{1/d})$ then the empirical measure $d_n(x)$ concentrates around $d_P(x)$ uniformly over $x \in \Omega$, 
	\begin{equation*}
	\sup_{x \in \Omega}|d_n(x) - d_P(x)| = o_P(\varepsilon^d).
	\end{equation*}
	Moreover, it follows from the Lipschitz continuity of $p$ that for any $x \in \Omega$,
	\begin{equation*}
	d_P(x) = \int_{\Omega} \eta\biggl(\frac{\|x' - x\|}{\varepsilon}\biggr) p(x') \,dx' = p(x) d_{\nu}(x) + O(\varepsilon^{d + 1}),
	\end{equation*}
	and thus by the triangle inequality $\sup_{x \in \Omega}|d_n(x) - d_\nu(x) p(x)| = o_P(\varepsilon^d)$.
	
	As pointed out previously, if $x' \in \Omega_{\varepsilon}$ then $d_{\nu}(x') = \varepsilon^d$. On the other hand by assumption~\ref{asmp:domain} for any $x' \in \Omega, d_{\nu}(x') \geq \varepsilon^d/\alpha$. Thus 
	\begin{equation*}
	\int_{\Omega} \frac{\eta(\|x' - x\|/\varepsilon)}{d_n(x')} p(x') \,dx' = \int_{\Omega} \frac{\eta(\|x' - x\|/\varepsilon)}{d_{\nu}(x') - o_P(\varepsilon^d)} \,dx' \leq 
	\begin{dcases*}
	\frac{1}{1 - o_P(1)}, & \textrm{if $x \in \Omega_{2\varepsilon}$} \\
	\frac{1}{\alpha} \cdot \frac{1}{1 - o_P(1)}, & \textrm{otherwise.}
	\end{dcases*}
	\end{equation*}
	Plugging back in to~\eqref{pf:kernel_smoothing_laplacian_eigenmaps_2}, we deduce that
	\begin{equation*}
	\|T_{n,\varepsilon}u\|_P^2 \leq \sum_{i = 1}^{n} u_i^2 \biggl(\1\Bigl\{x \in \Omega_{2\varepsilon}\Bigr\} \frac{1}{1 - o_P(1)} + \1\Bigl\{x \in (\partial\Omega)_{2\varepsilon}\Bigr\} \frac{1}{\alpha} \frac{1}{1 - o_P(1)}\biggr).
	\end{equation*}
	This establishes~\eqref{pf:kernel_smoothing_laplacian_eigenmaps_1}, and in turn~\eqref{pf:kernel_smoothing_laplacian_eigenmaps_0}.
	
	Now, suppose $\eta(\cdot)$ can take on negative values. In this case, we have $\eta(\cdot) = \mathrm{sgn}(\eta(\cdot))K^{1/2}(\cdot)$. Let $d_n(x;K^{1/2})$ be the empirical degree with respect to kernel function $K^{1/2}$. Then
	\begin{align*}
	T_{n,\varepsilon}(u)(x) & = \frac{1}{d_n(x)n} \sum_{i = 1}^{n} \bigl(u(X_i)\bigr) \eta\biggl(\frac{\|X_i - x\|}{\varepsilon}\biggr) \\
	& = \frac{d_n(x;K^{1/2})}{d_n(x)}\frac{1}{d_n(x;K^{1/2})n} \sum_{i = 1}^{n} \bigl(u(X_i)\bigr) \mathrm{sgn}\biggl(\eta\biggl(\frac{\|X_i - x\|}{\varepsilon}\biggr)\biggr) K^{1/2}\biggl(\frac{\|X_i - x\|}{\varepsilon}\biggr).
	\end{align*}
	Taking the square and applying Jensen's inequality, we deduce that
	\begin{align*}
	\bigl(T_{n,\varepsilon}u(x)\bigr)^2 & \leq \biggl(\frac{d_n(x;K^{1/2})}{d_n(x)}\biggr)^2 \cdot \frac{1}{d_n(x;K^{1/2})n}\sum_{i = 1}^{n} \bigl(u(X_i)\bigr)^2 K^{1/2}\biggl(\frac{\|X_i - x\|}{\varepsilon}\biggr) \\
	& \leq \frac{(1 + o_P(1))\|\eta\|_{\infty}^2}{\alpha^2} \cdot \frac{1}{d_n(x;K^{1/2})n}\sum_{i = 1}^{n} \bigl(u(X_i)\bigr)^2 K^{1/2}\biggl(\frac{\|X_i - x\|}{\varepsilon}\biggr),
	\end{align*}
	with the latter inequality following by Lemma~\ref{lem:measure_of_ball_estimate} and assumption~\ref{asmp:kernel}. From here the proof proceeds along similar lines to the case where $\eta$ is non-negative.
	
	\textit{Proof of~\eqref{pf:kernel_smoothing_laplacian_eigenmaps_0.5}.}  
	We will begin by introducing some notation. For a function $u \in L^2(\Omega)$, let
	\begin{equation*}
	L_{\varepsilon,n}u(x) := T_{\varepsilon,n}u(x) - u(x) = \frac{1}{d_n(x)n} \sum_{i = 1}^{n} \bigl(u(x) - u(X_i)\bigr) \eta\biggl(\frac{\|X_i - x\|}{\varepsilon}\biggr)
	\end{equation*}
	and let
	\begin{equation*}
	\wt{L}_{\varepsilon,n}u(x) := \frac{d_n(x)}{d_P(x)}L_{\varepsilon,n}u(x) = \frac{1}{d_P(x)n} \sum_{i = 1}^{n} \bigl(u(x) - u(X_i)\bigr) \eta\biggl(\frac{\|X_i - x\|}{\varepsilon}\biggr).
	\end{equation*}
	We have already shown that $d_n(x)/d_P(x) = (1 + o_P(1))$, and so we will focus our analysis on upper bounding $\Ebb[\|\wt{L}_{\varepsilon,n}f^{\ast}\|_P^2]$. We begin by providing expressions for the pointwise mean and variance of $\wt{L}_{\varepsilon,n}f^{\ast}(x)$:
	\begin{equation*}
	\Ebb\bigl[\wt{L}_{\varepsilon,n}f^{\ast}(x)\bigr] = T_{P,\varepsilon}f^{\ast}(x) - f^{\ast}(x)
	\end{equation*}
	and
	\begin{align*}
	\mathbb{V}\bigl[\wt{L}_{\varepsilon,n}f^{\ast}(x)\bigr] & = \frac{1}{(d_P(x))^2 n} \mathbb{V}\bigl[\bigl(f^{\ast}(X) - f^{\ast}(x)\bigr) \eta\bigl(\|X - x\|/\varepsilon\bigr)\bigr] \\
	& \leq \frac{1}{p_{\min}^2 \alpha^2 \varepsilon^{2d}} \mathbb{E}\bigl[\bigl(f^{\ast}(X) - f^{\ast}(x)\bigr)^2 K\bigl(\|X - x\|/\varepsilon\bigr)\bigr] \\
	& = \frac{1}{p_{\min}^2 \alpha^2 \varepsilon^{2d}} \int_{\Omega} \bigl(f^{\ast}(x') - f^{\ast}(x)\bigr)^2 K(\|x' - x\|/\varepsilon) p(x) \,dx.
	\end{align*}
	Then, integrating with respect to $P$ over $\Omega$, we obtain
	\begin{align*}
	\Ebb\bigl[\|\wt{L}_{\varepsilon,n}f^{\ast}\|_P^2\bigr]  & \leq \|T_{P,\varepsilon}f^{\star} - f^{\star}\|_{P}^2 + \frac{1}{p_{\min}^2 \alpha^2 \varepsilon^{2d}} E_{\varepsilon}(f^{\ast}) \\
	& \leq \|T_{P,\varepsilon}f^{\star} - f^{\star}\|_{P}^2 + \frac{C\varepsilon^2}{p_{\min}^2 \alpha^2 \varepsilon^{d}} |f|_{H^1(\Omega)}^2
	\end{align*}
	where $E_{\varepsilon}(f^{\ast}) := \int_{\Omega} \int_{\Omega} (f^{\ast}(x') - f^{\ast}(x))^2 K(\|x' - x\|/\varepsilon) p(x') p(x) ,\dx' \,dx$, and we have shown in our AISTATS paper that $E_{\varepsilon}(f^{\ast}) \leq C \varepsilon^{d + 2} |f|_{H^1(\Omega)}^2$, justifying the second inequality. Consequently, 
	\begin{align*}
	\|T_{n,\varepsilon}f^{\ast}_{1:n} - f^{\ast}\|_P^2 \leq (1 + o_P(1))^2 \|\wt{L}_{\varepsilon,n}f^{\ast}\|_P^2 & \leq \frac{\bigl(1 + o_P(1)\bigr)^2}{\delta} \mathbb{E}\bigl[\|\wt{L}_{\varepsilon,n}f^{\ast}\|_P^2\bigr] \\ 
	& \leq \frac{\bigl(1 + o_P(1)\bigr)^2}{\delta}\biggl(\|T_{P,\varepsilon}f^{\star} - f^{\star}\|_{P}^2 + \frac{C\varepsilon^2}{p_{\min}^2 \alpha^2 \varepsilon^{d}} |f|_{H^1(\Omega)}^2\biggr)
	\end{align*}
	with probability at least $1 - \delta$, establishing~\eqref{pf:kernel_smoothing_laplacian_eigenmaps_0.5}.
\end{proof}

\begin{lemma}
	\label{lem:measure_of_ball_estimate}
	With probability at least $1 - C \varepsilon^{-d}\exp\{-Cn\varepsilon^d\}$, it holds that
	\begin{equation*}
	\sup_{x \in \Omega}|d_n(x) - d_P(x)| \leq C \sqrt{\frac{\varepsilon^d}{n}},
	\end{equation*}
	and
	\begin{equation*}
	\sup_{x \in \Omega}|d_n(x;K^{1/2}) - d_P(x;K^{1/2})| \leq C \sqrt{\frac{\varepsilon^d}{n}}
	\end{equation*}
\end{lemma}

\begin{proof}
	\textcolor{red}{(TODO)}
\end{proof}

\section{Nystr\"{o}m extension of Laplacian eigenvectors}
The Nystr\"{o}m extension $\varphi_k$ of an eigenvector $v_k$ of the graph Laplacian $L$ is
\begin{equation}
\label{eqn:nystrom}
\varphi_k(x) := \frac{1}{d_n(x) - \lambda_k} \sum_{i = 1}^{n} v_{k,i} K\Bigl(\frac{\|X_i - x\|}{r}\Bigr)
\end{equation}
where $d_n(x) := \sum_{i = 1}^{n} K(\|X_i - x\|/r)$, and we assume $d_n(x) > \lambda_k$ for all $x \in \Omega$ and $k = 1,\ldots,K$. There are certain nice properties of the Nystr\"{o}m extension. It is an interpolant---that is, for any $\theta \in \mathrm{span}\{v_1,\ldots,v_K\}$ the function $f \in \Leb^2(\Omega)$ will agree with $\theta$ at sample design points---and it requires no information on $\Omega$ to be well-defined. However it is also tricky to analyze, because it involves taking $\theta \in \Leb^2(X)$, transforming it into eigenspace, and then applying a \emph{different} smoother to each eigenvector. To build some intuition, we will warm up by analyzing piecewise constant extension, which is much easier to tackle theoretically, if suboptimal for $f^{\ast} \in H^s(\mc{X})$ when $s > 1$.

\subsection{Piecewise constant extension}
Both $\|\Sigma\|$ and $\|f_{\perp}^{\ast}\|_P^2$ are easy to analyze when $\varphi_1,\ldots,\varphi_k$ are taken to be piecewise constant extensions of eigenvectors $v_k$ of the Laplacian matrix $L$.
Let $\{V_1,\ldots,V_n\}$ partition the domain $\Omega$ into $n$ cells, each of probability mass $\Pbb(V_j) = 1/n$, such that $x_j \in V_j$. The piecewise constant extension operator $P^{\star}: L^2(X)\to L^2(\Omega)$ is
\begin{equation}
\label{eqn:piecewise_extension}
\bigl(P^{\star}\theta\bigr)(x) = \sum_{j = 1}^{n} \theta_j \1\{x \in V_j\},
\end{equation}
and we take $\varphi_k = P^{\star}v_k$ for $k = 1,\ldots,K$.

It is clear that $\dotp{{\varphi}_k}{{\varphi}_{\ell}}_{P} = \dotp{v_k}{v_{\ell}}_{n} = \delta_{ij}$, so that $\Sigma = I$ and $\|\Sigma\| = 1$. Moreover, we have that $f_K^{\ast} = P^{\ast}\wt{f}_K$, so that
\begin{equation*}
\|P^{\star}f_{1:n}^{\ast} - {f}_K^{\ast}\|_P = \|f^{\ast} - \wt{f}_K\|_n = \|f^{\ast} - f_K^{\ast}\|_n;
\end{equation*}
the term $\|f^{\ast} - \wt{f}_K\|_n$ is an in-sample bias term that we have already analyzed. It remains to upper bound $\|P^{\star}f^{\ast} - f^{\ast}\|_P$, which we can do using crude but sufficient techniques when $f^{\ast} \in C^1(\Omega)$. 

\subsection{Nystr\"{o}m extension}

The analysis of the Nystr\"{o}m extension seems challenging. At present I have suitable bounds on zero out of three of the functionals in question ($\|\Sigma\|$, $\|f_{\perp}^{\ast}\|_n^2$ and $\|f_{\perp}^{\ast}\|_P^2$). 

\paragraph{Energy of population-level covariance $(\|\Sigma\|)$.}
The following representation of $\|\Sigma\|$ seems useful:
\begin{equation}
\label{eqn:covariance_energy}
\|\Sigma\| = \sup_{\substack{f \in \mathrm{span}\{\varphi_1,\ldots,\varphi_K\} \\ \|f\|_n = 1}} \|f\|_P.
\end{equation}
The equivalence~\eqref{eqn:covariance_energy} suggests a useful intermediary step in upper bound $\|\Sigma\|$ is an upper bound on $\|\varphi_k\|_P^2$. To do so, it is useful express the Nystrom extension of $v_k$ in terms of a kernel smoother. Recall that $T_{r,n}$ is the kernel smoother,
\begin{equation*}
\bigl(T_{r,n}\theta\bigr)(x) := \frac{1}{d_n(x)} \sum_{i = 1}^{n} \theta_i K\biggl(\frac{\|X_i - x\|}{r}\biggr).
\end{equation*}
Then constructing $\varphi_k$ is equivalent to first passing a kernel smoother over $v_k$, then applying a multiplication operator, i.e
\begin{equation}
\label{eqn:nystrom_extension_kernel_smoother_representation}
\varphi_k(x) = \frac{d_n(x)}{d_n(x) - \lambda_k} \bigl(T_{r,n}v_k\bigr)(x).
\end{equation}  
The operator $T_{r,n}$ is a contraction, in the sense that $\|T_{r,n}\theta\|_P \leq C \|\theta\|_n$, as the following derivations make clear:
\begin{align*}
\|T_{r,n}\theta\|_P^2 & = \int \Biggl(\frac{1}{d_n(x)} \sum_{i = 1}^{n} \theta_i K\biggl(\frac{\|X_i - x\|}{r}\biggr)\Biggr)^2 \,dP(x) \\
& \leq \sum_{i = 1}^{n} \theta_i^2 \int \frac{1}{d_n(x)}  K\biggl(\frac{\|X_i - x\|}{r}\biggr) \,dP(x) \\
& \leq \|\theta\|_n^2 \cdot \frac{d_{\max}(P)}{d_{\min}(P_n)}.
\end{align*}
(One can do better than this, actually, but this suffices for our purposes).

Therefore from~\eqref{eqn:nystrom_extension_kernel_smoother_representation}, and Holder's inequality, we have that
\begin{equation}
\label{eqn:nystrom_extension_l2p_norm}
\|\varphi_k\|_P \leq \sup_{x \in \Omega} \biggl\{\frac{d_n(x)}{d_n(x) - \lambda_k} \biggr\} \cdot  \|T_rv_k\|_P \leq \frac{d_{\min}(P_n)}{d_{\min}(P_n) - \lambda_k/n} \cdot \sqrt{ \frac{d_{\max}(P)}{d_{\min}(P_n)}}.
\end{equation}
For \textcolor{red}{typical} choices of $K$ and $r$, we have that $d_{\min}(P_n) \gg \lambda_K/n$, and therefore the upper bound~\eqref{eqn:nystrom_extension_l2p_norm} is $O(1)$, which suffices for our purposes.

Unfortunately, notwithstanding my prior claim, I don't yet see the right way to leverage this into a bound on $\|\Sigma\|$. For a given $\theta = \sum_{k = 1}^{K} \beta_k v_k$, and the resulting extrapolant $f = \sum_{k = 1}^{K} \beta_k \varphi_k(x)$, we can write $f$ in terms of $T_r\theta$ and a remainder term as follows,
\begin{equation*}
f(x) = \bigl(T_r\theta\bigr)(x) + \sum_{k = 1}^{K} \beta_k \frac{\lambda_k}{d_n(x) - \lambda_k} \bigl(T_rv_k\bigr)(x).
\end{equation*}
However, my best analysis of the remainder term, which I state here, is doomed by accumulation of error:
\begin{align*}
\biggl\|\sum_{k = 1}^{K} \beta_k \frac{\lambda_k}{d_n(x) - \lambda_k} \bigl(T_rv_k\bigr)(x)\biggr\|_P^2 & \overset{(i)}{\leq} K \cdot \sum_{k = 1}^{K} \beta_k^2  \biggl(\frac{\lambda_k/n}{d_{\min}(P_n) - \lambda_k/n}\biggr)^2 \cdot  \|T_rv_k\|_P^2 \\
& \overset{(ii)}{\leq} K \biggl(\frac{\lambda_K/n}{d_{\min}(P_n) - \lambda_K/n}\biggr)^2 \cdot \frac{d_{\max}(P)}{d_{\min}(P_n)} \\
& \asymp \frac{K \lambda_K^2}{n^2d_{\min}(P_n)}.
\end{align*}
To see that this bound is insufficient: recall the optimal choice $K_{\star} \asymp n^{d/(d + 2s)}$, that the eigenvalue $\lambda_K \asymp nr^{d + 2}K^{2/d}$ and that $d_{\min}(P_n) \asymp r^d$. Thus the bound reduces to $n^{\gamma}$ with exponent $\gamma = 3d/(2s + d) - 4/d$, which is greater than $0$ for most values of $d$. Moreover, the looseness in inequality $(ii)$ did not kill us---replacing $\lambda_K$ by $\lambda_2$ on the right hand side of $(ii)$ still results in a bound growing with $n$. On the other hand, the inequality $(i)$ was rather gross---it is tight only when $\varphi_k \propto \varphi_{\ell}$ for all $k,\ell = 1,\ldots,K$, and it seems likely that we can do better.

\paragraph{Distance between $f^{\ast}$ and subspace spanned by $\varphi_1,\ldots,\varphi_K$ $(f_{\perp}^{\ast})$.}

It seems like a useful starting place would be to take $f^{\ast} \propto \psi_k$, where $\psi_k$ is the $k$th eigenvector of the continuum weighted Laplace-Beltrami operator, and show that this is close to $\mathrm{span}\{\varphi_1,\ldots,\varphi_K\}$. I do not know an upper bound on the distance between $\psi_k$ and $\mathrm{span}\{\varphi_1,\ldots,\varphi_K\}$, but I do know such a bound for a different extension of $v_1,\ldots,v_K$. 

\paragraph{$(\|f_{\perp}^{\ast}\|_n)$.}

This one I frankly have no thoughts on how to analyze. I don't even understand it all that well, as evidenced by the fact that I can't name it.


\clearpage

\section{Prediction error for least-squares estimators: random design, fixed basis.}
In this section, we will assume that $X_1,\ldots,X_n$ are independent samples from a distribution $P$ supported over $\Omega \subseteq \Rd$, and that $\phi_1,\ldots,\phi_k$ are deterministic functions which in particular do not depend on $X_1,\ldots,X_n$.

The prediction error of $\wh{f}$ is $\mathbb{E}_{(X,Y)}[(Y - \wh{f}(X))^2]$, where $X \sim P$ and $\mathbb{E}[Y|X] = f^{\ast}(X)$ are independent of $X_1,\ldots,X_n$, and we additionally assume $Y - \mathbb{E}[Y|X]$ is independent of $X$ (i.e. homoskedastic noise). The prediction error can be decomposed into the sum of the $L^2(P)$ error and an irreducible error,
\begin{equation}
\mathbb{E}_{(X,Y)}[(Y - \wh{f}(X))^2] = \|\wh{f} - f^{\ast}\|_P^2 + \mathbb{E}_{(X,Y)}[(Y - f^{\ast}(X))^2],
\end{equation}
justifying our interest in the $\|\cdot\|_P^2$ error.

We will apply Lemma~\ref{lem:l2p_error_fixed_design}, and then use our assumptions on the design points and basis functions to control the right hand side of~\eqref{eqn:l2p_error_fixed_design}. In particular, we have the following. 
\begin{lemma}
	content...
\end{lemma}

\section{Miscellaneous}
I put things here which do not belong in any other section.

\subsection{Proof of Lemma~\ref{lem:kernel_smoothing_bias}, $d = 1$ and $p$ uniform}
\label{subsec:kernel_smoothing_bias_pf_1d_uniform_p}
Without loss of generality take the domain $\Omega = (0,1)$. Recall that we are interested in upper bounding
\begin{equation*}
\|T_{P,\varepsilon}u - u\|_{P}^2 = \int_{\Omega} \bigl(L_{\varepsilon,P}u(x)\bigr)^2 \,dP(x) \leq C\varepsilon^{2k}
\end{equation*}
where $L_{P,\varepsilon}$ is a non-local approximation to the continuum weighted Laplacian, given by
\begin{equation*}
L_{P,\varepsilon}u(x) := \frac{1}{d_P(x)}\int_{\Omega} \bigl(u(z) - u(x)\bigr) \eta\biggl(\frac{\|z - x\|}{\varepsilon}\biggr) \,dP(z).
\end{equation*}
Let $\Omega_{\varepsilon} := \{x: B(x,\varepsilon) \in \Omega\}$ and let $\partial_{\varepsilon}\Omega := \Omega \setminus \Omega_{\varepsilon}$. Clearly
\begin{equation*}
\|L_{P,\varepsilon}u\|_{P}^2 = \underbrace{\int_{\Omega_{\varepsilon}} \bigl(L_{\varepsilon,P}u(x)\bigr)^2 \,dP(x)}_{:=I_1} + \underbrace{\int_{\partial_{\varepsilon}\Omega} \bigl(L_{\varepsilon,P}u(x)\bigr)^2 \,dP(x)}_{:=I_2}
\end{equation*}
and we now separately upper bound $I_1$ and $I_2$.

\paragraph{Upper bound on $I_1$.}
We begin with a pointwise analysis of $L_{p,\varepsilon}u(x)$ at a given $x \in \Omega_{\varepsilon}$. We may assume without loss of generality that $u \in C_0^{\infty}(\Omega)$, since $C_0^{\infty}(\Omega)$ is dense in $H_0^k(\Omega)$ and $\|T_{P,\varepsilon}u - u\|_P^2$ is continuous with respect to $\|\cdot\|_{H_0^k}$. Since $u$ is smooth, for every $x,z \in \Omega$, we have the $k$th-order Taylor expansion
\begin{equation*}
u(z) - u(x) = \sum_{j = 1}^{k - 1} (z - x)^{j} u^{(j)}(x)  + \frac{(z - x)^k}{(k - 1)!}\int_{0}^{1}(1 - t)^{k - 1}u^{(k)}(x + t(z - x))\,dt.
\end{equation*}
As a result,
\begin{equation}
\label{pf:kernel_smoothing_bias_1d_1}
\begin{aligned}
L_{P,\varepsilon}u(x) = \frac{1}{d_P(x)} & \Biggl\{\sum_{j = 1}^{k - 1} u^{(j)}(x) \int_{\Omega}(z - x)^{j}\eta\biggl(\frac{\|z - x\|}{\varepsilon}\biggr)\,dz + \\ 
& \frac{1}{(k - 1)!} \int_{\Omega} \int_{0}^{1} (1 - t)^k (z - x)^k u^{(k)}(x + t(z - x)) \eta\biggl(\frac{\|z - x\|}{\varepsilon}\biggr) \,dt \,dz \Biggr\}
\end{aligned}
\end{equation} 

Now, since $x \in \Omega_{\varepsilon}$ then all $z \in B(x,\varepsilon)$ also belong to $\Omega$. For one thing, this implies
\begin{equation*}
d_P(x) = \int_{\Omega} \eta\biggl(\frac{\|z - x\|}{\varepsilon}\biggr) \,dz = \varepsilon \cdot \int_{B(0,1)}  \eta(v) \,dv = \epsilon.
\end{equation*}
Additionally, we have that all $j = 1,\ldots,k - 1$, 
\begin{equation*}
\int_{\Omega}(z - x)^{k}\eta\biggl(\frac{\|z - x\|}{\varepsilon}\biggr) \,dz = \varepsilon^j \cdot \int_{B(0,1)} v^j \eta(v) \,dv = 0,
\end{equation*}
where the second equality follows because $\eta$ is a $k$th-order kernel. Consequently $L_{P,\varepsilon}$ depends only on the $k$th derivative of $u$: changing variables to $v = (z - x)/\varepsilon$, we have
\begin{align*}
L_{P,\varepsilon}u(x) & = \frac{1}{\varepsilon} \cdot \frac{1}{(k - 1)!} \int_{0}^{1} \int_{\Omega} (1 - t)^k (z - x)^k u^{(k)}(x + t(z - x)) \eta\biggl(\frac{\|z - x\|}{\varepsilon}\biggr) \,dz \,dv \\
& = \frac{\varepsilon^{k}}{(k - 1)!} \int_{0}^{1} \int_{B(0,1)} (1 - t)^k v^k u^{(k)}(x + t\varepsilon v) \eta(|v|) \,dv \,dt
\end{align*} 
and as a result
\begin{equation*}
\Bigl(L_{P,\varepsilon}u(x)\Bigr)^2 = \frac{\varepsilon^{2k}}{(k!)^2} {\sigma_\eta}^2 \bigl(u^{(k)}(x)\bigr)^2 + \frac{\varepsilon^{2k}}{((k - 1)!)^2}A_1(x)A_{2}(x),
\end{equation*}
where $\sigma_\eta = \int v^k \eta(|v|) \,dv$ and 
\begin{align*}
A_1(x) & := \int_{0}^{1} \int_{B(0,1)} (1 - t)^k v^k \bigl(u^{(k)}(x + t\varepsilon v) - u^{(k)}(x)\bigr) \eta(|v|) \,dv \,dt \\
A_2(x) & := \int_{0}^{1} \int_{B(0,1)} (1 - t)^k v^k \bigl(u^{(k)}(x + t\varepsilon v) + u^{(k)}(x)\bigr) \eta(|v|) \,dv \,dt.
\end{align*}
Integrating over all $x \in \Omega_{\varepsilon}$, we have that
\begin{align*}
I_1 = \int_{\Omega_{\varepsilon}} \bigl(L_{P,\varepsilon}u(x)\bigr)^2 \,dx & = \frac{\varepsilon^{2k}}{(k!)^2} \sigma_\eta^2 \int_{\Omega_{\varepsilon}} \bigl(u^{(k)}(x)\bigr)^2 \,dx + \frac{\varepsilon^{2k}}{((k - 1)!)^2} \int_{\Omega_{\varepsilon}}A_1(x) A_2(x) \,dx \\
& \leq \frac{\varepsilon^{2k}}{(k!)^2} \Bigl(\sigma_\eta^2 \|u^{(k)}\|_{P}^2 + k^2 \|A_1\|_P \|A_2\|_P\Bigr).
\end{align*}
It remains to show that $\|A_1\|_P\|A_2\|_P = o(1)$.

\textcolor{red}{(TODO): Follow through on this calculation, referring to pages 192 - 195 of Tsybakov's book.}

\paragraph{Upper bound on $I_2$.}
We will prove the upper bound
\begin{equation}
\label{pf:kernel_smoothing_bias_1d_2}
\int_{0}^{\varepsilon} \bigl|L_{P,\varepsilon}u(x)\bigr|^2 \,dx \leq C \varepsilon^{2k + 1}\|u^{(k)}\|_P^2,
\end{equation}
with the bound on $\int_{1 - \varepsilon}^{1} \bigl|L_{P,\varepsilon}u\bigr|^2$ following in a completely analogous fashion.

Unlike the previous case, when $x$ is near the boundary of $\Omega$ we cannot use the higher-order property of $\eta$ to cancel the order-$1$ up through order-$(k - 1)$ derivatives of $u$. Instead, we will use the zero-trace property to show that $u$ itself must be quite small. Taking a Taylor expansion of $u$ around $0$, we have that for any $x \in \Omega$,
\begin{align}
u(x) & = u(0) + \sum_{j = 1}^{k - 1} x^j u^{(j)}(0) + \frac{1}{(k - 1)!} \int_{0}^{x}u^{(k)}(s)(x - s)^{k - 1}\,ds \nonumber \\
& = \frac{1}{(k - 1)!} \int_{0}^{x}u^{(k)}(s)(x - s)^{k - 1}\,ds. \label{pf:kernel_smoothing_bias_1d_3}
\end{align}
with the second inequality following since $u \in C_c^{\infty}(\Omega)$ implies $u(0) = 0$ as well as $u^{(j)}(0) = 0$ for $j = 1,\ldots,k - 1$. To make use of this upper bound on $u(x)$, we upper bound $|L_{P,\varepsilon}u(x)|^2$ using the Cauchy-Schwarz inequality,
\begin{align*}
|L_{P,\varepsilon}u(x)|^2 & = \Biggl(\frac{1}{d_P(x)} \int_{\Omega} \bigl(u(x) - u(z)\bigr) \eta\biggl(\frac{|x - z|}{\varepsilon}\biggr) \,dz\Biggr)^2 \\
& \leq \frac{1}{d_P(x)}\biggl(\int_{\Omega} K^{1/2}\biggl(\frac{x - z}{\varepsilon}\biggr) \,dz\biggr) \cdot \frac{1}{d_P(x)}\int_{\Omega} \bigl(u(x) - u(z)\bigr)^2 K^{1/2}\biggl(\frac{|x - z|}{\varepsilon}\biggr) \,dz \\
& \leq 2\frac{\|\eta\|_{\infty}}{\alpha^2\varepsilon} \cdot \int_{\Omega} \bigl(u(x)^2 + u(z)^2\bigr) K^{1/2}\biggl(\frac{|x - z|}{\varepsilon}\biggr) \,dz,
\end{align*}
where in the final inequality we have used the estimates $d_P(x;K^{1/2}) \leq \|\eta\|_{\infty} \varepsilon$ and $d_P(x) \geq (\varepsilon\alpha)$. 

We use these pointwise estimates to upper bound~\eqref{pf:kernel_smoothing_bias_1d_2}. Since $\eta$ is compactly supported, if $x \in (0,\varepsilon]$ and $\eta(|x - z|/\varepsilon) > 0$ then $z \in (0,2\varepsilon]$. As a result,
\begin{align*}
\int_{0}^{\varepsilon} |L_{P,\varepsilon}u(x)|^2 \,dP(x) & \leq 2\frac{\|\eta\|_{\infty}}{\alpha^2\varepsilon} \int_{0}^{\varepsilon}  \int_{\Omega} \bigl(u(x)^2 + u(z)^2\bigr) K^{1/2}\biggl(\frac{|x - z|}{\varepsilon}\biggr) \,dz \,dx \\
& \leq 2\frac{\|\eta\|_{\infty}}{\alpha^2\varepsilon} \int_{0}^{2\varepsilon}  \int_{0}^{2\varepsilon} \bigl(u(x)^2 + u(z)^2\bigr) K^{1/2}\biggl(\frac{|x - z|}{\varepsilon}\biggr) \,dz \,dx \\
& \leq 4\frac{\|\eta\|_{\infty}^2}{\alpha^2} \int_{0}^{2\varepsilon} (u(x))^2 \,dx,
\end{align*}
where the last inequality follows by Fubini's Theorem.

It remains to upper bound the $\Leb^2$ norm of $u$ over $(0,2\varepsilon]$. Here we use the bound~\eqref{pf:kernel_smoothing_bias_1d_3} and the Cauchy-Schwarz inequality to conclude that 
\begin{equation*}
\int_{0}^{2\varepsilon} \bigl(u(x)\bigr)^2 \,dx \leq \int_{0}^{2\varepsilon} \biggl(\int_{0}^{x} u^{(k)}(s)(x - s)^{k - 1} \,ds\biggr)^2 \,dx \leq \varepsilon^{2k - 1} \int_{0}^{2\varepsilon} \int_{0}^{x} \bigl(u^{(k)}(s)\bigr)^2 \,ds \,dx. 
\end{equation*}
Finally, switching the order of integration and substituting $v = s/(2\varepsilon)$ gives
\begin{align*}
\int_{0}^{2\varepsilon} \int_{0}^{x} \bigl(u^{(k)}(s)\bigr)^2 \,ds \,dx = \int_{0}^{2\varepsilon} \int_{x}^{2\varepsilon} \bigl(u^{(k)}(s)\bigr)^2 \,dx \,ds \leq 2\varepsilon \int_{0}^{2\varepsilon} \bigl(u^{(k)}(s)\bigr)^2 \,ds = 4\varepsilon^2 \|u^{(k)}\|_P^2,
\end{align*}
establishing~\eqref{pf:kernel_smoothing_bias_1d_2} upon choosing $C = \frac{16\|\eta\|_{\infty}^2}{\alpha^2}$ in the statement of the inequality.

\subsection{Proof of Lemma~\ref{lem:kernel_smoothing_bias}, $d = 1$ and $p \in C^k$.}
\label{subsec:kernel_smoothing_bias_pf_1d_smooth_p}
As in~\ref{subsec:kernel_smoothing_bias_pf_1d_uniform_p}, we take $\Omega = (0,1)$, let
\begin{equation*}
\|L_{P,\varepsilon}u\|_{P}^2 = \underbrace{\int_{\Omega_{\varepsilon}} \bigl(L_{\varepsilon,P}u(x)\bigr)^2 \,dP(x)}_{:=I_1} + \underbrace{\int_{\partial_{\varepsilon}\Omega} \bigl(L_{\varepsilon,P}u(x)\bigr)^2 \,dP(x)}_{:=I_2},
\end{equation*}
and turn to separately upper bounding $I_1$ and $I_2$.

\paragraph{Upper bound on $I_1$.}
We will show that for all $x \in \Omega_{\varepsilon}$,
\begin{equation*}
L_{P,\varepsilon}u(x) = \varepsilon^{k} \frac{\sigma_\eta}{k!}\Bigl( \frac{d^k}{dx^k}(up)(x) - p^{(k)}(x)u(x) + r(x)\Bigr)
\end{equation*}
for a remainder term $r$ satisfying $\int_{\Omega_{\varepsilon}} \bigl(r(x)\bigr)^2 p(x) \,dx  = o(1)$ as $\varepsilon \to 0$. This in turn will imply
\begin{align*}
I_1 = \int_{\Omega_{\varepsilon}} |L_{P,\varepsilon}u(x)|^2 \,dx & = \varepsilon^{2k} \frac{\sigma_\eta^2}{(k!)^2}\Bigl(\int_{\Omega_{\varepsilon}} \Bigl( \frac{d^k}{dx^k}(up)(x) - p^{(k)}(x)u(x) + r(x)\Bigr)^2  p(x) \,dx \\
& \leq 3 \varepsilon^{2k} \frac{\sigma_\eta^2}{(k!)^2} \bigl( |up|_{H^k(P)}^2 + |p^{(k)}u|_{\Leb^2(P)}^2 + o(1)\bigr) \\
& \leq C \varepsilon^{2k} \frac{\sigma_\eta^2}{(k!)^2} \Bigl( \|p\|_{C^k(P)}^2\bigl(|u|_{H^k(P)}^2 + |u|_{\Leb^2(P)}^2\bigr) + o(1) \Bigr),
\end{align*}
with the last line following from Young's inequality.

It remains to show the pointwise estimate \textcolor{red}{(...)}.

\end{document}