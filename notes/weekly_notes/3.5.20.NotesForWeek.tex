\documentclass{article}
\usepackage{amsmath}
\usepackage{amsfonts, amsthm, amssymb}
\usepackage{graphicx}
\usepackage[colorlinks]{hyperref}
\usepackage[parfill]{parskip}
\usepackage{enumerate}
\usepackage[shortlabels]{enumitem}
\usepackage{fullpage}
\usepackage{mathtools}

\usepackage{natbib}
\renewcommand{\bibname}{REFERENCES}
\renewcommand{\bibsection}{\subsubsection*{\bibname}}

\DeclareFontFamily{U}{mathx}{\hyphenchar\font45}
\DeclareFontShape{U}{mathx}{m}{n}{<-> mathx10}{}
\DeclareSymbolFont{mathx}{U}{mathx}{m}{n}
\DeclareMathAccent{\wb}{0}{mathx}{"73}

\DeclarePairedDelimiterX{\norm}[1]{\lVert}{\rVert}{#1}

\makeatletter
\newcommand{\vast}{\bBigg@{4}}
\newcommand{\Vast}{\bBigg@{5}}
\makeatother

\newcommand{\eqdist}{\ensuremath{\stackrel{d}{=}}}
\newcommand{\Graph}{\mathcal{G}}
\newcommand{\Reals}{\mathbb{R}}
\newcommand{\Identity}{\mathbb{I}}
\newcommand{\Xsetistiid}{\overset{\text{i.i.d}}{\sim}}
\newcommand{\convprob}{\overset{p}{\to}}
\newcommand{\convdist}{\overset{w}{\to}}
\newcommand{\Expect}[1]{\mathbb{E}\left[ #1 \right]}
\newcommand{\Risk}[2][P]{\mathcal{R}_{#1}\left[ #2 \right]}
\newcommand{\Prob}[1]{\mathbb{P}\left( #1 \right)}
\newcommand{\iset}{\mathbf{i}}
\newcommand{\jset}{\mathbf{j}}
\newcommand{\myexp}[1]{\exp \{ #1 \}}
\newcommand{\abs}[1]{\left \lvert #1 \right \rvert}
\newcommand{\restr}[2]{\ensuremath{\left.#1\right|_{#2}}}
\newcommand{\ext}[1]{\widetilde{#1}}
\newcommand{\set}[1]{\left\{#1\right\}}
\newcommand{\seq}[1]{\set{#1}_{n \in \N}}
\newcommand{\Xsetotp}[2]{\langle #1, #2 \rangle}
\newcommand{\floor}[1]{\left\lfloor #1 \right\rfloor}
\newcommand{\Var}{\mathrm{Var}}
\newcommand{\Cov}{\mathrm{Cov}}
\newcommand{\Xsetiam}{\mathrm{diam}}

\newcommand{\emC}{C_n}
\newcommand{\emCpr}{C'_n}
\newcommand{\emCthick}{C^{\sigma}_n}
\newcommand{\emCprthick}{C'^{\sigma}_n}
\newcommand{\emS}{S^{\sigma}_n}
\newcommand{\estC}{\widehat{C}_n}
\newcommand{\hC}{\hat{C^{\sigma}_n}}
\newcommand{\vol}{\text{vol}}
\newcommand{\spansp}{\mathrm{span}~}
\newcommand{\1}{\mathbf{1}}

\newcommand{\Linv}{L^{\Xsetagger}}
\DeclareMathOperator*{\argmin}{argmin}
\DeclareMathOperator*{\argmax}{argmax}

\newcommand{\emF}{\mathbb{F}_n}
\newcommand{\emG}{\mathbb{G}_n}
\newcommand{\emP}{\mathbb{P}_n}
\newcommand{\F}{\mathcal{F}}
\newcommand{\D}{\mathcal{D}}
\newcommand{\R}{\mathcal{R}}
\newcommand{\Rd}{\Reals^d}
\newcommand{\Nbb}{\mathbb{N}}

%%% Vectors
\newcommand{\thetast}{\theta^{\star}}
\newcommand{\betap}{\beta^{(p)}}
\newcommand{\betaq}{\beta^{(q)}}
\newcommand{\vardeltapq}{\varDelta^{(p,q)}}


%%% Matrices
\newcommand{\X}{X} % no bold
\newcommand{\Y}{Y} % no bold
\newcommand{\Z}{Z} % no bold
\newcommand{\Lgrid}{L_{\grid}}
\newcommand{\Xsetgrid}{D_{\grid}}
\newcommand{\Linvgrid}{L_{\grid}^{\Xsetagger}}

%%% Sets and classes
\newcommand{\Xset}{\mathcal{X}}
\newcommand{\Sset}{\mathcal{S}}
\newcommand{\Hclass}{\mathcal{H}}
\newcommand{\Pclass}{\mathcal{P}}
\newcommand{\Leb}{\mathcal{L}}

%%% Distributions and related quantities
\newcommand{\Pbb}{\mathbb{P}}
\newcommand{\Ebb}{\mathbb{E}}
\newcommand{\Qbb}{\mathbb{Q}}

%%% Operators
\newcommand{\Tadj}{T^{\star}}
\newcommand{\Xsetive}{\mathrm{div}}
\newcommand{\Xsetif}{\mathop{}\!\mathrm{d}}
\newcommand{\gradient}{\mathcal{D}}
\newcommand{\Hessian}{\mathcal{D}^2}
\newcommand{\dotp}[2]{\langle #1, #2 \rangle}

%%% Misc
\newcommand{\grid}{\mathrm{grid}}
\newcommand{\critr}{R_n}
\newcommand{\Xsetx}{\,dx}
\newcommand{\Xsety}{\,dy}
\newcommand{\Xsetr}{\,dr}
\newcommand{\Xsetxpr}{\,dx'}
\newcommand{\Xsetypr}{\,dy'}
\newcommand{\wt}[1]{\widetilde{#1}}
\newcommand{\ol}[1]{\overline{#1}}
\newcommand{\spec}{\mathrm{spec}}

%%% Order of magnitude
\newcommand{\soom}{\sim}



\newtheoremstyle{alden}
{6pt} % Space above
{6pt} % Space below
{} % Body font
{} % Indent amount
{\bfseries} % Theorem head font
{.} % Punctuation after theorem head
{.5em} % Space after theorem head
{} % Theorem head spec (can be left empty, meaning `normal')

\theoremstyle{alden} 


\newtheoremstyle{aldenthm}
{6pt} % Space above
{6pt} % Space below
{\itshape} % Body font
{} % Indent amount
{\bfseries} % Theorem head font
{.} % Punctuation after theorem head
{.5em} % Space after theorem head
{} % Theorem head spec (can be left empty, meaning `normal')

\theoremstyle{aldenthm}
\newtheorem{theorem}{Theorem}
\newtheorem{conjecture}{Conjecture}
\newtheorem{lemma}{Lemma}
\newtheorem{example}{Example}
\newtheorem{corollary}{Corollary}
\newtheorem{proposition}{Proposition}
\newtheorem{assumption}{Assumption}
\newtheorem{remark}{Remark}


\theoremstyle{definition}
\newtheorem{definition}{Definition}[section]

\theoremstyle{remark}

\begin{document}
\title{Notes for Week 2/28/20 - 3/5/20}
\author{Alden Green}
\date{\today}
\maketitle

Let $N$ be an integer, $n = N^d$, and 
\begin{equation*}
\wb{X} := \Bigl\{\frac{1}{N}(j_1,\ldots,j_d): j \in [N]^d\Bigr\}
\end{equation*}
consist of $n$ total evenly spaced grid points on $[0,1]^d$. We observe $n$ samples according to regression model
\begin{equation}
\label{eqn:grid_regression_model}
y_j = f(\wb{x}_j) + \varepsilon_j, ~\varepsilon_j \overset{\mathrm{i.i.d}}{\sim} \mathcal{N}(0,1) ~~\textrm{for each $j \in [N]^d$,}
\end{equation}
and our goal is to perform the goodness-of-fit testing
\begin{equation*}
\mathbb{H}_0 : f = f_0 := 0 ~~\textrm{vs.}~~ \mathbb{H}_a: f \neq f_0.
\end{equation*}

For an arbitrary graph $G$ on $n$ nodes with graph Laplacian $L_G = D_G - A_G$, let $\Lambda(G)$ consist of the eigenvalues $0 = \lambda_1(G) \leq \lambda_2(G) \leq \cdots \leq \lambda_n(G)$ of $L_G$, with $v_k(G) = (v_{k,1}(G),\ldots,v_{k,n}(G)) \in \Reals^n$ denoting the eigenvector corresponding to the $i$th eigenvalues $\lambda_k(G)$. The graph Laplacian eigenvector projection test is a linear projection test; letting $\kappa$ be some integer between $1$ and $n$, we define $\Pi_{k,G}(\theta) := \frac{1}{\sqrt{n}}\sum_{k = 1}^{\kappa} \dotp{v_k}{\theta}$ to be a suitable rescaling of the projection of $\theta$ onto the span of $v_1(G),\ldots,v_k(G)$. 

Our test statistic will be
\begin{equation*}
T_{\mathrm{spec}}(G) := \norm{\Pi_{\kappa,G}(Y)}_n^2 - \frac{\kappa}{n},
\end{equation*}
a simple function of the projection of the data $Y$ onto the span of the first $\kappa$ eigenvectors of the graph Laplacian $L(G)$; we will suitably choose $G$ based on the assumptions we make on $f$.

\section{Sobolev testing with fixed design.}
For now, we confine our attention to the periodic Sobolev space $H_{\textrm{per}}^{s}([0,1]^d)$ which consists of all those functions $f \in \Leb^2([0,1]^d)$ whose tensor product Fourier coefficients decay at a sufficiently fast rate. When $d = 1$ and $x \in [0,1]$, let
\begin{align*}
\phi_1(x) & = 1 \\
\phi_{2k}(x) & = \cos(2k\pi x),~~ k \in \mathbb{N} \\
\phi_{2k + 1}(x) & = \sin(2k\pi x),~~ k \in \mathbb{N}
\end{align*}
be the trigonometric basis over $[0,1]$. The tensor product Fourier basis is defined on $[0,1]^d$ as
\begin{equation*}
\phi_k(x) = \prod_{i = 1}^{d} \phi_{k_i}(x_i),
\end{equation*}
the Sobolev space $H_{\textrm{per}}^{s}([0,1]^d)$ as 
\begin{equation*}
H^{\textrm{per}}([0,1]^d) = \set{\sum_{k \in \mathbb{N}^d} \theta_k \phi_k^{(d)}: \sum_{k \in \mathbb{N}^d} \theta_k^2 \abs{k}^{2s} < \infty},
\end{equation*}
and the unit ball $H_{\textrm{per}}^{s}([0,1]^d, L)$ as 
\begin{equation*}
H^{\textrm{per}}([0,1]^d;L) = \set{\sum_{k \in \mathbb{N}^d} \theta_k \phi_k^{(d)}: \sum_{k \in \mathbb{N}^d} \theta_k^2 \abs{k}^{2s} \leq L},
\end{equation*}

Intuitively, a good choice of graph $G$ for our test statistic $T_{\spec}(G)$ will be one which keeps the approximation error
\begin{equation*}
\norm{f}_{\Leb^2}^2 - \Ebb\Bigl[\norm{\Pi_{\kappa,G}(Y)}_{n}^2\Bigr] =  \norm{f}_{\Leb^2}^2 - \norm{\Pi_{\kappa,G}(f)}_{n}^2
\end{equation*}
small when $\kappa$ is not too large.

\subsection{Torus Graph}

We will see that the torus graph $\wt{G}_d$ fulfills this purpose. The $1$-dimensional torus (ring) graph is defined as
\begin{equation*}
\wt{G}_1 = (\wb{X}; \wt{E}),~~ \wt{E} = \Bigl\{(\wb{x}_i,\wb{x}_j): i \in [n], j = (i + 1) \mod n\Bigr\};
\end{equation*}
when $d > 1$, we define $\wt{G}_d = \wt{G}_1 \otimes \cdot \wt{G}_1$ as the tensor product of $\wt{G}_1$ iterated $d$ times.

\begin{lemma}
	\label{lem:grid_sobolev_approximation_error}
	Let $K := \kappa^{1/d}$. For any $2s > d$ and $f \in H_{\textrm{per}}^s([0,1]^d;L)$, 
	\begin{equation}
	\label{eqn:grid_sobolev_approximation_error}
	\norm{f}_{\Leb^2}^2 - \norm{\Pi_{\kappa,G}(f)}_{n}^2 \leq c n^{1/2 - s/d} \kappa^{1/2} \norm{f}_2 + \sum_{k \in \mathbb{N}^d - [K]^d} \theta_k^2
	\end{equation}
	for a constant $c$ which depends only on $L$ and $d$. 
\end{lemma}

It is possible to improve on Lemma~\ref{lem:grid_sobolev_approximation_error} under \textcolor{red}{(relatively mild)} additional assumptions. Specifically, we assume the existence of $\varepsilon_j$ satisfying $j \varepsilon_j$ is monotone non-increasing in $j$, such that for every $f \in H_{\textrm{per}}^{s}([0,1]^d)$, $f = \sum_{k \in \mathbb{N}^d} \theta_k \phi_k$,
\begin{equation}
\label{asmp:fourier_coefficient_decay}
\abs{\theta_k} \leq \prod_{i = 1}^{d} \varepsilon_{k_i}~~\textrm{for all $k \in \mathbb{N}^d$},~~ \sum_{j = 1}^{\infty} \varepsilon_j = C,~~ \sum_{j = 1}^{\infty} \varepsilon_j^2 \leq C \biggl(\sum_{k \in \mathbb{N}^d} \theta_k^2\biggr)^{1/d}
\end{equation}
for some constant $C < \infty$. 

\textcolor{red}{(TODO:)} {Unfortunately these additional assumptions are not mild at all, but instead relatively close to assuming the function is $s$-Holder.}

\begin{lemma}
	\label{lem:grid_sobolev_approximation_error_2}
	Suppose that~\eqref{asmp:fourier_coefficient_decay} holds with respect to $H_{\textrm{per}}^{s}([0,1]^d)$. Then for every $f \in H_{\textrm{per}}^{s}([0,1]^d)$,
	\begin{equation}
	\label{eqn:grid_sobolev_approximation_error_2}
	\norm{f}_{\Leb^2}^2 - \norm{\Pi_{\kappa,G}(f)}_{n}^2 \leq c \norm{f}_{\Leb^2}^2 \Biggl( \frac{K^{1/2}}{N \norm{f}_{\Leb^2}^{1/d}} + \frac{K^{d/2}}{n\norm{f}_{\Leb^2}}\Biggr) + \sum_{k \in \mathbb{N}^d - [K]^d} \theta_k^2
	\end{equation}
\end{lemma}
When we choose $\kappa = n^{2d/(4s + d)}$, and additionally $\norm{f}_{\Leb^2} \geq n^{-2s/(4s + d)}$, we note that 
\begin{equation*}
\frac{K^{1/2}}{N \norm{f}_{\Leb^2}^{1/d}} + \frac{K^{d/2}}{n\norm{f}_{\Leb^2}} = o(1)
\end{equation*}
and therefore the discretization error is asymptotically negligible.

Now we will attempt to find a sharper bound on discretization error than Lemma~\ref{lem:grid_sobolev_approximation_error}, but that requires less severe assumptions than Lemma~\ref{lem:grid_sobolev_approximation_error_2}.
\begin{lemma}
	\label{lem:grid_sobolev_approximation_error_3}
	For any $2s > d$ and $f \in H_{\textrm{per}}^s([0,1]^d;L)$, 
	\begin{equation*}
	\norm{f}_{\Leb^2}^2 - \norm{\Pi_{\kappa,G}(f)}_{n}^2 \leq c n^{1/2 - s/d} \norm{f}_2 + \sum_{k \in \mathbb{N}^d - [K]^d} \theta_k^2
	\end{equation*}
\end{lemma}

\subsection{Grid graph}
In this section we will find it convenient to assume the grid points are symmetric in $[0,1]^d$, so we now treat
\begin{equation*}
\wb{X} := \set{\frac{1}{2N}(2j_1 - 1,\ldots,2j_d - 1): j \in [N]^d}.
\end{equation*}

The grid graph $\wb{G}_d = \wb{G} \otimes \cdots \otimes \wb{G}$, where
\begin{equation*}
\wb{G} = (\wb{X}; \wb{E}),~~ \wb{E} = \Bigl\{(\wb{x}_i,\wb{x}_j): i \in [n - 1], j = (i + 1)\Bigr\}; 
\end{equation*}
is the path (i.e. the torus with the edge connecting $1$ to $n$ removed.) The eigenvectors of the path can be written as follows:
\begin{equation*}
v_{k,i}(\wb{G}) = \cos\Bigl(\frac{\pi (k - 1) (i - 1/2)}{n}\Bigr), \textrm{for $i = 1,\ldots,n$ and $k = 1,\ldots,n - 1$}
\end{equation*}
with the usual consequence for the eigenvectors of the grid:
\begin{equation*}
v_{k,i}(\wb{G}_d) = \prod_{j = 1}^d v_{k_j,i_j}(\wb{G}),~~ \textrm{for $i \in [N]^d$, $j \in [N]^d$.}
\end{equation*}
We extend these eigenvectors to an orthonormal system over $\Leb^2([0,1]^d)$ as follows. Let
\begin{equation*}
\varphi_k(x) = \sqrt{2}\cos\biggl(\pi (k - 1) x\biggr),~~\textrm{for $k \in \Nbb, x \in [0,1]$},~~\textrm{and}~~ \varphi_k(x) = \prod_{j = 1}^{d} \varphi_{k_j}(x_j),~~\textrm{for $k \in \Nbb^d, x \in [0,1]^d$},
\end{equation*}
and note that by shifting the location of the design points $\wb{X}$, we have that $\varphi_k(\wb{x}_i) = \sqrt{n} v_{k,i}(\wb{G}_d)$ for all $k,i \in [N]^d$.

We let the space 
\begin{equation*}
\wt{H}^{s}([0,1]^d;L) = \Bigl\{ \sum_{k \in \mathbb{N}^d} \theta_k \varphi_k: \sum_{k \in \mathbb{N}^d} \theta_k^2 a_k^{2} < L \Bigr\}
\end{equation*}
where in the one-dimensional case,
\begin{equation*}
a_{k - 1} =
\begin{cases*}
(k - 1)^s,~~ \textrm{if $k - 1$ is even} \\
(k - 2)^s,~~ \textrm{if $k - 1$ is odd}
\end{cases*}
\end{equation*}
and for general $d$, $a_k = \sqrt{\sum_{j = 1}^{d} a_{k_j}^2}$. It is not hard to show that $\wt{H}^s([0,1]^d)$ consists of those functions in $H^s([0,1]^d)$ which satisfy some boundary conditions
\begin{equation*}
\wt{H}^s([0,1]^d;L) = \Bigl\{f \in H^{s}([0,1]^d;L'): \forall x_0 \in \partial [0,1]^d, \frac{\partial^{\ell} f}{\partial \mathbf{n}^{\ell}}(x_0) = 0,~~\textrm{for $\ell = 1,\ldots,s-1$} \Bigr \} \supseteq H_0^s([0,1]^d;L')
\end{equation*}
for a constant $L'$ which depends only on $L$ and $s'$; we work through the details for the case $d = 1$ in Section~\ref{subsec:sobolev_class_equivalence}.

For functions $f \in \wt{H}^{s}([0,1]^d;L)$, the error in approximating the continuum norm $\norm{f}_{\Leb^2}$ by the discrete norm of a projection of $f$ onto the eigenvectors of the grid is not too large. 

\begin{lemma}
	\label{lem:grid_sobolev_approximation_error_4}
	Suppose $2s < d$ and $f = \sum_{k \in \mathbb{N}^k} \theta_k \varphi_k$ satisfies $f \in \wt{H}_{\textrm{per}}^{s}([0,1]^d;L)$. Then
	\begin{equation*}
	\norm{f}_{\Leb^2}^2 - \norm{\Pi_{\kappa,\wb{G}_d}(f)}_{n}^2 \leq c n^{1/2 - s/d} \norm{f}_2 + \sum_{k \in \mathbb{N}^d - [K]^d} \theta_k^2
	\end{equation*}
\end{lemma}

\section{Proofs}
\subsection{Proof of Lemma~\ref{lem:grid_sobolev_approximation_error}}
Recall that the eigenvectors of the torus graph $v_k := v_k(\wt{G}_d)$ satisfy
\begin{equation*}
v_{k}(\wb{x}_i) = \frac{1}{\sqrt{n}} \phi_{k}(\wb{x}_i)
\end{equation*}
for each $i \in [N]^d$ and $k \in \mathbb{N}^d$. Rewriting $\Pi_{\kappa,G}$ as a function of the tensor product Fourier basis, we obtain
\begin{align*}
\Pi_{\kappa,G}(f) & = \frac{1}{n} \sum_{k \in [K]^d} \Bigl\{\sum_{i \in [N]^d} f(\wb{x}_i) \phi_k(\wb{x}_i)\Bigr\} \phi_k \\
& :=  \sum_{k \in [K]^d} \wt{\theta}_k \phi_k
\end{align*}
where the right hand side is interpreted as a function over $\wb{X}$. Since $\phi_k$ are $L_2(\wb{X})$ orthonormal, we obtain
\begin{align*}
\norm{\Pi_{\kappa,G}(f)}_n^2 & = \sum_{k \in [K]^d} \wt{\theta}_k^2 \\ 
& = \sum_{k \in [K]^d} {\theta}_k^2 + \sum_{k \in [K]^d} \wt{\theta}_k^2 - \theta_k^2 \\
& = \norm{f}_{\Leb^2([0,1]^d)}^2 - \sum_{k \in \mathbb{N}^d - [K]^d} \theta_k^2 + \sum_{k \in [K]^d} \wt{\theta}_k^2 - \theta_k^2
\end{align*}
and rearranging, we have
\begin{equation*}
\norm{f}_{\Leb^2([0,1]^d)}^2 - \norm{\Pi_{\kappa,G}(f)}_n^2 = \sum_{k \in [K]^d} \theta_k^2 - \wt{\theta}_k^2 + \sum_{k \in \mathbb{N}^d - [K]^d} \theta_k^2
\end{equation*}
The second term in the above equation corresponds to the second term in~\eqref{eqn:grid_sobolev_approximation_error}, and so we focus our attention on the first term, which represents discretization error. Applying the parallelogram law and the Cauchy-Schwarz inequality gives
\begin{equation}
\label{eqn:grid_sobolev_approximation_error_pf2}
\begin{aligned}
\sum_{k \in [K]^d} \theta_k^2 - \wt{\theta}_k^2 & \leq \sum_{k \in [K]^d} \max\Bigl\{(\theta_k + \wt{\theta}_k)(\theta_k - \wt{\theta}_k),0\Bigr\} \\
& \leq \left(\sum_{k \in [K]^d} \max\Bigl\{(\theta_k + \wt{\theta}_k)^2,0\Bigr\}\right)^{1/2} \left(\sum_{k \in [K]^d} (\theta_k - \wt{\theta}_k)^2\right)^{1/2} \\
& \leq \sqrt{2} \norm{f}_{\Leb^2([0,1]^d)} \left(\sum_{k \in [K]^d} (\theta_k - \wt{\theta}_k)^2\right)^{1/2}
\end{aligned}
\end{equation}

We focus on the differences $\theta_k - \wt{\theta}_k$, and using the same steps as \textcolor{red}{(Tsybakov)} Lemma 1.8 we obtain
\begin{equation}
\label{eqn:grid_sobolev_approximation_error_pf1}
\max_{k \in [K]^d}\abs{\theta_k - \wt{\theta}_k} \leq 2 L^{1/2} \left(\sum_{j \in \mathbb{N}^d - [N]^d} \abs{j}^{-2s}\right)^{1/2}
\end{equation}
so long as $K^d \leq n - 1$. Upper bounding the Riemann sum in~\eqref{eqn:grid_sobolev_approximation_error_pf1} by a corresponding integral, and converting to spherical coordinates, we have
\begin{align*}
\sum_{j \in \mathbb{N}^d - [N]^d} \abs{j}^{-2s} & \leq d \sum_{j \in \mathbb{N}^d: j_1 \geq N + 1} \abs{j}^{-2s} \\
& \leq d \int_{\Reals^d \setminus B(0,N+1)} \abs{x}^{-2s} \,dx \\
& \leq c \int_{N + 1}^{\infty} r^{-2s} r^{d - 1} \,dr \\
& \leq c N^{d - 2s} = c n^{1 - 2s/d}
\end{align*}
where the last inequality holds since $2s > d$, and $c$ is a constant which depends only on $d$. By plugging back in to~\eqref{eqn:grid_sobolev_approximation_error_pf1} we obtain
\begin{equation*}
\max_{k \in [K]^d}\abs{\theta_k - \wt{\theta}_k} \leq   c n^{1/2 - s/d}
\end{equation*}
and the claim then follows from~\eqref{eqn:grid_sobolev_approximation_error_pf2}.

\subsection{Proof of Lemma~\ref{lem:grid_sobolev_approximation_error_2}}
As shown in the proof of Lemma~\ref{lem:grid_sobolev_approximation_error}, 
\begin{equation}
\norm{f}_{\Leb^2}^2 - \norm{\Pi_{\kappa,G}(f)}_n^2 \leq \sqrt{2} \norm{f}_{\Leb^2([0,1]^d)} \left(\sum_{k \in [K]^d} (\theta_k - \wt{\theta}_k)^2\right)^{1/2} + \sum_{k \in \mathbb{N}^d - [K]^d} \theta_k^2 \label{eqn:grid_sobolev_approximation_error_2_pf2}
\end{equation}
We rewrite the discrete Fourier coefficients $\wt{\theta}_k$ as a weighted sum of continuum Fourier coefficients $\theta_k$,
\begin{align}
\wt{\theta}_k - \theta_k & = \frac{1}{n} \sum_{i \in [N]^d} \biggl\{\sum_{l \in \Nbb^d} \theta_l \phi_{l}(\wb{x}_i)\biggr\} \phi_k(\wb{x}_i)  - \theta_k \nonumber \\
& = \sum_{l \in \Nbb^d}  \theta_l \biggl\{\frac{1}{n} \sum_{i \in [N]^d} \phi_{l}(\wb{x}_i) \phi_k(\wb{x}_i)\biggr\}  - \theta_k \label{eqn:grid_sobolev_approximation_error_2_pf1}
\end{align}
Using the tensor decomposition of $\phi_k$, we obtain
\begin{align*}
\frac{1}{n} \sum_{i \in [N]^d} \phi_{l}(\wb{x}_i) \phi_k(\wb{x}_i) & = \frac{1}{n} \sum_{i \in [N]^d} \biggl\{\prod_{j = 1}^{d} \phi_{l_j}(\wb{x}_{ij}) \phi_{k_j}(\wb{x}_{ij}) \biggr\} \\
& = \frac{1}{n} \prod_{j = 1}^{d} \biggl\{\sum_{i_j = 1}^{N}  \phi_{l_j}(\wb{x}_{ij}) \phi_{k_j}(\wb{x}_{ij})  \biggr\} \\
& = \prod_{j = 1}^{d} \Bigl(\delta_{l_j k_j} + \Delta_{l_j k_j}^{(N)}\Bigr)
\end{align*}
where the last line follows from Lemma~\ref{lem:polyak90} (a restatement of results of \textcolor{red}{(Polyak 90)}) and $\Delta_{l_jk_j}$ is as in that Lemma. Inserting this into~\eqref{eqn:grid_sobolev_approximation_error_pf2} gives
\begin{align*}
\abs{\wt{\theta}_k - \theta_k} & = \abs{\sum_{l \in \Nbb^d}  \theta_l \prod_{j = 1}^{d}\Bigl( \delta_{l_j k_j} + \Delta_{l_jk_j}^{(N)}\Bigr)  - \theta_k} \\
& \overset{(i)}{=} \abs{\sum_{l \in \Nbb^d \setminus [N]^d}  \theta_l \prod_{j = 1}^{d}\Bigl( \delta_{l_j k_j} + \Delta_{l_jk_j}^{(N)}\Bigr)} \\
& \overset{(ii)}{\leq} \sum_{l \in \Nbb^d \setminus [N]^d} \prod_{j = 1}^{d} \varepsilon_{l_j}\Bigl(\delta_{l_j k_j} + \abs{\Delta_{l_jk_j}^{(N)}}\Bigr) \\
& \overset{(iii)}{=} \sum_{b \in \{0,1\}^d: \abs{b} \geq 1}\prod_{j: b_j = 0} \varepsilon_{k_j}\prod_{j: b_j = 1} \biggl(\sum_{m = N + 1}^{\infty} \varepsilon_{m} \abs{\Delta_{mk_j}^{(N)}}\biggr) \\
& \overset{(iv)}{\leq} c \sum_{b \in \{0,1\}^d : \abs{b} \geq 1} N^{-\abs{b}} \prod_{j: b_j = 0} \varepsilon_{k_j} 
\end{align*}
where $(i)$ and $(iv)$ follow from Lemma~\ref{lem:polyak90}, $(ii)$ follows from the condition~\eqref{asmp:fourier_coefficient_decay}, and $(iii)$ follows since for each index $l_j$ at most one of $\delta_{l_j k_j}$ or $\Delta_{l_j k_j}^{(N)}$ can be non-zero, and $\delta_{l_j k_j}$ must equal $0$ for at least one $j = 1,\ldots,d$. 

We can now upper bound the sum of squared differences between discrete Fourier coefficients and Fourier coefficients,
\begin{align*}
\sum_{k \in [K]^d} (\wt{\theta}_k - \theta_k)^2 & \leq c \sum_{k \in [K]^d} \biggl(\sum_{b \in \{0,1\}^d : \abs{b} \geq 1} N^{-\abs{b}} \prod_{j: b_j = 0} \varepsilon_{k_j} \biggr) \\
& \leq c \sum_{k \in [K]^d} \sum_{b \in \{0,1\}^d : \abs{b} \geq 1} N^{-2\abs{b}} \prod_{j: b_j = 0} \varepsilon_{k_j}^2 \\
& \leq c \sum_{b \in \{0,1\}^d : \abs{b} \geq 1} N^{-2\abs{b}} \sum_{k \in [K]^d} \prod_{j: b_j = 0} \varepsilon_{k_j}^2 \\
& \overset{(v)}{\leq} c \sum_{b \in \{0,1\}^d : \abs{b} \geq 1} N^{-2\abs{b}} \norm{f}_{\Leb^2}^{2(d - \abs{b})/d} K^{\abs{b}} \\
& \overset{(vi)}{\leq} c \norm{f}_{\Leb^2}^2 \Bigl( \frac{K}{N^2 \norm{f}_{\Leb^2}^{2/d}} + \frac{K^d}{N^{2d} \norm{f}_{\Leb^2}^{2}} \Bigr)
\end{align*}
where $(v)$ follows from condition~\eqref{asmp:fourier_coefficient_decay}, $(vi)$ since the summand achieves its maximum either when $b = 1$ or $b = d$, and in the above expression $c \geq 0$ is a constant which changes from line to line. Plugging back into~\eqref{eqn:grid_sobolev_approximation_error_pf2} implies Lemma~\ref{lem:grid_sobolev_approximation_error_2}.

\subsection{Proof of Lemma~\ref{lem:grid_sobolev_approximation_error_3}}
As shown in the proof of Lemma~\ref{lem:grid_sobolev_approximation_error}, 
\begin{equation}
\norm{f}_{\Leb^2}^2 - \norm{\Pi_{\kappa,G}(f)}_n^2 \leq \sqrt{2} \norm{f}_{\Leb^2([0,1]^d)} \left(\sum_{k \in [K]^d} (\theta_k - \wt{\theta}_k)^2\right)^{1/2} + \sum_{k \in \mathbb{N}^d - [K]^d} \theta_k^2 \label{eqn:grid_sobolev_approximation_error_3_pf1}
\end{equation}

By Lemma~\ref{lem:polyak90}, we have
\begin{align*}
\abs{\wt{\theta}_k - \theta_k} & = \abs{\sum_{l \in \Nbb^d}  \theta_l \prod_{j = 1}^{d}\Bigl( \delta_{l_j k_j} + \Delta_{l_jk_j}^{(N)}\Bigr)  - \theta_k} \\
& = \abs{\sum_{l \in \Nbb^d \setminus [N]^d}  \theta_l \prod_{j = 1}^{d}\Bigl( \delta_{l_j k_j} + \Delta_{l_jk_j}^{(N)}\Bigr)} \\
& = \sum_{b \in \{0,1\}^d: \abs{b} < N} \sum_{l \in I_b(k)} \abs{\theta_l} \prod_{j: b_j = 1}^{d} \abs{\Delta_{l_jk_j}^{(N)}}
\end{align*}
and therefore, upper bounding $l_2$ norm by $l_1$ norm and exchanging sums, we have
\begin{align*}
\biggl(\sum_{k \in [K]^d} (\wt{\theta}_k - \theta_k)^2\biggr)^{1/2} & \leq \sum_{k \in [K]^d} \abs{\wt{\theta}_k - \theta_k} \\
& \leq \sum_{b \in \{0,1\}^d: \abs{b} < N} \sum_{k \in [K]^d}  \sum_{l \in I_b(k)} \abs{\theta_l} \prod_{j: b_j = 1}^{d} \abs{\Delta_{l_jk_j}^{(N)}}.
\end{align*}
In Lemma~\ref{lem:double_counting}, we establish that the following upper bound holds for each $b \in \{0,1\}^d$ such that $\abs{b} < N$:
\begin{equation*}
\sum_{k \in [K]^d}  \sum_{l \in I_b(k)} \abs{\theta_l} \prod_{j: b_j = 1}^{d} \abs{\Delta_{l_jk_j}^{(N)}} \leq 2^d \sum_{l \in \Nbb^d - [N]^d} \abs{\theta_l}
\end{equation*}
and therefore,
\begin{align*}
\biggl(\sum_{k \in [K]^d} (\wt{\theta}_k - \theta_k)^2\biggr)^{1/2} & \leq 4^d \sum_{l \in \mathbb{N}^d - [N]^d} \abs{\theta_l} \\
& \leq 4^d \biggl(\sum_{l \in \mathbb{N}^d - [N]^d} \abs{\theta_l}^2 \abs{l}^{2s}\biggr)^{1/2} \biggl(\sum_{l \in \mathbb{N}^d - [N]^d} \abs{l}^{-2s}\biggr)^{1/2}  \\
& \leq 4^d L^{1/2} \biggl(\sum_{l \in \mathbb{N}^d - [N]^d} \abs{l}^{-2s}\biggr)^{1/2}.
\end{align*}
Note that this upper bound~\eqref{eqn:grid_sobolev_approximation_error_pf1}, which upper bounds the maximum over $k \in [K]^d$ rather than the sum over all $k \in [K]^d$ by the same term, and therefore incurs an additional loss of $K^{d/2}$. Continuing as in the proof of Lemma~\ref{eqn:grid_sobolev_approximation_error} yields the claimed result.

\subsection{Proof of Lemma~\ref{lem:grid_sobolev_approximation_error_4}}
This proof follows the along the same lines as the proof of Lemma~\ref{lem:grid_sobolev_approximation_error_3}, replacing functions $\phi$ in the Fourier basis by functions $\varphi$ in the modified Fourier basis when necessary.

\section{Supporting Theory}
\begin{lemma}[Restatement and Extension of Lemma 1 of \textcolor{red}{(Polyak 90)}]
	\label{lem:polyak90}
	\begin{equation*}
	\frac{1}{n}\sum_{i = 1}^{n} \phi_{k}(\wb{x}_i)\phi_{l}(\wb{x}_i) = \delta_{kl} + \Delta_{kl}^{(n)}
	\end{equation*}
	where $\Delta_{kl}^{(n)} = 0$ if $k,l \leq n$ or $k$ and $l$ are of different parities, and otherwise
	\begin{equation*}
	\abs{\Delta_{kl}} \leq \sum_{m = 1}^{\infty} \delta_{\abs{[k/2] - [l/2]} = mn} + \delta_{\abs{[k/2] + [l/2]} = mn}
	\end{equation*}
	
	The same statement holds true when replacing $\phi_k$ by $\varphi_k$ and $\phi_l$ by $\varphi_l$, and using symmetric grid points $\wb{X}_i$.
\end{lemma}
\begin{proof}
	We prove only the statement involving $\varphi_k$ and $\varphi_l$. Note that for $k = 1,\ldots,n$, and $i = 1,\ldots,n$,
	\begin{equation*}
	v_{k,i} = \frac{1}{\sqrt{n}}\varphi_k(\wb{x}_i)
	\end{equation*}
	and $\Delta_{kl}^{(n)} = 0$ for $1 \leq k,l \leq n$ immediately follows. 
	
	By restricting the basis $\varphi_k$ to the nodes $\wb{x}_i$ we induce periodicity. In particular, for $k = mn + (j + 1)$ where $m$ and $j$ are non-negative integers and $0 \leq j < n$, we have
	\begin{equation*}
	\varphi_k(\wb{x}_i) \propto
	\begin{cases*}
	\varphi_{j + 1}(\wb{x}_i),& ~~\textrm{if $m \mod 2 = 0$} \\
	\varphi_{n - j}(\wb{x}_i),& ~~\textrm{if $m \mod 2 = 1$}
	\end{cases*}
	\end{equation*}
	Writing $l = m'n + j' + 1$, we therefore see that $\dotp{\varphi_k}{\varphi_l}_n \in \{-1,+1\}$ if and only if (a) $m \mod 2 = m' \mod 2$ and $j = j'$, or (b) $m \mod 2 \neq m' \mod 2$ and $j + 1 = n - j'$; otherwise, $\dotp{\varphi_k}{\varphi_l}_n = 0$. Condition (a) is equivalent to $|[k/2] - [l/2]| = mn$ for some $m \in \mathbb{N}$, and Condition (b) is equivalent to $[k/2] + [m/2] = mn$ for some $m \in \Nbb$, thus proving the claim.
	
\end{proof}

\begin{lemma}
If $(\theta_k) \in \ell^2(\Nbb)$ satisfies~\eqref{asmp:fourier_coefficient_decay}, then additionally
\begin{equation*}
\max_{1 \leq k \leq n} \biggl(\sum_{l = n + 1}^{\infty} \theta_l \Delta_{kl}\biggr)^2 \leq \frac{C'}{n^2}
\end{equation*}
for some $C'$ which depends only on $C$.
\end{lemma}	

For $b \in \{0,1\}^d$, let $I_b(k) = \set{l \in \Nbb^d: l_ib_i = k_ib_i~~\textrm{for each}~~i = 1,\ldots,d}$.

\begin{lemma}
	\label{lem:double_counting}
	For any $\theta: \mathbb{N}^d \to \Reals^+$, any $K \leq N$, and any $b \in \{0,1\}^d$ such that $\abs{b} < N$, 
	\begin{equation}
	\label{eqn:double_counting}
	\abs{\sum_{k \in [K]^d} \sum_{l \in I_b(k)} \theta_l \prod_{j:b_j = 0} \Delta_{l_jk_j}^{(N)}} \leq 2^d \sum_{\mathbb{N}^d -[N]^d} \abs{\theta_l}
	\end{equation}
\end{lemma}
\begin{proof}
	Without loss of generality, suppose $b_1 = \ldots = b_{\abs{b}} = 1$, and $b_{\abs{b} + 1} = \ldots = b_{d} = 0$. We can upper bound the left hand side of ~\eqref{eqn:double_counting} as 
	\begin{equation*}
	\abs{\sum_{k \in [K]^d} \sum_{l \in I_b(k)} \theta_l \prod_{j:b_j = 0} \Delta_{l_jk_j}^{(N)}} \leq \sum_{k_1,\ldots,k_{\abs{b}} = 1}^{K} \sum_{k_{\abs{b} + 1},\ldots,k_{d} = 1}^{K} \sum_{l_{\abs{b} + 1}, \ldots, l_d = N + 1}^{\infty} \abs{\theta_{k_1\ldots k_{\abs{b}} l_{\abs{b} + 1} \ldots l_d}} \prod_{j = \abs{b} + 1}^{d} \abs{\Delta_{l_jk_j}^{(N)}}
	\end{equation*}
	and it is therefore sufficient to show that
	\begin{equation}
	\label{eqn:double_counting_pf1}
	\sum_{k_{\abs{b} + 1},\ldots,k_{d} = 1}^{K} \sum_{l_{\abs{b} + 1}, \ldots, l_d = N + 1}^{\infty} \abs{\theta_{k_1\ldots k_{\abs{b}} l_{\abs{b} + 1} \ldots l_d}} \prod_{j = \abs{b} + 1}^{d} \abs{\Delta_{l_jk_j}^{(N)}} \leq \sum_{l_{\abs{b} + 1},\ldots,l_{d} = N + 1}^{\infty} \abs{\theta_{k_1\ldots k_{\abs{b}} l_{\abs{b} + 1} \ldots l_d}}
	\end{equation}
	Using the upper bound on $\Delta_{kl}$ given by Lemma~\ref{lem:double_counting}, for a given $k_{\abs{b} + 1},\ldots,k_d$ we have
	\begin{align}
	\sum_{l_{\abs{b} + 1}, \ldots, l_d = N + 1}^{\infty} & \abs{\theta_{k_1\ldots k_{\abs{b}} l_{\abs{b} + 1} \ldots l_d}} \prod_{j = \abs{b} + 1}^{d} \abs{\Delta_{l_jk_j}^{(N)}} \nonumber \\
	& \leq \sum_{l_{\abs{b} + 1}, \ldots, l_d = N + 1}^{\infty} \abs{\theta_{k_1\ldots k_{\abs{b}} l_{\abs{b} + 1} \ldots l_d}} \sum_{m_{\abs{b} + 1,\ldots, m_d} = 1}^{\infty} \prod_{j = \abs{b} + 1}^{d} \delta_{\abs{[k_j/2] - [l_j/2]} = m_jN} + \delta_{\abs{[k_j/2] + [l_j/2]} = m_jN} \nonumber \\
	& = \sum_{m_{\abs{b} + 1,\ldots, m_d} = 1}^{\infty} \sum_{l_{\abs{b} + 1}, \ldots, l_d = N + 1}^{\infty} \abs{\theta_{k_1\ldots k_{\abs{b}} l_{\abs{b} + 1} \ldots l_d}} \prod_{j = \abs{b} + 1}^{d} \delta_{\abs{[k_j/2] - [l_j/2]} = m_jN} + \delta_{\abs{[k_j/2] + [l_j/2]} = m_jN} \nonumber \\
	& = \sum_{m_{\abs{b} + 1,\ldots, m_d} = 1}^{\infty} \sum_{l \in L(m,k)} \abs{\theta_l}. \label{eqn:double_counting_pf2}
	\end{align}
	In the above lines, the sums are always over $l_j$ such that $l_j$ and $k_j$ have the same parity, since otherwise $\Delta_{l_j k_j}^{(N)} = 0$ by Lemma~\ref{lem:polyak90}, and $L(m,k) := \cap_{j = 1}^{d} S_j(m,k) \cap \cap_{j = 1}^{d} R_j(l)$ for
	\begin{align*}
	S_j(m,k) & = 
	\begin{cases*}
	\Bigl\{l_j \in \mathbb{N}: l_j = k_j\Bigr\}, & ~\textrm{for $j = 1,\ldots,\abs{b}$} \\
	\Bigl\{l_j \in \mathbb{N} \setminus [N]: [l_j/2] - [k_j/2] = m_j N \Bigr\} \cup \Bigl\{l_j \in \mathbb{N} \setminus{N}: [l_j/2] + [k_j/2] = m_j N \Bigr\}, & ~\textrm{for $j = \abs{b} + 1,\ldots,d$}.
	\end{cases*} \\
	R_j(k) & = \Bigl\{l_j \in \mathbb{N}: l_j \mod 2 = k_j \mod 2\Bigr\}
	\end{align*}
	The key point is that for any distinct $k_j,k_j' \in [K]$ the sets $S_j(m,k)$ and $S_j(m,k')$ are disjoint unless (a) $[k_j/2] = [k_j'/2]$, in which case $R_j(k)$ and $R_j(k')$ are disjoint since $k_j$ and $k_j'$ are of different parity, or (b) $[k_j/2] = N - [k_j/2]$ and $k_j,k_j'$ are of the same parity. Therefore, for any $k \in [K]^d$ and $m \in \mathbb{N}^d$, there are at most $2^{d - \abs{b}} < 2^d$ choices of $k'$ such that $L(m,k)$ and $L(m,k')$ are not disjoint. Moreover, for any $m \in \mathbb{N}^d$ the inclusion $l_j \in S_j(m,k)$ implies that $\abs{l_j} \geq N + 1$. As a result,
	\begin{equation*}
	\bigcup_{k_{\abs{b} + 1},\ldots,k_d = 1}^{N}\Bigl\{\bigcup_{m \in \mathbb{N}^d} L(m,k)\Bigr\} \subseteq \bigcup_{l_{\abs{b} + 1},\ldots, l_{d} = N}^{\infty} (k_1,\ldots,k_{\abs{b}},l_{\abs{b} + 1},\ldots,l_d)
	\end{equation*} Plugging~\eqref{eqn:double_counting_pf2} back into the left hand side of~ \eqref{eqn:double_counting_pf1}, we have
	\begin{align*}
	\sum_{k_{\abs{b} + 1},\ldots,k_{d} = 1}^{K} \sum_{l_{\abs{b} + 1}, \ldots, l_d = N + 1}^{\infty} \abs{\theta_{k_1\ldots k_{\abs{b}} l_{\abs{b} + 1} \ldots l_d}} \prod_{j = \abs{b} + 1}^{d} \abs{\Delta_{l_jk_j}^{(N)}} & \leq \sum_{k_{\abs{b} + 1},\ldots,k_{d} = 1}^{K} \sum_{m_{\abs{b} + 1,\ldots, m_d} = 1}^{\infty} \sum_{l \in L(m,k)} \abs{\theta_l} \\
	& \leq 2^d \sum_{l_{\abs{b} + 1},\ldots,l_{d} = N + 1}^{\infty} \abs{\theta_{k_1\ldots k_{\abs{b}} l_{\abs{b} + 1} \ldots l_d}}
	\end{align*}
	and the proof~\eqref{eqn:double_counting_pf1}--and therefore the proof the Lemma--is complete.
\end{proof}

\subsection{Equivalence between Sobolev classes}
\label{subsec:sobolev_class_equivalence}

\textcolor{red}{Work through details.}


\end{document}