\documentclass{article}
\usepackage{amsmath}
\usepackage{amsfonts, amsthm, amssymb}
\usepackage{graphicx}
\usepackage[colorlinks]{hyperref}
\usepackage[parfill]{parskip}
\usepackage{algpseudocode}
\usepackage{algorithm}
\usepackage{enumerate}
\usepackage[shortlabels]{enumitem}
\usepackage{fullpage}
\usepackage{mathtools}

\usepackage{natbib}
\renewcommand{\bibname}{REFERENCES}
\renewcommand{\bibsection}{\subsubsection*{\bibname}}

\DeclarePairedDelimiterX{\norm}[1]{\lVert}{\rVert}{#1}

\newcommand{\eqdist}{\ensuremath{\stackrel{d}{=}}}
\newcommand{\Graph}{\mathcal{G}}
\newcommand{\Reals}{\mathbb{R}}
\newcommand{\Identity}{\mathbb{I}}
\newcommand{\distiid}{\overset{\text{i.i.d}}{\sim}}
\newcommand{\convprob}{\overset{p}{\to}}
\newcommand{\convdist}{\overset{w}{\to}}
\newcommand{\Expect}[1]{\mathbb{E}\left[ #1 \right]}
\newcommand{\Risk}[2][P]{\mathcal{R}_{#1}\left[ #2 \right]}
\newcommand{\Prob}[1]{\mathbb{P}\left( #1 \right)}
\newcommand{\iset}{\mathbf{i}}
\newcommand{\jset}{\mathbf{j}}
\newcommand{\myexp}[1]{\exp \{ #1 \}}
\newcommand{\abs}[1]{\left \lvert #1 \right \rvert}
\newcommand{\restr}[2]{\ensuremath{\left.#1\right|_{#2}}}
\newcommand{\ext}[1]{\widetilde{#1}}
\newcommand{\set}[1]{\left\{#1\right\}}
\newcommand{\seq}[1]{\set{#1}_{n \in \N}}
\newcommand{\dotp}[2]{\langle #1, #2 \rangle}
\newcommand{\floor}[1]{\left\lfloor #1 \right\rfloor}
\newcommand{\Var}{\mathrm{Var}}
\newcommand{\Cov}{\mathrm{Cov}}
\newcommand{\diam}{\mathrm{diam}}

\newcommand{\emC}{C_n}
\newcommand{\emCpr}{C'_n}
\newcommand{\emCthick}{C^{\sigma}_n}
\newcommand{\emCprthick}{C'^{\sigma}_n}
\newcommand{\emS}{S^{\sigma}_n}
\newcommand{\estC}{\widehat{C}_n}
\newcommand{\hC}{\hat{C^{\sigma}_n}}
\newcommand{\vol}{\text{vol}}
\newcommand{\spansp}{\mathrm{span}~}
\newcommand{\1}{\mathbf{1}}

\newcommand{\Linv}{L^{\dagger}}
\DeclareMathOperator*{\argmin}{argmin}
\DeclareMathOperator*{\argmax}{argmax}

\newcommand{\emF}{\mathbb{F}_n}
\newcommand{\emG}{\mathbb{G}_n}
\newcommand{\emP}{\mathbb{P}_n}
\newcommand{\F}{\mathcal{F}}
\newcommand{\D}{\mathcal{D}}
\newcommand{\R}{\mathcal{R}}
\newcommand{\Rd}{\Reals^d}

%%% Vectors
\newcommand{\thetast}{\theta^{\star}}

%%% Matrices
\newcommand{\X}{X} % no bold
\newcommand{\Y}{Y} % no bold
\newcommand{\Z}{Z} % no bold
\newcommand{\Lgrid}{L_{\grid}}
\newcommand{\Dgrid}{D_{\grid}}
\newcommand{\Linvgrid}{L_{\grid}^{\dagger}}

%%% Sets and classes
\newcommand{\Xset}{\mathcal{X}}
\newcommand{\Sset}{\mathcal{S}}
\newcommand{\Hclass}{\mathcal{H}}
\newcommand{\Pclass}{\mathcal{P}}
\newcommand{\domain}{\mathcal{X}}

%%% Distributions and related quantities
\newcommand{\Pbb}{\mathbb{P}}
\newcommand{\Ebb}{\mathbb{E}}
\newcommand{\Qbb}{\mathbb{Q}}

%%% Operators
\newcommand{\Tadj}{T^{\star}}
\newcommand{\dive}{\mathrm{div}}
\newcommand{\dif}{\mathop{}\!\mathrm{d}}
\newcommand{\Partial}{\mathcal{D}}
\newcommand{\Hessian}{\mathcal{D}^2}

%%% Misc
\newcommand{\grid}{\mathrm{grid}}
\newcommand{\critr}{R_n}
\newcommand{\dx}{\,dx}
\newcommand{\dy}{\,dy}
\newcommand{\dr}{\,dr}
\newcommand{\dxpr}{\,dx'}
\newcommand{\dypr}{\,dy'}
\newcommand{\wt}[1]{\widetilde{#1}}

%%% Order of magnitude
\newcommand{\soom}{\sim}

% \newcommand{\span}{\textrm{span}}

\newtheoremstyle{alden}
{6pt} % Space above
{6pt} % Space below
{} % Body font
{} % Indent amount
{\bfseries} % Theorem head font
{.} % Punctuation after theorem head
{.5em} % Space after theorem head
{} % Theorem head spec (can be left empty, meaning `normal')

\theoremstyle{alden} 


\newtheoremstyle{aldenthm}
{6pt} % Space above
{6pt} % Space below
{\itshape} % Body font
{} % Indent amount
{\bfseries} % Theorem head font
{.} % Punctuation after theorem head
{.5em} % Space after theorem head
{} % Theorem head spec (can be left empty, meaning `normal')

\theoremstyle{aldenthm}
\newtheorem{theorem}{Theorem}
\newtheorem{conjecture}{Conjecture}
\newtheorem{lemma}{Lemma}
\newtheorem{example}{Example}
\newtheorem{corollary}{Corollary}
\newtheorem{proposition}{Proposition}
\newtheorem{assumption}{Assumption}
\newtheorem{remark}{Remark}


\theoremstyle{definition}
\newtheorem{definition}{Definition}[section]

\theoremstyle{remark}

\begin{document}
\title{Summary of Progress}
\author{Alden Green}
\date{\today}
\maketitle

A brief summary of the progress we've made so far.
\section{Progress to date.}
We observe samples $x_1,\ldots,x_n \sim P$ and $y_1,\ldots,y_m \sim Q$, (for convenience, assume $P$ and $Q$ are supported on $[0,1]^d$ absolutely continuous with respective densities $p$ and $q$ which are lower bounded away from $0$, and pper bounded away from $\infty$, over their support.) We started with the following IPM-based test statistic; construct a neighborhood graph $G$ with incidence matrix $D$, and let
\begin{equation*}
T_C := \sup_{\theta: \norm{D\theta}_2 \leq C} \dotp{\theta}{z},
\end{equation*}
where $z = 1_X - 1_Y$ is a contrast vector and $C$ is a scaling parameter which controls the size of the function class we compute the IPM over. We made the following alterations to the above setup and test statistic:
\begin{itemize}
	\item We modified the setup of the problem in several ways to ease tractability of analysis. First, we assumed we observed values $z_i = f(x_i) + \varepsilon_i$, where $x_i \sim (p + q)/2$, and $f = (p - q)/(p + q)$, and $\varepsilon_i$ represented the error of a Bernoulli random variable minus its expectation. Then, we let $\varepsilon_i$ be normally distributed. 
	
	\item More fundamentally, we changed the norm dictating the function class over which we take supremum. Let $C$ be such that if $f = (p - q) \in W^{2,1}(L)$, then $\norm{Df}_2 \lesssim C$. This choice of $C$ results in the IPM being computed over too rich of a function class, so must restrict the function class further. To do so, we modified the IPM as follows: letting $D^T D = L = V \Lambda V^T$ be the eigendecomposition of the Laplacian matrix $L$, let $g_C(\lambda) = \infty \cdot \1(\lambda > C)$, and let $g_C(L) = V g_C(\Lambda) V^T$. Then, consider the following choice of test statistic:
	\begin{equation*}
	T_C := \sup_{\theta: \theta^T g_C(L) \theta \leq C} \dotp{\theta}{z}
	\end{equation*}
	or equivalently
	\begin{equation*}
	T_C := \sum_{k: \lambda_k \leq C^2} \dotp{v_k}{z} 
	\end{equation*}
\end{itemize}

For an appropriately formed neighborhood graph, and an appropriate choice of $C$, does $T_C$ achieve non-trivial power uniformly over a nonparametric function class , that is assume that
\begin{equation*}
\norm{f}_{W^{2,1}} := \sum_{\abs{\alpha} \leq 1} \norm{\Partial f}_{L^2}^2 \leq L
\end{equation*}


\end{document}