\documentclass{article}
\usepackage{amsmath}
\usepackage{amsfonts, amsthm, amssymb}
\usepackage{graphicx}
\usepackage[colorlinks]{hyperref}
\usepackage[parfill]{parskip}
\usepackage{algpseudocode}
\usepackage{algorithm}
\usepackage{enumerate}
\usepackage[shortlabels]{enumitem}
\usepackage{fullpage}

\usepackage{natbib}
\renewcommand{\bibname}{REFERENCES}
\renewcommand{\bibsection}{\subsubsection*{\bibname}}

\newcommand{\eqdist}{\ensuremath{\stackrel{d}{=}}}
\newcommand{\Graph}{\mathcal{G}}
\newcommand{\Reals}{\mathbb{R}}
\newcommand{\Identity}{\mathbb{I}}
\newcommand{\distiid}{\overset{\text{i.i.d}}{\sim}}
\newcommand{\convprob}{\overset{p}{\to}}
\newcommand{\convdist}{\overset{w}{\to}}
\newcommand{\Expect}[1]{\mathbb{E}\left[ #1 \right]}
\newcommand{\Risk}[2][P]{\mathcal{R}_{#1}\left[ #2 \right]}
\newcommand{\Prob}[1]{\mathbb{P}\left( #1 \right)}
\newcommand{\iset}{\mathbf{i}}
\newcommand{\jset}{\mathbf{j}}
\newcommand{\myexp}[1]{\exp \{ #1 \}}
\newcommand{\norm}[1]{\left\lVert#1\right\rVert}
\newcommand{\abs}[1]{\left \lvert #1 \right \rvert}
\newcommand{\restr}[2]{\ensuremath{\left.#1\right|_{#2}}}
\newcommand{\ext}[1]{\widetilde{#1}}
\newcommand{\set}[1]{\left\{#1\right\}}
\newcommand{\seq}[1]{\set{#1}_{n \in \N}}
\newcommand{\dotp}[2]{\langle #1, #2 \rangle}
\newcommand{\floor}[1]{\left\lfloor #1 \right\rfloor}
\newcommand{\Var}{\mathrm{Var}}
\newcommand{\diam}{\mathrm{diam}}

\newcommand{\emC}{C_n}
\newcommand{\emCpr}{C'_n}
\newcommand{\emCthick}{C^{\sigma}_n}
\newcommand{\emCprthick}{C'^{\sigma}_n}
\newcommand{\emS}{S^{\sigma}_n}
\newcommand{\estC}{\widehat{C}_n}
\newcommand{\hC}{\hat{C^{\sigma}_n}}
\newcommand{\vol}{\text{vol}}
\newcommand{\spansp}{\mathrm{span}~}
\newcommand{\1}{\mathbb{I}}

\newcommand{\Linv}{L^{\dagger}}
\DeclareMathOperator*{\argmin}{argmin}
\DeclareMathOperator*{\argmax}{argmax}

\newcommand{\emF}{\mathbb{F}_n}
\newcommand{\emG}{\mathbb{G}_n}
\newcommand{\emP}{\mathbb{P}_n}
\newcommand{\F}{\mathcal{F}}
\newcommand{\D}{\mathcal{D}}
\newcommand{\R}{\mathcal{R}}
\newcommand{\Rd}{\Reals^d}

%%% Vectors
\newcommand{\thetast}{\theta^{\star}}

%%% Matrices
\newcommand{\X}{X} % no bold
\newcommand{\Y}{Y} % no bold
\newcommand{\Z}{Z} % no bold
\newcommand{\Lgrid}{L_{\grid}}
\newcommand{\Dgrid}{D_{\grid}}
\newcommand{\Linvgrid}{L_{\grid}^{\dagger}}

%%% Sets and classes
\newcommand{\Xset}{\mathcal{X}}
\newcommand{\Sset}{\mathcal{S}}
\newcommand{\Hclass}{\mathcal{H}}
\newcommand{\Pclass}{\mathcal{P}}

%%% Distributions and related quantities
\newcommand{\Pbb}{\mathbb{P}}
\newcommand{\Ebb}{\mathbb{E}}
\newcommand{\Qbb}{\mathbb{Q}}

%%% Operators
\newcommand{\Tadj}{T^{\star}}
\newcommand{\dive}{\mathrm{div}}
\newcommand{\dif}{\mathop{}\!\mathrm{d}}
\newcommand{\gradient}{\mathcal{D}}
\newcommand{\Hessian}{\mathcal{D}^2}

%%% Misc
\newcommand{\grid}{\mathrm{grid}}
\newcommand{\critr}{R_n}
\newcommand{\dx}{\,dx}
\newcommand{\dy}{\,dy}
\newcommand{\dr}{\,dr}

%%% Order of magnitude
\newcommand{\soom}{\sim}

% \newcommand{\span}{\textrm{span}}

\newtheoremstyle{alden}
{6pt} % Space above
{6pt} % Space below
{} % Body font
{} % Indent amount
{\bfseries} % Theorem head font
{.} % Punctuation after theorem head
{.5em} % Space after theorem head
{} % Theorem head spec (can be left empty, meaning `normal')

\theoremstyle{alden} 


\newtheoremstyle{aldenthm}
{6pt} % Space above
{6pt} % Space below
{\itshape} % Body font
{} % Indent amount
{\bfseries} % Theorem head font
{.} % Punctuation after theorem head
{.5em} % Space after theorem head
{} % Theorem head spec (can be left empty, meaning `normal')

\theoremstyle{aldenthm}
\newtheorem{theorem}{Theorem}
\newtheorem{conjecture}{Conjecture}
\newtheorem{lemma}{Lemma}
\newtheorem{example}{Example}
\newtheorem{corollary}{Corollary}
\newtheorem{proposition}{Proposition}
\newtheorem{assumption}{Assumption}
\newtheorem{remark}{Remark}


\theoremstyle{definition}
\newtheorem{definition}{Definition}[section]

\theoremstyle{remark}

\begin{document}
\title{Notes for Week 6/9/19 - 6/11/19}
\author{Alden Green}
\date{\today}
\maketitle

Let $f,g$ be $L^2$ density functions supported on the $d$-dimensional unit cube $[0,1]^d$, meaning
\begin{equation*}
\int_{[0,1]^d} f(x)^2 dx, \int_{[0,1]^d} g(x)^2 dx < \infty, \quad \text{and} \quad \int_{[0,1]^d} f(x) dx, \int_{[0,1]^d} g(x) dx = 1.
\end{equation*}
with $f$,$g$ bounded away from $0$, $f(x), g(x) > p_{\min}$ for all $x \in [0,1]^d$.

We observe data $(X,\ell)$, a design matrix and associated labels, specified as follows. We let $x_1, \ldots, x_n$ be the rows of $X$, each sampled independently from $\mu = (f + g)/2$. For each $i = 1,\ldots,n$, we then sample $\ell_i$ according to
\begin{equation*}
\ell_i =
\begin{cases}
1,~ \textrm{with probability} \frac{f(x_i)}{f(x_i) + g(x_i)} \\
-1,~ \textrm{with probability} \frac{g(x_i)}{f(x_i) + g(x_i)}
\end{cases}
\end{equation*}
and let $\ell = (\ell_i)$. 

Our statistical goal is hypothesis testing: that is, we wish to construct a test function $\phi$ which differentiates between
\begin{equation*}
\mathbb{H}_0: f = g \text{ and } \mathbb{H}_1: f \neq g.
\end{equation*}

For a given function class $\Hclass$, some $\epsilon > 0$, and test function $\phi$ a Borel measurable function of the data with range $\set{0,1}$, we evaluate the quality of the test using \emph{worst-case risk}
\begin{equation*}
R_{\epsilon}^{(t)}(\phi; \Hclass) = \sup_{f \in \Hclass} \Ebb_{f,f}^{(t)}(\phi) + \sup_{ \substack{f,g \in \Hclass \\ \delta(f,g) \geq \epsilon } } \Ebb_{f,g}^{(t)}(1 - \phi)
\end{equation*} 
where 
\begin{equation*}
\delta^2(f,g) = \int_{\D} (f - g)^2 dx.
\end{equation*}
We will consider those density functions which belong to $\mathcal{H} := \mathcal{H}_{1}^d(L)$, the $d$-dimensional Lipschitz ball with norm $L$. Formally, a function $f \in \mathcal{H}$ if for all $x,y \in [0,1]^d$, $\abs{f(x) - f(y)} \leq L \norm{x - y}$.\footnote{Note that the Lipschitz requirement, along with the restriction that $f$ be a density function over a bounded domain, imply that $f$ has finite $L^2$ norm.}

\section{Sieve-based test.}

For a radius $r > 0$ to be determined later, the $r$-neighborhood graph $G = (V,E)$ consists of vertices $V = [n]$ corresponding to the $n$ data points, and edges $E = \set{(u,v): \norm{x_u - x_v} \leq r}$. Write $D$ for the incidence matrix of this graph, and $L = D^T D$ for the graph Laplacian. Denote the singular value decomposition of $D$ by $D = U \Lambda^{1/2} V^T$, where $U$ and $V$ are orthonormal matrices and $\Lambda$ is a diagonal matrix with entries $0 = \lambda_1 \leq \lambda_2 \leq \lambda_3 \leq \ldots \leq \lambda_n$, so that $L = V \Lambda V^T$. For $k = 1,\ldots,n$, write $V_k = (v_1 \ldots v_k)$ for the $n \times k$ matrix containing the first $k$ columns of $V$--that is, the first $k$ eigenvectors of $L$. Our first test statistic will be as follows:

\begin{equation}
\label{eqn: laplacian_eigenmaps_statistic}
T_{LE}^{(k)} := \sup_{\substack{\theta \in \mathrm{col}(V_k), \\ \norm{\theta} = 1} } \dotp{\theta}{\ell}^2 = \ell^T V_k V_k^T \ell.
\end{equation}

\subsection{Assumptions}
The following are statements I wish to hold in order for the later theory to follow, but I cannot yet prove:
\begin{enumerate}
	\item There exists a constant $c_1 > 0$ -- which may depend on dimension $d$ but not on $n$ --  such that for all $h \in 1, \ldots,k$ and all $i = 1,\ldots,n$, 
	\begin{equation}
	\label{eqn: asmp_1}
	\abs{v_{hi}} \leq \frac{c}{\sqrt{n}}
	\end{equation}
\end{enumerate}

If $v_k$ were eigenvectors of the grid--rather than of the neighborhood graph--the first item would hold with $c = 2^{d/2}$.

\subsection{Main result.}
Let $d = 3,4$, choose $r = c(\log(n)/n)^{1/d}$ to satisfy the condition of Lemma \ref{lem: approximation_error}, and let $k = n^{2d/(d + 4)}$. Let 
\begin{equation*}
\phi_{LE} =
\begin{cases}
1,~ & \text{if}~ T_{LE}^{(k)} \geq k + a\sqrt{2k + c^4\left(k + \frac{k^2}{n}\right)} \\
0,~ & \text{otherwise}.
\end{cases}
\end{equation*}
The following theorem holds.
\begin{theorem}
	There exists some number $c > 0$ constant in sample size $n$ such that for any $\epsilon \geq c n^{-2/(4 + d)}$, we have
	\begin{equation*}
	R_{\epsilon}^{(t)}(\phi_{LE}; \Hclass_1^{d}(L)) \leq \frac{1}{a^2}.
	\end{equation*}
	when $d = 3$ or $4$. 
\end{theorem}
\begin{proof}
	
\end{proof}

\subsection{Type I error}

Under the null hypothesis, $\ell$ consists of $n$ independent and identically distributed Rademacher random variables. Using this fact, we can bound the first two moments of $T_{LE}$ under the null hypothesis.

\begin{lemma}
	If $f = g$, then
	\begin{equation}
	\Ebb(T_{LE}^{(k)}) = k
	\end{equation}
	and
	\begin{equation}
	\label{eqn: variance_noise_typeI}
	\Var(T_{LE}^{(k)}) \leq 2k + c_1^4\left(k + \frac{2k^2}{n}\right)
	\end{equation}
\end{lemma}
\begin{proof}
	Taking the expectation first, we have
	\begin{align*}
	\Ebb(T_{LE}^{(k)}) & = \sum_{j = 1}^{k} \Ebb((\ell^T v_j)^2) \\
	& = \sum_{j = 1}^{k} \sum_{i = 1}^{n} \sum_{i' = 1}^{n} \Ebb(\ell_i \ell_{i'}) v_{ji} v_{ji'} \\
	& =  \sum_{j = 1}^{k} \sum_{i = 1}^{n} v_{ji}^2 = k.
	\end{align*}
	
	We defer the proof of \eqref{eqn: variance_noise_typeI} until the following subsection, where we show it in greater generality.
\end{proof}

\subsection{Type II error.}

Write $\ell =: \thetast + w$, where $\thetast = (\thetast_i)$ is defined by
\begin{equation*}
\thetast_i = \frac{f(x_i) - g(x_i)}{f(x_i) + g(x_i)}
\end{equation*}
and $w = (w_i)$ therefore consists of $n$ independent (although not identically distributed) zero-mean noise terms. We expand
\begin{equation}
\label{eqn: LE_expansion}
\ell^T V_k V_k^T \ell = (\thetast)^T V_k V_k^T \thetast + w^T V_k V_k w + 2 w^T V_k V_k^T \thetast,
\end{equation}
and focus at first our attention on the second term in \eqref{eqn: LE_expansion}.

\begin{lemma}
	Assuming \eqref{eqn: asmp_1} holds,
	\begin{equation}
	\label{eqn: expectation_noise_typeII}
	\Ebb \left( w^T V_k V_k w\right) \geq k - \frac{kc_1^2 \delta^2(f,g)}{p_{\min}^2}
	\end{equation}
	and
	\begin{equation}
	\label{eqn: variance_noise}
	\Var(T_{LE}^{(k)}) \leq 2k + c_1^4\left(k + \frac{2k^2}{n}\right)
	\end{equation}
\end{lemma}
\begin{proof}
	We note that, conditional on $X$,
	\begin{equation*}
	w_i = 
	\begin{cases}
	2\frac{g(x_i)}{f(x_i) + g(x_i)},~ & \text{with probability}~ \frac{f(x_i)}{f(x_i) + g(x_i)} \\
	-2\frac{f(x_i)}{f(x_i) + g(x_i)},~ & \text{with probability}~ \frac{g(x_i)}{f(x_i) + g(x_i)}
	\end{cases}
	\end{equation*}
	
	We first show \eqref{eqn: expectation_noise_typeII}. Expanding the quadratic form as a double sum, and using the law of iterated expectation, we have
	\begin{align}
	\label{eqn: expectation_noise_typeII_1}
	\Ebb \left( w^T V_k V_k w\right) & = \sum_{h = 1}^{k} \sum_{i = 1}^{n} \sum_{j = 1}^n \Ebb \bigl(w_i w_j v_{hi} v_{hj} \bigr) \nonumber \\
	& = \sum_{h = 1}^{k} \sum_{i = 1}^{n} \sum_{j = 1}^n \Ebb \biggl( \Ebb\Bigl(w_i w_j \bigl\vert x_i, x_j\Bigr) v_{hi} v_{hj}\biggr) \nonumber \\
	& = \sum_{h = 1}^{k} \sum_{i = 1}^{n} \Ebb \biggl( \Ebb\Bigl(w_i^2 \bigl\vert x_i\Bigr) v_{hi}^2\biggr).
	\end{align}
	The conditional expectation $\Ebb\Bigl(w_i^2 \bigl\vert x_i\Bigr)$ can be directly computed,
	\begin{align*}
	\Ebb\Bigl(w_i^2 \bigl\vert x_i\Bigr) & = 4\left(\frac{g(x_i)}{f(x_i) + g(x_i)}\right)^2 \frac{f(x_i)}{f(x_i) + g(x_i)} + 4\left(\frac{f(x_i)}{f(x_i) + g(x_i)}\right)^2 \frac{g(x_i)}{f(x_i) + g(x_i)} \\
	& = 4 \frac{f(x_i)g(x_i)(f(x_i) + g(x_i))}{(f(x_i) + g(x_i))^3} \\
	& = 4 \frac{f(x_i)g(x_i)}{(f(x_i) + g(x_i))^2}.
	\end{align*}
	and plugging this back into \eqref{eqn: expectation_noise_typeII_1}, we obtain
	\begin{align*}
	\Ebb \left( w^T V_k V_k w\right) & = \sum_{h = 1}^{k} \sum_{i = 1}^{n} \Ebb \left(4 \frac{f(x_i) g(x_i)}{(f(x_i) + g(x_i))^2} v_{hi}^2\right) \\
	& = \sum_{h = 1}^{k} \sum_{i = 1}^{n} \Ebb \left(v_{hi}^2 \left(1 - \left(\frac{f(x_i) - g(x_i)}{f(x_i) + g(x_i)} \right)^2 \right)\right) \\
	& \geq k - \frac{c^2}{n}\sum_{h = 1}^{k} \sum_{i = 1}^{n} \Ebb \left(\frac{f(x_i) - g(x_i)}{f(x_i) + g(x_i)} \right)^2 \\
	& \geq k - \frac{kc_1^2 \delta^2(f,g)}{p_{\min}^2}
	\end{align*}
	and \eqref{eqn: expectation_noise_typeII} is shown.
	
	We now show \eqref{eqn: variance_noise}. It will be helpful to note that for all $i,i',j,j' \in [n]$,
	\begin{equation*}
	\mathrm{Cov}(w_iw_j, w_{i'}w_{j'} | X) = 
	\begin{cases}
	0, & \text{if}~ \set{i,i',j,j'}~ \textrm{has four distinct elements} \\
	0, & \text{if}~ \set{i,i',j,j'}~ \textrm{has three distinct elements} \\
	0, & \text{if}~ \set{i,i',j,j'}~ \textrm{has two distinct elements and}~ i = j \\
	\Ebb(w_i^2|X) \Ebb(w_j^2|X), & \text{if}~ \set{i,i',j,j'}~ \textrm{has two distinct elements and}~ i \neq j \\
	\Var(w_i^4|X), & \text{if}~ \set{i,i',j,j'}~ \textrm{has one distinct element.}
	\end{cases}
	\end{equation*}
	and additionally
	\begin{equation*}
	\Ebb(w_iw_j|X) =
	\begin{cases}
	0, & \text{if}~ i \neq j \\
	E(w_i^2|X), & \text{if}~ i = j.
	\end{cases}
	\end{equation*}
	Then, by the law of total covariance,
	\begin{align*}
	\mathrm{Var}(w^T V_k V_k^T w) & = \sum_{h,h' = 1}^{k} \sum_{i,i' = 1}^{n} \sum_{j,j' = 1}^{n} \mathrm{Cov} \bigl(w_iw_jv_{hi} v_{hj}, w_{i'}w_{j'}v_{h'i'} v_{h'j'}\bigr) \\
	& = \sum_{h,h' = 1}^{k} \sum_{i,i' = 1}^{n} \sum_{j,j' = 1}^{n} \Ebb \Bigl(  v_{hi} v_{hj} v_{h'i'} v_{h'j'} \mathrm{Cov} \bigl(w_iw_j, w_{i'}w_{j'} |X\bigr)\Bigr) + \\
	& \quad \quad \mathrm{Cov} \Bigl( v_{hi}v_{hj} \Ebb(w_iw_j|X), v_{h'i'}v_{h'j'} \Ebb(w_i'w_j'|X)  \Bigr) \\
	& = 2 \sum_{h,h' = 1}^{k} \sum_{i \neq j}^{n} \Ebb\Bigl( v_{hi} v_{hj} v_{h'i} v_{h'j} \Ebb(w_i^2|X) \Ebb(w_j^2|X) \Bigr) + \\
	& \quad \quad \sum_{h,h' = 1}^{k} \sum_{i = 1}^{n} \Ebb \left(v_{hi}^2 v_{h'i}^2 \Var(w_i^4|X)\right) +  \\
	& \quad \quad \quad \sum_{h,h' = 1}^{k} \sum_{i,i' = 1}^{n} \mathrm{Cov} \bigl( v_{hi} ^2 \Ebb(w_i^2|X), v_{h'i'}^2 \Ebb(w_{i'}^2|X)\bigr)
	\end{align*}
	Then, using the upper bound $\Ebb(w_i^2|X),\Ebb(w_j^2|X) \leq 1$, we can bound the first term
	\begin{align*}
	\sum_{h,h' = 1}^{k} \sum_{i \neq j}^{n} \Ebb\Bigl( v_{hi} v_{hj} v_{h'i} v_{h'j} \Ebb(w_i^2|X) \Ebb(w_j^2|X) \Bigr) & \leq \sum_{h,h' = 1}^{k} \sum_{i \neq j}^{n} \Ebb\Bigl( v_{hi} v_{hj} v_{h'i} v_{h'j} \Bigr) \\
	& \leq \sum_{h,h' = 1}^{k} \Ebb\Bigl( \sum_{i = 1}^{n}v_{h'i} v_{hi} \sum_{j = 1}^{n} v_{h'j} v_{hj} \Bigr) \\
	& = \sum_{h = 1}^{k} 1 = k.
	\end{align*}
	
	To bound the second term, we note that $\Var(w_i^4|X) \leq 1$, and along with \eqref{eqn: asmp_1} this yields
	\begin{align*}
	\sum_{h,h' = 1}^{k} \sum_{i = 1}^{n} \Ebb \left(v_{hi}^2 v_{h'i}^2 \Var(w_i^4|X)\right) \leq \sum_{h,h' = 1}^{n} \sum_{i = 1}^{n} \frac{c_1^4}{n^2} = \frac{c_1^4k^2}{n}.
	\end{align*}
	
	Finally, to bound the third term we note that if $h \neq h'$ and $i \neq i'$,
	$\mathrm{Cov} \bigl( v_{hi} ^2 \Ebb(w_i^2|X), v_{h'i'}^2 \Ebb(w_{i'}^2|X)\bigr) = 0$. Therefore,
	\begin{align*}
	\sum_{h,h' = 1}^{k} \sum_{i,i' = 1}^{n} \mathrm{Cov} \bigl( v_{hi} ^2 \Ebb(w_i^2|X), v_{h'i'}^2 \Ebb(w_{i'}^2|X)\bigr) & = \sum_{h = 1}^{k} \sum_{i = 1}^{n} \sum_{j = 1}^{n} \mathrm{Cov} \bigl( v_{hi} ^2 \Ebb(w_i^2|X), v_{hi'}^2 \Ebb(w_{i'}^2|X)\bigr) + \\
	& \quad \quad \sum_{h \neq h'}^{k} \sum_{i = 1}^{n} \mathrm{Cov} \bigl( v_{hi} ^2 \Ebb(w_i^2|X), v_{h'i}^2 \Ebb(w_{i}^2|X)\bigr) \\
	& \leq \sum_{h = 1}^{k} \sum_{i = 1}^{n} \sum_{j = 1}^{n} \frac{c_1^4}{n^2} + \sum_{h \neq h'}^{k} \sum_{i = 1}^{n} \frac{c_1^4}{n^2} \\
	& = c_1^4 k + c_1^4 \frac{k^2}{n},
	\end{align*}
	and putting the pieces together yields \eqref{eqn: variance_noise}.
\end{proof}

Now, we turn our attention to the third term of \eqref{eqn: LE_expansion}.
\begin{lemma}
	\begin{equation}
	\Ebb(w^T V_k V_k^T \thetast) = 0
	\end{equation}
	and assuming \eqref{eqn: asmp_1} holds,
	\begin{equation*}
	\Var(w^T V_k V_k^T \thetast) \leq k \frac{c_1^2 \delta^2(f,g)}{p_{\min}^2}
	\end{equation*}
\end{lemma}

Finally, we examine the first term in \eqref{eqn: LE_expansion}. We will need the following result.
\begin{lemma}
	For any $0 < \delta < 1$, there exists a number $c_2 > 0$ depending only on $d$ and $\delta$ such that, if $r \geq c_2(\log n/n)^{1/d}$, then
	\begin{equation*}
	\norm{D \thetast}^2 \leq \frac{16 L^4}{p_{\min}^2} n^2 r^{d + 2}
	\end{equation*}
	and
	\begin{equation*}
	\lambda_k \geq \frac{4 k^{2/d}}{\pi^2 n^{2/d}}
	\end{equation*}
	with probability at least $1 - \delta$.
\end{lemma}
\begin{lemma}
	\label{lem: approximation_error}
	For any $\delta > 0$, there exists a number $c_2 > 0$ constant in sample size such that if $r \geq c_2 (\log n / n)^{1/d}$,
	\begin{equation*}
	(\thetast)^T V_k V_k^T \thetast \geq n (\delta(f,g))^2 - \frac{n^2r^{d + 2} n^{2/d}}{k^{2/d}}
	\end{equation*}
	with probability at least $1 - \delta$.
\end{lemma}
\begin{proof}
	Letting $z = D \thetast$, we have
	\begin{align*}
	(\thetast)^T V_k V_k^T \thetast & = \norm{\thetast}^2 - (\thetast)^T \left(I - V_k V_k^T \right) \thetast \\
	& = \norm{\thetast}^2- z^T (D^{\dagger})^T (I - V_k V_k^T) D^{\dagger} z \\
	& \geq \norm{\thetast}^2 - z^T z \lambda_{\max} \Bigl((D^{\dagger})^T (I - V_k V_k^T) D^{\dagger} \Bigr) \\
	& \geq \norm{\thetast}^2 - \frac{z^T z}{\lambda_{k}}
	\end{align*}
	Examining $\norm{\thetast}^2$, we have that
	\begin{equation*}
	\Ebb(\norm{\thetast}^2) = n \delta^2(f,g) \quad \text{and} \quad \Var(\norm{\thetast}^2) \leq n \delta^2(f,g)
	\end{equation*}
	Turning to the second term, we have given our choice of $r$, along with the fact $f - g \in \mathcal{H}_1^{d}(2L)$,
	\begin{equation*}
	z^T z = \norm{D \thetast}^2 \leq \frac{16 L^4}{p_{\min}^2} n^2 r^{d + 2}
	\end{equation*}
	and additionally
	\begin{equation*}
	\lambda_k \geq \frac{4 k^{2/d}}{\pi^2 n^{2/d}},
	\end{equation*}
	each with probability $1 - \delta$.
\end{proof}






\end{document}