\documentclass{article}
\usepackage{amsmath}
\usepackage{amsfonts, amsthm, amssymb}
\usepackage{graphicx}
\usepackage[colorlinks]{hyperref}
\usepackage[parfill]{parskip}
\usepackage{enumerate}
\usepackage[shortlabels]{enumitem}
\usepackage{fullpage}
\usepackage{mathtools}

\usepackage{natbib}
\renewcommand{\bibname}{REFERENCES}
\renewcommand{\bibsection}{\subsubsection*{\bibname}}

\DeclareFontFamily{U}{mathx}{\hyphenchar\font45}
\DeclareFontShape{U}{mathx}{m}{n}{<-> mathx10}{}
\DeclareSymbolFont{mathx}{U}{mathx}{m}{n}
\DeclareMathAccent{\wb}{0}{mathx}{"73}

\DeclarePairedDelimiterX{\norm}[1]{\lVert}{\rVert}{#1}

\makeatletter
\newcommand{\vast}{\bBigg@{4}}
\newcommand{\Vast}{\bBigg@{5}}
\makeatother

\newcommand{\eqdist}{\ensuremath{\stackrel{d}{=}}}
\newcommand{\Graph}{\mathcal{G}}
\newcommand{\Reals}{\mathbb{R}}
\newcommand{\Identity}{\mathbb{I}}
\newcommand{\Xsetistiid}{\overset{\text{i.i.d}}{\sim}}
\newcommand{\convprob}{\overset{p}{\to}}
\newcommand{\convdist}{\overset{w}{\to}}
\newcommand{\Expect}[1]{\mathbb{E}\left[ #1 \right]}
\newcommand{\Risk}[2][P]{\mathcal{R}_{#1}\left[ #2 \right]}
\newcommand{\Prob}[1]{\mathbb{P}\left( #1 \right)}
\newcommand{\iset}{\mathbf{i}}
\newcommand{\jset}{\mathbf{j}}
\newcommand{\myexp}[1]{\exp \{ #1 \}}
\newcommand{\abs}[1]{\left \lvert #1 \right \rvert}
\newcommand{\restr}[2]{\ensuremath{\left.#1\right|_{#2}}}
\newcommand{\ext}[1]{\widetilde{#1}}
\newcommand{\set}[1]{\left\{#1\right\}}
\newcommand{\seq}[1]{\set{#1}_{n \in \N}}
\newcommand{\Xsetotp}[2]{\langle #1, #2 \rangle}
\newcommand{\floor}[1]{\left\lfloor #1 \right\rfloor}
\newcommand{\Var}{\mathrm{Var}}
\newcommand{\Cov}{\mathrm{Cov}}
\newcommand{\Xsetiam}{\mathrm{diam}}

\newcommand{\emC}{C_n}
\newcommand{\emCpr}{C'_n}
\newcommand{\emCthick}{C^{\sigma}_n}
\newcommand{\emCprthick}{C'^{\sigma}_n}
\newcommand{\emS}{S^{\sigma}_n}
\newcommand{\estC}{\widehat{C}_n}
\newcommand{\hC}{\hat{C^{\sigma}_n}}
\newcommand{\vol}{\text{vol}}
\newcommand{\spansp}{\mathrm{span}~}
\newcommand{\1}{\mathbf{1}}

\newcommand{\Linv}{L^{\Xsetagger}}
\DeclareMathOperator*{\argmin}{argmin}
\DeclareMathOperator*{\argmax}{argmax}

\newcommand{\emF}{\mathbb{F}_n}
\newcommand{\emG}{\mathbb{G}_n}
\newcommand{\emP}{\mathbb{P}_n}
\newcommand{\F}{\mathcal{F}}
\newcommand{\D}{\mathcal{D}}
\newcommand{\R}{\mathcal{R}}
\newcommand{\Rd}{\Reals^d}
\newcommand{\Nbb}{\mathbb{N}}

%%% Vectors
\newcommand{\thetast}{\theta^{\star}}
\newcommand{\betap}{\beta^{(p)}}
\newcommand{\betaq}{\beta^{(q)}}
\newcommand{\vardeltapq}{\varDelta^{(p,q)}}


%%% Matrices
\newcommand{\X}{X} % no bold
\newcommand{\Y}{Y} % no bold
\newcommand{\Z}{Z} % no bold
\newcommand{\Lgrid}{L_{\grid}}
\newcommand{\Xsetgrid}{D_{\grid}}
\newcommand{\Linvgrid}{L_{\grid}^{\Xsetagger}}
\newcommand{\Lap}{{\bf L}}

%%% Sets and classes
\newcommand{\Xset}{\mathcal{X}}
\newcommand{\Sset}{\mathcal{S}}
\newcommand{\Hclass}{\mathcal{H}}
\newcommand{\Pclass}{\mathcal{P}}
\newcommand{\Leb}{L}

%%% Distributions and related quantities
\newcommand{\Pbb}{\mathbb{P}}
\newcommand{\Ebb}{\mathbb{E}}
\newcommand{\Qbb}{\mathbb{Q}}

%%% Operators
\newcommand{\Tadj}{T^{\star}}
\newcommand{\Xsetive}{\mathrm{div}}
\newcommand{\Xsetif}{\mathop{}\!\mathrm{d}}
\newcommand{\gradient}{\mathcal{D}}
\newcommand{\Hessian}{\mathcal{D}^2}
\newcommand{\dotp}[2]{\langle #1, #2 \rangle}

%%% Misc
\newcommand{\grid}{\mathrm{grid}}
\newcommand{\critr}{R_n}
\newcommand{\Xsetx}{\,dx}
\newcommand{\Xsety}{\,dy}
\newcommand{\Xsetr}{\,dr}
\newcommand{\Xsetxpr}{\,dx'}
\newcommand{\Xsetypr}{\,dy'}
\newcommand{\wt}[1]{\widetilde{#1}}
\newcommand{\wh}[1]{\widehat{#1}}
\newcommand{\ol}[1]{\overline{#1}}
\newcommand{\mc}[1]{\mathcal{#1}}
\newcommand{\spec}{\mathrm{spec}}
\newcommand{\LS}{\mathrm{LS}}
\newcommand{\LE}{\mathrm{LE}}

%%% Order of magnitude
\newcommand{\soom}{\sim}



\newtheoremstyle{alden}
{6pt} % Space above
{6pt} % Space below
{} % Body font
{} % Indent amount
{\bfseries} % Theorem head font
{.} % Punctuation after theorem head
{.5em} % Space after theorem head
{} % Theorem head spec (can be left empty, meaning `normal')

\theoremstyle{alden} 


\newtheoremstyle{aldenthm}
{6pt} % Space above
{6pt} % Space below
{\itshape} % Body font
{} % Indent amount
{\bfseries} % Theorem head font
{.} % Punctuation after theorem head
{.5em} % Space after theorem head
{} % Theorem head spec (can be left empty, meaning `normal')

\theoremstyle{aldenthm}
\newtheorem{theorem}{Theorem}
\newtheorem{conjecture}{Conjecture}
\newtheorem{lemma}{Lemma}
\newtheorem{example}{Example}
\newtheorem{corollary}{Corollary}
\newtheorem{proposition}{Proposition}
\newtheorem{assumption}{Assumption}
\newtheorem{remark}{Remark}


\theoremstyle{definition}
\newtheorem{definition}{Definition}[section]

\theoremstyle{remark}

\begin{document}
\title{Notes for Week 4/10/20 - 4/16/20}
\author{Alden Green}
\date{\today}
\maketitle

Let $\mc{M}$ be a closed, connected, smooth manifold without boundary of dimension $m$ embedded in $\Rd$. We give to $\mc{M}$ the Riemannian structure induced by the ambient space $\Rd$, and denote the volume form by $\mu$.  Let $P$ be a distribution supported on $\mc{M}$, with density $p$ with respect to the volume form $\mu$. Suppose we observe $n$ samples $X_1, \ldots, X_n$ drawn independently from $P$. In addition, we observe responses
\begin{equation*}
Y_i = f_0(X_i) + \varepsilon_i
\end{equation*}
where $\varepsilon_i$ are i.i.d $N(0,1)$ noise, and $f_0:\mc{M} \to \Reals$ is an unknown function we wish to estimate.

We construct a neighborhood graph $G_{n,r}$ over the data $X_1,\ldots,X_n$ as follows. For a kernel $K: \Rd \to [0,\infty)$ and radius $r > 0$, we let
\begin{equation*}
K_r(X_i,X_j) = K\biggl(\frac{\norm{X_i - X_j}_{\Rd}}{r}\biggr)
\end{equation*}
be the weight of the edge connecting $X_i$ and $X_j$. The graph $G_{n,r}$ has an associated Laplacian matrix $L_{n,r}$, which is defined by the action
\begin{equation*}
\Lap_{n,r}f(X_i) := \sum_{i = 1}^{n} \Bigl(f(X_i) - f(X_j)\Bigr) K_r(X_i,X_j)
\end{equation*}
We write $(\lambda_1(G_{n,r}),v_1(G_{n,r})),\ldots,(\lambda_n(G_{n,r}), v_n(G_{n,r}))$ for the $n$ eigenvalue/eigenvector pairs of $\Lap_{n,r}$, and adopt the usual convention of arranging the eigenvalues in ascending order, meaning $0 = \lambda_1(G_{n,r}) \leq \lambda_1(G_{n,r}) \leq \cdots \leq \lambda_1(G_{n,r})$. The Laplacian eigenmaps estimator is a nonparametric estimator which projects the data onto the span of the first $\kappa$ eigenvectors
\begin{equation*}
\wh{f}_{\LE} = \sum_{k = 1}^{\kappa} \dotp{Y}{v_k(G_{n,r})}_n v_k(G_{n,r})
\end{equation*}
where $1 \leq \kappa \leq n$ is tuning parameter. We would like to show that $f_0 \in C^1(\mc{M},Q)$ for a value of $Q$ fixed in $n$, then for an appropriate choice of $\kappa$
\begin{equation*}
\norm{\wh{f}_{\LE} - f_0}_n^2 \lesssim n^{-2/(2+m)}
\end{equation*}
with high probability. 

\section{Theory}
Recall that the squared bias of $\wh{f}_{\LE}$ is upper bounded
\begin{equation}
\label{eqn:squared_bias}
\biggl\|\Ebb\Bigl[\wh{f}_{\LE} \Big | {\bf X}\Bigr] - f_0\biggr\|_n^2 \leq \frac{f_0^T \Lap_{n,r} f_0}{n \lambda_{\kappa}(G_{n,r})}
\end{equation}
(We have already supplied a sufficient bound on the variance which is independent of the dimension of $\mc{M}$). Under certain assumptions, we shall prove upper and lower bounds on the graph Sobolev seminorm $f_0^T \Lap_{n,r} f_0$ and the graph Laplacian eigenvalue $\lambda_{\kappa}(G_{n,r})$, respectively. These assumptions are as follows:
\begin{enumerate}[label=(A\arabic*)]
	\item 
	\label{asmp:max_radius}
	The radius $r$ satisfies the bounds
	\begin{equation*}
	(m + 5)d_{\infty}(P,P_n)\leq r \leq \min\biggl\{1,\frac{i_0}{10},\frac{1}{\sqrt{mK}},\frac{R}{\sqrt{27m}}\biggr\}
	\end{equation*}
	Here, $R$ is an upper bound on the reach of $\mc{M}$, $\mc{K}$ is an upper bound on the absolute values of the sectional curvatures of $\mc{M}$, $i_0$ is the injectivity radius of $\mc{M}$. 
	\item
	\label{asmp:kernel_shape}
	The kernel function $K: [0,\infty) \to [0,\infty)$ is supported on $[0,1]$ and is $Q_K$-Lipschitz continuous on its support, and satisfies $K(0) = 1$.
	\item
	\label{asmp:density}
	The density function $p$ is a $Q_p$-Lipschitz continuous function on $\mc{M}$, and is bounded away from $0$ and $\infty$,
	\begin{equation*}
	0 < p_{\min} \leq p(x) \leq p_{\max} < \infty
	\end{equation*}
	for all $x \in \mc{M}$.
\end{enumerate}

The bounds we prove are as follows.
\begin{enumerate}[(I)]
	\item \textbf{Graph Sobolev semi-norm:} Assume~\ref{asmp:max_radius}-\ref{asmp:density}. There exists a constant $C$ such that the following statement holds: for any $f_0 \in C^1(\mc{M},Q)$ and any $r > 0$,
	\begin{equation}
	\label{eqn:graph_sobolev_seminorm}
	f_0^T \Lap_{n,r} f_0 \leq \frac{C}{\delta}  Q^2 n^2 r^{m + 2}
	\end{equation}
	with probability at least $1 - \delta$.
	\item \textbf{Laplacian eigenvalue:}
	Assume~\ref{asmp:max_radius}-\ref{asmp:density}. There exist constants $c,C,k_{\star}$ and $\beta$ such that the following statement holds: for all $\kappa \in \mathbb{N}$ satisfying $\kappa \geq k_{\star}$, and any $r$ satisfying
	\begin{equation}
	\label{eqn:graph_eigenvalue_1}
	C \frac{\log(n)^{p_d}}{n^{1/d}} < r < c \kappa^{-1/m}
	\end{equation}
	the graph Laplacian eigenvalue satisfies the lower bound
	\begin{equation}
	\label{eqn:graph_eigenvalue_2}
	\lambda_{\kappa}(G_{n,r}) \geq c n r^{m + 2} \kappa^{2/m}.
	\end{equation}
	with probability at least $1 - C n^{-\beta}$.
\end{enumerate}

\subsection{(I): Graph Sobolev semi-norm.}

We shall prove the following upper bound on the expectation of the graph Sobolev semi-norm. 
\begin{lemma}
	\label{lem:expected_graph_sobolev_seminorm_manifold}
	Suppose assumptions~\ref{asmp:max_radius}-\ref{asmp:density} are satisfied. Then there exists a constant $C_1 > 0$ such that for any $f \in C^1(\mc{M}, Q)$,
	\begin{equation}
	\label{eqn:expected_graph_sobolev_seminorm_manifold}
	\Ebb\Bigl[f_0^T \Lap_{n,r} f_0 \Bigr] \leq C_1 p_{\max} Q^2  n^2 r^{m + 2} 
	\end{equation}
	with probability at least $1 - n\exp(c_0 n r^m p_{\min})$. 
\end{lemma}

Note that \eqref{eqn:graph_sobolev_seminorm} then follows by Markov's inequality.

\subsection{(II): Graph Laplacian eigenvalue}
Let $\Delta_{p}$ be defined for smooth functions $f: \mc{M} \to \Reals$ as
\begin{equation*}
\Delta_pf = -\frac{1}{p} \mathrm{div}\bigl(\nabla(p^2 f)\bigr)
\end{equation*}
$\Delta_p$ has a point spectrum, and we denote its eigenvalues by $\lambda_1(\mc{M}) \leq \lambda_2(\mc{M}) \leq \cdots$. 

In Lemma~\ref{lem:graph_eigenvalue_manifold}, we will prove a lower bound on the graph Laplacian eigenvalues $\lambda_k(G_{n,r})$ by the eigenvalues $\lambda_k(\mc{M})$. The lower bound will be a simple corollary of the developments in \citep{trillos2019}, which we summarize in~\ref{subsec:results_of_others}. 

\begin{lemma}[Lower bound on graph eigenvalue $\lambda_k(G_{n,r})$]
	\label{lem:graph_eigenvalue_manifold}
	Assume~\ref{asmp:max_radius}-\ref{asmp:density}. Then there exists some $k_{\star} \in \mathbb{N}$ such that for any $k \geq k_{\star}$, the following statement holds: if
	\begin{equation*}
	8 C_3 A \biggl(\frac{\log(n)^{p_d}}{n^{1/d}}\biggr) < r < \min\biggl\{\frac{C_2}{C_5 k^{1/m}}, \frac{1}{8C_3 L(p)}, \frac{1}{8C_3k^{1/m}}, \frac{1}{8c_3\sqrt{\mc{K}}}\biggr\}
	\end{equation*}
	then
	\begin{equation*}
	\lambda_k(G_{n,r}) \geq \frac{c_1}{2} n r^{m + 2} k^{2/m}
	\end{equation*}
	with probability at least $1 - C_4 n^{-\beta}$.
\end{lemma}

\subsection{(III): Higher order graph Sobolev semi-norms}

We have the following bound on $f^T \Lap_{n,r}^2 f$ for functions $f \in C^2(\mc{M})$.
\begin{lemma}
	\label{lem:graph_sobolev_seminorm_2}
	Assume~\ref{asmp:max_radius}-\ref{asmp:density}. Let $f \in C^2(\mc{M},Q)$. Then there exists constants $C_7$ and $c_2$ which depend only on $m,\mc{K},R$ and $i_0$ such that
	\begin{equation*}
	f^T \Lap_{n,r}^2 f \leq 2 C_7^2 Q^2 n^3 r^{2(m + 2)} p_{\max}^2
	\end{equation*}
	with probability at least $1 - 2n\exp(-2cnr^{m + 2})$. 
\end{lemma}

A similar bound holds on $f^T \Lap_{n,r}^3 f$ for functions $f \in C^3(\mc{M})$.
\begin{lemma}
	\label{lem:graph_sobolev_seminorm_3}
	Assume~\ref{asmp:max_radius}-\ref{asmp:density}. Let $f \in C^3(\mc{M},Q)$ and $p \in C^2(\mc{M},Q)$. Then there exist constants $C_8$ and $c_3$ which depend only on $m,\mc{K},R,i_0$ and $Q_K$ such that
	\begin{equation}
	\label{eqn:graph_sobolev_seminorm_3}
	f^T \Lap_{n,r}^3 f \leq \frac{C_8}{\delta} Q^2 n^4 r^{3(m + 2)} p_{\max}^3 (p_{\max} + 1)
	\end{equation}
	with probability at least $1 - 2n\exp(-c_2nr^{4 + m}) - n\exp(4 n\nu_mr^mp_{\min}/3) - \delta$. 
\end{lemma}

When we turn to $f^T \Lap_{n,r}^s f$ for $s \geq 4$, things become difficult, and at present I cannot obtain results analogous to~\ref{lem:graph_sobolev_seminorm_2} and~\ref{lem:graph_sobolev_seminorm_3}. I detail what I find difficult---in the $s = 4$ case---in Subsection~\ref{subsec:order4_is_difficult}.

Finally, we note the following bound on the graph Sobolev seminorm when $f \in H^1(P,Q)$, a direct corollary of Lemma 6 in \citep{trillos2019} (c.f. Lemma 3.3 \citep{burago2014}) plus Markov's inequality.

\begin{lemma}
	Assume~\ref{asmp:max_radius}-\ref{asmp:density}. Then there exists a universal constant $C > 0$ such that the following statement holds for any $f \in H^1(P)$:
	\begin{equation*}
	f^T \Lap_{n,r} f \leq \frac{1}{\delta}\bigl(1 + Q_p r\bigr) \bigl(1 + CmKr^2\bigr) \sigma_K r^{m + 2} |f|_{H^1(P)}
	\end{equation*}
	with probability at least $1 - \delta$. 
\end{lemma}




\section{Proof of Theorems and Major Lemmas}
We will often make use of the following geometric estimates, which hold under assumption~\ref{asmp:max_radius}. These help us to convert from integrals over $\mc{M}$ to integrals over a Euclidean space. First
\begin{equation}
\label{eqn:geodisic_euclidean_distance}
d_{\mc{M}}(y,x) \leq \norm{x - y}_{\Rd} + \frac{8}{R^2} \norm{x - y}_{\Rd}^3
\end{equation} 
and additionally
\begin{equation}
\label{eqn:manifold_ball_volume}
\abs{\mu(B(x,r)) - \nu_m r^m} \leq C_0r^{m + 2}
\end{equation}
See ((3.3) and (3.2) of (Calder and Garcia-Trillos 19)).

\subsection{Proof of Lemma~\ref{lem:expected_graph_sobolev_seminorm_manifold}}
In the following manipulations we invoke first the Holder property of $f$, then the upper bound in~\ref{eqn:geodisic_euclidean_distance}, and finally Lemma~\ref{lem:max_degree} with $\delta = 2$ to get
\begin{align*}
f^T \Lap_{n,r}f  & = \sum_{i,j = 1}^{n} \Bigl(f(X_i) - f(X_j)\Bigr)^2 K\biggl(\frac{\norm{X_i - X_j}_{\Rd}}{r}\biggr) \\
& \leq Q^2 \sum_{i,j = 1}^{n} \Bigl(d_{\mc{M}}(X_i,X_j)\Bigr)^2 K\biggl(\frac{\norm{X_i - X_j}_{\Rd}}{r}\biggr) \\
& \leq 2 Q^2 \sum_{i,j = 1}^{n} \Bigl(\norm{X_i - X_j}_{\Rd}^2 + \frac{64}{R^4} \norm{X_i - X_j}_{\Rd}^6\Bigr) K\biggl(\frac{\norm{X_i - X_j}_{\Rd}}{r}\biggr) \\ 
& \leq 2 Q^2 \Bigl(r^2 + \frac{64}{R^4} r^6\Bigr) n \deg_{\max}(G_{n,r}) \\
& \leq 4 Q^2 p_{\max}\Bigl(r^2 + \frac{64}{R^4} r^6\Bigr) \Bigl(\nu_mr^m + C_0r^{m + 2}\Bigr) n^2  
\end{align*}
with the last inequality holding with probability at least $1 - n\exp(-\frac{4}{3}n\nu_mr^mp_{\min})$. Lemma~\ref{lem:expected_graph_sobolev_seminorm_manifold} then follows from appropriate choice of constants $c_0$ and $C_1$.
\subsection{Proof of Lemma~\ref{lem:graph_eigenvalue_manifold}}

Applying Theorem~\ref{thm:trillos19_2}---which gives an estimate on the $\infty$-optimal transport distance between $P$ and $P_n$---to Theorem~\ref{thm:trillos19_1}---which gives a lower bound on $\lambda_k(G_{n,r})$ in terms of $\lambda_k(\mc{M})$ and $d_{\infty}(P,P_n)$--- we have that if
\begin{equation*}
2C_3 A \frac{\log(n)^{p_d}}{n^{1/d}} \leq r < \min\biggl\{\frac{C_2}{\sqrt{\lambda_k(\mc{M})}}, \frac{1}{8C_3 L(p)}, \frac{1}{8C_3\sqrt{\lambda_k(\mc{M})}}, \frac{1}{8C_3\sqrt{\mc{K}}}\biggr\}
\end{equation*}
then
\begin{equation*}
\lambda_k(G_{n,r}) \geq \frac{1}{2} nr^{m+2}\lambda_k(\mc{M})
\end{equation*}
with probability at least $1 - C_4 n^{-\beta}$. Lemma~\ref{lem:graph_eigenvalue_manifold} follows from replacing all instances of $\lambda_k(\mc{M})$ in the above expression with estimates given by Weyl's Law (Theorem~\ref{thm:weyl}).

\subsection{Proof of Lemma~\ref{lem:graph_sobolev_seminorm_2}}
Note that
\begin{equation*}
f^T \Lap_{n,r}^2 f = \sum_{i = 1}^{n} \Bigl(\Lap_{n,r}f(X_i)\Bigr)^2
\end{equation*}
Lemma~\ref{lem:graph_sobolev_seminorm_2} follows directly from the pointwise estimate Lemma~\ref{lem:pointwise_laplacian_ub} upon taking a union bound over $x = X_1,\ldots,X_n$.

\subsection{Proof of Lemma~\ref{lem:graph_sobolev_seminorm_3}}
We will consider $\Lap_{n,r}f(X_i)$ as an estimate of $\sigma_K \Delta_pf(X_i)$ (when appropriately scaled), and show that this deviation is small when $r$ is sufficiently large.  To begin, writing $\Lap_{n,r}f(X_i) = \sigma_{K}\Delta_{p}f(X_i)nr^{m + 2} + \Lap_{n,r}f(X_i) - \sigma_{K}\Delta_{p}f(X_i)nr^{m+2}$ inside the definition of the 3rd-order graph Sobolev seminorm, we obtain
\begin{align*}
f^T \Lap_{n,r}^3 f & = \sum_{i,j = 1}^{n} \Bigl(\Lap_{n,r}f(X_i) - \Lap_{n,r}f(X_j)\Bigr)^2 K_r(X_i,X_j) \\
& \leq 6 \sum_{i,j = 1}^{n} \Bigl(\Lap_{n,r}f(X_i) - \sigma_{K}\Delta_{p}f(X_i)nr^{m+2}\Bigr)^2 K_r(X_i,X_j)~+ \\
&~~3 \sigma_K^2 n^2 r^{2(m + 2)} \sum_{i,j = 1}^{n} \Bigl(\Delta_pf(X_i) - \Delta_pf(X_j)\Bigr)^2 K_r(X_i,X_j)
\end{align*}
Applying Theorem~\ref{thm:calder19_1} with $\delta = r$, and Lemma~\ref{lem:max_degree} with $\delta = 1$, we have
\begin{align*}
\sum_{i,j = 1}^{n} \Bigl(\Lap_{n,r}f(X_i) - \sigma_{K}\Delta_{p}f(X_i)nr^{m+2}\Bigr)^2  K_r(X_i,X_j) & \leq  C_6^2 \bigl([f]_{1;\mc{M}} + \norm{f}_{C^3(\mc{M})} + 1\bigr)^2 n^2r^{2m + 6} \sum_{i,j = 1}^{n} K_r(X_i,X_j) \\
& \leq  C_6^2 \bigl([f]_{1;\mc{M}} + \norm{f}_{C^3(\mc{M})} + \norm{p}_{C^2(\mc{M})}\bigr)^2 n^3r^{2m + 6} \deg_{\max}(G_{n,r}) \\
& \leq 4 C_6^2 \bigl([f]_{1;\mc{M}} + \norm{f}_{C^3(\mc{M})} + \norm{p}_{C^2(\mc{M})}\bigr)^2 n^4 r^{2m + 6} \Bigl[\nu_mr^m + C_0r^{m + 2}\Bigr] p_{\max} \\
& \leq C p_{\max}\bigl(Q + p_{\max}\bigr)^2n^4 r^{3m + 6},
\end{align*}
with probability at least $1 - 2n\exp(-c_2nr^{4 + m}) - n\exp(4 n\nu_mr^mp_{\min}/3)$. 

It remains to upper bound $(\Delta_pf)^T \Lap_{n,r} (\Delta_pf)$. Rewriting
\begin{equation*}
\Delta_pf = -\bigl(\nabla p^T \nabla f + p \Delta_{\mc{M}} f\bigr)
\end{equation*}
we observe that since $f \in C^3(\mc{M}, Q)$ and $p \in C^2(\mc{M}, p_{\max})$, $\Delta_pf \in C^1(\mc{M},4Qp_{\max})$. By Lemma~\ref{lem:expected_graph_sobolev_seminorm_manifold} we therefore have
\begin{equation*}
(\Delta_pf)^T \Lap_{n,r} (\Delta_pf) \leq \frac{16}{\delta} C_1 n^2 r^{m + 2} Q^2 p_{\max}^4
\end{equation*}
with probability at least $1 - \delta$. The claim of Lemma~\ref{lem:graph_sobolev_seminorm_3} then follows after an appropriate choice of constant in~\eqref{eqn:graph_sobolev_seminorm_3}.


\section{Additional Theory}

\subsection{Estimates on Graph Eigenvalues}
\label{subsec:results_of_others}


\begin{theorem}[(Part of) Theorem 4 of \citep{trillos2019}]
	\label{thm:trillos19_1}
	There exist constants $C_2$ and $C_3$ which depend only on $m, p_{\min}, p_{\max}, L(p)$ and $K$ such that if 
	\begin{equation*}
	\sqrt{\lambda_k(\mc{M})} r < C_2,
	\end{equation*}
	then,
	\begin{equation*}
	\lambda_k(G_{n,r}) \geq n r^{m + 2}\lambda_k(\mc{M}) \Biggl[1 - C_3\biggl(L(p)r + \frac{d_{\infty}(P,P_n)}{r} + \sqrt{\lambda_k(\mc{M})}r + \mc{K}r^2\biggr)\Biggr]
	\end{equation*}
\end{theorem}

In order to make use of Theorem~\ref{thm:trillos19_1}, we need an upper bound on $d_{\infty}(P,P_n)$. Theorem~\ref{thm:trillos19_2} provides a probabilistic estimate.

\begin{theorem}[Theorem 2 of \citep{trillos2019}]
	\label{thm:trillos19_2}
	Let $P$ satisfy~\ref{asmp:density}. Then for any $\beta > 1$ and every $n \in \mathbb{N}$, there exists a transport map $T_n: \mc{M} \to {\bf X}$ and a constant $A$ such that
	\begin{equation*}
	\sup_{x \in \mc{M}} d_{\infty}(x,T_n(x)) \leq A \cdot 
	\begin{cases*}
	\frac{(\log(n))^{3/4}}{n^{1/2}},& ~~\textrm{if $m = 2$,} \\
	\biggl(\frac{\log(n)}{n}\biggr)^{1/m},& ~~\textrm{if $m \geq 3$,}
	\end{cases*}
	\end{equation*}
	with probability at least $1 - C_4 n^{-\beta}$. The constant $A$ depends only on $\mc{K},i_0,m,\mu(\mc{M}), \alpha$ and $\beta$, and the constant $C_4$ depends only on $\mc{K},i_0,m,\mu(\mc{M})$.
\end{theorem}

\subsection{Weyl's Law}

Although Weyl's Law is traditionally stated with respect to the unweighted Laplace-Beltrami operator, the same asymptotics apply to $\Delta_{p}$ when $p$ is bounded away from $0$ and $\infty$ on $\mc{M}$. Let $N(\lambda)$ count the number of eigenvalues of $\mc{M}$ which are less than $\lambda$.
\begin{theorem}[Weyl's Law]
	\label{thm:weyl}
	Assume~\ref{asmp:density}. If $\mc{M}$ is a compact connected oriented manifold then
	\begin{equation*}
	N(\lambda) \sim \lambda^{m/2}
	\end{equation*}
	or, equivalently,
	\begin{equation*}
	\lambda_k(\mc{M}) \sim k^{2/m}.
	\end{equation*}
\end{theorem}

\subsection{Pointwise estimates on graph Laplacians}

We give some pointwise estimates on $\Lap_{n,r}f(x)$, many of which are reproduced from \citep{calder2019}. Define
\begin{equation*}
\Lap_{P,r}f(x) := \frac{1}{r^{m + 2}} \int_{B(x,r)} K_r(y,x) \Bigl( f(x) - f(y) \Bigr) p(y) \,d\mu(y)
\end{equation*}
to be a nonlocal Laplacian operator, which acts as an intermediary between $\Lap_{n,r}$ and $\Delta_p$.

\begin{theorem}[Theorem 3.3 of \citep{calder2019}]
	\label{thm:calder19_1}
	Assume~\ref{asmp:max_radius}-\ref{asmp:density}. Let $f \in C^3(\mc{M})$ and $p \in C^2(\mc{M})$. There exist constants \textcolor{red}{$C_6$} and $c_2$ such that for any $r \leq \delta \leq r^{-1}$, 
	\begin{equation*}
	\max_{1 \leq i \leq n} \abs{\Lap_{n,r}f(X_i) - \sigma_{K} \Delta_{p} f(X_i) nr^{m + 2}} \leq C_6 \bigl([f]_{1;\mc{M}} + \norm{f}_{C^3(\mc{M})} + \norm{p}_{C^2(\mc{M})}\bigr) \delta  nr^{m + 2} 
	\end{equation*}
	with probability at least $1 - 2n\exp(-c\delta^2 nr^{m+2})$.
\end{theorem}



\begin{lemma}[Lemma 3.4 of \citep{calder2019}]
	\label{lem:calder19}
	Assume~\ref{asmp:max_radius}-\ref{asmp:density}. Let $f \in C^1(\mc{M})$. Then for $x \in \mc{M}$ and $r \leq \delta \leq r^{-1}$,
	there exists constants $C_6$ and $c_2$ which depend only on $m,\mc{K},R$ and $i_0$ such that
	\begin{equation*}
	\Pbb\biggl[\Bigl|\frac{1}{nr^{m + 2}} \Lap_{n,r}f(x) - \Lap_{P,r}f(x)\Bigr| \geq C_6 [f]_{1,B(x,2r)} \delta \biggr] \leq 2\exp(-c_2\delta^2 n r^{m + 2})
	\end{equation*}
\end{lemma}

Most analysis on $\Lap_{n,r}f(x)$ provide pointwise estimates assuming $f \in C^{q + 2}(\mc{M})$ for some $q > 0$. We shall instead assume only $f \in C^2(\mc{M})$, and content ourselves with merely an upper bound on $|\Lap_{n,r}f(x)|$.
\begin{lemma}
	\label{lem:pointwise_laplacian_ub}
	Assume~\ref{asmp:max_radius}-\ref{asmp:density}. Let $f \in C^2(\mc{M})$. Then for $x \in \mc{M}$ and $r \leq \delta \leq r^{-1}$, there exists constants $C_7$ and $c_2$ which depend only on $m,\mc{K},R$ and $i_0$ such that
	\begin{equation*}
	\Bigl|\Lap_{n,r}f(x)\Bigr| \leq C_7 nr^{m + 2}\Bigl([f]_{1;B(x,2r)}\delta + p_{\max} \nu_d[f]_{1,B(x,r)} r + \norm{\nabla f(x)} +  \norm{f}_{C^2(B(x,r))}\Bigr)
	\end{equation*}
	with probability at least $1 - 2\exp(-c_2\delta^2 n r^{m + 2})$.
\end{lemma}
\begin{proof}
	We shall follow the proof of Theorem 3.3 of~\citep{calder2019}, deviating when necessary to fit our differing assumptions and ultimate goal. Let
	\begin{equation*}
	\Lap_{P,r}^{(i)}f(x) = \frac{1}{r^{m + 2}} \int_{B(x,r)} K\biggl(\frac{d_{\mc{M}}(y,x)}{r}\biggr)\bigl(f(x) - f(y)\bigr) p(y) \,d\mu(y)
	\end{equation*} 
	be an intrinsic analogue to $\Lap_{P,r}$. It follows from~\eqref{eqn:geodisic_euclidean_distance} and~\ref{asmp:kernel_shape} that 
	\begin{align}
	\abs{\Lap_{P,r}^{(i)}f(x) - \Lap_{P,r}f(x)} & \leq \frac{1}{r^{2 + m}} \int_{B(x,r)} \Biggl|K(\frac{d_{\mc{M}}(x,y)}{r}) - K(\frac{d_{\mc{M}}(x,y)}{r})\Biggr| \bigl|f(x) - f(y)\bigr| p(y) \,d\mu(y) \nonumber \\ 
	& \leq \frac{p_{\max}}{r^{m}} \int_{B(x,r)} \bigl|f(x) - f(y)\bigr| \,d\mu(y)\\
	& \leq p_{\max} \nu_d[f]_{1,B(x,r)} r \label{eqn:pointwise_laplacian_ub}
	\end{align}
	By assumption $r < i_0$, and therefore the exponential map $\exp_x: B(0,r) \subset \mc{T}_{x}(\mc{M}) \to \mc{Mc}$ is a diffeomorphism. For $v \in B(0,r)$ let $w(v) = f(\exp_x(v))$ and $\rho(v) = p(\exp_x(v))$, i.e. express $f$ and $p$ in terms of normal Riemmanian coordinates,  and let $J_x(v)$ be the Jacobian of $\exp_x$ at $v$.  Then
	\begin{align*}
	\Lap_{P,r}^{(i)}f(x) & = -\frac{1}{r^{m + 2}} \int_{B(0,r)} K\biggl(\frac{\norm{v}}{r}\biggr)\bigl(w(v) - w(0)\bigr) \rho(v) J_x(v) \,d\mu(v) \\
	& = -\frac{1}{r^2} \int_{B(0,1)} K(\norm{v})  \bigl(w(rv) - w(0)) \rho(rv) J_x(rv) \,d\mu(v).
	\end{align*}
	Now we plug in the Taylor expansions $w(rv) = w(0) + \nabla w(0) \cdot v + O(\norm{w}_{C^2(B(0,r))}r^2)$, $\rho(rv) = \rho(0) + O(r)$ and $J_x(rv) = 1 + O(r^2)$ to obtain
	\begin{align}
	\Lap_{P,r}^{(i)}f(x) & = -\frac{1}{r^2} \int_{B(0,1)} K(\norm{v})  \Bigl(r \nabla w(0)^T v + O\bigl(\norm{w}_{C^2(B(0,r))}r^2\bigr)\Bigr) \Bigl(\rho(0) + O(r)\Bigr) \Bigl(1 + O(r^2)\Bigr) \,d\mu(v) \nonumber \\
	& = O\Bigl(\norm{\nabla w(0)} +  \norm{w}_{C^2(B(0,r))} \Bigr) \label{eqn:pointwise_laplacian_ub_2}
	\end{align}
	The claim then follows by combining~\eqref{eqn:pointwise_laplacian_ub},~\eqref{eqn:pointwise_laplacian_ub_2} and Lemma~\ref{lem:calder19}.
\end{proof}

\subsection{Degree bounds}

Lemma~\ref{lem:max_degree} follows from the multiplicative form of Hoeffding's inequality along with~\eqref{eqn:manifold_ball_volume} and~\ref{asmp:kernel_shape}.
\begin{lemma}
	\label{lem:max_degree}
	Assume~\ref{asmp:max_radius}-\ref{asmp:density}. Then
	\begin{equation*}
	\deg_{\max}(G_{n,r}) \leq (1 + \delta)\Bigl[\nu_mr^m + C_0r^{m + 2}\Bigr] p_{\max} n
	\end{equation*}
	with probability at least $1 - n\exp(-\delta^2 n \nu_m r^m p_{\min}/3)$.
\end{lemma}

\subsection{Explaining what is difficult when $s = 4$.}
\label{subsec:order4_is_difficult}
Let me start by giving a rough summary of the difficulty, and then go into more detail .We would like to show
\begin{equation}
\label{eqn:order4_is_difficult_1}
f^T \Lap_{n,r}^4 f = \sum_{i = 1}^{n} \Bigl(\Lap_{n,r}^2f(X_i)\Bigr)^2 \lesssim n^{5}r^{8(m + 2)}
\end{equation}
for $f \in C^4(\mc{M})$.
Note that
\begin{equation*}
\Lap_{P,r}f(x) = \int_{B(x,r) \cap \mc{M}} \bigl(f(x) - f(y)\bigr) \,dP(y)
\end{equation*}
is (up to scaling by a factor of $nr^{m+2}$)the expectation of $\Lap_{n,r}f(x)$ . It is possible to show a sufficiently tight bound on $|\Lap_{n,r}^2f(x) - nr^{2+m}\Lap_{P,r}^2f(x)|$ when $r \gg n^{-1/(6 + m)}$, thus handling the variance. To handle the bias, we can show that
\begin{equation*}
\Lap_{P,r}f(x) = nr^{m + 2} \sigma_{K} \Delta_pf(x) + \mc{I}f(x)
\end{equation*}
where $\Delta_p$ is the weighted Laplace-Beltrami operator on $\mc{M}$, and $\mc{I}f$ is an error term satisfying $\mc{I}f(x) = O(r^{m + 3})$. This bound on the error term is sufficient for $s = 1$ up to $s = 3$, however, when $s \geq 4$ it is insufficient (check by plugging back in to~\eqref{eqn:order4_is_difficult_1}). In the Euclidean case, we had that $\mc{I}f(x)$ itself belong to $C^1$ and thus could argue $(\Lap_{P,r} \mc{I}f)(x) \ll \mc{I}f(x)$. In the manifold case, I no longer know how to do this.

The longer version: we can make the following progress towards upper bound $f^T \Lap_{n,r}^4 f$:
\begin{align*}
f^T \Lap_{n,r}^4 f & \leq n \max_{1 \leq i \leq n} \abs{\Lap_{n,r}^2 f(X_i)}^2 \\
& \leq n \max_{1 \leq i \leq n} \abs{\Lap_{n,r}(\Lap_{n,r} - nr^{m + 2}\Lap_{P,r})f(X_i)}^2 + n^2 r^{m + 2} \max_{1 \leq i \leq n} \abs{(\Lap_{n,r} - nr^{m + 2}\Lap_{P,r})\Lap_{P,r}f(X_i)}^2 +~\\
& ~~n^4 r^{4m + 8} \max_{1 \leq i \leq n} \abs{\Lap_{P,r}^2f(X_i)}
\end{align*}
We can upper bound the first term as follows:
\begin{align*}
\abs{\Lap_{n,r}(\Lap_{n,r} - nr^{m + 2}\Lap_{P,r})f(x)}^2 & \leq \max_{1 \leq j \leq n} \abs{(\Lap_{n,r} - nr^{m + 2}\Lap_{P,r})f(X_j)} \cdot \sum_{j = 1}^{n} K\Bigl(\frac{\norm{x - X_j}_{\Rd}}{r}\Bigr) \\
& \leq C nr^m \max_{1 \leq j \leq n} \abs{(\Lap_{n,r} - nr^{m + 2}\Lap_{P,r})f(X_j)} \\
& \leq C n^2r^{2m + 2} \delta 
\end{align*}
where the last inequality follows from Bernstein's inequality and holds with probability at least on the order of $1 - \exp(-nr^{2 + m}\delta^2)$. Choosing $\delta \asymp r^2$ gives the desired rate, assuming $n \gg r^{6 + m}$.

We can use a similar argument to upper bound the second term.

The third term---$\Lap_{P,r}^2f(X_i)$---is what poses the challenge. We want to convert the integral on $\mc{M}$ to an integral on a Euclidean space using exponential maps. Letting $\exp_x:T_x(\mc{M}) \to \mc{M}$ be the exponential map, letting $\wt{B}$ satisfy $\exp_x(\wt{B}) = B(x,r) \cap \mc{M}$, associating the tangent plane $T_x(\mc{M})$ with $\Reals^m$, and writing $g_x(v) = \norm{x - \exp_x(v)}_{\Rd} - \norm{v}_{\Reals^m}$ we have
\begin{align*}
\Lap_{P,r}f(x) & = \int_{\mc{M}} (f(x) - f(y)) K\Bigl(\frac{\norm{y - x}_{\Rd}}{r}\Bigr) p(y) \,d\mu(y) \\
& = \int_{\wt{B}} (w(0) - w(v)) K\Bigl(\frac{\norm{x - \exp_x(v)}_{\Rd}}{r}\Bigr) \rho(v) J_v(x) \,dv \\
& = \int_{\wt{B}} (w(0) - w(v)) K\Bigl(\frac{\norm{v}_{\Reals^m} + g_x(v)}{r}\Bigr) \rho(v) J_v(x) \,dv \\
& = \int_{B(0,r) \subset \Reals^m} (w(0) - w(v)) K\Bigl(\frac{\norm{v}_{\Reals^m}}{r}\Bigr) \rho(v) J_v(x) \,dv +~~\\
&~~ \underbrace{\int_{\wt{B}} (w(0) - w(v)) \biggl[K\Bigl(\frac{\norm{v}_{\Reals^m} + g_x(v)}{r}\Bigr) - K\Bigl(\frac{\norm{v}_{\Reals^m}}{r}\Bigr)\biggr] \rho(v) J_v(x) \,dv}_{\mc{I}(x)}
\end{align*}
The first of the two integrals is now an integral over a ball in a Euclidean domain, and is not a problem. The second integral is tricky. The only way I know how to analyze $\mc{I}(x)$ is to use the following facts: first, $\abs{w(v) - w(0)} \lesssim r$, second that $g_x(v) \lesssim r^3$ for all $\exp_x(v) \in B(0,r) \cap \mc{M}$, third that $\abs{\mu(B(x,r) \cap {\mc{M}}) - \nu_m r^m} \lesssim r^{m + 2}$, and fourth that $\abs{J_x(v)} \lesssim 1 + r^2$. Together, these give
\begin{align*}
\int_{\wt{B}} (w(0) - w(v)) \biggl[K\Bigl(\frac{\norm{v}_{\Reals^m} + g_x(v)}{r}\Bigr) - K\Bigl(\frac{\norm{v}_{\Reals^m}}{r}\Bigr)\biggr] \rho(v) J_v(x) \,dv  & \lesssim r \Bigl(\mu\bigl(B(x,r + r^3) \cap {\mc{M}} \bigr) - \mu\bigl(B(x,r) \cap {\mc{M}}\bigr)\Bigr)  \\
& \lesssim r^{m + 3}.
\end{align*}
But this only gives $\Lap_{P,r}^2f(x) \lesssim r^{2m + 3}$, which is not nearly enough. The problem is that I am not using any smoothness properties of $\mc{I}(x)$ in my upper bound on $\Lap_{P,r}\mc{I}f(x)$. I suspect it may have such properties, related to the smoothness assumptions we place on $\mc{M}$, but proving it seems non-trivial.




\section{Notation}

\begin{itemize}
	\item For any $x,y \in \mc{M}$ we write $d_{\mc{M}}(x,y)$ for the geodesic distance between $x$ and $y$, that is the length of the shortest path connecting $x$ and $y$. 
	\item For measures $P$ and $Q$ on $\mc{M}$, we write $d_{\infty}(P,Q)$ for the $\infty$-optimal transport distance between $P$ and $Q$. Formally,
	\begin{equation*}
	d_{\infty}(P,Q) = \inf\biggl\{ \mathrm{esssup}_{\gamma}\Bigl\{\abs{x - y}: x,y \in \mc{M} \Bigr\} : \gamma \in \Gamma(P,Q) \biggr\}
	\end{equation*}
	where $\Gamma(P,Q)$ is the set of joint distributions on $\mc{M} \times \mc{M}$ with marginals $P$ and $Q$.
	
\end{itemize}

\bibliographystyle{plainnat}
\bibliography{../../graph_testing_bibliography} 

\end{document}