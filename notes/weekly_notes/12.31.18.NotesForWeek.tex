\documentclass{article}
\usepackage{amsmath}
\usepackage{amsfonts, amsthm, amssymb}
\usepackage{bm}
\usepackage{graphicx}
\usepackage[colorlinks]{hyperref}
\usepackage[parfill]{parskip}
\usepackage{algpseudocode}
\usepackage{algorithm}
\usepackage{enumerate}

\usepackage{natbib}
\renewcommand{\bibname}{REFERENCES}
\renewcommand{\bibsection}{\subsubsection*{\bibname}}

\makeatletter
\newcommand{\leqnomode}{\tagsleft@true}
\newcommand{\reqnomode}{\tagsleft@false}
\makeatother

\newcommand{\eqdist}{\ensuremath{\stackrel{d}{=}}}
\newcommand{\Graph}{\mathcal{G}}
\newcommand{\Reals}{\mathbb{R}}
\newcommand{\Identity}{\mathbb{I}}
\newcommand{\distiid}{\overset{\text{i.i.d}}{\sim}}
\newcommand{\convprob}{\overset{p}{\to}}
\newcommand{\convdist}{\overset{w}{\to}}
\newcommand{\Expect}[1]{\mathbb{E}\left[ #1 \right]}
\newcommand{\Risk}[2][P]{\mathcal{R}_{#1}\left[ #2 \right]}
\newcommand{\Var}[1]{\mathrm{Var}\left( #1 \right)}
\newcommand{\Prob}[1]{\mathbb{P}\left( #1 \right)}
\newcommand{\iset}{\mathbf{i}}
\newcommand{\jset}{\mathbf{j}}
\newcommand{\myexp}[1]{\exp \{ #1 \}}
\newcommand{\norm}[1]{\left\lVert#1\right\rVert}
\newcommand{\dotp}[2]{\langle #1 , #2 \rangle}
\newcommand{\abs}[1]{\left \lvert #1 \right \rvert}
\newcommand{\restr}[2]{\ensuremath{\left.#1\right|_{#2}}}
\newcommand{\defeq}{\overset{\mathrm{def}}{=}}
\newcommand{\convweak}{\overset{w}{\rightharpoonup}}
\newcommand{\dive}{\mathrm{div}}

\newcommand{\emC}{C_n}
\newcommand{\emCpr}{C'_n}
\newcommand{\emCthick}{C^{\sigma}_n}
\newcommand{\emCprthick}{C'^{\sigma}_n}
\newcommand{\emS}{S^{\sigma}_n}
\newcommand{\estC}{\widehat{C}_n}
\newcommand{\hC}{\hat{C^{\sigma}_n}}
\newcommand{\vol}{\text{vol}}
\newcommand{\Bal}{\textrm{Bal}}
\newcommand{\Cut}{\textrm{Cut}}
\newcommand{\Ind}{\textrm{Ind}}
\newcommand{\set}[1]{\left\{#1\right\}}
\newcommand{\seq}[1]{\set{#1}_{n \in \N}}
\newcommand{\Perp}{\perp \! \! \! \perp}


\newcommand{\Linv}{L^{\dagger}}
\newcommand{\tr}{\text{tr}}
\newcommand{\h}{\textbf{h}}
% \newcommand{\l}{\ell}
\newcommand{\x}{\textbf{x}}
\newcommand{\y}{\textbf{y}}
\newcommand{\bl}{\bm{\ell}}
\newcommand{\bnu}{\bm{\nu}}
\newcommand{\Lx}{\mathcal{L}_X}
\newcommand{\Ly}{\mathcal{L}_Y}
\DeclareMathOperator*{\argmin}{argmin}


\newcommand{\emG}{\mathbb{G}_n}
\newcommand{\A}{\mathcal{A}}
\newcommand{\F}{\mathcal{F}}
\newcommand{\G}{\mathcal{G}}
\newcommand{\X}{\mathcal{X}}
\newcommand{\Rd}{\Reals^d}
\newcommand{\N}{\mathbb{N}}
\newcommand{\E}{\mathcal{E}}

\newtheoremstyle{alden}
{6pt} % Space above
{6pt} % Space below
{} % Body font
{} % Indent amount
{\bfseries} % Theorem head font
{.} % Punctuation after theorem head
{.5em} % Space after theorem head
{} % Theorem head spec (can be left empty, meaning `normal')

\theoremstyle{alden} 
\newtheorem{definition}{Definition}[section]

\newtheoremstyle{aldenthm}
{6pt} % Space above
{6pt} % Space below
{\itshape} % Body font
{} % Indent amount
{\bfseries} % Theorem head font
{.} % Punctuation after theorem head
{.5em} % Space after theorem head
{} % Theorem head spec (can be left empty, meaning `normal')

\theoremstyle{aldenthm}
\newtheorem{theorem}{Theorem}
\newtheorem{conjecture}{Conjecture}
\newtheorem{lemma}{Lemma}
\newtheorem{example}{Example}
\newtheorem{corollary}{Corollary}
\newtheorem{proposition}{Proposition}
\newtheorem{assumption}{Assumption}

\theoremstyle{remark}
\newtheorem{remark}{Remark}

\begin{document}
	
\title{Notes for Week of 12/24/18 - 12/31/18}
\author{Alden Green}
\date{\today}
\maketitle

\section{SETUP}
\paragraph{Data model.}

We are given two unknown distributions, $P$ and $Q$, both supported on $D \subset \Rd$, with continuous density functions $p$ and $q$, respectively. We have the capacity to sample from either distribution. Our goal is to test the nonparametric hypothesis $H_0: P = Q$ vs. the alternative $H_1: P \neq Q$. 

Under the \textbf{binomial data model}, we sample data $\set{z_1, \ldots, z_n}$ as follows. For each $i$, we draw an independent Rademacher label $\ell_i \in \set{1, -1}$, $\mathbb{P}(\ell_i = 1) = \mathbb{P}(\ell_i = -1) = 1/2$. Then, if $\ell_i = 1$ we sample $z_i \sim P$, whereas if $\ell_i = -1$ we samle $z_i \sim Q$ otherwise. Define $1_X$ to be the length $n$ indicator vector for $\ell_i = 1$
\begin{equation*}
1_X(i) = 
\begin{cases}
1, \ell_i = 1\\
0 \text{ otherwise } 
\end{cases}
\end{equation*}
and similarly for $1_Y$
\begin{equation*}
1_Y(i) = 
\begin{cases}
1, \ell_i= -1 \\
0 \text{ otherwise } 
\end{cases}
\end{equation*}
Denote the number of positive labels $N = \sum_{i = 1}^{n} 1_x(i)$, and let $M$ be such that $n = N + M$. Then, let the \textbf{normalized label vector} $a$ be given by $a = \frac{1}{N}1_X - \frac{1}{M} 1_Y$. 

\paragraph{Graph.}

Heuristically, our graph based test will attempt to identify whether samples from the same distribution are on average more similar than samples from different distributions. As a result, we introduce $K$, a \textbf{kernel function} which measures similarity, and make the following assumptions on $K$
\begin{itemize}
	\item $K: [0,\infty) \to [0,\infty)$ is non-increasing.
	\item The integral $\int_{0}^{\infty} K(r) r^d dr$ is finite.
\end{itemize}

and define 

\begin{equation*}
K_{\epsilon}(z) = \frac{1}{\epsilon^d} K\left(\frac{z}{\epsilon}\right)
\end{equation*}


Given a sequence $\seq{\epsilon_n}$, form the \textbf{$\epsilon$-radius neighborhood graph} $G_{n,r} = (V_n,E_n)$ with $V_n = \set{z_1, \ldots, z_n}$ and $E_n = \set{(i,j): K_{\epsilon}(\norm{z_i - z_j}) > 0}$. Let $A$ be the adjacency matrix associated with $G_{n,r}$. Take $L_n = D - A$ to be the (unnormalized) \textbf{Laplacian matrix} of $A$ (where $D$ is the diagonal degree matrix with $D_{ii} = \sum_{j \in [n]} A_{ij}$). Denote by $B$ the $\abs{E} \times n$ \textbf{incidence matrix} of $A$, where we denote the $i$th row of $B$ as $B_i$ and set $B_i$ to have entry $A_{ij}$ in position $i$, $-A_{ij}$ in position $j$, and $0$ everywhere else. 

\paragraph{Test Statistic.}
For a given neighborhood graph $G_{n,r}$, let $A$ be the adjacency matrix associated with $G_{n,r}$. Take $L_n = D - A$ to be the (unnormalized) \textbf{Laplacian matrix} of $A$ (where $D$ is the diagonal degree matrix with $D_{ii} = \sum_{j \in [n]} A_{ij}$). Denote by $B$ the $\abs{E} \times n$ \textbf{incidence matrix} of $A$, where we denote the $i$th row of $B$ as $B_i$ and set $B_i$ to have entry $A_{ij}$ in position $i$, $-A_{ij}$ in position $j$, and $0$ everywhere else. 
Now, we can define our \textbf{Laplacian smooth} test statistic
\begin{equation}
\label{eqn: laplacian_smooth_statistic}
T_2 := \left(\max_{\theta: \norm{B\theta}_2 \leq C_n} a^T \theta \right)^2
\end{equation}
for some sequence of positive numbers $\seq{C_n} \geq 0$.


\paragraph{Empirical Risk Minimization.}
To introduce the continuous limit of $T_2$, it will be useful to slightly recast the variational problem of (\ref{eqn: laplacian_smooth_statistic}) as an empirical risk minimization problem. (Really, all we are doing is introducing some new notation.) Let $\nu_n$ be the \textbf{empirical measure} induced by $\set{z_1, \ldots, z_n}$
\begin{equation*}
\nu_n \defeq \frac{1}{n} \sum_{i = 1}^{n} \delta_{z_i}
\end{equation*}

Then, for any mapping $u_n: \set{z_1, \ldots, z_n} \to \Reals$ such that $u_n \in L^2(\nu_n)$, let the \textbf{empirical risk functional} $R_n(u_n)$ be given by
\begin{equation}
\label{eqn: empirical_risk_functional}
R_n(u_n) \defeq - \sum_{i = 1}^{N} u_n(z_i) \widetilde{\ell}_n(z_i)
\end{equation}
where $\widetilde{\ell}_n: \set{z_1, \ldots, z_n} \to \set{0,1}$ is the \textbf{normalized label function} defined by $\widetilde{\ell}_n(z_i) := a_i$. 

To relate the risk functional of (\ref{eqn: empirical_risk_functional}) to the variational problem of (\ref{eqn: laplacian_smooth_statistic}), we introduce the \textbf{constrained empirical risk functional}, $R_n^{(con)}(u_n)$, defined by
\begin{equation*}
R_n^{(con)}(u_n) := 
\begin{cases}
R_n(u_n), \text{ if $\mathcal{E}_n^2(u_n) \leq 1$} \\
\infty, \text{ otherwise }
\end{cases}
\end{equation*}
where $\mathcal{E}_n^2(u_n)$ is the \textbf{Laplacian regularization functional} given by
\begin{equation*}
\mathcal{E}_n^2(u_n) \defeq \frac{1}{n^2 \epsilon_n^{d+1}} \sum_{i = 1}^{n} \sum_{j = 1}^{n} K\left(\frac{\norm{z_i - z_j}}{\epsilon_n}\right)(u_n(z_i) - u_n(z_j))^2.
\end{equation*}

Note that for $B$ the incidence matrix of the $\epsilon$-radius neighborhood graph $G_{n,\epsilon_n}$ as defined above, and for $\theta = \set{u_1(z_i), \ldots, u_n(z_i)}$, we have
\begin{equation*}
N^2 \epsilon_N^{d+1} \mathcal{E}_N^2(u_N) = \norm{B \theta}_2, ~~~ \mathcal{E}_N^2(u_N) \leq 1 \Leftrightarrow \norm{B \theta}_2 \leq N^2 \epsilon_N^{d+1}
\end{equation*}
and $a^T \theta = - R_n(u_n)$.
As a result, letting $C_n = N^2 \epsilon_N^{d+1}$, we have that for
\begin{equation}
\label{eqn: minimizer_of_constrained_empirical_risk_functional}
u_n^{\star} := \argmin_{u_n \in L^2(\nu_n)} R_n^{(con)}(u_n)
\end{equation}
the following relation holds:
\begin{equation*}
T_2^{1/2} = R_n(u_n^{\star})
\end{equation*}

\paragraph{Continuum limit.}

As the preceding manipulations make clear, the statistic $T_2$ can be seen as a constrained minimization problem with constraint enforced by a regularization functional over the neighborhood graph $G_{n,\epsilon}$. It is well known that, for an appropriate schedule of $\seq{\epsilon_n}$ and data generated from a density satisfying certain regularity conditions, such regularization functionals are well behaved in the limit. 

Let $\nu = \frac{p + q}{2}$. For $u \in L^2(\nu)$, define the \textbf{continuous risk functional} $R(u)$ via
\begin{equation*}
R(u) = -\int_{D} u(x) \bigl(p(x) - q(x) \bigr) dx
\end{equation*}
the \textbf{weighted $L^2$ regularization functional}
\begin{equation*}
\mathcal{E}_{\infty}^2(u) = \int_D \norm{\nabla u(x)}^2 \mu^2(x) dx
\end{equation*}
and the \textbf{constrained continuous risk functional} $R^{(con)}(u)$ as
\begin{equation*}
R^{(con)}(u) =
\begin{cases}
R(u), \text{if $\mathcal{E}_{\infty}^2(u) \leq 1$ } \\
\infty, \text{otherwise}
\end{cases}
\end{equation*}

Let $u^{\star}$ be defined analogously to $u_n^{\star}$,
\begin{equation}
\label{eqn: minimizer_of_constrained_continuous_risk_functional}
u^{\star} = \argmin_{u \in L^2(\nu)} R^{(con)}(u)
\end{equation}

\section{RESULTS}
\begin{theorem}
	\label{thm: consistency_of_empirical_risk_minimizer}
	Consider a sequence $\seq{\epsilon_n} \to 0$ satisfying
	\begin{equation*}
	\left(\frac{\log n}{n}\right)^{1/d} = o(\epsilon_n).
	\end{equation*}
	Then, for $u_n^{\star}$ satisfying (\ref{eqn: minimizer_of_constrained_empirical_risk_functional}) and likewise $u^{\star}$ satisfying (\ref{eqn: minimizer_of_constrained_continuous_risk_functional}), with probability one:
	\begin{equation}
	R_n(u_n^{\star}) \to R(u^{\star})
	\end{equation}
\end{theorem}

\begin{proof}[Proof attempt of Theorem \ref{thm: consistency_of_empirical_risk_minimizer}]
	
We begin by restricting ourselves to the case where $\underset{n \to \infty}{\limsup}~ \E_n^2(u_n^{\star}) \leq 1$.
As a result of Lemma 
\end{proof}
\subsection{Technical results.}

Lemma \ref{lem: conv_of_risk_functional} is very similar to Proposition 2.7 in \textcolor{red}{cite (Garcia-Trillos 16)}.  
\begin{lemma}
	\label{lem: conv_of_risk_functional}
	With probability one the following statement holds: Let $\seq{u_n}$ be a sequence of $[-1,1]$-valued functions, with $u_n \in L^1(\nu_n)$. If $u_n \overset{TL^1}{\to} u$ as $n \to \infty$, then
	\begin{equation*}
	\lim_{n \to \infty} R_n(u_n) = R(u).
	\end{equation*}
\end{lemma}

Lemma \ref{lem: weak_limit_of_label_function} is needed to prove Lemma \ref{lem: conv_of_risk_functional}. It is very similar to a result from \textcolor{red}{cite (Garcia-Trillos 16)}. To understand Lemma \ref{lem: weak_limit_of_label_function}, we must introduce a notion of weak convergence of functions.

\begin{definition}
	Given a sequence of functions $g_n \in L^1(\nu)$, and $g \in L^1(\nu)$, we say \textbf{$g_n$ converges weakly to $g$}, $g_n \rightharpoonup g$ if for all $L^{\infty}(\nu)$,
	\begin{equation*}
	\lim_{n \to \infty} \int_D g_n(x) f(x) d\nu(x) = \int_D g(x) f(x) d\nu(x)
	\end{equation*}
	Given a sequence $\seq{u_n}$ with $u_n \in L^1(\nu_n)$, we say that $\seq{u_n}$ \textbf{converges weakly} to $u \in L^1(\nu)$, $u_n \rightharpoonup u$,  if the sequence of functions $\set{u_n \circ T_n} \in L^1(\nu)$ converges weakly to $u$, for $T_n$ stagnating transportation maps.
\end{definition}

\begin{lemma}
	\label{lem: weak_limit_of_label_function}
	For the label function $\ell_n: \set{z_1, \ldots, z_n} \to \set{-1,1}$ defined by
	\begin{equation}
	\label{eqn: label_function}
	\ell_n(z_i) := \ell_i, ~~~ i \in 1,\ldots,n
	\end{equation}
	with probability one $\ell_n \rightharpoonup \frac{p - q}{2 \mu}$. 
\end{lemma}

\begin{lemma}
	\label{lem: gamma_convergence_of_energy_function}
	The Laplacian regularization functional $\E_n^2(u_n)$ satisfies the following three properties, 
\end{lemma}

\section{PROOFS}
\begin{proof}[Proof of Lemma \ref{lem: conv_of_risk_functional}]
	We begin by removing the effect of the random normalization in $\widetilde{\ell}_n(z_i)$ via
	\begin{equation*}
	- R_n(u_n) = \frac{2}{n} \sum_{i = 1}^{n} u_n(z_i) \ell_n(z_i) + \sum_{i = 1}^{n} \left(\frac{2}{n} \ell_n(z_i) - \widetilde{\ell}_n(z_i) \right)u_n(z_i)
	\end{equation*}
	First, we show the second term converges to zero with probability one,
	\begin{align*}
	\abs{\sum_{i = 1}^{n} \left(\frac{2}{n} \ell_n(z_i) - \widetilde{\ell}_n(z_i) \right)u_n(z_i)} & \leq \sum_{i = 1}^{n} \abs{\frac{2}{n} \ell_n(z_i) - \widetilde{\ell}_n(z_i)} \abs{u_n(z_i)} \\
	& \leq \left(\abs{\frac{1}{N} - \frac{1}{n/2}} + \abs{\frac{1}{M} - \frac{1}{n/2}}\right) \sum_{i = 1}^{n} \abs{u_n(z_i)} \\
	& \leq \left(\abs{\frac{N - n/2}{N}} + \abs{\frac{M - n/2}{M}}\right) \cdot \frac{2}{n} \sum_{i = 1}^{n}\abs{u_n(z_i)}
	\end{align*}
	Then, the $TL^1$ convergence of $u_n$ to $u \in L^1(\nu)$ implies
	\begin{equation*}
	\frac{2}{n} \sum_{i = 1}^{n}\abs{u_n(z_i)} \overset{n}{\to} 2 \norm{u}_{L^1(\nu)}
	\end{equation*}
	and by standard concentration results of binomial random variables, with probability one
	\begin{equation*}
	\abs{\frac{N - n/2}{N}}, \abs{\frac{M - n/2}{M}} \overset{n}{\to} 0,
	\end{equation*}
	and so with probability one
	\begin{equation*}
	\abs{\sum_{i = 1}^{n} \left(\frac{2}{n} \ell_n(z_i) - \widetilde{\ell}_n(z_i) \right)u_n(z_i)} \overset{n}{\to} 0.
	\end{equation*}
	
	Now, we rewrite the first term in the summand on the right hand side using transportation maps,
	\begin{align*}
	\frac{2}{n} \sum_{i = 1}^{n} u_n(z_i) \ell_n(z_i) & = 2 \int_D u_n(z) \ell_n(z) d\nu_n(z) \\
	& \overset{(i)}{=} 2 \int_D  \bigl(u_n \circ T_n (z) \bigr) \bigl( \ell_n \circ T_n (z)  \bigr) d\nu(z) \\
	& = 2 \int_D  \bigl(u_n \circ T_n (z)  - u(z) \bigr) \bigl( \ell_n \circ T_n (z)  \bigr) d\nu(z) + 2 \int_D  \bigl(u(z) \bigr) \bigl( \ell_n \circ T_n (z)  \bigr) d\nu(z) \\
	\end{align*}
	with $(i)$ following from the change of variables formula
	\begin{equation*}
	\int_D f(T(x)) d \theta(x) = \int_D f(z) d T_{\sharp} \theta x
	\end{equation*}
	where $f: D \to \Reals$ is an arbitrary Borel function, $\theta$ a Borel measure, and $T_{\sharp} \theta$ the push-forward measure of $\theta$. 
	
	Now, the first term converges to zero,
	\begin{equation*}
	\abs{2 \int_D  \bigl(u_n \circ T_n (z)  - u(z) \bigr) \bigl( \ell_n \circ T_n (z)  \bigr) d\nu(z)} \leq 2 \int_D \abs{u_n \circ T_n (z)  - u(z)} d\nu(z) \overset{n}{\to} 0.
	\end{equation*}
	by the boundedness of $\ell_n$ and the $TL^1$ convergence of $u_n$ to $u$.
	
	By Lemma \ref{lem: weak_limit_of_label_function}, with probability one the 2nd term converges to (negative of) the risk functional,
	\begin{equation*}
	2 \int_D  \bigl(u(z) \bigr) \bigl( \ell_n \circ T_n (z)  \bigr) d\nu(z) \overset{n}{\to} \int_D u(z) \left(\frac{p(z) - q(z)}{\mu(z)}\right) d\nu(z) = -R(u).
	\end{equation*}
\end{proof}

\begin{proof}[Proof of Lemma \ref{lem: conv_of_risk_functional}]
	Lemma \ref{lem: weak_limit_of_label_function} is essentially a restatement of Lemma 2.5 from \textcolor{red}{cite (Garcia-Trillos 16)}. For completeness purposes, we restate that result here.
	
	\begin{lemma}
		\label{lem: weak_conv_label_function}
		Let $f(z) = \mathbb{E}(\ell \vert Z = z)$ be the conditional expectation of $\ell$ given $Z = z$.  
		For the label function $\ell_n: \set{x_1, \ldots, x_n} \to \set{-1,1}$ defined as in (\ref{eqn: label_function}), with probability one $\ell_n \rightharpoonup f$.
	\end{lemma}
	Given Lemma \ref{lem: weak_conv_label_function}, all that is needed to show Lemma \ref{lem: weak_limit_of_label_function} is that $f = \frac{p - q}{2\mu}$. This follows from a simple application of Bayes Rule. 
\end{proof}
	
\end{document}