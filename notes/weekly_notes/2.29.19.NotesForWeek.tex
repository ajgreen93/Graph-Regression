\documentclass{article}
\usepackage{amsmath}
\usepackage{amsfonts, amsthm, amssymb}
\usepackage{graphicx}
\usepackage[colorlinks]{hyperref}
\usepackage[parfill]{parskip}
\usepackage{algpseudocode}
\usepackage{algorithm}
\usepackage{enumerate}
\usepackage[shortlabels]{enumitem}

\usepackage{natbib}
\renewcommand{\bibname}{REFERENCES}
\renewcommand{\bibsection}{\subsubsection*{\bibname}}

\newcommand{\eqdist}{\ensuremath{\stackrel{d}{=}}}
\newcommand{\Graph}{\mathcal{G}}
\newcommand{\Reals}{\mathbb{R}}
\newcommand{\Identity}{\mathbb{I}}
\newcommand{\distiid}{\overset{\text{i.i.d}}{\sim}}
\newcommand{\convprob}{\overset{p}{\to}}
\newcommand{\convdist}{\overset{w}{\to}}
\newcommand{\Expect}[1]{\mathbb{E}\left[ #1 \right]}
\newcommand{\Risk}[2][P]{\mathcal{R}_{#1}\left[ #2 \right]}
\newcommand{\Prob}[1]{\mathbb{P}\left( #1 \right)}
\newcommand{\iset}{\mathbf{i}}
\newcommand{\jset}{\mathbf{j}}
\newcommand{\myexp}[1]{\exp \{ #1 \}}
\newcommand{\norm}[1]{\left\lVert#1\right\rVert}
\newcommand{\abs}[1]{\left \lvert #1 \right \rvert}
\newcommand{\restr}[2]{\ensuremath{\left.#1\right|_{#2}}}
\newcommand{\ext}[1]{\widetilde{#1}}
\newcommand{\set}[1]{\left\{#1\right\}}
\newcommand{\seq}[1]{\set{#1}_{n \in \N}}
\newcommand{\dotp}[2]{\langle #1, #2 \rangle}
\newcommand{\floor}[1]{\left\lfloor #1 \right\rfloor}
\newcommand{\Var}{\mathrm{Var}}
\newcommand{\diam}{\mathrm{diam}}

\newcommand{\emC}{C_n}
\newcommand{\emCpr}{C'_n}
\newcommand{\emCthick}{C^{\sigma}_n}
\newcommand{\emCprthick}{C'^{\sigma}_n}
\newcommand{\emS}{S^{\sigma}_n}
\newcommand{\estC}{\widehat{C}_n}
\newcommand{\hC}{\hat{C^{\sigma}_n}}
\newcommand{\vol}{\text{vol}}
\newcommand{\spansp}{\mathrm{span}~}
\newcommand{\1}{\mathbb{I}}

\newcommand{\Linv}{L^{\dagger}}
\DeclareMathOperator*{\argmin}{argmin}
\DeclareMathOperator*{\argmax}{argmax}

\newcommand{\emF}{\mathbb{F}_n}
\newcommand{\emG}{\mathbb{G}_n}
\newcommand{\emP}{\mathbb{P}_n}
\newcommand{\F}{\mathcal{F}}
\newcommand{\D}{\mathcal{D}}
\newcommand{\R}{\mathcal{R}}
\newcommand{\Rd}{\Reals^d}

%%% Matrices
\newcommand{\Xbf}{\mathbf{X}}
\newcommand{\Ybf}{\mathbf{Y}}
\newcommand{\Zbf}{\mathbf{Z}}

%%% Sets and classes
\newcommand{\Xset}{\mathcal{X}}
\newcommand{\Sset}{\mathcal{S}}
\newcommand{\Hclass}{\mathcal{H}}
\newcommand{\Pclass}{\mathcal{P}}

%%% Distributions and related quantities
\newcommand{\Pbb}{\mathbb{P}}
\newcommand{\Ebb}{\mathbb{E}}
\newcommand{\Qbb}{\mathbb{Q}}

%%% Operators
\newcommand{\Tadj}{T^{\star}}
\newcommand{\dive}{\mathrm{div}}

% \newcommand{\span}{\textrm{span}}

\newtheoremstyle{alden}
{6pt} % Space above
{6pt} % Space below
{} % Body font
{} % Indent amount
{\bfseries} % Theorem head font
{.} % Punctuation after theorem head
{.5em} % Space after theorem head
{} % Theorem head spec (can be left empty, meaning `normal')

\theoremstyle{alden} 


\newtheoremstyle{aldenthm}
{6pt} % Space above
{6pt} % Space below
{\itshape} % Body font
{} % Indent amount
{\bfseries} % Theorem head font
{.} % Punctuation after theorem head
{.5em} % Space after theorem head
{} % Theorem head spec (can be left empty, meaning `normal')

\theoremstyle{aldenthm}
\newtheorem{theorem}{Theorem}
\newtheorem{conjecture}{Conjecture}
\newtheorem{lemma}{Lemma}
\newtheorem{example}{Example}
\newtheorem{corollary}{Corollary}
\newtheorem{proposition}{Proposition}
\newtheorem{assumption}{Assumption}
\newtheorem{remark}{Remark}


\theoremstyle{definition}
\newtheorem{definition}{Definition}[section]

\theoremstyle{remark}

\begin{document}
\title{Notes for Week 2/23/19 - 2/29/19}
\author{Alden Green}
\date{\today}
\maketitle

Consider distributions $\Pbb$ and $\Qbb$ supported on $\D \subset \Reals^d$ which are absolutely continuous with density functions $f$ and $g$, respectively. For fixed $n \geq 0$, Let $\Zbf = (z_1, \ldots,z_n)$, where for $i = 1,\ldots,t$, $z_i \sim \frac{\Pbb + \Qbb}{2}$ are independent. Given $\Zbf$, for $i = 1,...,t$ let
\begin{equation*}
\ell_i = 
\begin{cases}
1~ \text{with probability $\frac{f(z_i)}{f(z_i) + g(z_i)}$} \\
-1~ \text{with probability $\frac{g(z_i)}{f(z_i) + g(z_i)}$}
\end{cases}
\end{equation*} 
be conditional independent labels, and write
\begin{equation*}
1_X = 
\begin{cases}
1,~ l_i = 1\\
0,~ \text{otherwise}
\end{cases}
1_Y = 
\begin{cases}
1,~ l_i = -1 \\
0,~ \text{otherwise.}
\end{cases}
\end{equation*}
We will write $\Xbf = \set{x_1, \ldots,x_{N_X}} := \set{z_i: \ell_i = 1}$ and similarly $\Ybf = \set{y_1, \ldots,y_{N_Y}} := \set{y_i: \ell_i = -1}$, where $N_X$ and $N_Y$ are of course random but $N_X + N_Y = n$. 

Our statistical goal is hypothesis testing: that is, we wish to construct a test function $\phi$ which differentiates between
\begin{equation*}
\mathbb{H}_0: f = g \text{ and } \mathbb{H}_1: f \neq g.
\end{equation*}

For a given function class $\Hclass$, some $\epsilon > 0$, and test function $\phi$ a Borel measurable function of the data with range $\set{0,1}$, we evaluate the quality of the test using \emph{worst-case risk}
\begin{equation*}
R_{\epsilon}^{(t)}(\phi; \Hclass) = \sup_{f \in \Hclass} \Ebb_{f,f}^{(t)}(\phi) + \sup_{ \substack{f,g \in \Hclass \\ \delta(f,g) \geq \epsilon } } \Ebb_{f,g}^{(t)}(1 - \phi)
\end{equation*} 
where 
\begin{equation*}
\delta^2(f,g) = \int_{\D} (f - g)^2 dx.
\end{equation*}

\section{Laplacian smooth test statistic}

For $r \geq 0$, define the \emph{$r$-graph} $G_r = (V,E_r)$ to have vertex set $V = \set{1,\ldots,t}$ and edge set $E_r$ which contains the pair $(i,j)$ if and only if $\norm{z_i - z_j}_2 \leq r$. Let $D_{G_r}$ denote the incidence matrix of $G_r$. 

Define the \emph{$r$-Laplacian Smooth} test statistic to be
\begin{equation*}
T_{LS} = \sup_{\theta: \norm{D_{G_r}\theta}_2 \leq C_{n,r}} \dotp{\theta}{\frac{1_X}{N_X} - \frac{1_Y}{N_Y}}
\end{equation*}

We would like to relate the graph $G_r$ to a graph with a more easily accessible spectrum. For $\kappa = n^{1/d}$, consider the \emph{grid graph}
\begin{equation*}
G_{grid} = (V_{grid},E_{grid}),~~ V_{grid} = \set{\frac{k}{\kappa}: k \in [\kappa]^d},~~ E_{grid} = \set{(k,k'): k, k' \in V_{grid}, \norm{k - k'}_1 = \frac{1}{\kappa^d}}
\end{equation*}
with associated incidence matrix $D_{grid}$.

\begin{lemma}[Spectral similarity of $r$-graph to grid]
	\label{lem: spectral_similarity_rgraph_grid}
	Fix $r \geq 2 \left(\frac{\log t}{n}\right)^{1/d} + (\frac{1}{n})^{1/d}$, and let $\ell(r,n) = \biggl(\sqrt{d} r n^{1/d} + 2\sqrt{d}(\log n)^{1/d}\biggr)^{1/2}$. For any $\theta \in \Reals^n$, the following relations hold:
	\begin{equation}
	\label{eqn: spectral_similarity_rgraph_grid}
	\frac{\norm{D_{G_r}\theta}_2}{\ell(t)} \leq \norm{D_{grid}\theta}_2 \leq \norm{D_{G_r}\theta}_2
	\end{equation}
	with probability at least $1 - n^{-\alpha}$ where $\alpha = c_1 (\log n)^{1/2}$ for some constant $c_1 > 0$.
\end{lemma}
\begin{proof}
	We begin by mapping the data $\Zbf$ to the grid points $[\kappa]^d$ in such a way that as little mass as possible is disturbed:
	\begin{lemma}
		\label{lem: mass_transport_mapping}
		There exists a bijective mapping $T: \Zbf \to [\kappa]^d$ for $\kappa = t^{1/d}$ such that
		\begin{equation*}
		\max_{i} \norm{T(z_i) - z_i}_2 \leq C\left(\frac{\log t}{t}\right)^{1/d}
		\end{equation*}
		with probability at least $1 - n^{-\alpha}$ where $\alpha = c_1 (\log n)^{1/2}$ for some constant $c_1 > 0$.
	\end{lemma}
	Hereafter, we assume there exists $T$ such that Lemma \ref{lem: mass_transport_mapping} holds.
	
	We first prove the second bound in \eqref{eqn: spectral_similarity_rgraph_grid}.  Consider grid points $k ~ k'$ connected in the grid graph. Then, there exist $z_i$ and $z_j$ such that $T(z_i) = k$ and $T(z_j) = k'$. By the triangle inequality,
	\begin{align*}
	\norm{z_i - z_j}_2 & \leq \norm{T(z_i) - z_i}_2 + \norm{T(z_i) - T(z_j)}_2 + \norm{T(z_j) - z_j}_2 \\
	& \leq 2C\left(\frac{\log t}{t}\right)^{1/d} + \frac{1}{t^{1/d}}
	\end{align*}
	and so by our choice of $r$, $i \sim j$ in $G_r$.
	
	Now, we turn to the first bound. Assume $i \sim j$ in the graph $G_r$. By a similar set of steps to the above, we have
	\begin{equation*}
	\norm{T(z_i) - T(z_j)}_2 \leq 2C\left(\frac{\log t}{t}\right)^{1/d} + r
	\end{equation*}
	As a result, using the simple relation $\norm{x}_1 \leq \sqrt{d} \norm{x}_2$ for any $x \in \Rd$, we have
	\begin{equation*}
	\norm{T(z_i) - T(z_j)}_1 \leq \sqrt{d}(2C\left(\frac{\log t}{t}\right)^{1/d} + r)
	\end{equation*}
	Since each edge in the grid graph is of length $n^{1/d}$, it is easy to see that there exists a path between $T(z_i)$ and $T(z_j)$ in $G_{grid}$, $P(T(Z_i) \to T(Z_j))$ with no more than
	\begin{equation*}
	\frac{\sqrt{d}(2C\left(\frac{\log t}{t}\right)^{1/d} + r)}{t^{1/d}}
	\end{equation*}
	edges. The bound follows by Lemma \ref{lem: graph_ordering}.
\end{proof}

\section{Poincare inequalities}

Let $G$ and $\widetilde{G}$ be undirected, unweighted graphs over vertex set $V$, with edge sets $E_G$ and $E_{\widetilde{G}}$, respectively. Let $\widetilde{\mathcal{P}}$ be the space of all paths over $E_{\widetilde{G}}$; that is, $\mathcal{P}$ consists of $\widetilde{P} \in \widetilde{\mathcal{P}}$
\begin{equation*}
P = (\widetilde{e}_1, \ldots, \widetilde{e}_m) \tag{$\widetilde{e_i} \in E_{\widetilde{G}}$}
\end{equation*}
for some integer $m \geq 1$. We say $\abs{P} = m$ is the \emph{path length}.

\begin{lemma}[Poincare inequality for general graphs.]
	Define a mapping $\gamma: E_G \to \mathcal{P}$ where for each $e = (\ell,\ell')$ in $E_G$
	\begin{equation*}
	\gamma(e) = ((\ell,u), \ldots, (v,\ell'))
	\end{equation*}
	meaning $e$ is mapped to a path which begins at $\ell$ and ends at $\ell'$. Then
	\begin{equation*}
	G \preceq \widetilde{G} \cdot \max_{e \in E_G} \abs{\gamma(e)}  \cdot b_{\gamma}
	\end{equation*}
	where $b_{\gamma}$ is a bottleneck parameter given by
	\begin{equation*}
	b_{\gamma} = \max_{\widetilde{e} \in E_{\widetilde{G}}} \abs{\set{e \in E: \widetilde{e} \in \widetilde{P}_e}}
	\end{equation*}
\end{lemma}
\begin{proof}
	Let $G_e = (V, \set{e})$ and $P_e = (V, \set{\widetilde{e}: \widetilde{e} \in \gamma(e)})$ be the graphs associated with $e$ and $\gamma(e)$, respectively. By Lemma \ref{lem: path_poincare}, we have
	\begin{equation*}
	G_{e} \preceq \abs{P_e} P_e
	\end{equation*}
	Summing over all $e \in E_G$, we obtain
	\begin{align*}
	G & \preceq \sum_{e \in E_G} \abs{P_e} P_e \\
	& \preceq \max_{e \in E_G} \abs{\gamma(e)} \sum_{e \in E_G} P_e \\
	& \preceq \max_{e \in E_G} \abs{\gamma(e)} b_{\gamma}\cdot \widetilde{G}
	\end{align*}
\end{proof}

\begin{lemma}[Poincare inequality for path graphs.]
	\label{lem: path_poincare}
	Fix $m \geq 0$. For vertices $V = \set{1, \ldots,m}$ define the path $P(1 \to m) = ((1,2),(2,3),\ldots, (m-1,m))$ and $G_{(1,m)}$ to be the graph consisting only of an edge between $1$ and $m$. Then,
	\begin{equation*}
	(m - 1) \cdot P(1 \to m) \succeq G_{(1,m)}
	\end{equation*}
\end{lemma}

Decompose $\frac{1_X}{N_X} - \frac{1_Y}{N_Y} := \theta^{\star} + w$, where
\begin{equation*}
(\theta^{\star})_i := \frac{f(x) - g(x)}{f(x) + g(x)}
\end{equation*}

The upper bound in Lemma \ref{lem: spectral_similarity_rgraph_grid} allows us the following upper bound on the empirical process
\begin{equation*}
\sup_{\theta: \norm{D_{r}\theta}_2 \leq C_{n,r}} \dotp{\theta}{w} \leq  \sup_{\theta: \norm{D_{grid}\theta}_2 \leq C_{n,r}} \dotp{\theta}{w} = C_{n,r} w^T{\Linv_{grid}}w 
\end{equation*}
whereas the lower bound helps us with the approximation error term,
\begin{equation*}
\sup_{\widetilde{\theta}: \norm{D_{r}\theta}_2 \leq C_{n,r}} \dotp{\widetilde{\theta}}{\theta^{\star}} \geq \sup_{\theta: \norm{D_{grid}\theta}_2 \leq C_{n,r}/ \ell(n,r)} \dotp{\theta}{\theta^{\star}} \geq \frac{C_{n,r}}{\ell(n,r)} \theta^{\star}\Linv_{grid} \theta^{\star}
\end{equation*}

\end{document}