\documentclass{article}

% Packages
\usepackage[utf8]{inputenc} % allow utf-8 input
\usepackage[T1]{fontenc}    % use 8-bit T1 fonts
\usepackage{booktabs}       % professional-quality tables
\usepackage{nicefrac}       % compact symbols for 1/2, etc.
\usepackage{microtype}      % microtypography
\usepackage{times}             % times font
\usepackage{mathrsfs}      % Added by Alden, for script font.

\usepackage[round]{natbib}
\usepackage{amssymb,amsmath,amsthm,bbm}
\usepackage[margin=1in]{geometry}
\usepackage{verbatim,float,url,dsfont}
\usepackage{graphicx,subfigure,psfrag}
\usepackage{algorithm,algorithmic}
\usepackage{mathtools,enumitem}
\usepackage[colorlinks=true,citecolor=blue,urlcolor=blue,linkcolor=blue]{hyperref}
\usepackage{multirow}

% Theorems and such
\newtheorem{theorem}{Theorem}
\newtheorem{lemma}{Lemma}
\newtheorem{corollary}{Corollary}
\newtheorem{proposition}{Proposition}
\theoremstyle{definition}
\newtheorem{remark}{Remark}
\newtheorem{definition}{Definition}
\newtheorem{example}{Example} % Added by Alden

% Assumption
\newtheorem*{assumption*}{\assumptionnumber}
\providecommand{\assumptionnumber}{}
\makeatletter
\newenvironment{assumption}[2]{
  \renewcommand{\assumptionnumber}{Assumption #1#2}
  \begin{assumption*}
  \protected@edef\@currentlabel{#1#2}}
{\end{assumption*}}
\makeatother

% Widebar
\makeatletter
\newcommand*\rel@kern[1]{\kern#1\dimexpr\macc@kerna}
\newcommand*\widebar[1]{%
  \begingroup
  \def\mathaccent##1##2{%
    \rel@kern{0.8}%
    \overline{\rel@kern{-0.8}\macc@nucleus\rel@kern{0.2}}%
    \rel@kern{-0.2}%
  }%
  \macc@depth\@ne
  \let\math@bgroup\@empty \let\math@egroup\macc@set@skewchar
  \mathsurround\z@ \frozen@everymath{\mathgroup\macc@group\relax}%
  \macc@set@skewchar\relax
  \let\mathaccentV\macc@nested@a
  \macc@nested@a\relax111{#1}%
  \endgroup
}
\makeatother

% Min and max
\newcommand{\argmin}{\mathop{\mathrm{argmin}}}
\newcommand{\argmax}{\mathop{\mathrm{argmax}}}
\newcommand{\minimize}{\mathop{\mathrm{minimize}}}
\newcommand{\st}{\mathop{\mathrm{subject\,\,to}}}
\DeclareMathOperator*{\esssup}{ess\,sup} % Added by Alden

% Shortcuts
\def\R{\mathbb{R}}
\def\C{\mathbb{C}}
\def\E{\mathbb{E}}
\def\P{\mathbb{P}}
\def\T{\mathsf{T}}
\def\Cov{\mathrm{Cov}}
\def\Var{\mathrm{Var}}
\def\half{\frac{1}{2}}
\def\tr{\mathrm{tr}}
\def\df{\mathrm{df}}
\def\dim{\mathrm{dim}}
\def\col{\mathrm{col}}
\def\row{\mathrm{row}}
\def\nul{\mathrm{null}}
\def\rank{\mathrm{rank}}
\def\nuli{\mathrm{nullity}}
\def\spa{\mathrm{span}}
\def\sign{\mathrm{sign}}
\def\supp{\mathrm{supp}}
\def\diag{\mathrm{diag}}
\def\aff{\mathrm{aff}}
\def\conv{\mathrm{conv}}
\def\dom{\mathrm{dom}}
\def\hy{\hat{y}}
\def\hf{\hat{f}}
\def\hmu{\hat{\mu}}
\def\halpha{\hat{\alpha}}
\def\hbeta{\hat{\beta}}
\def\htheta{\hat{\theta}}
\def\cA{\mathcal{A}}
\def\cB{\mathcal{B}}
\def\cD{\mathcal{D}}
\def\cE{\mathcal{E}}
\def\cF{\mathcal{F}}
\def\cG{\mathcal{G}}
\def\cK{\mathcal{K}}
\def\cH{\mathcal{H}}
\def\cI{\mathcal{I}}
\def\cL{\mathcal{L}}
\def\cM{\mathcal{M}}
\def\cN{\mathcal{N}}
\def\cP{\mathcal{P}}
\def\cS{\mathcal{S}}
\def\cT{\mathcal{T}}
\def\cW{\mathcal{W}}
\def\cX{\mathcal{X}}
\def\cY{\mathcal{Y}}
\def\cZ{\mathcal{Z}}

%%% Begin Alden's additions
\newcommand{\Ebb}{\mathbb{E}}
\newcommand{\Pbb}{\mathbb{P}}
\newcommand{\dotp}[2]{\langle #1, #2 \rangle}
\newcommand{\wt}[1]{\widetilde{#1}}
\newcommand{\wh}[1]{\widehat{#1}}
\newcommand{\mc}[1]{\mathcal{#1}}
\newcommand{\ttt}[1]{\texttt{#1}}
\newcommand{\Reals}{\mathbb{R}} % Same thing as Ryan's \R
\newcommand{\Rd}{\Reals^d}
\newcommand{\wb}[1]{\widebar{#1}}
\newcommand{\floor}[1]{\left\lfloor #1 \right\rfloor}
\newcommand{\1}{\mathbf{1}}
\newcommand{\bj}{{\bf j}}
\newcommand{\restr}[2]{\ensuremath{\left.#1\right|_{#2}}}
\newcommand{\TV}{\mathrm{TV}}

\DeclareFontFamily{U}{mathx}{\hyphenchar\font45}
\DeclareFontShape{U}{mathx}{m}{n}{<-> mathx10}{}
\DeclareSymbolFont{mathx}{U}{mathx}{m}{n}
\DeclareMathAccent{\wc}{0}{mathx}{"71}
%%% End Alden's Additions

\begin{document}
	% Front Matter
	\title{Estimation over Graph Sobolev Spaces}
	\author{Alden Green}
	\maketitle
	
	\tableofcontents
	
	\section{Introduction}
	We consider a Normal means problem. We observe ${\bf Y} = (Y_1,\ldots,Y_n) \in \Reals^n$ according to the signal plus Gaussian noise model,
	\begin{equation}
	Y_i = \theta_i^{\ast} + w_i,
	\end{equation}
	where $w_i \sim N(0,1)$ are independent Gaussian noise variables. The task is to recover the mean vector $\theta_i^{\ast} \in \Reals^n$, which is generally impossible unless some structure is assumed.
	
	The classical approach assumes that $\theta_i^{\ast}$ exhibits \emph{decay} as a function of $i$. We take a different route, and assume \emph{smoothness} of the mean vector $\theta^{\ast}$ with respect to some graph $\texttt{G} = (\texttt{V},\texttt{E})$. This smoothness is defined with respect to the edge incidence matrix $D \in \Reals^{m \times n}$---which has rows $D_{e} = (0,\ldots,1,\ldots,-1,\ldots,0)$ for each edge $e = (i,j) \in \ttt{E}$, and $m = |E|$ is the number of edges in $\ttt{G}$---and the Laplacian $L = D^{\top}D$. In particular, we assume $\theta^{\ast}$ belongs to a \emph{graph Sobolev class}, which we now define.
	
	\begin{definition}[Graph Sobolev Space]
		For a given $s \in \mathbb{N}, s \geq 1$ and $p \in [1,\infty)$, the \emph{graph Sobolev seminorm} of a vector $\theta \in \Reals^n$ is given by
		\begin{equation}
		|\theta|_{s,p} := 
		\begin{cases}
		\|DL^{(s - 1)/2}\|_p, & \quad\textrm{$s$ odd.}\\
		\|L^{s/2}\|_p, & \quad\textrm{$s$ even.}
		\end{cases}
		\end{equation}
		The graph Sobolev class $W_n^{s,p} \subseteq \Reals^n$ consists of all vectors $\theta \in \Reals^n$ for which $\|\theta\|_{s,p} < \infty$. The graph Sobolev ball $W_{n}^{s,p}(M) = \{\theta \in W_n^{s,p}: |\theta|_{s,p} < M\}$.
	\end{definition} 
	We address the following questions:
	\begin{itemize}
		\item \emph{Minimax rates}. The risk of an estimator $\wh{\theta} \in \Reals^n$--i.e. a measurable function of ${\bf Y}$---with respect to mean-squared error loss is
		\begin{equation}
		\label{eqn:risk}
		R\bigl(\wh{\theta},\theta^{\ast}\bigr) = \frac{1}{n}\Ebb[\|\wh{\theta} - \theta^{\ast}\|_2^2]. 
		\end{equation}
		The minimax risk over balls $W_{n}^{s,p}(M)$ is
		\begin{equation}
		R(W_{n}^{s,p}(M)) = \inf_{\wh{\theta}} \sup_{\theta^{\ast} \in W_n^{s,p}(M)} 	R\bigl(\wh{\theta},\theta^{\ast}\bigr).
		\end{equation}
		We would like to find a function $\psi(n,M)$ such that $R(W_{n}^{s,p}(M)) \asymp \psi(n,M)$.
		\item \emph{Adaptivity}. \textcolor{red}{(TODO)}.
		\item \emph{Fractional Sobolev spaces} \textcolor{red}{(TODO)}.
	\end{itemize}
	Our particular angle will be to get as far as we can using only discrete analogues to classical results of Fourier analysis.
	
	\section{Discrete Fourier Analysis}
	The Laplacian $L$ is a positive semi-definite matrix, and according to the spectral theorem can be characterized as $L = \sum_{k = 1}^{n} \lambda_k v_k v_k^{\top}$, where the eigenvalue/eigenvector pairs $(\lambda_k,v_k)$ satisfy
	\begin{equation}
	L v_k = \lambda_k v_k, \quad \|v_k\|_2^2 = 1,
	\end{equation} 
	and are index in ascending order of eigenvalues $0 = \lambda_1 \leq \lambda_2 \leq \ldots \leq \lambda_n$. We assume the spectrum of $L$ exhibits some structure characteristic of certain geometric graphs.
	\begin{assumption}{A}{1}
		\label{asmp:weyl}
		There exist positive numbers $a$ and $A$ such that
		\begin{equation}
		\label{eqn:weyl}
		a k^{2/d} \leq \lambda_k \leq A K^{2/d}, \textrm{for all $k \in \{2,\ldots,n\}$.}
		\end{equation}
	\end{assumption}
	\begin{assumption}{A}{2}
		\label{asmp:laplacian-spectrum} 
		The graph Laplacian $L$ of $\ttt{G}$ satisfies the following: there exists a positive number $B$ such that
		\begin{equation}
		\label{eqn:eigenvector-incoherence}
		\max_{k = 1,\ldots,n} \|v_k\|_{\infty} \leq \frac{B}{\sqrt{n}}.
		\end{equation}
	\end{assumption}
	Trivially~\eqref{eqn:weyl} is satisfied for any connected graph $\ttt{G}$, and~\eqref{eqn:eigenvector-incoherence} for any graph, connected or not. However, we will obtain sharp rates for estimation only when $a \asymp A$ and $B \asymp 1$. 
	
	Let $F: \Reals^n \to \Reals^n$ denote the Graph Fourier Transform $F\theta := (\theta^{\top} v_1,\ldots,\theta^{\top}v_n)$.  Under Assumption~\ref{asmp:laplacian-spectrum}, we can reproduce many results of classical Fourier analysis.
	\begin{proposition}
		For any graph $\ttt{G}$ with incidence matrix $D$, Laplacian $L$, and GFT $F$, the following statements hold for all $\theta \in \Reals^n$.
		\begin{itemize}
			\item \emph{Parseval}. $\|\theta\|_2 = \|F\theta\|_2$
			\item \emph{Laplacian to Multiplication}. $F(L\theta) = \vec{\lambda} \odot F(\theta)$. Here $\vec{\lambda} := (\lambda_1,\ldots,\lambda_n)$ and for vectors $v,u \in \Reals^n, v \odot u = (v_1u_1,\ldots,v_nu_n)$ stands for pointwise multiplication.
		\end{itemize}
		Suppose additionally that $\ttt{G}$ satisfies~\ref{asmp:laplacian-spectrum}. Then
		\begin{itemize}
			\item \emph{Duality}. $\|\theta\|_1 \leq (B/\sqrt{n}) \|F\theta\|_{\infty}$.
			\item \emph{Hausdorff-Young}.  
		\end{itemize}
	\end{proposition}
\end{document}