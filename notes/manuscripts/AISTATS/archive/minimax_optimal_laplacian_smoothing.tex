\documentclass{article}
\usepackage{amsmath}
\usepackage{amsfonts, amsthm, amssymb}
\usepackage{graphicx}
\usepackage[colorlinks]{hyperref}
\usepackage[parfill]{parskip}
\usepackage{algpseudocode}
\usepackage{algorithm}
\usepackage{enumerate}
\usepackage[shortlabels]{enumitem}
\usepackage{fullpage}
\usepackage{mathtools}
\usepackage{tikz}

\usepackage{natbib}
\renewcommand{\bibname}{REFERENCES}
\renewcommand{\bibsection}{\subsubsection*{\bibname}}

\DeclareFontFamily{U}{mathx}{\hyphenchar\font45}
\DeclareFontShape{U}{mathx}{m}{n}{<-> mathx10}{}
\DeclareSymbolFont{mathx}{U}{mathx}{m}{n}
\DeclareMathAccent{\wb}{0}{mathx}{"73}

\DeclarePairedDelimiterX{\norm}[1]{\lVert}{\rVert}{#1}
\DeclarePairedDelimiterX{\seminorm}[1]{\lvert}{\rvert}{#1}

\newcommand{\eqdist}{\ensuremath{\stackrel{d}{=}}}
\newcommand{\Graph}{\mathcal{G}}
\newcommand{\Reals}{\mathbb{R}}
\newcommand{\iid}{\overset{\text{i.i.d}}{\sim}}
\newcommand{\convprob}{\overset{p}{\to}}
\newcommand{\convdist}{\overset{w}{\to}}
\newcommand{\Expect}[1]{\mathbb{E}\left[ #1 \right]}
\newcommand{\Risk}[2][P]{\mathcal{R}_{#1}\left[ #2 \right]}
\newcommand{\Prob}[1]{\mathbb{P}\left( #1 \right)}
\newcommand{\iset}{\mathbf{i}}
\newcommand{\jset}{\mathbf{j}}
\newcommand{\myexp}[1]{\exp \{ #1 \}}
\newcommand{\abs}[1]{\left \lvert #1 \right \rvert}
\newcommand{\restr}[2]{\ensuremath{\left.#1\right|_{#2}}}
\newcommand{\ext}[1]{\widetilde{#1}}
\newcommand{\set}[1]{\left\{#1\right\}}
\newcommand{\seq}[1]{\set{#1}_{n \in \N}}
\newcommand{\floor}[1]{\left\lfloor #1 \right\rfloor}
\newcommand{\Var}{\mathrm{Var}}
\newcommand{\Cov}{\mathrm{Cov}}
\newcommand{\diam}{\mathrm{diam}}

\newcommand{\emC}{C_n}
\newcommand{\emCpr}{C'_n}
\newcommand{\emCthick}{C^{\sigma}_n}
\newcommand{\emCprthick}{C'^{\sigma}_n}
\newcommand{\emS}{S^{\sigma}_n}
\newcommand{\estC}{\widehat{C}_n}
\newcommand{\hC}{\hat{C^{\sigma}_n}}
\newcommand{\vol}{\text{vol}}
\newcommand{\spansp}{\mathrm{span}~}
\newcommand{\1}{\mathbf{1}}

\newcommand{\Linv}{L^{\dagger}}
\DeclareMathOperator*{\argmin}{argmin}
\DeclareMathOperator*{\argmax}{argmax}

\newcommand{\emF}{\mathbb{F}_n}
\newcommand{\emG}{\mathbb{G}_n}
\newcommand{\emP}{\mathbb{P}_n}
\newcommand{\F}{\mathcal{F}}
\newcommand{\D}{\mathcal{D}}
\newcommand{\R}{\mathcal{R}}
\newcommand{\Rd}{\Reals^d}
\newcommand{\Nbb}{\mathbb{N}}

%%% Vectors
\newcommand{\thetast}{\theta^{\star}}
\newcommand{\betap}{\beta^{(p)}}
\newcommand{\betaq}{\beta^{(q)}}
\newcommand{\vardeltapq}{\varDelta^{(p,q)}}


%%% Matrices
\newcommand{\X}{X} % no bold
\newcommand{\Y}{Y} % no bold
\newcommand{\Z}{Z} % no bold
\newcommand{\Lgrid}{L_{\grid}}
\newcommand{\Dgrid}{D_{\grid}}
\newcommand{\Linvgrid}{L_{\grid}^{\dagger}}
\newcommand{\Lap}{L}
\newcommand{\NLap}{{\bf N}}
\newcommand{\PLap}{{\bf P}}
\newcommand{\Id}{I}

%%% Sets and classes
\newcommand{\Xset}{\mathcal{X}}
\newcommand{\Vset}{\mathcal{V}}
\newcommand{\Sset}{\mathcal{S}}
\newcommand{\Hclass}{\mathcal{H}}
\newcommand{\Pclass}{\mathcal{P}}
\newcommand{\Leb}{L}
\newcommand{\mc}[1]{\mathcal{#1}}

%%% Distributions and related quantities
\newcommand{\Pbb}{\mathbb{P}}
\newcommand{\Ebb}{\mathbb{E}}
\newcommand{\Qbb}{\mathbb{Q}}
\newcommand{\Ibb}{\mathbb{I}}

%%% Operators
\newcommand{\Tadj}{T^{\star}}
\newcommand{\dive}{\mathrm{div}}
\newcommand{\dif}{\mathop{}\!\mathrm{d}}
\newcommand{\gradient}{\mathcal{D}}
\newcommand{\Hessian}{\mathcal{D}^2}
\newcommand{\dotp}[2]{\langle #1, #2 \rangle}
\newcommand{\Dotp}[2]{\Bigl\langle #1, #2 \Bigr\rangle}

%%% Misc
\newcommand{\grid}{\mathrm{grid}}
\newcommand{\critr}{R_n}
\newcommand{\dx}{\,dx}
\newcommand{\dy}{\,dy}
\newcommand{\dr}{\,dr}
\newcommand{\dxpr}{\,dx'}
\newcommand{\dypr}{\,dy'}
\newcommand{\wt}[1]{\widetilde{#1}}
\newcommand{\wh}[1]{\widehat{#1}}
\newcommand{\ol}[1]{\overline{#1}}
\newcommand{\spec}{\mathrm{spec}}
\newcommand{\LE}{\mathrm{LE}}
\newcommand{\LS}{\mathrm{LS}}
\newcommand{\OS}{\mathrm{OS}}
\newcommand{\PLS}{\mathrm{PLS}}

%%% Order of magnitude
\newcommand{\soom}{\sim}

% \newcommand{\span}{\textrm{span}}

\newtheoremstyle{alden}
{6pt} % Space above
{6pt} % Space below
{} % Body font
{} % Indent amount
{\bfseries} % Theorem head font
{.} % Punctuation after theorem head
{.5em} % Space after theorem head
{} % Theorem head spec (can be left empty, meaning `normal')

\theoremstyle{alden} 


\newtheoremstyle{aldenthm}
{6pt} % Space above
{6pt} % Space below
{\itshape} % Body font
{} % Indent amount
{\bfseries} % Theorem head font
{.} % Punctuation after theorem head
{.5em} % Space after theorem head
{} % Theorem head spec (can be left empty, meaning `normal')

\theoremstyle{aldenthm}
\newtheorem{theorem}{Theorem}
\newtheorem{conjecture}{Conjecture}
\newtheorem{lemma}{Lemma}
\newtheorem{example}{Example}
\newtheorem{corollary}{Corollary}
\newtheorem{proposition}{Proposition}
\newtheorem{assumption}{Assumption}
\newtheorem{remark}{Remark}


\theoremstyle{definition}
\newtheorem{definition}{Definition}[section]

\theoremstyle{remark}

\begin{document}
\title{Minimax optimal Laplacian smoothing}
\author{Alden Green}
\date{\today}
\maketitle

\textbf{Disclaimer:} This largely follows the same setup as in the draft I have written which combines Laplacian smoothing and Laplacian eigenmaps results. That draft --- document name \textit{graph\_regression2} --- should for the moment be considered obsolete.

\section{Introduction}

\textbf{(1)} Laplacian smoothing is a graph-based approach to nonparametric regression. 

\begin{itemize}
	\item In the random design nonparametric regression problem, we observe data $(X_1,Y_1),\ldots,(X_n,Y_n)$, where $X_1,\ldots,X_n$ are independent samples from a distribution $P$ supported on a domain $\Xset \subset \Rd$, and 
	\begin{equation}
	\label{eqn:random_design_regression}
	Y_i = f_0(X_i) + \varepsilon_i
	\end{equation}
	with $\varepsilon_i \sim N(0,1)$ independent Gaussian noise. Our goal is to perform statistical inference on the unknown regression function $f_0: \Xset \to \Reals$, by which we mean either (a) \emph{estimating} $f_0$ by $\wh{f}$, an estimator constructed from the data $(X_1,Y_1),\ldots,(X_n,Y_n)$ or (b) simply \emph{testing} whether $f_0 = 0$, i.e whether there is any signal present. 
	\item In graph-based nonparametric regression, we perform the above inferential tasks by first building a neighborhood graph $G_{n,r}$ which captures the geometry of $P$ and $\mc{X}$ in an appropriate sense, and then constructing an estimate $\wh{f}$ which is smooth with respect to the graph $G_{n,r}$. The neighborhood graph $G_{n,r} = ([n],{\bf W})$ is a weighted, undirected graph on vertices $[n] = \{1,...,n\}$, which we associate with the samples $\{X_1,\ldots,X_n\}$. The $n \times n$ weight matrix ${\bf W} = ({\bf W}_{ij})_{ij}$ encodes proximity between pairs of samples; for a kernel function $K: [0,\infty) \to \Reals$ and connectivity radius $r > 0$, the entries $\mathbf{W}_{ij}$ are given by
	\begin{equation*}
	\label{eqn:neighborhood_graph}
	{\bf W}_{ij} = K\Biggl(\frac{\norm{X_i - X_j}_{\Rd}}{r}\Biggr).
	\end{equation*}
	Then the degree matrix ${\bf D}$ is the $n \times n$ diagonal matrix with entries ${\bf D}_{ii} = \sum_{j = 1}^{n}{\bf W}_{ij}$, and the graph Laplacian can be written as
	\begin{equation}
	\label{eqn:graph_Laplacian}
	\Lap_{n,r} = \bf{D} - \bf{W}
	\end{equation}
	\item The Laplacian smoothing estimator $\wh{f}_{\LS}$ \citep{smola2003} is a penalized least squares estimator, given by
	\begin{equation}
	\label{eqn:laplacian_smoothing}
	\wh{f}_{\LS} := \min_{f \in \Reals^n} \biggl\{\sum_{i = 1}^{n}(Y_i - f_i)^2 + \rho \cdot f^T \Lap_{n,r} f \biggr\}
	\end{equation}
	where $\rho > 0$ acts a tuning parameter on the penalty $f^T \Lap_{{n,r}} f$.
	Assuming~\eqref{eqn:laplacian_smoothing} is a reasonable estimator of $f_0$, the squared empirical norm
	\begin{equation}
	T_{\LS} = \frac{1}{n}\Bigl\|\wt{f}_{\LS}\Bigr\|_2^2 \label{eqn:laplacian_smoothing_test}
	\end{equation}
	is in turn a reasonable statistic to assess whether or not $f_0 = 0$. 
\end{itemize}


\textbf{(2.a)} Laplacian smoothing has many advantages compared to other nonparametric regression methods. For instance:

\begin{itemize}
	\item It is fast, easy, and stable to compute. Letting ${\bf Y} = (Y_1,\ldots,Y_n) \in \Reals^n$, it is not hard to see that the minimizer $\wt{f}_{\LS} = (\rho \Lap_{n,r} + I)^{-1}{\bf Y}$ can be computed by solving a linear equation; in practice, one typically chooses $r$ to be small, with the result being that the matrix $\Lap_{n,r}$ is sparse. There exist known fast solvers of exactly this system.
	\item Is is generalizable to non-standard data---for instance, text or images, or really any data modality on which it is possible to define a kernel.
	\item It is easily adaptable to the semi-supervised learning setting. 
\end{itemize}

\textbf{(2.b)} For this reason a considerable body of work has emerged analyzing the consistency of the graph Laplacian $\Lap_{{n,r}}$ as $n \to \infty$ and $r(n) \to 0$. This is meant in various senses:
\begin{itemize}
	\item Pointwise consistency, meaning $\Lap_{n,r}f \to \Delta_Pf$ for an appropriate limiting operator $\Delta_P$.
	\item Spectral consistency, meaning the eigenvalues $\lambda_k(\Lap_{n,r}) \to \lambda_k(\Delta_P)$.
	\item Consistency of norms, meaning $f^T \Lap_{n,f} f \to \langle f,\Delta_Pf \rangle_{\Leb^2(P)}$. 
\end{itemize}
However, the statistical optimality of Laplacian smoothing is still not well understood.

\textbf{(2.c)} Our main contributions lie in this latter direction, and they can all be summarized as follows: Laplacian smoothing methods are minimax optimal over Sobolev spaces. In more detail, we will show that when $f_0$ belongs to the Sobolev ball $H^1(\Xset;M)$:
\begin{itemize}
	\item \textbf{Minimax optimal estimation.} With high probability, $\frac{1}{n}\norm{\wt{f}_{\LS} - f_0}_{2}^2 \lesssim n^{-2/(2 + d)}$.
	\item \textbf{Minimax optimal testing.}
	A level-$\alpha$ test constructed using $T_{\LS}$ has non-trivial power whenever $\norm{f_0}_{\Leb^2(\Xset)}^2 \gtrsim n^{-4/(4 + d)}$. 
	\item \textbf{Manifold adaptivity.}
	If $\mc{X} \subset \Rd$ is a submanifold of dimension $m < d$, both of the aforementioned rates hold with $d$ replaced by $m$.
\end{itemize}
Finally, we will also establish that the Laplacian smoothing estimator can be altered to take advantage of higher-order smoothness assumptions on $f_0$. This last conclusion will hold only under very limited circumstances.

\paragraph{Organization.}

List out what is to come.

\section{Minimax-optimal regression over Sobolev spaces}

Before we get to our main results, we briefly review minimax optimal estimation and testing rates over Sobolev classes. We also single out a notable method which achieves these rates: \emph{smoothing splines}. As we will see, Laplacian smoothing can be seen as a graph-based---i.e. discrete---counterpart to the ``continuous-time'' approach of smoothing splines. 

\subsection{Sobolev spaces}
\label{subsec:sobolev_spaces}

We start by briefly introducing Sobolev spaces, emphasizing only the material relevant to our statistical context. Roughly speaking, Sobolev spaces contain functions $f \in \Leb^p(\Xset)$ with derivatives which themselves belong to $\Leb^p(\Xset)$. More formally, for given integers $s$ and $p > 0$, the Sobolev space $W^{s,p}(\Xset)$ consists of all functions $f \in \Leb^p(\Xset)$ such that for each multiindex $\alpha = (\alpha_1,\ldots,\alpha_d) \in \mathbb{N}^d$ satisfying $\abs{\alpha} := \sum_{i = 1}^{d} \alpha_i \leq s$, the weak derivative $D^{\alpha}f$ exists and belongs to $\Leb^p(\Xset)$. For such functions, the $(s,p)$-Sobolev seminorm $\seminorm{f}_{W^{s,p}(\Xset)}$ and norm $\norm{f}_{W^{s,p}(\Xset)}$ are given by 
\begin{equation*}
\seminorm{f}_{W^{s,p}(\Xset)}^p = \sum_{\abs{\alpha} = s}\int_{\mathcal{X}} \Bigl|\bigl(D^{\alpha}f\bigr)(x)\Bigr|^p \,dx, ~~ \norm{f}_{W^{s,p}(\Xset)}^p = \sum_{k = 0}^{s} \seminorm{f}_{W^{k,p}(\Xset)}^p
\end{equation*}
and for a given $M > 0$, the $(s,p)$-Sobolev ball is $W^{s,p}(\Xset, M) = \set{f: \norm{f}_{W^{s,p}(\Xset)} \leq M}$. In the special case when $p = 2$, the Sobolev space $W$ is a Hilbert space, and we adopt the usual convention of writing $H^s(\Xset) = W^{s,2}(\Xset)$ and $H^s(\Xset,M) = W^{s,2}(\Xset,M)$; we will confine our attention to this case hereafter.

\textbf{(4)} In fact, an alternative way to motivate Laplacian smoothing nonparametric regression methods---as opposed to the advantages mentioned in \textbf{(2.a)}---comes from viewing them as discrete-time versions of smoothing splines. This is made more rigorous by the consistency results of \textbf{(2.b)}. Viewed in this light, an additional advantage of Laplacian smoothing emerges: it automatically adapts to the geometry of $\Xset$ and $p$. 

\textbf{(5)} Smoothing splines are known to have strong minimax optimal properties. Reasoning by analogy, we might hope that Laplacian smoothing would share these properties. 

\section{Graph Sobolev classes}

\textbf{(6)} The minimax optimality of smoothing splines rests on two crucial facts: one, the a priori assumption that $|f|_{H^1(\Xset)}$ is bounded; two, an upper bound on the metric entropy of $H^1(\Xset)$, characterized by the decay of its eigenvalues. \textcolor{red}{(TODO)}:(I know this isn't well-stated, and I need to clean it up.)  We now show that qualitatively similar statements hold with respect to 

\quad \textbf{(6.1)} the graph Sobolev semi-norm $f_0^T \Lap_{n,r} f_0$ and,
 
\quad \textbf{(6.2)} the graph eigenvalues $\lambda_k(\Lap_{n,r})$. 

\section{Minimax optimality of Laplacian smoothing}

\textbf{(7)} Using \textbf{(6.1)} and \textbf{(6.2)}, we establish upper bounds on the estimation and testing error of Laplacian smoothing, which confirm that Laplacian smoothing is asymptotically minimax rate-optimal over the Sobolev class $H^1(\Xset)$. 

\textbf{(8)} While we motivated graph Laplacians by suggesting they might replicate the statistical properties of smoothing splines, we have in fact managed to show something stronger. In particular:

\quad \textbf{(8.1)} When $d \geq 2$, there is no Sobolev embedding of $H^1(\Xset) \subseteq C^s(\Xset)$ for any $s > 0$. Therefore the smoothing spline estimator is ill-posed.

\quad \textbf{(8.2)} Additionally, the metric entropy of $C^1(\Xset)$ is too large to obtain optimal rates for \textcolor{red}{Holder-smoothing splines} when $d \geq 2$. Thus, we see a gap emerge between the optimality properties of the discrete-time (Laplacian smoothing) and continuous-time (smoothing splines) estimators, in favor of the latter.

\quad \textbf{(8.3)} On the other hand, when $d \geq 4$, we no longer get the known optimal rates in the estimation problem. When $d = 4$, our rates are suboptimal by a factor of $\log n$. When $d > 5$ they are very suboptimal. We believe this is related to classical results regarding the entropy of low-smoothness Holder classes, i.e. the results used to justify \textbf{(8.2)}. The interesting point is that the parameter spaces $C^1(\Xset)$ and $H^1(G_{n,r})$ become ``too large'' for different values of $d$.
 
\quad \textbf{(8.4)} \textcolor{red}{(TODO)}: Translate these results into error in $\Leb^2(P)$ error, if possible. 

\subsection{Manifold adaptivity}

\quad \textbf{(9)} We now show that when $\Xset$ is an $m$-dimensional submanifold of $\Reals^d$, Laplacian smoothing approaches automatically ``feel'' the intrinsic dimension $m$ of $\Xset$, which translate into improved upper bounds on estimating and testing error. This confirms the statement in \textbf{(4)} that Laplacian smoothing automatically adapts to the geometry of $\Xset$ and $p$. 

\subsection{Higher-order smoothness classes}

\quad \textbf{(10)} Analogously to smoothing splines, Laplacian smoothing can also be adapted to take advantage of additional assumed regularity on $f_0$, i.e. $f_0 \in H^s(\mc{\Xset})$ for $s > 1$. In very limited circumstances, we can show that this adapted estimator achieves the sharper minimax rates over these classes. However, the general story--for all combinations of $s$ and $d$---remains beyond our reach.

\section{Simulations}

\quad \textbf{(11)} Empirically, we demonstrate that the risk for Laplacian smoothing is comparable to that of smoothing splines, in the full dimensional case. Although one cannot carry out a comparison in the case of an unknown manifold --- since the smoothing spline estimator must be changed in a non-obvious manner --- we do show that when the density $p$ is concentrated close to a manifold and $f_0$ is smooth where $p$ is peaked, we have improved performance for $\wh{f}_{\LS}$ compared to $\wh{f}_{\mathrm{SM}}$. 

\section{Discussion}

\clearpage

\bibliographystyle{plainnat}
\bibliography{../../../graph_testing_bibliography} 

\end{document}