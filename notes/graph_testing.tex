\documentclass{article}
\usepackage{amsmath}
\usepackage{amsfonts, amsthm, amssymb}
\usepackage{bm}
\usepackage{graphicx}
\usepackage[colorlinks]{hyperref}
\usepackage[parfill]{parskip}
\usepackage{algpseudocode}
\usepackage{algorithm}
\usepackage{enumerate}

\usepackage{natbib}
\renewcommand{\bibname}{REFERENCES}
\renewcommand{\bibsection}{\subsubsection*{\bibname}}

\makeatletter
\newcommand{\leqnomode}{\tagsleft@true}
\newcommand{\reqnomode}{\tagsleft@false}
\makeatother

\newcommand{\eqdist}{\ensuremath{\stackrel{d}{=}}}
\newcommand{\Graph}{\mathcal{G}}
\newcommand{\Reals}{\mathbb{R}}
\newcommand{\Identity}{\mathbb{I}}
\newcommand{\distiid}{\overset{\text{i.i.d}}{\sim}}
\newcommand{\convprob}{\overset{p}{\to}}
\newcommand{\convdist}{\overset{w}{\to}}
\newcommand{\Expect}[1]{\mathbb{E}\left[ #1 \right]}
\newcommand{\Risk}[2][P]{\mathcal{R}_{#1}\left[ #2 \right]}
\newcommand{\Var}[1]{\mathrm{Var}\left( #1 \right)}
\newcommand{\Prob}[1]{\mathbb{P}\left( #1 \right)}
\newcommand{\iset}{\mathbf{i}}
\newcommand{\jset}{\mathbf{j}}
\newcommand{\myexp}[1]{\exp \{ #1 \}}
\newcommand{\norm}[1]{\left\lVert#1\right\rVert}
\newcommand{\dotp}[2]{\langle #1 , #2 \rangle}
\newcommand{\abs}[1]{\left \lvert #1 \right \rvert}
\newcommand{\restr}[2]{\ensuremath{\left.#1\right|_{#2}}}
\newcommand{\defeq}{\overset{\mathrm{def}}{=}}
\newcommand{\convweak}{\overset{w}{\rightharpoonup}}
\newcommand{\dive}{\mathrm{div}}
\newcommand{\Bin}{\mathrm{Bin}}

\newcommand{\emC}{C_n}
\newcommand{\emCpr}{C'_n}
\newcommand{\emCthick}{C^{\sigma}_n}
\newcommand{\emCprthick}{C'^{\sigma}_n}
\newcommand{\emS}{S^{\sigma}_n}
\newcommand{\estC}{\widehat{C}_n}
\newcommand{\hC}{\hat{C^{\sigma}_n}}
\newcommand{\vol}{\text{vol}}
\newcommand{\Bal}{\textrm{Bal}}
\newcommand{\Cut}{\textrm{Cut}}
\newcommand{\Ind}{\textrm{Ind}}
\newcommand{\set}[1]{\left\{#1\right\}}
\newcommand{\seq}[1]{\set{#1}_{n \in \N}}
\newcommand{\Perp}{\perp \! \! \! \perp}
\newcommand{\Naturals}{\mathbb{N}}


\newcommand{\Linv}{L^{\dagger}}
\newcommand{\tr}{\text{tr}}
\newcommand{\h}{\textbf{h}}
% \newcommand{\l}{\ell}
\newcommand{\x}{\textbf{x}}
\newcommand{\y}{\textbf{y}}
\newcommand{\bl}{\bm{\ell}}
\newcommand{\bnu}{\bm{\nu}}
\newcommand{\Lx}{\mathcal{L}_X}
\newcommand{\Ly}{\mathcal{L}_Y}
\DeclareMathOperator*{\argmin}{argmin}


\newcommand{\emG}{\mathbb{G}_n}
\newcommand{\A}{\mathcal{A}}
\newcommand{\F}{\mathcal{F}}
\newcommand{\G}{\mathcal{G}}
\newcommand{\X}{\mathcal{X}}
\newcommand{\Rd}{\Reals^d}
\newcommand{\N}{\mathbb{N}}
\newcommand{\E}{\mathcal{E}}

%%% Matrix related notation
\newcommand{\Xbf}{\mathbf{X}}
\newcommand{\Ybf}{\mathbf{Y}}
\newcommand{\Zbf}{\mathbf{Z}}
\newcommand{\Abf}{\mathbf{A}}
\newcommand{\Dbf}{\mathbf{D}}
\newcommand{\Wbf}{\mathbf{W}}
\newcommand{\Lbf}{\mathbf{L}}
\newcommand{\Ibf}{\mathbf{I}}
\newcommand{\Bbf}{\mathbf{B}}

%%% Vector related notation
\newcommand{\lbf}{\bm{\ell}}
\newcommand{\fbf}{\mathbf{f}}

%%% Set related notation
\newcommand{\Dset}{\mathcal{D}}
\newcommand{\Aset}{\mathcal{A}}
\newcommand{\Wset}{\mathcal{W}}

%%% Distribution related notation
\newcommand{\Pbb}{\mathbb{P}}
\newcommand{\Qbb}{\mathbb{Q}}
% \newcommand{\Pr}{\mathrm{Pr}}}

%%% Functionals
\newcommand{\1}{\mathbf{1}}


\newtheoremstyle{alden}
{6pt} % Space above
{6pt} % Space below
{} % Body font
{} % Indent amount
{\bfseries} % Theorem head font
{.} % Punctuation after theorem head
{.5em} % Space after theorem head
{} % Theorem head spec (can be left empty, meaning `normal')

\theoremstyle{alden} 
\newtheorem{definition}{Definition}[section]

\newtheoremstyle{aldenthm}
{6pt} % Space above
{6pt} % Space below
{\itshape} % Body font
{} % Indent amount
{\bfseries} % Theorem head font
{.} % Punctuation after theorem head
{.5em} % Space after theorem head
{} % Theorem head spec (can be left empty, meaning `normal')

\theoremstyle{aldenthm}
\newtheorem{theorem}{Theorem}
\newtheorem{conjecture}{Conjecture}
\newtheorem{lemma}{Lemma}
\newtheorem{example}{Example}
\newtheorem{corollary}{Corollary}
\newtheorem{proposition}{Proposition}
\newtheorem{assumption}{Assumption}

\theoremstyle{remark}
\newtheorem{remark}{Remark}

\begin{document}
	
\title{\textcolor{red}{Testing with Graphs}}
\author{Alden Green}
\date{\today}
\maketitle

\paragraph{Two-sample non-parametric hypothesis-testing problem.}

For fixed integers $n_1 + n_2 = n$, let $\Xbf = \set{x_1, \ldots, x_{n_1}} \subset \Rd$ and $\Ybf = \set{y_1, \ldots, y_{n_2}}$ be sampled i.i.d from distributions $\Pbb$ and $\Qbb$ with density functions $p$ and $q$, respectively, both with support on $D \subset \Rd$. Our statistical problem is testing the null hypothesis $H_0: \Pbb = \Qbb$ vs. the alternative $H_1: \Pbb \neq \Qbb$, where our knowledge of $\Pbb$ and $\Qbb$ come from the samples $\Xbf$ and $\Ybf$. 

\paragraph{Integral Probability Metric.}
Given $\F$ a class of real-valued bounded measurable functions on $D$, the \emph{integral probability metric} between $\Pbb$ and $\Qbb$ with respect to $\F$ is
\begin{equation*}
\gamma_{\F}(\Pbb,\Qbb) = \sup_{f \in \F}\abs{\int_{D} f d\Pbb - \int_{D} f d\Qbb}
\end{equation*}

One such IPM that has not received close attention until surprisingly recently is the \emph{(weighted) Sobolev IPM}. For $f \in L^2(\Dset)$ and $\rho$ a density function over $\Dset$, the ($\rho^2$-weighted) Sobolev 1,2 norm of $f$ is given by
\begin{equation*}
\norm{f}_{1,2,\rho^2} := 
\begin{cases}
\int_{\Dset}\norm{\nabla_x f(x)}^2 \rho^2 dx,~ & \text{$f \in H^1(\Dset)$} \\
\infty,~ & \text{$f \in L^2(\Dset) \setminus H^1(\Dset)$}
\end{cases}
\end{equation*}
where $H^1(\Dset)$ is \textcolor{red}{the Sobolev space of $L^2(\Dset)$ functions with weak derivative $\nabla_x f(x) \in L^2(\Dset)$.}

Consider the unit ball of $\norm{\cdot}_{1,2,\rho^2}$,
\begin{equation*}
\Wset^{1,2}(\Dset,\rho^2) := \set{f: \norm{f}_{1,2,\rho^2} \leq 1}.
\end{equation*}
The weighted Sobolev IPM is simply $\gamma_{\Wset^{1,2}(\Dset,\rho^2)}(\Pbb,\Qbb)$.

\paragraph{Neighborhood graph.}

Let $G_{n,r_n} = (V,E,w)$ denote a weighted, undirected graph constructed from the samples $\Zbf = \set{z_1, \ldots, z_n} = (\Xbf,\Ybf)$ where $V = \set{1, \ldots, n}$, and $w_{uv} = K(z_u, z_v):= k(\frac{\norm{z_u - z_v}}{\epsilon_n}) \geq 0$ for $u,v \in V$, and a particular kernel function $k$. Here $(u,v) \in E$ if and only if $w_{uv} > 0$.

Motivated by the integral probability metric, our test statistics will be of the form
\begin{equation*}
\gamma_{\F_n}(\Pbb_n,\Qbb_n) = \sup_{f_n \in \F_n}\abs{\int_{D} f_n d\Pbb_n - \int_{D} f_n d\Qbb_n}
\end{equation*}
where
\begin{equation*}
\Pbb_n := \frac{1}{n_1}\sum_{i = 1}^{n_1} \delta_{x_i}, ~~ \Qbb_n := \frac{1}{n_2}\sum_{i = 1}^{n_2} \delta_{y_i}
\end{equation*}
are the empirical distributions of $\Xbf$ and $\Ybf$, respectively, and $\F_n$ is a \textcolor{red}{class of functions $f_n: \Zbf \to \Reals$ exhibiting some regularity with respect to the neighborhood graph $G_{n,r}$.}

\paragraph{Laplacian smoothing and Total Variation denoising.}
 
For convenience, we number the edges $E  = (e_1, \ldots, e_m)$. We denote by $\Bbf \in \Reals^{m \times n}$ the edge incidence matrix of $G_{n,r}$, which for $k$th edge $e_k = (u,v)$ has $k$th row $\Bbf_{k} = (0,\ldots,-w_{uv},\ldots,w_{uv},\ldots,0)$ with a $-w_{uv}$ in the $u$th location, and a $w_{uv}$ in the $v$th location. The random (unnormalized) Laplacian matrix is then $\Lbf = \Bbf^T \Bbf$. We also introduce a \emph{label vector}, given by $\lbf = (\ell_1, \ldots, \ell_n)$ with
\begin{equation}
\label{eqn: label_vector}
\ell_k = 
\begin{cases}
\frac{n}{n_1},~~~ \text{$z_k \in \Xbf$} \\
-\frac{n}{n_2}, ~ \text{$z_k \in \Ybf$}
\end{cases}
\end{equation}


Our test statistics $T_1(\lbf; G_{n,r})$ and $T_2(\lbf; G_{n,r})$ are defined as follows:
\begin{align*}
	T_1(\lbf; G_{n,r}) := \sup_{\fbf \in \Reals^n: \norm{\Bbf \fbf}_1 \leq C_{n,r}} ~ \frac{1}{n} \sum_{k = 1}^{n} \ell_k f_k \\
	T_2(\lbf; G_{n,r}) := \sup_{\fbf \in \Reals^n: \norm{\Bbf \fbf}_2^2 \leq C_{n,r}} ~ \frac{1}{n} \sum_{k = 1}^{n} \ell_k f_k
\end{align*}
where $C_{n,r} = \frac{\sigma_k}{n^2 r_n^{d+2}}$ and we write $\fbf = (f_1,\ldots, f_n)$. We note that, as promised, these satisfy the form
\begin{align*}
T_1(\lbf; G_{n,r}) & = \sup_{f_n \in TV_n}\abs{\int_{D} f_n d\Pbb_n - \int_{D} f_n d\Qbb_n} \\
T_2(\lbf; G_{n,r}) & = \sup_{f_n \in \Wset_n}\abs{\int_{D} f_n d\Pbb_n - \int_{D} f_n d\Qbb_n}
\end{align*}
where $TV_n = \set{\fbf: \norm{\Bbf \fbf}_1 \leq C_{n,r}}$ and $\Wset_n = \set{\fbf: \norm{\Bbf \fbf}_2 \leq C_{n,r}}$.



\section{Consistency under fixed alternative}

\paragraph{Binomialized data model.}
For technical reasons, we would like $z_1, \ldots, z_n$ to be independent and identically distributed. We consider the following generative model, which we term the \emph{binomialized data model}: 

Fix $n \in \Naturals > 0$, and $n_1 \sim \Bin(n,1/2),~ n_2 = n - n_1$. Then, let $x_1, \ldots, x_{n_1} \in \Rd$ be a sequence of i.i.d random points chosen according to $\Pbb$, and $y_1, \ldots, y_{n_2} \in \Rd$ a separate sequence of i.i.d random points chosen according to $\Qbb$, with $x_j \perp y_k$ for all $j,k$. Fix $\widetilde{\Zbf} = (\widetilde{z}_1,\ldots,\widetilde{z}_n) := (x_1, \ldots, x_{n_1}, y_1, \ldots, y_{n_2})$.  Finally, for a permutation $\pi: [n] \to [n]$ chosen uniformly at random among all such permutations, let $\Zbf = (z_1,\ldots,z_n) =  (\widetilde{z}_{\pi(1)}, \ldots, \widetilde{z}_{\pi(n)})$. 

The label vector $\lbf$ remains defined as in \eqref{eqn: label_vector} with respect to $\Zbf$. Note that now $z_i \overset{i.i.d}{\sim} \frac{\Pbb}{2} + \frac{\Qbb}{2}$, as we desired, with density function $\mu(x) := \frac{p(x) + q(x)}{2}$.

\begin{theorem}[Pointwise limit of Laplacian smooth test statistic.]
	\label{thm: LLN_for_t2}
	Let $d \geq 2$ and let $\Dset \subset \Rd$ be an open, bounded, connected set with Lipschitz boundary. Let $\mu$ satisfy
	\begin{equation*}
	m \leq \mu(x) \leq M \tag{$\forall x \in D$}
	\end{equation*}
	for some $0 < m \leq M$. Let $(r_n)$ be a sequence of positive numbers converging to $0$ and satisfying
	\begin{align*}
	& \lim_{n \to \infty} \frac{(\log n)^{3/4}}{n^{1/2}} \frac{1}{r_n} = 0 ~~ \text{if $d = 2$} \\
	& \lim_{n \to \infty} \frac{(\log n)^{1/d}}{n^{1/d}} \frac{1}{r_n} = 0 ~~ \text{if $d \geq 3$}
	\end{align*}
	Assume the kernel $k$ satisfies conditions: 
	\begin{align*}
	& k(0) > 0 ~ \text{and $k$ is continuous at $0$.} \tag{\textbf{K1}} \\
	& k~ \text{is non-increasing.} \tag{\textbf{K2}} \\
	& \text{The integral $\int_{0}^{\infty} k(r) r^{d+1} dr$ is finite.} \tag{\textbf{K3}}
	\end{align*}
	
	Then with probability one the following statement holds: For $(z_1, \ldots, z_n)$ chosen under the binomialized data model,
	\begin{equation*}
	\lim_{n \to \infty} T_2(\lbf; G_{n,r_n}) = \gamma_{\Wset^{1,2}(\Dset,\mu^2)}(\Pbb,\Qbb).
	\end{equation*}
\end{theorem}

\section{Proofs}

Fix $\mu_n = \frac{d\Pbb_n + d\Qbb_n}{2}$. \textcolor{red}{We will show a variational form of convergence of $\mu_n$ to $\mu$.}

\subsection{Gamma convergence of constraint}

\begin{definition}[$TL^2$ convergence]
	Denote by $\mathfrak{B}(\Dset)$ the Borel $\sigma$-algebra of $\Dset$ and $\mathcal{P}(\Dset)$ the set of all Borel probability measures on $\Dset$. Given a Borel map $T: \Dset \to \Dset$, the \emph{push-forward} of $\mu$ by $T$ is given by
	\begin{equation*}
	T_{\star}\mu(\Aset) := \mu(T^{-1}(\Aset)), ~~ \Aset \in \mathfrak{B}(\Dset).
	\end{equation*}
	
	Given $\widetilde{\mu} \in \mathcal{P}(\Dset)$, we say that $T$ is a \emph{transportation map} between $\mu$ and $\widetilde{\mu}$ if $T_{\star}\mu = \widetilde{\mu}$. If for a sequence $(T_n)$ of transportation maps
	\begin{equation*}
	\int_{\Dset} \abs{x - T_n(x)}^2 d\mu(x) \to 0,~ \text{as $n \to \infty$}
	\end{equation*}
	we refer to the sequence as \emph{stagnating.} 
	
	Take $f \in L^2(\mu)$ and a sequence $(f_n)$ with $f_n \in L^2(\mu_n)$ for $n = 1,2,\ldots$. If there exists a stagnating sequence of transportation maps $(T_n)$ such that
	\begin{equation*}
	\int_{\Dset} \abs{ f - T_n(f_n(x)) }^2 d\mu(x) \to 0,~ \text{as $n \to \infty$}
	\end{equation*}
	we say that $(f_n)$ converges $TL^2$ to $f$, and write $f_n \overset{TL^2}{\to} f$.
\end{definition}

We restate Theorem 1.4 of \cite{trillos2018}, changing notation to match the rest of this paper.

\begin{theorem}[Theorem 1.4 of \cite{trillos2018}]
	\label{thm: gamma_convergence_energy_functional}
	Under the setup and conditions of Theorem \ref{thm: LLN_for_t2}, with probability one the following statements hold:
	\begin{itemize}
		\item \emph{Liminf inequality:}
		For all $f \in L^2(\mu)$ and all sequences $(f_n)$ with $f_n \in L^2(\mu_n)$ and $f_n \overset{TL^2}{\to} f$,
		\begin{equation*}
		\liminf_{n \to \infty} \frac{1}{n^2 r_n^{d + 2}} \norm{Bf}_2^2 \geq \textcolor{red}{\sigma_{k}} \norm{f}_{1,2,\rho^2}
		\end{equation*}
		
		\item \emph{Limsup inequality:} For all $f \in L^2(\mu)$, there exists a sequence $(f_n)$ with $f_n \in L^2(\mu_n)$ and $f_n \overset{TL^2}{\to} f$ such that
		\begin{equation*}
		\limsup_{n \to \infty} \frac{1}{n^2 r_n^{d + 2}} \norm{\Bbf \fbf_n }_2^2 \leq \textcolor{red}{\sigma_{k}} \norm{f}_{1,2,\rho^2}
		\end{equation*}
		where $\fbf_n = (f_n(z_1), \ldots, f_n(z_n))$.
		
		\item \emph{Compactness property:} Every sequence $(f_n)$ with $f_n \in L^2(\mu_n)$ satisfying
		\begin{equation*}
		\sup_{n \in \Naturals} \frac{1}{n^2 r_n^{d + 2}} \norm{Bf}_2^2 < \infty
		\end{equation*}
		is precompact in $TL^2$, that is, every subsequence of $(f_n)$ has a further subsequence which converges in the $TL^2$-sense to an element of $L^2(\Dset)$.
	\end{itemize}
	
\end{theorem}

\subsection{Continuity of risk functional}



\begin{lemma}
	\label{lem: continuity_of_risk_functional}
	With probability one the following statement holds: If a sequence $(f_n)$ where $f_n: \set{z_1, \ldots,z_n} \to [-1,1]^{n}$ converges $TL^2$ to $f \in L^2(\mu)$, then
	\begin{equation*}
	\lim_{n \to \infty} \frac{1}{n} \sum_{k = 1}^{n} \ell_k f_n(z_k) = \int_{\Dset} f d\Pbb(x) - \int_{\Dset} fd\Qbb(x).
 	\end{equation*}
\end{lemma}

\subsection{Proof of Theorem \ref{thm: LLN_for_t2}}

Let $f^{\star}$ be the witness function for $\gamma_{\Wset^{1,2}(\Dset,\mu^2)}(\Pbb,\Qbb)$, meaning
\begin{equation*}
\abs{\int_{\Dset}f^{\star}d\Pbb - \int_{\Dset}f^{\star}d\Qbb} = \gamma_{\Wset^{1,2}(\Dset,\mu^2)}(\Pbb,\Qbb)
\end{equation*}
where $f^{\star} \in \Wset^{1,2}(\Dset,\mu^2)$ implies $f^{\star} \in L^2(\mu)$. By the limsup inequality in Theorem \ref{thm: gamma_convergence_energy_functional}, there exists some $(f_n) \overset{TL^2}{\to} f$ with $f_n \in L^2(\mu_n)$ such that
\begin{equation}
\label{eqn: limsup_discrete_energy}
\limsup_{n \to \infty} \frac{1}{n^2 r_n^{d+2}} \norm{\Bbf \fbf_n}_2^2 \leq \textcolor{red}{\sigma_k} \norm{f}_{1,2,\mu^2} \leq \sigma_k.
\end{equation} 
From Lemma \ref{lem: continuity_of_risk_functional}
\begin{equation*}
\lim_{n \to \infty} \frac{1}{n} \sum_{k = 1}^{n} \ell_k f_n(z_k) = \int_{\Dset} f d\Pbb(x) - \int_{\Dset} f d\Qbb(x) = \gamma_{\Wset^{1,2}(\Dset,\mu^2)}.
\end{equation*}
and along with \eqref{eqn: limsup_discrete_energy}, this implies
\begin{equation*}
\lim_{n \to \infty} T_2(\lbf; G_{n,r}) \leq \gamma_{\Wset^{1,2}(\Dset,\mu^2)}(\Pbb,\Qbb).
\end{equation*}

Let $\Lbf^{\dagger}$ be the pseudoinverse of the Laplacian matrix $\Lbf$. We introduce $f_n^{\star} = \lbf^T \Lbf^{\dagger} \lbf$, which satisfies
\begin{equation*}
\abs{\int_{\Dset}f_n^{\star}d\Pbb_n - \int_{\Dset}f_n^{\star}d\Qbb_n} = T_2(\lbf; G_{n,r})
\end{equation*}
Note that
\begin{equation*}
\norm{\Bbf \fbf_n^{\star}}_2^2 \leq C_n \Longrightarrow \frac{1}{n^2 r_n^{d+2}} \norm{\Bbf \fbf_n^{\star}}_2^2  \leq \sigma_k < \infty
\end{equation*}
and so by the compactness property in Theorem \ref{thm: gamma_convergence_energy_functional}, every subsequence of $f_n^{\star}$ has a further subsequence which is $TL^2$-convergent. For simplicity, and without loss of generality, let us work along a subsequence which is convergent (and call it $f_n^{\star}$) to $f \in L^2(\mu)$. Then
\begin{equation}
\label{eqn: lower_bound_T2_1}
\lim_{n \to \infty} \frac{1}{n} \sum_{k = 1}^{n} \ell_k f_n^{\star}(z_k) = \abs{\int_{\Dset}f d\Pbb(x) - \int_{\Dset}f d\Qbb(x)}
\end{equation}
and by the liminf inequality of Theorem \ref{thm: gamma_convergence_energy_functional},
\begin{equation*}
\liminf_{n \to \infty} \frac{1}{n^2 r_n^{d+2}} \norm{\Bbf \fbf_n}_2^2 \geq \sigma_k \norm{f}_{1,2,\mu^2}
\end{equation*}
which implies $\norm{f}_{1,2,\mu^2} \leq 1$. This, along with \eqref{eqn: lower_bound_T2_1}, implies
\begin{equation*}
\lim_{n \to \infty} \frac{1}{n} \sum_{k = 1}^{n} \ell_k f_n^{\star}(z_k) \geq \gamma_{\Wset^{1,2}(\Dset,\mu^2)}(\Pbb,\Qbb)
\end{equation*}

\clearpage

\bibliography{../graph_testing_bibliography}
\bibliographystyle{plain}

\end{document}